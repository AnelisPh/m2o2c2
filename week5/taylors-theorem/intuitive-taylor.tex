\documentclass{ximera}
\title{Intuitively}

\begin{document}
	\begin{abstract}
		The second derivative allows us to approximate functions better than just the first derivative
	\end{abstract}
	
	\begin{question}
	Let $f:\R^2 \to \R$ be a function.  All we know about $f$ at the point $(1,2)$ is the following:
		\begin{itemize}
			\item $f(1,2) = 6$
			\item \(Df(1,2) = \begin{bmatrix} 4 & 5 \end{bmatrix}\)
			\item \( D^2f(1,2)  = \begin{bmatrix} 1 & -2 \\ -2 & 3\end{bmatrix}\)
		\end{itemize} 
		
		Suppose that we want to approximate $f(1.1,2.2)$ as accurately as we can given this information.
		We can simply use the linear approximation to $f$ at $(1,2)$:
		
		\begin{solution}
			\begin{hint}
				\begin{align*}
					f(1.1,2.2 &\approx 6+\begin{bmatrix} 4 & 5 \end{bmatrix} \verticalvector{0.1\\0.2}\\
							&=6+0.4+1\\
							&=7.4
				\end{align*}
			\end{hint}
			Using the linear approximation to $f$ at $(1,2)$, $f(1.1,2.2) \approx$ \answer{$7.4$}
		\end{solution}
		
		This approximation ignores the second order data provided by the second derivative: we have essentially 
		assumed that the first derivative is constant along the line from $(1,2)$ to $(1.1,2.2)$.  Since we know the second 
		derivative at the point $(1,2)$ we can estimate how the derivative is changing along this line, and get better approximations.
		
		For example, we could use the following three step process:
		
		\begin{itemize}
			\item Use the linear approximation to $f$ at $(1,2)$ to approximate $f(1.05,2.1)$
			\item Use the second derivative to approximate $Df(1.05,2.1)$
			\item Use the linear approximation to $f$ at $(1.05,2.1)$ to approximate $f(1.1,2.2)$
		\end{itemize}
		
		\begin{solution}
			\begin{hint}
				\begin{align*}
					f(1.05,2.1 &\approx 6+\begin{bmatrix} 4 & 5 \end{bmatrix} \verticalvector{0.05\\0.1}\\
							&=6+0.2+0.5\\
							&=6.7
				\end{align*}
			\end{hint}
			Let's try that here:  $f(1.05,2.1) \approx$ \answer{$6.7$}
		\end{solution}
		\begin{solution}
			\begin{hint}
				\begin{align*}
					Df(1.05,2.1) &\approx Df(1,2)+\begin{bmatrix} 0.05 & 0.1\end{bmatrix}\begin{bmatrix} 1 & -2 \\ -2 & 3\end{bmatrix}\\
										&= \begin{bmatrix} 4 & 5 \end{bmatrix} + \begin{bmatrix} 0.05(1)+0.1(-2) & 0.05(-2)+0.1(3)\end{bmatrix}\\
										&=\begin{bmatrix} 3.85 & 4.8\end{bmatrix}
				\end{align*}
			\end{hint}
			Using the second derivative, $Df(1.05,2.05) $ is approximately:
			\begin{matrix-answer}
				correctMatrix = [['3.85','4.8']]
			\end{matrix-answer}
		\end{solution}
		\begin{solution}
				\begin{hint}
				\begin{align*}
					f(1.1,2.2) &\approx 6.45+\begin{bmatrix} 3.85 & 4.8 \end{bmatrix} \verticalvector{0.05\\0.1}\\
							&=6.7+3.85(0.05)+4.8(0.1)\\
							&=7.1225
				\end{align*}
				\end{hint}
			Using the linear approximation to $f$ at $(1.05,2.1)$, $f(1.1,2.2) \approx$ \answer{$7.1225$ }
		\end{solution}
		
		So this method allowed us to get a slightly better approximation of $f(1.1,2.2)$ using the fact that the derivative is decreasing
		in the direction $\verticalvector{1\\2}$. 
		
		We do not have to limit ourselves to only using a two step approximation:  we could get better and better approximations of $f(1.1,2.2)$
		by using more and more partitions of the line segment from $(1,2)$ to $(1.1,2.2)$.  For example, we could use ten partitions by
		\begin{itemize}
			\item Use the linear approximation to $f$ at $(1,2)$ to approximate $f(1.01,2.02)$
			\item Use the second derivative to approximate $Df(1.01,2.02)$
			\item Use the linear approximation to $f$ at $(1.01,2.02)$ to approximate $f(1.02,2.04)$
			\item Use the second derivative to approximate $Df(1.02,2.04)$
			\item Use the linear approximation to $f$ at $(1.02,2.04)$ to approximate $f(1.03,2.06)$
			\item $\vdots$
			\item  Use the linear approximation to $f$ at $(1.09,2.18)$ to approximate $f(1.1,2.2)$
		\end{itemize}
		
		This kind of process, where we are summing more and more of smaller and smaller values to approximate something, 
		furiously demands to be phrased as an integral.
		
		\begin{solution}
			\begin{hint}
				Notice that \begin{align*}
					D^2f\left(\right)\verticalvector{0.1\frac{1}{n}\\0.2\frac{1}{n}},\verticalvector{0.1\frac{1}{n}\\0.2\frac{1}{n}}\right)
					&=\begin{bmatrix} 0.1 \frac{1}{n}, 0.2\frac{1}{n}\end{bmatrix}  \begin{bmatrix} 1 & -2 \\ -2 & 3\end{bmatrix} \verticalvector{0.1\frac{1}{n}\\0.2\frac{1}{n}}\\
					&=\left(0.1(0.1)(1)+0.1(0.2)(-2)+(0.1)(0.2)(-2)+0.2(0.2)(3)\right)\frac{1}{n^2}\\
					&=0.05\frac{1}{n^2}
				\end{align*}
			\end{hint}
			\begin{hint}
				By partitioning $[0,1]$ into $n$ little pieces of equal width, the contribution to the sum 
					\begin{itemize}
						\item over $[0,\frac{1}{n}]$ is \(Df(1,2)\left( 0.1\frac{1}{n} \\ 0.2\frac{1}{n}\right) = \begin{bmatrix} 4 &5\end{bmatrix} \verticalvector{0.1\frac{1}{n} \\ 0.2\frac{1}{n}} = 1.4\frac{1}{n}\)
						\item over $[\frac{1}{n}, \frac{2}{n}]$ is 
							\begin{align*}
								Df(1+0.1\frac{1}{n},2+0.2\frac{1}{n})\left( \verticalvector{0.1\frac{1}{n}\\0.2\frac{1}{n}} \right) 
								&\approx Df(1,2)\left( \verticalvector{0.1\frac{1}{n}\\0.2\frac{1}{n}}+D^2f\left(\right)\verticalvector{0.1\frac{1}{n}\\0.2\frac{1}{n}},\verticalvector{0.1\frac{1}{n}\\0.2\frac{1}{n}}\right)\\
								&= 1.4\frac{1}{n} + 0.05\frac{1}{n^2}
							\end{align*}
						\item over $[\frac{2}{n},\frac{3}{n}]$ is
							\begin{align*}
								Df(1+2(0.1\frac{1}{n}),2+2(0.2\frac{1}{n}))\left( \verticalvector{0.1\frac{1}{n}\\0.2\frac{1}{n}} \right) 
								&approx  Df(1+0.1\frac{1}{n},2+0.2\frac{1}{n})\left( \verticalvector{0.1\frac{1}{n}\\0.2\frac{1}{n}} \right) +D^2f\left(\right)\verticalvector{0.1\frac{1}{n}\\0.2\frac{1}{n}},\verticalvector{0.1\frac{1}{n}\\0.2\frac{1}{n}}\right)\\
								&approx 1.4\frac{1}{n} + 0.5\frac{1}{n^2}+ 0.5\frac{1}{n^2}\\
								&=1.4\frac{1}{n}+0.05\frac{2}{n}\frac{1}{n}
							\end{align*}
							\item $\vdots$
							\item over $[\frac{k+1}{n},\frac{k+2}{n}]$ is
								\begin{align*}
								Df(1+(k+1)(0.1\frac{1}{n}),2+(k+1)(0.2\frac{1}{n}))\left( \verticalvector{0.1\frac{1}{n}\\0.2\frac{1}{n}} \right) 
								&approx  Df(1+(k)0.1\frac{1}{n},2+(k)0.2\frac{1}{n})\left( \verticalvector{0.1\frac{1}{n}\\0.2\frac{1}{n}} \right) +D^2f\left(\right)\verticalvector{0.1\frac{1}{n}\\0.2\frac{1}{n}},\verticalvector{0.1\frac{1}{n}\\0.2\frac{1}{n}}\right)\\
								&approx 1.4\frac{1}{n} + (k-1)0.05\frac{1}{n^2}+ 0.05\frac{1}{n^2}\\
								&=1.4\frac{1}{n}+0.05\frac{k}{n}\frac{1}{n}
							\end{align*}
					\end{itemize}
				
			\end{hint}
			\begin{hint}
				So \(f(1.1,2.2) \approx 6+ \displaystyle\sum_{k=0}^{n} 1.4\frac{1}{n} +0.05\frac{k}{n}\frac{1}{n} \)
			\end{hint}
			\begin{hint}
				By definition of the integral we have \(\displaystyle\lim_{n \to \infty} \displaystyle\sum_{k=0}^{n} 1.4\frac{1}{n} +0.05\frac{k}{n}\frac{1}{n} = \displaystyle\int_0^1 (1.4+0.05t) dt\)
			\end{hint}
			In this case, we get that $f(1.1,2.2) \approx 6+ \displaystyle\int_0^{1} f(t) dt$, where $f(t)=$\answer{ $1.4+0.05t$}
		\end{solution}
		
		\begin{solution}
			\begin{hint}
				\begin{align*}
					f(1.1,2.2) &\approx 6+ \displaystyle\int_0^1 (1.4+0.05t) dt\\
						&=6+\left(1.4t+\frac{1}{2}(0.05)t^2 \right)\big|_{0}^{1}\\
						&=6+1.4+0.025\\
						&=7.425
				\end{align*}
			\end{hint}
			Evaluating this integral we have $f(1.1,2.2) \approx $ \answer{ $7.425$}.  This is the best approximation we can really expect to get given only this
			information about $f$ at $(1,2)$.
		\end{solution}
		
		Notice that this approximation of $f$ is really just 
		$f(1,2)+Df(1,2) \left(\verticalvector{0.1\0.2} \right)+ \frac{1}{2} D^2f(1,2)\( \verticalvector{0.1\\0.2}, \verticalvector{0.1\\0.2}\\)$.
		
		The first two terms are just the regular linear approximation to $f$ at $(1,2)$, but the next term arose from integrating the function 
		$D^2f( \verticalvector{0.1t\\0.2t},\verticalvector{0.1t\\0.2t})$ from $t=0$ to $t=1$.  This is  exactly
		$\displaystlye \int D^2f(\verticalvector{0.1\\0.2},\verticalvector{0.1\\0.2}) t = \frac{1}{2} D^2f(\verticalvector{0.1\\0.2},\verticalvector{0.1\\0.2})$.
		
		Generalizing, we might expect in general that 
		
		\begin{theorem}
			\(f(\mathbf{p} + \vec{h}) \approx f(\mathbf{p}) + Df(\mathbf{p})(\vec{h})+ \frac{1}{2} D^2f(\mathbf{p})\left( \vec{h},\vec{h}\right) \)
		\end{theorem}
		
		This is the second order taylor approximation of $f$ at $\mathbf{p}$. 
		
		Hopefully this (admittedly long) discussion has helped you to understand where this approximation comes from!  We will give a rigorous statement
		and proof of the theorem in the next section.
		
	\end{question}
\end{document}