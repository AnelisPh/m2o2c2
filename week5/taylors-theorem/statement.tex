\documentclass{ximera}
\title{Rigorously}

\begin{document}
	\begin{abstract}
		The second derivative enables quadratic approximation
	\end{abstract}
	
	You should know the statement of the following theorem for this course:
	
	\begin{theorem}[Second order Taylor's theorem]
		If $f:\R^n \to \R$ is a twice differentiable function, $\mathbf{p} \in \R^n$ then we have
		\[
			f(\mathbf{p} + \vec{h}) = f(\mathbf{p})+ Df(\mathbf{p})\vec{h}+\frac{1}{2}D^2f(\mathbf{b+\xi\vec{h}})(\vec{h},\vec{h})
		\]
		
		for some $\xi \in [0,1]$.
		
		It follows (after a lot of work!) that
		
		\[
			f(\mathbf{p} + \vec{h}) = f(\mathbf{p})+ Df(\mathbf{p})\vec{h}+\frac{1}{2}D^2f(\mathbf{p})(\vec{h},\vec{h})+ \textrm{Error}(\vec{h})
		\]
		
		with $\displaystyle\lim_{\vec{h} \to \vec{0}} \frac{|\textrm{Error}(\vec{h})|}{|\vec{h}|^2} = 0$
	\end{theorem}
	
	This approximation is also sometimes phrased as 
	$f(\mathbf{x}) \approx f(\mathbf{p}) + Df(\mathbf{p})(\mathbf{x}-\mathbf{p})+D^2f(\mathbf{p})(\mathbf{x}-\mathbf{p},\mathbf{x}-\mathbf{p})$
	
	The rest of this section consists of a proof of the above theorem.  You do not have to know the proof, and you can safely skip it at first if you want.  
	It might even be better to wait until
	the end of the week to read this proof, if you want to.  If you do want to read the proof, you should at a minimum make sure you have already worked through 
	the other two optional sections on \href{http://ximera.osu.edu/course/kisonecat/m2o2c2/course/activity/week2/limits/formal-limit/}{the formal definition of a limit} 
	and also \href{http://ximera.osu.edu/course/kisonecat/m2o2c2/course/activity/week2/chain-rule/proof/}{ the proof of the chain rule} where operator norms are introduced.
	
	\begin{proof}
	\end{proof}
	
	
	
	
	
\end{document}