\documentclass{ximera}
\title{Rigorously}

\begin{document}
	\begin{abstract}
		The second derivative allows approximations to second order accuracy.
	\end{abstract}
	
	\begin{theorem}
	If $f$ has continuous second partial derivatives everywhere, then 
	\[
		D^2f|_\mathbf{p}  = \Sum_{i,j=1}^n \frac{\partial^2 f}{\partial x^i \partial x^j} dx_i \otimes dx_j
	\]
	
	The matrix associated to this bilinear form is the matrix:
	
		\[ 
		
		\mathcal{H}(f)\big|_\mathbf{p} = [\frac{\partial^2 f}{\partial x_i \partial x_j}]
	
		\]
		
		and is called the Hessian matrix of $f$ at $\mathbf{p}$.
\end{theorem}

\begin{question}
	If $f:\R^n \to \R$ is a function and $Df = BLAH$ what is the hessian matrix of $f$?
\end{question}

\begin{question}
	What is the hessian of $f(x,y) = x\sin(xy)$?
\end{question}

In working out Hessians of various functions, you might have noticed that the matrix you get is symmetric, i.e.  that ``mixed partials commute'' 
$frac{\partial^2 f}{\partial x^i \partial x^j}$

\begin{theorem}
	If $f$ has continuous second partial derivatives, then $D^2f$ is a symmetric bilinear form.
\end{theorem}
 
\end{document}