\documentclass{ximera}

%\newcommand{\R}{R}
%\usepackage{amsmath}
%\newcommand{\verticalvector}[1]{\begin{bmatrix}#1\end{bmatrix}}
%\newenvironment{definition}{}{}
%\newcommand{\answer}[1]{#1}
%\newenvironment{question}{}{}
%\newenvironment{matrix-answer}{}{}
%\newenvironment{hint}{}{}
%\newenvironment{solution}{}{}
%\newenvironment{theorem}{}{}
%\newenvironment{multiple-choice}{}{}
%\newcommand{\choice}{}

\title{Taylor series}

\begin{document}

\begin{abstract}
  The second derivative allows us to approximate functions better than just the first derivative
\end{abstract}\maketitle

As it stands, the second derivative lets us get approximations of the first derivative.  The first derivative allows us to get approximations of the original function.
In the following extended question, we will see how we can use the second derivative to get more information about the first derivative, which then lets us get more
information about the original function.  This will lead to approximations with second order accuracy, rather than just first order accuracy.  This is the essence of 
the second order Taylor's theorem.

\end{document}
