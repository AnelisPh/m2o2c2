\documentclass{ximera}
\title{Intuitively}

\begin{document}

\begin{abstract}
	The second derivative is a bilinear form.
\end{abstract}\maketitle
	
	From our perspective, the second derivative of a function $f: \R^n \to \R$ at a point will be a bilinear form on $\R^n$.  
	Let us take some time to understand, intuitively, why that should be the case.
	
	Let $f:\mathbb{R}^2 \to \mathbb{R}$ be defined by $f(x,y) = x^2y$.

$D(f)\big|_{(x,y)}$ is the linear map given by the matrix $\left[ \begin{matrix} 2xy&x^2\end{matrix} \right]$.  
That is to say, $D(f)\big|_{(x,y)}(\verticalvector{\Delta x\\\Delta y}) = 2xy\Delta x + x^2\Delta y \approx f(x+\Delta x,y+\Delta y) - f(x,y)$.  

The second derivative should now tell you how much the derivative changes from point to point.  
If we increment $(x,y)$ by a little bit to $(x+\Delta x,y)$ then we should expect the derivative to 
increase by about $\begin{bmatrix} \Delta x \frac{\partial}{ \partial x} ( 2xy) & \Delta x\frac{\partial}{\partial x}(x^2) \end{bmatrix} = \begin{bmatrix} 2y\Delta x&2x \Delta x\end{bmatrix}$.  
Similarly, when we increase $y$ by $\Delta y$, the derivative should change by about 
$\begin{bmatrix} \Delta y \frac{\partial}{ \partial y} ( 2xy) & \Delta y\frac{\partial}{\partial y}(x^2) \end{bmatrix} = \begin{bmatrix} 2x \Delta y&0\Delta y\end{bmatrix}$.

By linearity, if we change from $(x,y)$ to $(x+\Delta x,y+\Delta y)$, 
we expect the derivative to change by
 \[\begin{bmatrix} 2y\Delta x + 2x\Delta y & 2x\Delta x + 0 \end{bmatrix}  = \begin{bmatrix} \Delta x&\Delta y\end{bmatrix} \begin{bmatrix} 2y&2x\\2x&0\end{bmatrix} \]

This gives a matrix which is the approximate change in the derivative.  You can then apply this to another vector if you so wish.  

Summing it up, if you wanted to see approximately how much the derivative changes from $p = (x,y)$ to $(x+\Delta x_2,y+\Delta y_2) = p+\vec{h_2}$ 
($\vec{h_2} = \verticalvector{\Delta x_2\\\Delta y_2}$)
when both are evaluated in the same direction  $\vec{h_1} = \verticalvector{\Delta x_1\\\Delta y_1}$, you would perform the computation:

\[ 
Df_{p+\vec{h_2}}(\vec{h_1})  -Df_p(\vec{h_1}) \approx
\begin{bmatrix} 
\Delta x_2 & \Delta y_2\end{bmatrix} 
 \begin{bmatrix} 2x&2x\\2x&0\end{bmatrix} \begin{bmatrix} \Delta x_1\\ \Delta y_1\end{bmatrix}
 \]
 
 This is exactly using the matrix $\begin{bmatrix} 2x&2x\\2x&0\end{bmatrix}$ as a bilinear form applied to the two vectors 
 $\vec{h_1} = \verticalvector{\Delta x_1\\ \Delta y_1}$ and $\vec{h_2} = \verticalvector{\Delta x_2\\ \Delta y_2}$.
 
With all of this as motivation, we make the following wishy washy "definition"

\begin{definition}
	The second derivative of a function $f:\R^n \to \R$ at a point $\mathbf{p}\in \R^n$  
	is a bilinear form $D^2f\big|_\mathbf{p} : \R^n \times \R^n \to \R$ enjoying the following approximation property:
	
	\[
		Df\big|_{\mathbf{p}+\vec{h_1}}(\vec{h_2})  \approx Df\big|_\mathbf{p}(\vec{h_2}) + D^2f\big|_\mathbf{p} (\vec{h}_1,\vec{h}_2)
	\]
	
\end{definition} 

We will make the sense in which this approximation holds precise in another section, but for now this is good enough.

\begin{question}
If $f: \R^2 \to \R$ is a function, and $Df\big|_{\mathbf{(1,2)}}
= \begin{bmatrix} -1 & 1\end{bmatrix}$ and $Hf\big|_{\mathbf{(1,2)}} = \begin{bmatrix} 0 & 4\\4 &
    1\end{bmatrix}$. $Df\big|_{(1.2,1.1)}(\verticalvector{-0.2\\0.3})$.
    \begin{solution}
      \begin{hint}
	By the fundamental approximating property, we have
        \(Df\big|_{\mathbf{(1,2)}+\verticalvector{0.2\\0.1}}(-0.2\\0.3)
          \approx Df\Big|_\mathbf{(1,2)}{\verticalvector{-0.2\\0.3}} +
          D^2f\big|_\mathbf{(1,2)} \left(
          \verticalvector{0.2\\0.1},\verticalvector{-0.2\\0.3}\right)\)
      \end{hint}
      \begin{hint}
	Thus \(Df\big|_{\mathbf{(1,2)}+\verticalvector{0.2\\0.1}}(-0.2\\0.3)  \approx
	  \begin{bmatrix}  -1 & 1\end{bmatrix}  \verticalvector{-0.2\\0.3}
	    + \begin{bmatrix} 0.2\\0.1\end{bmatrix} \begin{bmatrix} 0 & 4\\4 & 1\end{bmatrix} \verticalvector{-0.2\\0.3}\)
      \end{hint}
      \begin{hint}
	\begin{align*}
	  Df\big|_{(1.2,1.1)}(\verticalvector{-0.2\\0.3}) &\approx \begin{bmatrix}  -1 & 1\end{bmatrix}  \verticalvector{-0.2\\0.3}
	    + \begin{bmatrix} 0.2\\0.1\end{bmatrix} \begin{bmatrix} 0 & 4\\4 & 1\end{bmatrix} \verticalvector{-0.2\\0.3}\\
		&=-1(-0.2)+1(0.3)+
		+ (0.2)(0)(-0.2)+(0.2)(4)(0.3)+(0.1)(4)(-0.2)+(0.1)(1)(0.3)\\
		&= 0.2+0.3+0+0.12-0.08+0.03\\
		&=0.57
	\end{align*}	
      \end{hint}
      $Df\big|_{(1.2,1.1)}(\verticalvector{-0.2\\0.3}) \approx$ \answer{0.57}
    \end{solution}
		
		Note that the computation really splits into a first order change ($Df|_p(\vec{h})$) and a second order change ($D^2f(\vec{h_1},\vec{h_2})$).  In this case
		the first order change was $0.5$, and the second order change was $0.07$.  
		This should be a better approximation to the real value than if we had used the first derivative alone.
\end{question}

\begin{question}
	Let $f:\R^2 \to \R$ be a function with  \(Df|_p =  \begin{bmatrix}  3 & 4\end{bmatrix}\) and 
	\(Hf|_p = \begin{bmatrix}  1 & 3 \\ 3 & -2\end{bmatrix}\), approximate $Df_{p+\verticalvector{0.01\\0.02}}$.
	\begin{solution}
		\begin{hint}
			By the fundamental approximation property, \(Df_{p+\verticalvector{0.01\\0.02}}(\vec{h}) \approx Df_{p}(\vec{v})+\begin{bmatrix} 0.01 & 0.02\end{bmatrix} Hf|_p \vec{v} \).  So
			$Df_{p+\verticalvector{0.01\\0.02}} \approx Df_{p}+\begin{bmatrix} 0.01 & 0.02\end{bmatrix} Hf|_p$ as linear maps from $\R^2 \to \R$
		\end{hint}
		\begin{hint}
			\begin{align*}
				\begin{bmatrix} 0.01 & 0.02\end{bmatrix} Hf|_p  &= \begin{bmatrix} 0.01 & 0.02\end{bmatrix} \begin{bmatrix}  1 & 3 \\ 3 & -2\end{bmatrix}\\
					&=\begin{bmatrix} 0.01(1)+0.02(3) & 0.01(3)+0.02(-2)\end{bmatrix}\\
					&=\begin{bmatrix} 0.07 & -0.01\end{bmatrix}
			\end{align*}
		\end{hint}
		\begin{hint}
			So $Df_{p+\verticalvector{0.01\\0.02}} \approx \begin{bmatrix}  3 & 4\end{bmatrix} + \begin{bmatrix} 0.07 & -0.01\end{bmatrix} = \begin{bmatrix} 3.07 & 3.99\end{bmatrix}$
		\end{hint}
	\begin{matrix-answer}
		correctMatrix = [['3.07','3.99']]
	\end{matrix-answer}
	\end{solution}
\end{question}


Following the development at the beginning of this activity, we can anticipate how to compute the second derivative as a bilinear form:

\begin{warning}
	This is an intuitive development, not a rigorous proof
\end{warning}

Let $f:\R^n \to \R$.

Since \[Df\big|_p =  \begin{bmatrix}  f_{x_1}(p) & f_{x_2}(p) & ...& f_{x_n}(p) \end{bmatrix}\] 

It is reasonable to think that
 \[ Df\big|_{\mathbf{p}+\vec{h}_1} \approx Df\big|_{\mathbf{p}}+\begin{bmatrix} Df_{x_1}(p)(\vec{h}_1) & Df_{x_2}(p)(\vec{h}_1) & ...& Df_{x_n}(p)(\vec{h}_1) \end{bmatrix}\]
 
 but \[Df_{x_i}(\vec{h_1})  = \begin{bmatrix} f_{x_1x_1}(p)  & f_{x_2x_1}(p) & ... & f_{x_nx_1}(p)\end{bmatrix} \vec{h_1}\]
 
 We can rewrite this as $\vec{h_1}^\top \verticalvector{ f_{x_1x_1}(p) \\ f_{x_2x_1}(p) \\ \vdots \\ f_{x_nx_1}(p)}$, so we obtain the rather pleasing formula
 
 
 \[ 
 Df\big|_{\mathbf{p}+\vec{h}_1} \approx Df\big|_{\mathbf{p}}+\vec{h_1}^\top
 \begin{bmatrix}
		 f_{x_1x_1}(p) &  f_{x_1x_2}(p) & ... &  f_{x_1x_n}(p)\\ f_{x_2x_1}(p) &  f_{x_2x_2}(p) & ... &  f_{x_2x_n}(p)\\ \vdots \\ f_{x_nx_1}(p) &  f_{x_nx_2}(p) & ... &  f_{x_nx_n}(p)
 \end{bmatrix} 
 \]
 
 
 So 
 
  \[ 
 Df\big|_{\mathbf{p}+\vec{h}_1}(\vec{h}_2) \approx Df\big|_{\mathbf{p}}(\vec{h}_2)+\vec{h_1}^\top
 \begin{bmatrix}
		 f_{x_1x_1}(p) &  f_{x_1x_2}(p) & ... &  f_{x_1x_n}(p)\\ f_{x_2x_1}(p) &  f_{x_2x_2}(p) & ... &  f_{x_2x_n}(p)\\ \vdots \\ f_{x_nx_1}(p) &  f_{x_nx_2}(p) & ... &  f_{x_nx_n}(p)
 \end{bmatrix} \vec{h}_2
 \]
 
 So it looks like we have:
 
 \begin{theorem}
 	If $f: \R^n \to \R$, the matrix of the bilinear form $D^2f\big|_p : \R^n \times \R^n \to \R$ is \[\begin{bmatrix}
		 f_{x_1x_1}(p) &  f_{x_1x_2}(p) & ... &  f_{x_1x_n}(p)\\ f_{x_2x_1}(p) &  f_{x_2x_2}(p) & ... &  f_{x_2x_n}(p)\\ \vdots \\ f_{x_nx_1}(p) &  f_{x_nx_2}(p) & ... &  f_{x_nx_n}(p)
 \end{bmatrix}\]
 
 This matrix is also called the \textbf{Hessian} matrix of $f$.
 
 We could also express this in the following convenient notation:
 
 \[D^2f\big|_p  = \sum_{i,j=1}^n \frac{\partial^2 f}{\partial x_i \partial x_j} dx_i \otimes dx_j\]
 \end{theorem}
 
 By the equality of mixed partial derivatives, this bilinear form is actually symmetric!  So all of the theory we developed about self adjoint linear operators and symmetric bilinear forms
 can (and will) be brought to bear on the study of the second derivative.
 
 \begin{question}
 	Let $f:\R^2 \to \R$ be defined by $f(x,y) = \frac{x}{y}$.  
 	\begin{solution}
 		\begin{hint}
 					\begin{question}
 						
 						\begin{solution}
 						\begin{hint}
 							\begin{align*}
 								f_{xx} &= \frac{\partial}{\partial x} \frac{\partial }{\partial x} \frac{x}{y}\\
 									&= \frac{\partial}{\partial x} \frac{1}{y}\\
 									&= 0
 							\end{align*}
 						\end{hint}
 							$f_{xx}=$\answer{$0$}
 						\end{solution}
 						\begin{solution}
 						\begin{hint}
 							\begin{align*}
 								f_{xy} &= \frac{\partial}{\partial x} \frac{\partial }{\partial y} \frac{x}{y}\\
 									&= \frac{\partial}{\partial x} \frac{-x}{y^2}\\
 									&= \frac{-1}{y^2}
 							\end{align*}
 						\end{hint}
 							$f_{xy}=$\answer{$-1/y^2$}
 						\end{solution}
 						\begin{solution}
 						\begin{hint}
 							\begin{align*}
 								f_{yx} &= f_{xy} \text{ by equality of mixed partials}\\
 									&= \frac{-1}{y^2}
 							\end{align*}
 						\end{hint}
 							$f_{yx}=$\answer{$-1/y^2$}
 						\end{solution}
 						\begin{solution}
 						\begin{hint}
 							\begin{align*}
 								f_{yy} &= \frac{\partial}{\partial y} \frac{\partial }{\partial y} \frac{x}{y}\\
 									&= \frac{\partial}{\partial y} \frac{-x}{y^2}\\
 									&= \frac{2x}{y^3}
 							\end{align*}
 						\end{hint}
 							$f_{yy}=$\answer{$2x/y^3$}
 						\end{solution}
 					\end{question}
 				\end{hint}
 		\begin{hint}
 					\(\mathcal{H} = \begin{bmatrix} 0 & \frac{-1}{y^2}\\ \frac{-1}{y^2} &  \frac{2x}{y^3} \end{bmatrix} \)
 				\end{hint}
 		What is the Hessian matrix of $f$ at the point $(x,y)$?
 			\begin{matrix-answer}
 				correctMatrix = [['0','-1/y^2'],['-1/y^2','2x/y^3']]
 			\end{matrix-answer}
 	\end{solution}
 \end{question}
 
 \begin{question}
 	Let $f:\R^3 \to \R$ be defined by $f(x,y,z) = xy+yz$.
 	\begin{solution}
 		\begin{hint}
 			The only second partials which are not zero are $f_{xy} = f_{yx}$ and $f_{yz} =f_{zy}$
 		\end{hint}
 		\begin{hint}
 			$f_{xy} = f_{yx} = 1$
 			
 			and
 		
 			$f_{yz} = f_{zy} = 1$
 		\end{hint}
 		\begin{hint}	
 			\(\mathcal{H} = \begin{bmatrix} 0 & 1&0\\1&0&1\\0&1&0\end{bmatrix}\)
 		\end{hint}
 		What is the Hessian matrix of $f$ at the point $(x,y,z)$?
 			\begin{matrix-answer}
 				correctMatrix = [['0','1','0'],['1','0','1'],['0','1','0']]
 			\end{matrix-answer}
 	\end{solution}
 		Notice that the second derivative of this function is the same at every point because $f$ was a quadratic function.  Any other polynomial of
 		degree $2$ in $n$ variables would also have a constant second derivative. For example $f(x,y,z,t) = 1+2x+3z+4z^2+zx+xt+t^2+yx$ would also have
 		constant second derivative.
 \end{question}
 
% Would be nice to have a video here where we explain how we can think of $Df: \R^n \to Hom(\R^n, \R)$, so $D^2f: \R^n \to Hom( \R^n, Hom(\R^n,\R))$
% Which then naturally corresponds to $D^2f: \R^n \to \R^n^* \otimes \R^n^*$
%
\end{document}
