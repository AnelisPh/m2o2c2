\documentclass{ximera}

\title{Optimization}

\begin{document}

\begin{abstract}
  Optimization means finding a biggest (or smallest) value.
\end{abstract}

Suppose $A$ is a subset of $\R^n$, meaning that each element of $A$ is
a vector in $\R^n$.  Maybe $A$ contains all vectors in $\R^n$, maybe
not.  Further suppose $f : A \to \R$ is a function.

\begin{question}
  Is there a vector $\vec{v} \in A$ so that $f(\vec{v})$ is at least
  as large as any other output of $f$?

  \begin{solution}
    \begin{multiple-choice}
      \choice{This is always the case.}
      \choice[correct]{This is not necessarily the case.}
    \end{multiple-choice}
  \end{solution}

  It really does depend on $A$ and on $f$.

  For example, suppose $f(\vec{v}) = \langle \vec{v}, \vec{v}
  \rangle$, meaning $f$ sends $\vec{v}$ to the square of the length of
  $\vec{v}$.  Further suppose that $A = \{ \vec{v} \in \R^n : |v| \le
  1 \}$.

  Then for all $\vec{v} \in A$, it is the case that $f(\vec{v}) \le
  1$.  And yet, there is not a single vector $\vec{v} \in A$ so that
  $f(\vec{v})$ is at least as large as all outputs of $f$.

  If you claim that you have found a vector $\vec{v}$ so that
  $f(\vec{v})$ is as large as any output of $f$, then you should consider the input
  $$
  \vec{w} = \frac{1 + |\vec{v}|}{2} \cdot \vec{v},
  $$
  and note that $f(\vec{w}) \gt f(\vec{v})$.
\end{question}

\hrule

\begin{question}
  Let's consider an example.  Let $g : \R^2 \to \R$ be the function given by
  $$
  g \left( \begin{bmatrix} x \\ y \end{bmatrix} \right) = 10 - (x+1)^2 - (y-2)^2.
  $$  

  \begin{solution}
    \begin{hint}
      No matter what $x$ is, $(x+1)^2 \geq 0$.
    \end{hint}

    \begin{hint}
      No matter what $y$ is, $(y-2)^2 \geq 0$.
    \end{hint}

    \begin{hint}
      No matter what $x$ and $y$ are, $(x+1)^2 + (y-2)^2 \geq 0$.
    \end{hint}

    \begin{hint}
      No matter what $x$ and $y$ are, $- (x+1)^2 - (y-2)^2 \leq 0$.
    \end{hint}

    \begin{hint}
      No matter what $x$ and $y$ are, $10 - (x+1)^2 - (y-2)^2 \leq 10$.
    \end{hint}

    \begin{hint}
      If $(x,y) = (-1,2)$, then $10 - (x+1)^2 - (y-2)^2 = 10$.
    \end{hint}

    \begin{hint}
      Consequently, the largest possible output of $g$ is $10$.
    \end{hint}

    The largest possible output of $g$ is \answer{$10$}.
  \end{solution}

  \begin{solution}
    This largest possible output occurs when $x$ is \answer{$-1$}.
  \end{solution}

  \begin{solution}
    This largest possible output occurs when $y$ is \answer{$2$}.
  \end{solution}

  In this case, we were able to think through the situation by
  considering some algebra---namely the fact that when we square a
  real number, the result is nonnegative.

  Here is the key idea that motivates everything we are about to do:
  \textbf{using the second derivative, we can approximate complicated
    functions by ``quadratic'' functions, and quadratic functions we
    can analyze just as we analyzed this example.}

\end{question}

\end{document}
