\documentclass{ximera}

\title{Definitions}

\begin{document}

\begin{abstract}
  ``Local'' means ``after restricting to a small neighborhood.''
\end{abstract}\maketitle

\begin{definition}
  Let $X \subset \R^n$ and $f : X \to \R$.  To say that the
  \textbf{maximum value} of $f$ occurs at the point $\mathbf{p} \in X$
  is to say that, for all $\mathbf{q} \in X$, we have $f(\mathbf{p})
  \geq f(\mathbf{q})$.

  Conversely, to say that the \textbf{minimum value} of $f$ occurs at
  the point $\mathbf{p} \in X$ is to say that, for all $\mathbf{q} \in
  X$, we have $f(\mathbf{p}) \leq f(\mathbf{q})$.

  Sometimes people use the term ``extremum value'' to speak of both
  maximum values and minimum values.  Sometimes people say ``maxima''
  instead of maximums and ``minima'' instead of minimums.
\end{definition}

A function need not achieve a maximum or a minimum value.

Our goal will be to use calculus to search for maximums and minimums,
but that raises a problem.  The derivative at a point is only
describing what is happening around that point, so if we use calculus
to search for extreme values, then we will only see ``local''
extremes.

\begin{definition}
  Let $X \subset \R^n$ and $f : X \to \R$.  To say that a
  \textbf{local maximum} of $f$ occurs at the point $\mathbf{p} \in X$
  is to say that there is an $\epsilon > 0$ so that for all
  $\mathbf{q} \in X$ within $\epsilon$ of $\mathbf{p}$, we have
  $f(\mathbf{p}) \geq f(\mathbf{q})$.

  Conversely, to say that a \textbf{local minimum} of $f$ occurs at
  the point $\mathbf{p} \in X$ is to say that there is an $\epsilon >
  0$ so that for all $\mathbf{q} \in X$ within $\epsilon$ of
  $\mathbf{p}$, we have $f(\mathbf{p}) \leq f(\mathbf{q})$.
\end{definition}

Here's an example of how this works out in practice.  Let $g : \R^2
\to \R$ be the function given by
$$
g(x,y) = x^2 + y^2 + y^3
$$
\begin{question}
  Does this function $g$ achieve a minimum value?

  \begin{solution}
    \begin{multiple-choice}
      \choice[correct]{No.}
      \choice{Yes.}
    \end{multiple-choice}
  \end{solution}

  That's correct: there is no ``global'' minimum.  No matter how
  negative you want the output to $g$ to be, you can achieve it by
  looking at $g(0,y)$ where $y$ is a very negative number.

  On the other hand, if we restrict our attention near the point
  $(0,0)$, then $g$ is nonnegative there.
  \begin{solution}
    Whenever $(x,y)$ is within \begin{expression-answer}
      function validator(f) {
        if (f.variables().length != 0) {
          feedback('Your answer should be a number.');
          return 0;
        }

        if (f.evaluate({}) <= 0) {
          feedback('Your answer should be nonnegative.');
          return 0;
        }

        if (f.evaluate({}) > 1) {
          feedback('Your answer cannot be larger than 1.');
          return 0;
        }
        
        return 1;
      }
    \end{expression-answer} of $(0,0)$, then $g(x,y) \geq g(0,0) = 0$.
  \end{solution}

  As a result, $g$ achieves a local minimum at $(0,0)$, in spite of
  the fact that there is no global maximum.
\end{question}

\end{document}
