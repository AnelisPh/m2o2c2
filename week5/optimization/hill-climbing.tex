\documentclass{ximera}

\title{Hill climbing in Python}

\begin{document}

\begin{abstract}
  Hill climbing is a computational technique for finding a local maximum.
\end{abstract}

Let's try \textbf{hill climbing} to---at least numerically---attempt
to find a local maximum.

The idea is the following: \textbf{the gradient points up hill} so if
I want to find a local maximum, I should start somewhere and follow
the gradient up.  Hopefully I'll find a point where the gradient
vanishes (i.e., a critical point).

\begin{question}
  Here's the procedure that I'd like you to code in Python:
  \begin{enumerate}
  \item Start with some point $\mathbf{p}$.
  \item Replace $\mathbf{p}$ with $\mathbf{p}$ plus a small multiple of $\nabla f(\mathbf{p})$.
  \item If $\nabla f(\mathbf{p})$ is very small, stop!
  \item Otherwise, repeat.
  \end{enumerate}


  \begin{solution}
    \begin{hint}
       You might want to use some code to add and scale vectors, like
\begin{verbatim}
def add_vector(v,w):
  return [sum(v) for v in zip(v,w)]
def scale_vector(c,v):
  return [c*x for x in v]
\end{verbatim}
    \end{hint}
    \begin{hint}
      You may also want some code to compute the gradient numerically.
\begin{verbatim}
epsilon = 0.001
def gradient(f, p):
  n = len(p)
  ei = lambda i: [0]*i + [epsilon] + [0]*(n-i-1)
  return [ (f(add_vector(p, ei(i))) - f(p)) / epsilon for i in range(n) ]      
\end{verbatim}
    \end{hint}
    \begin{hint}
To do the hill climbing, we can put together these pieces.
\begin{verbatim}
def climb_hill(f, starting_point):
  p = starting_point
  nabla = gradient(f,p)
  while vector_length(nabla) > epsilon:
    p = add_vector(p, scale_vector(epsilon,nabla))
    nabla = gradient(f,p)
  return p
\end{verbatim}
    \end{hint}
    \begin{python}
def climb_hill(f, starting_point):
  p = starting_point
  # while gradient of f is pretty big at p
  #   p = p + multiple of gradient f
  # return p
#
# here's an example to try
p = [3,6,2]
f = lambda x: 10 - (x[0] + x[1])**2 - (x[0] - 3)**2 - (x[2] - 4)**2
print(climb_hill(f, p))

def validator():
  f = lambda x: 10 - (x[0] - 2)**4 - (x[1] - 3)**2 - (x[2] - 4)**2
  p = climb_hill(f, [3,6,2])
  return abs(p[0] - 2) < 0.1 and abs(p[1] - 3) < 0.1 and abs(p[2] - 4) < 0.1
    \end{python}
  \end{solution}

Use your program to find the maximum value of the function $f : \R^3 \to \R$ given by
$$
f(x,y,z) = 10 - (x+y)^2 - (x-3)^2 - (z-4)^2.
$$

\end{question}

\end{document}
