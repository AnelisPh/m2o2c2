\documentclass{ximera}
\title{Second derivative test}

\begin{document}
	\begin{abstract}
		Definiteness of the second derivative determines extreme behavior at critical points.
	\end{abstract}
	
	In this section, we apply the second derivative to extreme value problems.
	
	\begin{theorem}[Second derivative test]
		Let $f:\R^n \to \R$ be a $\mathcal{C}^2$ function.  Let $\mathbf{p} \in \R^n$ be a critical point of $f$.  Then 
			\begin{itemize}
				\item If $D^2f(\mathbf{p})$ is positive definite, $\mathbf{p}$ is a local minimum
				\item If $D^2f(\mathbf{p})$ is negative definite, $\mathbf{p}$ is a local maximum
				\item If $D^2f(\mathbf{p})$ is indefinite, then $\mathbf{p}$ is a saddle point
				\item If $D^2f(\mathbf{p})$ is only positive semidefinite or negative semidefinite, we get no information
			\end{itemize}
	\end{theorem}
	
	\begin{proof}
		By the second order taylor's theorem, we have 
		\[
			f(\mathbf{p} + \vec{h}) = f(\mathbf{p}) + Df(\mathbf{p})(\vec{h})+D^2f(\mathbf{p}+\xi\vec{h})(\vec{h},\vec{h}) \text{ for some } \xi \in [0,1]$
		\]
		
		Since $\mathbf{p}$ is a critical point,
		
		\[
			f(\mathbf{p} + \vec{h}) = f(\mathbf{p}) + D^2f(\mathbf{p}+\xi\vec{h})(\vec{h},\vec{h}) \text{ for some } \xi \in [0,1]$
		\]
		
		If  $D^2f(\mathbf{p})$ is positive definite, then because $f$ is $\mathcal{C}^2$,  $D^2f(\mathbf{p}+\xi\vec{h})$ is also positive semidefinite for 
		small enough $\vec{h}$ (this just uses continuity of the second derivative).  Thus $D^2f(\mathbf{p}+\xi\vec{h})(\vec{h},\vec{h}) > 0$ for
		all small enough values of $\vec{h}$, say $|\vec{h}| \leq \epsilon$.  But this just says that $f(\mathbf{p}+\vec{h}) > f(\mathbf{p})$, so $\mathbf{p}$
		is a local minimum.
		
		If $D^2f(\mathbf{p})$ is negative definite, a completely analogous proof works.
		
		If $D^2f(\mathbf{p})$ is indefinite, then there are directions $\vec{h}_1$ where $D^2f(\mathbf{p})(\vec{h}_1,\vec{h}_1) > 0$, 
		and $\vec{h}_2$ where $D^2f(\mathbf{p})(\vec{h}_2,\vec{h}_2) < 0$.  By continuity again, for $|\vec{h_i}| < \epsilon$ , we have
		$D^2f(\mathbf{p}+\xi\vec{h}_1)(\vec{h}_1,\vec{h}_1) > 0$ and $D^2f(\mathbf{p}+\xi\vec{h}_2)(\vec{h}_2,\vec{h}_2) < 0$.
		
		So we have $f(\mathbf{p}+\xi_1\vec{h_1}) > f(\mathbf{p})$ and $f(\mathbf{p}+\xi_2\vec{h_2}) < f(\mathbf{p})$.   Thus $\mathbf{p}$ is neither a local 
		maximum nor a minimum, and so is a saddle point.
		
		The method of proof show why the semidefinite cases might break down.  Without strict positivity or negativity, continuity does not guarantee  
		that $D^2f$ is still positive definite or negative definite for nearby points:  you might slip into an indefinite case, for instance.
		
		For a concrete counterexample in the semidefinite case, consider $f(x,y) = x^2+y^3$.  $(0,0)$ is a critical point, but
		and the Hessian \(\begin{bmatrix} 2 & 0\\0&0\end{bmatrix}\) is positive semidefinite, but $(0,0)$ is not a local minimum, 
		since $f(0,-\epsilon) = -\epsilon^3$ is always less than $f(0,0)$.  The trouble here is really that the Hessian at nearby points $(x,y)$ is
		\(\begin{bmatrix} 2 & 0\\0&6y\end{bmatrix}\), which is indefinite for $y<0$.  				
	\end{proof}
	
	We have already proven in the section on bilinear forms that the definiteness of a bilinear form is completely determined by the sign
	of the eigenvalues of the associated linear operator.
	
	For the following exercises, use whatever means necessary to compute the eigenvalues of the Hessian, use that information to determine the 
	definiteness of the second derivative, and use this to draw extremum information about $f$.  I recommend using a computer algebra system like 
	\href{http://www.sagemath.org/}{Sage} to compute the eigenvalues, 
	but you can also use a free online app like \href{http://www.bluebit.gr/matrix-calculator/}{this one}. 
	
	\begin{question}
		Let $f(x,y) = x^3+e^{3y}-3xe^{y}$
		\begin{solution}
			\begin{hint}
				\begin{align*}
					Df(x,y)  &= \begin{bmatrix} 0 & 0\end{bmatrix}\\
					\begin{bmatrix} 3x^2-3xe^y & 3e^{3y}-3xe^y\end{bmatrix} &= \begin{bmatrix} 0 & 0\end{bmatrix}\\
					&\begin{cases}
						3x^2-3e^y = 0\\
						3e^{3y}-3xe^y = 0
					\end{cases}\\
					&\begin{cases}
						x^2=e^y\\
						e^{3y}=xe^y
					\end{cases}\\
					&\begin{cases}
						x^2=e^y\\
						x=e^{2y}
					\end{cases}\\
					&\begin{cases}
						x^4=x\\
						x=e^{2y}
					\end{cases}
				\end{align*}
				
				To satisfy the first solution, either $x=0 $ or $x=1$.  If $x=0$, then $x=e^{2y}$ has no solutions, so we must have $x=1$.  Thus $(1,0)$ is
				the only critical point. 
			\end{hint}
			What is the critical point of $f$?  Give your answer as a vertical vector.
			\begin{matrix-answer}
				correctMatrix = [['1'],['0']]
			\end{matrix-answer}
		\end{solution}
		\begin{solution}
			\begin{hint}
				\(\mathcal{H}(x,y) =  \begin{bmatrix} 6x & -3e^y \\ -3e^y & 9e^y \end{bmatrix}\)
			\end{hint}
			\begin{hint}
				\(\mathcal{H}(0,0) =  \begin{bmatrix} 6 & -3 \\ -3 & 9 \end{bmatrix}\)
			\end{hint}
			What is the Hessian matrix of $f$ at $(1,0)$?
			\begin{matrix-answer}
				correctMatrix = [['6','-3'],['-3','9']]
			\end{matrix-answer}
		\end{solution}
		\begin{solution}
			\begin{hint}
				By using a computer algebra system, we see that the eigenvalues of the Hessian are approximately $4.146$ and $10.854$.  
			\end{hint}
			\begin{hint}
				So $D^2f(1,0)$ is positive definite.
			\end{hint}
			\begin{hint}
				Thus $(1,0)$ is a local minimum.
			\end{hint}
			\begin{multiple-choice}
				\choice{$(1,0)$ is a local maximum}
				\choice[correct]{$(1,0)$ is a local minimum}
				\choice{$(1,0)$ is a saddle point}
				\choice{The second derivative gives no information in this case}
			\end{multiple-choice}
		\end{solution}
		
		Observe that even though $(1,0)$ is the only local extremum, and this is a local minimum, $(1,0)$ is \textbf{not} a global minimum because $f(1,0) = -1$ but
		 $f(-10,0) = -1000+1-3(-10)= -969$.  Contemplate this carefully.
	\end{question}
	
	\begin{question}
		Let $f:\R^3 \to \R$ be defined by $f(x,y,z) = e^{x+y+z}  - x - y - z+ 4z^2+xy$.  $f$ has a critical point at $(0,0,0)$.
		\begin{solution}
			\begin{hint}
				\begin{question}
					\begin{solution}
						\begin{hint}
							\(
								\begin{bmatrix}
									1 &2 &1\\
									2& 1&1\\
									1&1&9
								\end{bmatrix}
							\)
						\end{hint}
						The Hessian matrix of $f$ at $(0,0,0)$ is
						\begin{matrix-answer}
							correctMatrix = [['1','2','1'],['2','1','1'],['1','1','9']]
						\end{matrix-answer}
					\end{solution}
				\end{question}
			\end{hint}
			\begin{hint}
				According to computer algebra software, the eigenvalues of this matrix are  approximately $4.414$, $-1$ and $1.586$, so the second derivative
				is indefinite
			\end{hint}
			\begin{hint}
				Thus $f$ has a saddle point at $(0,0,0)$
			\end{hint}
			\begin{multiple-choice}
				\choice{$(0,0,0)$ is a local maximum}
				\choice{$(0,0,0)$ is a local minimum}
				\choice[correct]{$(0,0,0)$ is a saddle point}
				\choice{The second derivative gives no information in this case}
			\end{multiple-choice}
		\end{solution}
	\end{question}
	
	\begin{question}
		Let $f(x,y) = \cos(x+y^2)-x^2$.  $(0,0)$ is a critical point of $f$.
			\begin{solution}
				\begin{hint}
					The hessian matrix of $f$ at $(0,0)$ is \( \begin{bmatrix} -1 & 0\\0&-2\end{bmatrix}\) which plainly has eigenvalues $-1$ and $-2$.
				\end{hint}
				\begin{hint}
					Thus $D^2f(0,0)$ is negative definite
				\end{hint}
				\begin{hint}
					Thus $f$ has a local maximum at $(0,0)$
				\end{hint}
				\begin{hint}
					You can actually see that this is a global maximum, since the largest $\cos$ could ever be is $1$, and the largest $-x^2$ could ever be is $0$,
					so $1$ is the largest value that $f$ ever attains, and this is attained at $(0,0)$.  In fact this value is attained at infinitely many points on the line $x=0$.
				\end{hint}
				\begin{multiple-choice}
				\choice[correct]{$(0,0)$ is a local maximum}
				\choice{$(0,0)$ is a local minimum}
				\choice{$(0,0)$ is a saddle point}
				\choice{The second derivative gives no information in this case}
			\end{multiple-choice}
			\end{solution}
	\end{question}
	
	
	
	
	
\end{document}
