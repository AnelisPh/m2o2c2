\documentclass{ximera}
\title{Lagrange multipliers}

\begin{document}
	\begin{abstract}
		Lagrange multipliers enable constrained optimization
	\end{abstract}
	
	In the previous section we considered unconstrained optimization problems.  Sometimes we want to find extreme values of a function $f:\R^n \to \R$
	subject to some constraints.  
	
	\begin{example}
		Say we want to maximize $f(x,y) = x^2y$ subject to the contraint that $x^2+y^2=1$.  In  words, we want to know among all points on the unit circle,
		which of them has the greatest product of the square of the first coordinate with the second coordinate.  One way we can do this is to reduce it to a single variable
		calculus problem:  we can parameterize the unit circle by $\gamma(t) = (\cos(t),\sin(t))$, and try to maximize $f(\gamma(t)):[0,2\pi] \to \R$. In other words, we
		are maximizing $\cos^2(t)\sin(t)$ on the interval $[0,2\pi]$.
	\end{example}
	
	\begin{definition}
		Let $f:\R^n \to \R$ be a function, and $g:\R^n \to \R^m$ be another function.  A point $\mathbf{p}$ is called a \textbf{local maximum of $f$ with constraint 
		$g(\mathbf{x}) = \vec{0}$} if there is an $\epsilon>0$ such that if $|\mathbf{x}-\mathbf{p}| < \epsilon$ and $g(\mathbf{x}) = 0$, 
		then $f(\mathbf{x})<f(\mathbf{p})$
	\end{definition}
	
	Give a definition of a constrained local minimum:
	
	\begin{free-response}
		Let $f:\R^n \to \R$ be a function, and $g:\R^n \to \R^m$ be another function.  A point $\mathbf{p}$ is called a \textbf{local minimum of $f$ with constraint 
		$g(\mathbf{x}) = \vec{0}$} if there is an $\epsilon>0$ such that if $|\mathbf{x}-\mathbf{p}| < \epsilon$ and $g(\mathbf{x}) = 0$, 
		then $f(\mathbf{x})>f(\mathbf{p})$
	\end{free-response}
	
	The method we outlined above is a great way to find constrained local extrema: If you can parameterize $g^{-1}(\{0\})$, by some function
	 $M: U \subset \R^k \to \R^n$, 
	then you can just try to find the unconstrained extrema  of $f\circ M$.  The problem is that, although finding a parameterization of the circle was easy, more general
	sets might be harder to parameterize.  Some of them might not be parameterizable by just one open set:  you might have to use several patches.  We will not pursue
	this method further.  Instead we will develop an alternative method, the method of lagrange multipliers.
	
	\begin{theorem}
		Let $f:\R^n \to \R$ and $g:\R^n \to \R^m$. Assume $g$ has the $m$ component functions $g_1,g_2,...,g_n:\R^n \to \R$. If $\mathbf{p}$ is a constrained extrema of $f$ with the constraint $g(\mathbf{x}) = 0$, 
		and $\textrm{Rank}(Dg(\mathbf{p})) = m$, then there exist $\lambda_1,\lambda_2,...\lambda_m$ with 
		$Df(\mathbf{p}) = \lambda_1Dg_1(\mathbf{p})+ \lambda_2Dg_2(\mathbf{p})+...+\lambda_mDg_m(\mathbf{p})$.  The scalars $\lambda_i$ are
		called \textbf{Lagrange Multipliers}.
	\end{theorem}
	
	\begin{proof}
			The full proof of this theorem would require the \href{http://en.wikipedia.org/wiki/Implicit_function_theorem}{implicit function theorem}.  We will 
			make one intuitive assumption to get around this.
			
			A vector  $\vec{v}$ should be  tangent to the set $g^{-1}(\{\vec{0}\})$ at the point $\mathbf{p}$  if and only if 
			$Dg(\mathbf{p})(\vec{v}) = 0$.  This is just because moving ``infinitesmally'' in the direction of a tangent vector to 
			$g^{-1}(\{\vec{0}\})$ should not change the value of $g$ to first order.  
			
			Since $\mathbf{p}$ is a constrained maximum, moving in one of these tangent directions should not effect the value of $f$ to first order either.
			
			We can summarize these intuitive statements as $\textrm{Null}(Dg(\mathbf{p})) \subset \textrm{Null}(Df(\mathbf{p}))$.  This is the assumption whose
			formal proof would require the implicit function theorem.
			
			Given this assumption, the result follows essentially formally from our work with linear algebra:
			
			\begin{align*}
				\textrm{Null}(Dg(\mathbf{p})) &\subset \textrm{Null}(Df(\mathbf{p}))\\
				\textrm{Null}(Df(\mathbf{p}))^\perp &\subset \textrm{Null}(Dg(\mathbf{p}))^\perp\\
				\textrm{Image}(Df(\mathbf{p})^\top) &\subset \textrm{Image}(Dg(\mathbf{p})^\top)\\
			\end{align*}
			
			This last line is exactly what we are trying to prove!
		\end{proof}
		
	Now let's get some practice actually using this theorem as a practical tool.

	\hrule

	First some questions with only one constraint equation:  $g:\R^n \to \R$
	
	\begin{question}
		\begin{solution}
		\begin{hint}
			At a point constrained maximum point $(x,y)$ we would need $Df(x,y) = \lambda Dg(x,y)$ for some $\lambda \in \R$.
		\end{hint}
		\begin{hint}
			\begin{question}
			\begin{solution}
				What is $Df(x,y)$?
					\begin{matrix-answer}
						correctMatrix = [['2xy','x^2']]
					\end{matrix-answer}
			\end{solution}
			\begin{solution}
				What is $Dg(x,y)$?
					\begin{matrix-answer}
						correctMatrix = [['2x','2y']]
					\end{matrix-answer}
			\end{solution}
			\end{question}
		\end{hint}
		\begin{hint}
			So we must have \(\begin{bmatrix} 2xy & x^2\end{bmatrix} = \lambda \begin{bmatrix} 2\lambda x & 2\lambda y\end{bmatrix}\)
		\end{hint}
		\begin{hint}
			&\begin{align*}
				\begin{cases}
				2xy = 2\lambda x\\
				x^2 =  2\lambda y
				\end{cases}\\
				\begin{cases}
				y = \lambda \text{  $x \neq 0$ for if it were then $y=0$ by the second equation}\\
				x^2 =  2\lambda y
				\end{cases}\\
				\begin{cases}
				y = \lambda\\
				x^2 =  2\lambda^2
				\end{cases}
			\end{align*}
			But $x^2+y^2=1$, so we have $3\lambda^2 = 1$, or $\lambda =\frac{\pm 1}{\sqrt{3}}$
		\end{hint}
		\begin{hint}	
			This results in only $4$ possible extrema at $(\sqrt{\pm \frac{2}{3}}, \pm \frac{1}{\sqrt{3}})$
		\end{hint}
		\begin{hint}
			The values of $f$ at these points are $\pm \sqrt{\frac{4}{9}}$.  So the maximum value of $f$ on the unit circle 
			is $\sqrt{\frac{4}{9}}$ and the minimum value is $-\sqrt{\frac{4}{9}}$. 
		\end{hint}
		The maximum value of $f(x,y) = x^2y$ subject to the constraint $x^2+y^2=1$ is \answer{$sqrt(4/9)$}
		\end{solution}
	\end{question}

	\hrule

	\begin{question}
		Let $f:\R^3 \to \R$ be defined by $f(x,y,z) = x^2+y^2+x^2$ subject to the constraint that $g(x,y,z) = xyz -4 = 0 $.  
		\begin{solution}
			\begin{hint}
			    At a point constrained maximum point $(x,y)$ we would need $Df(x,y) = \lambda Dg(x,y)$ for some $\lambda \in \R$.
		    \end{hint}
			\begin{hint}
			\begin{question}
			\begin{solution}
				What is $Df(x,y)$?
					\begin{matrix-answer}
						correctMatrix = [['2x','2y','2z']]
					\end{matrix-answer}
			\end{solution}
			\begin{solution}
				What is $Dg(x,y)$?
					\begin{matrix-answer}
						correctMatrix = [['yz','xz','xy']]
					\end{matrix-answer}
			\end{solution}
			\end{question}
		\end{hint}
			\begin{hint}
				So we must have \(\begin{bmatrix} 2x & 2y & 2z\end{bmatrix} = \lambda \begin{bmatrix} \lambda yz & \lambda xz & \lambda xy\end{bmatrix}\)
			\end{hint}
			\begin{hint}
				\(
					\begin{cases}
						2x &= \lambda yz\\
						2y &= \lambda xz\\
						2z &= \lambda xy
					  \end{cases}
					\)
			\end{hint}
			\begin{hint}
				Multiplying all of these equations together, we have $8xyz  = \lambda^3(xyz)^2$.  Since $xyz=1$, we have $\lambda^3 = 8$, so $\lambda =2$
			\end{hint}
			\begin{hint}
				\(
					\begin{cases}
						x &=  yz\\
						y &=  xz\\
						z &=  xy
					  \end{cases}\\
					  \\
					  \begin{cases}
						x &=  xz^2\\
						y &=  yx^2\\
						z &=  zy^2
					  \end{cases}\\
					
					  
					\)
			\end{hint}
			\begin{hint}
				So $x=\pm 1$, $y= \pm 1$, $z = \pm 1$.  So the only possible location of constrained extrema are $(1,1,1),(1,-1,-1),(-1,1,-1),(-1,-1,1)$.  At each of these points
				$f(x,y,z) =1$.  These are all local minima.  $f$ has no local or global maxima.
			\end{hint}
			The minimum value of $f$ subject to this constraint is \answer{$1$}
		\end{solution}
		
	\end{question}

	\hrule

	Here is a question with two constraint equations:  $g:\R^n \to \R^2$
	
	\begin{question}
		\begin{hint}
			Here $g(x,y,z) = \verticalvector{x^2+y^2+z^2-1\\x+y+z-1}$
		\end{hint}
		\begin{hint}
			We need \( \begin{bmatrix} 1 & -1 & 0\end{bmatrix} = \lambda_1 \begin{bmatrix} 2x & 2y & 2z\end{bmatrix}+ \lambda_2\begin{bmatrix}  1 & 1 & 1\end{bmatrix}\)
		\end{hint}
		\begin{hint}
				\(\begin{cases}
					2\lambda_1 x+\lambda_2 = 1\\
					2\lambda_1 y+\lambda_2 = -1\\
					2\lambda_1 z+\lambda_2 = 0 
				\end{cases}\)
				
				Adding these all together and using that $(x+y+z)=1$, we have $2\lambda_1+3\lambda_2 = 0$.
				\\
				\\
				So the last equation becomes $-3\lambda_2 z+\lambda_2 = 0$, or $\lambda_2(1-3z) = 0$.  $\lambda_2 \neq 0$, for otherwise $\lambda_1=0$, which leads to 
				a contradiction.
				\\
				\\
				So We know that $z = \frac{1}{3}$
		\end{hint}
		\begin{hint}
			There are only two points satisfying both constraints and with $z = \frac{1}{3}$
		\end{hint}
		\begin{hint}
			Solving the system of $2$ equations
			
			\(
				\begin{cases}
					x^2+y^2+(\frac{1}{3})^2 = 1\\
					x+y+\frac{1}{3} = 0
				\end{cases}
			\) 
			
			we obtain that the only two points which work are $(\frac{1-\sqrt{2}}{3},\frac{1+\sqrt{2}}{3},\frac{1}{3})$ and $(\frac{1+\sqrt{2}}{3},\frac{1-\sqrt{2}}{3},\frac{1}{3})$.
		\end{hint}
		\begin{hint}
			So the maximum value of $x-y$ is $\frac{2\sqrt{2}}{3}$
		\end{hint}
		The maximum value of  $f(x,y,z) = x-y$ subject to the two constraints that $x^2+y^2+z^2 = 1$ and $x+y+z=1$ is \answer{2sqrt(2)/3}

	\end{question}		
		
\end{document}
