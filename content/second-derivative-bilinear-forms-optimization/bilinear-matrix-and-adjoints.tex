\begin{document}
\section{Linear maps associated with bilinear forms and adjoints}
	
	It turns out that we will be able to use the inner product on $\R^n$ to rewrite any bilinear form on $\R^n$ in a special form. 
	
	\begin{question}
		Let $L: \R^n \to \R^m$ be a linear map.  Prove that $B_L:\R^n \to \R^m \to \R$ given by $B_L(\vec{v},\vec{w}) = L(\vec{v}) \cdot \vec{w}$ is a bilinear form.
	\end{question}	
	
	\begin{question}
		Let $A: \R^n \to \R^m $ be a linear map.  If $M$ is the matrix of $L$, show that $B_L(\vec{v},\vec{w}) = w^\top M \vec{v}$.
	\end{question}
	
	\begin{question}
		Show that any bilinear form $B: \R^n \times \R^m \to \R$ can be written as $B_L$ for a unique linear map $L: \R^n \to \R^m$.  Note that this representation of a 
		bilinear form as a matrix is very special:  if the codomain is something other than $\R$, there is no notion of a matrix of the bilinear map.
	\end{question}
	
	\begin{question}
		If the matrix of the bilinear form $B$ is BLAH, what is $B(BLAH,BLAH)$?
	\end{question}
	
	\begin{question}
		Show that the matrix of the bilinear form $\sum a_{i,j} dx_i \otimes dx_j$ is the matrix $[a_{i,j}]$.
	\end{question}
	
	
	Let $L: \R^n \to \R^m$ be a linear map.  
		Then there is an associated bilinear form $B_L  : \R^n \times \R^m \to \R$ given by $B_L(\vec{v},\vec{w}) = \langle L(\vec{v}), \vec{w} \rangle$
		Now we can just as easily view $B$ as a bilinear map from $\R^m \times \R^n$, namely $B_L^* (\vec{w},\vec{v}) = B_L(\vec{v},\vec{w})$.
		So we get an associated linear map $L^* : \R^m \to \R^n$.
		
		\begin{definition}
			If $L:\R^n \to \R^m$ is a linear map, we call the associated map $L^* : \R^m \to \R^n$ given by the line of reasoning above the \textit{Adjoint} of $L$.
		\end{definition}
		
		\begin{question}
			Let $L: \R^n \to \R^m$ is a linear maps.  
			Show that $\langle L(\vec{v}), \vec{w}\rangle = \langle  \vec{v},L^*(\vec{w})\rangle$ for every $v \in \R^n$ and $w \in \R^m$
		\end{question}
	
		\begin{question}
			Let $L: \R^n \to \R^m$ be a linear map.
			If $L$ has matrix $M$ with respect to the standard basis, show that $L^*$ has matrix $M^\top$.
		\end{question}
		
		\begin{question}
			Show that if $\vec{v}$ is an eigenvector of $L:\mathbb{R} \to \mathbb{R}$, with eigenvalue $\lambda$ then
			there is an eigenvector $\vec{u}$ of $L^*$ with eigenvector $\lambda$, and $\vec{u} \perp \vec{v}$. 
		\end{question}
		
		\begin{definition}
			A linear operator $L:\R^n \to \R^n$ is called \textit{self adjoint} if $L = L^*$.
		\end{definition}
		
		\begin{question}
			Show that a linear map is self adjoint if and only if its matrix with respect to the standard basis is satisfies $M_{ij} = M_{ji}$ for each $1 \leq i,j \leq n$.  We call
			such matrices \textit{symmetric matrices} because they are symmetric about the main diagonal.
		\end{question}

\begin{definition}
	A bilinear form on $\R^n$ is \textit{symmetric} if $B(\vec{v},\vec{w}) = B(\vec{w},\vec{v})$ for all $\vec{v},\vec{w} \in \R^n$.
\end{definition}

\begin{question}
	Show that the dot product on $\R^n$ is a symmetric bilinear form.
\end{question}

\begin{question}
	If $B$ is a symmetric bilinear form on $\R^3$, and (BADBAD give action of B on half of pairs of basis vectors), then $B((x_1,x_2,x_3),(y_1,y_2,y_3)) = $?
\end{question}

\begin{question}
	Let $B$ be the symmetric bilinear form defined by $BLAH$.  What is the matrix of $B$?  What do you notice about this matrix?
\end{question}

\begin{question}
	Show that $L$ is self adjoint if and only if the bilinear form associated to it is symmetric, i.e. if its matrix is a symmetric matrix.
\end{question}

\end{document}