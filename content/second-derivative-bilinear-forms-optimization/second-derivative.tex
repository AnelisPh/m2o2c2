\begin{document}
\section{The Second Derivative}
	From our perspective, the second derivative of a function $f: \R^n \to \R$ at a point will be a bilinear form on $\R^n$.  
	Let us take some time to understand, intuitively, why that should be the case.
	
	Let $f:\mathbb{R}^2 \to \mathbb{R}$ be defined by $f(x,y) = x^2y$.

$D(f)\big|_{(x,y)}$ is the linear map given by the matrix $\left[ \begin{matrix} 2xy&x^2\end{matrix} \right]$.  
That is to say, $D(f)\big|_{(x,y)}(\Delta x,\Delta y) = 2xy\Delta x + x^2\Delta y \approx f(x+\Delta x,y+\Delta y) - f(x,y)$.  


The second derivative should now tell you how much the derivative changes from point to point.  
If we increment $(x,y)$ by a little bit to $(x+\Delta x,y)$ then we should expect the derivative to 
increase by about $\left[ \begin{matrix} 2y\Delta x&2x \Delta x\end{matrix} \right]$.  
Similarly, when we increase $y$ by $\Delta y$, the derivative should change by about $\left[ \begin{matrix} 2x \Delta y&0\Delta y\end{matrix} \right]$.

By linearity, if we change from $(x,y)$ to $(x+\Delta x,y+\Delta y)$, 
we expect the derivative to change by
 $$\left[ \begin{matrix} \Delta x&\Delta y\end{matrix} \right] \left[ \begin{matrix} 2y&2x\\2x&0\end{matrix} \right]$$

This gives a matrix which is the approximate change in the derivative.  You can then apply this to another vector if you so wish.  

Summing it up, if you wanted to see approximately how much the derivative changes from $p = (x,y)$ to $(x+\Delta x_2,y+\Delta y_2) = p+\vec{h_2}$ 
($\vec{h_2} = \verticalvector{\Delta x_2,\Delta y_2}$)
when both are evaluated in the same direction  $\vec{h_1} = (\Delta x_1,\Delta y_1)$, you would perform the computation:

\[ 
Df_{p+\vec{h_2}}(\vec{h_1})  -Df_p(\vec{h_1}) \approx
\begin{bmatrix} 
\Delta x_2&\Delta y_2\end{bmatrix} 
 \begin{bmatrix} 2x&2x\\2x&0\end{bmatrix} \begin{bmatrix} \Delta x_1\\ \Delta y_1\end{matrix}
 \]
 
 This is exactly using the matrix $\begin{bmatrix} 2x&2x\\2x&0\end{bmatrix} \begin{bmatrix}$ as a bilinear form applied to the two vectors 
 $\vec{h_1} = \verticalvector{\Delta x_1,\Delta y_1}$ and $\vec{h_2} = \verticalvector{\Delta x_2,\Delta y_2}$.
 
With all of this as motivation, we define

\begin{definition}
	The second derivative of a function $f:\R^n \to \R$ is a bilinear form at a point $\mathbf{p}\in \R^n$  
	is a bilinear form $D^2f\big|_\mathbf{p} : \R^n \times \R^n \to \R$ enjoying the following estimates:
	
	\[
		Df\big|_{\mathbf{p}+\vec{h_1}}(\vec{h_2})  = Df\Big|_\mathbf{p}{\vec{h_2}} + Error(\vec{h_1},\vec{h_2})
	\]
	
	with $\lim_{\vec{h_1},\vec{h_2} \to \vec{0}} \frac{Error(\vec{h_1},\vec{h_2})}{\vec{h_1},\vec{h_2}}$
\end{definition} 

\begin{theorem}
	If $f$ has continuous second partial derivatives everywhere, then 
	\[
		D^2f|_\mathbf{p}  = \Sum_{i,j=1}^n \frac{\partial^2 f}{\partial x^i \partial x^j} dx^i \otimes dx^j
	\]
	
	The matrix associated to this bilinear form is the matrix:
	
		\[ 
		
		\mathcal{H}(f)\big|_\mathbf{p} = [\frac{\partial^2 f}{\partial x^i \partial x^j}]
	
		\]
		
		and is called the Hessian matrix of $f$ at $\mathbf{p}$.
\end{theorem}

\begin{question}
	If $f:\R^n \to \R$ is a function and $Df = BLAH$ what is the hessian matrix of $f$?
\end{question}

\begin{question}
	If $Df|_p = BLAH$ and $Hf|_p = BLAH$, approximate $Df_{p+blah}$.
\end{question}

\begin{question}
	If $Df\big|_{\mathbf{(1,2)}} = \blah$ and $Hf\big|_{\mathbf{p}} = blah$, approximate $Df\big|_{(1.2,1.1)}(\verticalvector{-0.2,0.3})$.
\end{question}

\begin{question}
	What is the hessian of $f(x,y) = x\sin(xy)$?
\end{question}

In working out Hessians of various functions, you might have noticed that the matrix you get is symmetric, i.e.  that ``mixed partials commute'' 
$frac{\partial^2 f}{\partial x^i \partial x^j}$

\begin{theorem}
	If $f$ has continuous second partial derivatives, then $D^2f$ is a symmetric bilinear form.
\end{theorem}
 
\end{document}