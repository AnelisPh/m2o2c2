\begin{document}
\section{Linear Maps}

\begin{definition}
	A function $L: \R^n \to \R^m$ is called a \textit{linear map} if it "respects addition and scalar multiplication".
	Symbolically, for a map to be linear, we must have that $L(v+w) = L(v)+L(w)$ for all $v,w \in \R^n$ and also
	$L(av) = a L(v)$ for all $a \in \R$ and $v\in \R^n$
\end{definition}

	\begin{question}
		Which of the following functions are linear?
		\begin{itemize}
			\item $f: \R^2 \to \R^1$ defined by $f(\verticalvector(x,y)) = x+2y$
			\item $g: \R^3 \to R^2$ defined by $g(\verticalvector(x,y,z)) = \verticalvector(x,xy)$
			\item $h:\R \to \R^4$ defined by $h(x) = \verticalvector(x,x,x,4x)$
			\item $G: \R^4 \to  \R^3$ defined by $G(\verticalvector(x,y,z,t)) = \verticalvector(e^{x+y},x+y,sin(x+y))$
			\item $A: \R^2 \to R^2$ defined by $A(\verticalvector(x,y))=(\verticalvector(0,0)$
		\end{itemize}
	\end{question}
	
	\begin{question}
	 	Let $L:\R^3 \to \R^2$ be a linear function.  Suppose $L(\verticalvector(1,0,0)) = \verticalvector(3,4)$, 
	 	 $L(\verticalvector(0,1,0)) = \verticalvector(-2,0)$,  and  $L(\verticalvector(0,0,1)) = \verticalvector(1,-1)$.
	 	 
	 	What is $L(4,-1,2)$?
	\end{question}
	
	\begin{question}
	 	Let $L:\R^3 \to \R^2$ be a linear function.  Suppose $L(\verticalvector(1,0,0)) = \verticalvector(3,4)$, 
	 	 $L(\verticalvector(0,1,0)) = \verticalvector(-2,0)$,  and  $L(\verticalvector(0,0,1)) = \verticalvector(1,-1)$.
	 	 
	 	What is $L(x,y,z)$?
	\end{question}
	
	As you have already discovered a linear map $L: \R^n \to \R^m$  is fully determined by
	its action on the "standard basis vectors" $e_1 = \verticalvector(1,0,0,...,0), e_2 = \verticalvector(0,1,0,...,0), e_3 = (0,0,1,0,...,0)$, and $e_n = (0,0,0,...,0,1)$.
	
	To make writing a linear map a little less cumbersome, we will develop a compact notation for them using the observation above. 
	
	\begin{definition}
		A $m \times n$ \textit{matrix} is an array of numbers which has $m$ rows and $n$ columns.  The numbers in a matrix are called \textit{enteries}. If $A$ is a matrix, 
		we will often write $a_ij$ for the entry in the $i^{th}$  row and $j^{th}$ column of the matrix.
	\end{definition}
	
	\begin{definition}
		To each linear map $L: \R^n \to \R^m$  we associate a $m \times n$ matrix $A_L$ called the \textit{matrix of the linear map}.  It is defined 
		by letting $a_{i,j}$ be the $i^{th}$ component of $L(e_j)$.  In other words, the $j^{th}$ column of the matrix $A_L$ is the vector $L(e_i)$.  We also associate to each 
		matrix $m \times n$ matrix $M$ a linear map $L_M: \R^n \to \R^m$ by requiring that $L(e_i)$ is the $j^{th}$ column of the matrix $M$. 
	\end{definition}
	
	\begin{question}
		$A = \begin{bmatrix}
		1&-1\\2&4\\3&-5
		\end{bmatrix}$
		is an $n \times m$ matrix.  What are $n$ and $m$? 
	\end{question}
	
	\begin{question}
		The $3 \times 4$ matrix $A$ has $a_{i,j} = i+j$.  What is $A$?
	\end{question}
	
	\begin{question}
		$A = \begin{bmatrix}
		1&-1\\2&4\\3&-5
		\end{bmatrix}$
		 What is $a_{3,2}$?
	\end{question}
	
	\begin{question}
		The linear map $L:\mathbb{R^2}\to\mathbb{R^3}$ satisfies $L(\verticalvector(1,0)) = \verticalvector(3,-5,2)$ and $L(\verticalvector(0,1)) = \verticalvector(1,1,1)$. 
		 What is the matrix of $L$?
	\end{question}
	
	\begin{question}
		The matrix of $L$ is 
		$A = \begin{bmatrix}
		1&-1\\2&4\\3&-5
		\end{bmatrix}$
		What is the dimension of the domain of  $L$?  The codomain?
	\end{question}
	
	\begin{question}
		The matrix of $L$ is 
		$A = \begin{bmatrix}
		1&-1\\2&4\\3&-5
		\end{bmatrix}$
		
		What is $L(0,1)$?
	\end{question}
	
	\begin{question}
		The matrix of $L$ is 
		$A = \begin{bmatrix}
		1&-1\\2&4\\3&-5
		\end{bmatrix}$
		
		What is $L(4,5)$?
	\end{question}
	
	\begin{question}
		The matrix of $L$ is 
		$A = \begin{bmatrix}
		1&-1\\2&4\\3&-5
		\end{bmatrix}$
		
		What is $L(x,y)$?
	\end{question}

	\begin{question}
		Prove that if $S:\R^n \to \R^m$ is a linear map, and $T:\R^m \to \R^k$ is a linear map, then the composite function $T\circ S:\R^n \to R^k$ is also linear.
	\end{question}
	
	\begin{question}
		If the matrix of $S$ is $M_S = BLAH$ and the matrix of $T$ is $M_T = BLAH$, find the matrix of $T \circ S$.
	\end{question}
	
	\begin{question}
		If the matrix of $S$ is $M_S = BLAH$ and the matrix of $T$ is $M_T = BLAH$, find the matrix of $T \circ S$.
	\end{question}
	
	\begin{definition}
		If $M$ is a $m\times n$ matrix and $N$ is a $k \times m$ matrix, then the \textit{product} $NM$ of the matrices is
		defined as the matrix of the composition of the linear maps defined $M$ and $N$.  In other words  $MN$ is the matrix of 
		$L_M\ circ L_N$.
	\end{definition}
	
	WARNING:  you may have seen another definition for matrix multiplication in the past.  That definition could be seen as a shortcut for how
	to compute the product, but it is usually presented devoid of mathematical meaning.  Hopefully our definition seems properly motivated:  matrix definition is 
	just what you do to compose linear maps.  We suggest working out the problems here using our definition:  you will develop your own efficient shortcuts in time.
	
	\begin{question}
		If $M = BLAH$ and $N=BLAH$, compute $NM$.
	\end{question}
	
	Note:  it may be easier to think about the following questions if you think about linear maps first, and convert everything over to matrices after.
	
	\begin{question}
		Find $2\times 2$ matrices $A$ and $B$ with $AB \neq BA$.
	\end{question}
	
	\begin{question}
		Find $A \neq 0$  with $AA = 0$.
	\end{question}
	
	\begin{question}
		If $A = BLAH$, find $v \neq 0$ with $Av = 0$ 
	\end{question}
	
	\begin{question}
		If $A = BLAH$, find $v$ with $Av = BLAH$ 
	\end{question}
	
	In the last two exercises, you found that solving matrix equations is equivalent to solving systems of linear equations.
	
	\begin{question}
	Rewrite  $System of linear equations$ as $matrix equation$.
	\end{question}
	
	\begin{python}
		#We will store a matrix as a list of lists.   For example the list [[1,2],[3,4],[5,6]] will represent the matrix 
		# 1 3 5
		# 2 4 6
		
		#write a function multiply(M_1,M_2) which takes two matrices stored in the above format, and returns the matrix of their product
		
		def multiply(M_1,M_2):
			#your code here
		
		#Write a function makeFunction(M) which takes a matrix and returns the linear map associated to the matrix.
		#For example, makeFunction([[1,2],[3,4]])([2,5]) = [12 32]
		
		def makeFunction(M):
			#your code here
		
		#Write a function makeMatrix(L) which takes a linear function L ( given as a python function which takes and returns lists) and returns the matrix associated to L.
		#For example if  
		#def L(v):
		#  return [2*v[0]+3*v[1], -4*v[0]]
		#Then makeMatrix(L) = BLAH (polymorphism is weird here)
		
		def makeMatrix(L):
			#your code here
		
		#You should double check for some examples that the following is true (as it ought to be)
		# makeMatrix(makeFunction(M_1)(makeFunction(M_2))) = multiply(M_1,M_2)
		
		
	\end{python}
	
	
	
	
	

\end{document}