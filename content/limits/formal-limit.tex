\section{The $\epsilon -\delta$ definition of a limit}

This optional section explores limits a formal and rigorous point of view.   
The level of mathematical maturity required to get through this section is much higher than others. 
If you get through it and understand everything, you can consider yourself "hardcore".

\begin{definition}
	Let $U \subset \R^n$.  The \textit{closure} $\overline{U}$ of $U$ is defined to be the set of all $\mathbf{p} \in \R^n$ such that
	every solid ball centered at $p$ contains at least one point of $U$.  Symbolically 
	$\overline{U} = \{ \mathbf{p} \in \R^n : \forall r>0 \exists \mathbf{x} \in U \textit{ with } |x-p|<r\}$.
\end{definition}

\begin{question}
Prove that $U \subset \overline{U}$ for any subset $U$ of $\R^n$.
\end{question}

\begin{question}
Prove that the closure of the open unit ball is the closed unit ball.  That is show that if $U = \{\mathbf{x}:|x|<1\}$, then the closure of $U$ is $\{\mathbf{x}:|x|\leq 1\}$
\end{question}

\begin{definition}
	Let $f:U \to  V$  with $U \subset \R^n$, $V \subset \R^m$ and $\mathbf{p} \in \overline{U}$.  We say that $\lim_{\mathbf{x} \to \mathbf{p}}f(x) = \mathbf{L}$ if for every $\epsilon >0$ we can find a 
	$\delta>0$ so that if $0<|\mathbf{x}-\mathbf{p}|<\delta$ and $\mathbf{x} \in U$, then $|\mathbf{p}-\mathbf{x}|<\epsilon$.
\end{definition}

BADBAD: Prove sum, product, composition are continuous.  Division away from zeros of denominator, etc

\begin{question}
Give an epsilon delta proof that lim BLAH =BLAH.
\end{question}
