\begin{document}
\section{$\R^n$ as an inner product space}
In order to do geometry in $\R^n$, we will want to be able to compute the length of a vector, and the angle between two vectors.   Miraculously, a single operation will allow us to
compute both quantities.

\begin{definition}
	A \textit{covector} on $\R^n$ is a linear map from $\R^n \to R$.  As a matrix, it is a single row of length $n$.
\end{definition}

\begin{example}
	$\begin{bmatrix} 2 & -1 & 3 \end{bmatrix}$ is the matrix of a covector on $\R^3$.
\end{example}

\begin{question}
	What is $\begin{bmatrix} 2 & -1 & 3 \end{bmatrix} \begin{bmatrix} 3 \\5\\7 \end{bmatrix}$?
\end{question}

\begin{question}
	What is $\begin{bmatrix} x_1 & x_2 & x_3 \end{bmatrix} \begin{bmatrix} y_1 \\y_2\\y_3 \end{bmatrix}$?
\end{question}

There is a natural way to turn a vector into a covector, or a covector into a vector:  just turn the matrix $90^\circ$ one direction or the other!

\begin{definition}
	We define the transpose of a vector $v = \begin{bmatrix} x_1 \\x_2\\ .\\.\\. \\ x_n\ \end{bmatrix}$ to be the covector $v^\top$ with matrix 
	$\begin{bmatrix} x_1 &x_2& ... &x_n \end{bmatrix}$.
	
	Similarly we define the transpose of a covector $\omega: \begin{bmatrix} x_1 &x_2& ... &x_n\ \end{bmatrix}$ to be the vector $\omega^\top$ with matrix
	$\begin{bmatrix} x_1 \\x_2\\ .\\.\\. \\ x_n\ \end{bmatrix}$.  
\end{definition}

Observe that $(v^\top)^\top = v$ and $(\omega^\top)^\top = \omega$ for any vector $v$ and covector $\omega$.

\begin{question}
	If $v = \begin{bmatrix}  5 \\ 3 \\ 1\end{bmatrix}$ and $w = \begin{bmatrix}  2 \\ -2 \\ 7\end{bmatrix}$ what is $v^\top(w)$? What is $w \circ v^\top$?
\end{question}

\begin{definition}
	Given two vectors $v,w \in \R^n$, we define their \textit{standard inner product} $\langle v, w\rangle$ by $\langle v,w \rangle = v^\top(w) \in \R$.  We sometimes use the notation 
	$v \cdot w$ for $\langle v , w \rangle$, and call the operation a textit{dot product}. 
\end{definition}


\begin{question}
	Make sure for yourself , by using the definition, that 
	$\begin{bmatrix} x_1 \\x_2\\ .\\.\\. \\ x_n\ \end{bmatrix} \cdot \begin{bmatrix} y_1 \\y_2\\ .\\.\\. \\ y_n\ \end{bmatrix}  = x_1y_1+x_2y_2+x_3y_3 + ...+x_ny_n$
\end{question}

\begin{question}
	Prove the following facts about the dot product.  $u,v,w \in \R^n$ and $a \in \R$
		\begin{itemize}
			\item $v \cdot w = w \cdot v$ (The dot product is commutative)
			
			\item $(u+v)\cdot w = u\cdot w + v\cdot w$ 	and $(av)\cdot w = a(v \cdot w)$ (The dot product is linear in the first argument)
			
			\item $u \cdot (v+w) = u\cdot v+ u\cdot w$ and  $v \cdot (aw) = a(v \cdot  w)$ (The dot product is linear in the second argument)
			
			\item $v\cdot v \geq 0$ (The dot product is positive definite)
			
			\item if $v \cdot z = 0$ for all $z \in \R^n$, then $v =0$ (The dot product is nondegenerate)
		\end{itemize}
\end{question}

So far, the inner product feels like it belongs to the realm of pure algebra.  In the next few exercises, we will start to see some hints of its geometric meaning.

\begin{question}
	Let $v  = \begin{bmatrix}  5  \\ 1\end{bmatrix}$.  What is $\langle v,v\rangle$?  
	
	Using plane geometry, compute the distance from the point $(0,0)$ to the point $(5,1)$.  What is this distance?
	
	How are these two related?
\end{question}

\begin{question}
	$v  = \begin{bmatrix}  x  \\ y\end{bmatrix}$.  What is $\langle v,v\rangle$?  
	
	Using plane geometry, compute the distance from the point $(0,0)$ to the point $(x,y)$.  What is this distance?
	
	How are these two related?
\end{question}

You should have discovered that $v\cdot v$ is the square of the length of the vector $v$ when viewed as an arrow based at the origin.  So far, you have only shown this in the 
$2$ dimensional case.  See if you can do it in three dimensions:

\begin{problem}
	Show that the length of the line segment from $(0,0,0)$ to $(x,y,z)$ is $\sqrt{v \cdot v}$, where $v  = \begin{bmatrix}  x  \\ y \\ z\end{bmatrix}$.
\end{problem}

Until now, you have probably not seen a treatment of length in higher dimensions.  Generalizing the results of the previous two problems, we define

\begin{definition}
	The \textit{length} of a vector $v \in \R^n$ is defined by $|v| = \sqrt{v \cdot v}$.
\end{definition}

\begin{question}
	What is the length of the vector $\verticalvector(6,2,3,1)$?
\end{question}

\begin{question}
	What is the distance between the points $(2,3)$ and $(5,9)$?
\end{question}

\begin{question}
	What is the distance between the points $(2,7,3,1)$ and $(5,6,9,8)$?
\end{question}

\begin{question}
	Write an equation for the sphere in $\R^4$ of radius $r$ using the coordinates $x,y,z,w$ on $\R^4$.
\end{question}

\begin{question}
	Write an inequality stating that the point $(x,y,z,w)$ is more than $4$ units away from the point $(2,3,1,9)$
\end{question}

\begin{question}
	Show that $|a\vec{v}| = |a| |\vec{v}|$ for every $a \in \mathbb{R}$. (Note these two uses of $|\cdot|$ are distinct:  $|a|$ means the absolute value of $a$, and $|\vec{v}|$ is the 
	length of $\vec{v}$)
\end{question}

\begin{question}
	Give a vector of length $1$ which points in the same direction as $\verticalvector(1,2)$.
\end{question}

\begin{question}
	Given a nonzero vector $\vec{v} \in \R^n$, show that $\frac{1}{|\vec{v}|}\vec{v}$ is a unit vector.
\end{question}

Now that we understand the relationship between the inner product and length of vectors, we will attempt to establish a connection between the inner product and the 
angle between two vectors.

Do you remember the law of cosines?  It states the following:

\begin{theorem}
	BADBAD PICTURE
	If a triangle has side lengths $a$,$b$, and $c$, then $c^2 = a^2+b^2 - 2ab\cos(\theta)$, where $\theta$ is the angle opposite the side with length $c$.
\end{theorem}

\begin{question}
Prove the law of cosines.
\end{question}

We can rephrase this in terms of vectors, since geometrically if $\vec{v}$ and $\vec{w}$ are vectors, the third side of the triangle is the vector $w-v$.

BADBAD PICTURE

\begin{theorem}
	For any two vectors $v,w \in \R^n$, $|w-v|^2 = |w|^2 +|v^2| - 2|v| |w| \cos(\theta) $, where $\theta$ is the angle between $v$ and $w$.
\end{theorem}

(For you sticklers, this is really being taken as the definition of the angle between two vectors in arbitrary dimension)

\begin{question}
	Rewrite the theorem above by using our definition of length in terms of the dot product.  Performing some algebra you should obtain a nice expression for $v\cdot w$
	in terms of $|v|, |w|$, and $\cos(\theta)$.
 \end{question}
 
 You should have discovered the following theorem:
 
 \begin{theorem}
 	For any two vectors $v,w \in \R^n$,  $v \cdot w = |v||w|\cos(\theta)$.  In words, the dot product of two vectors is the product of the lengths of the two vectors,
 	times the cosine of the angle between them.
 \end{theorem}
 
 This gives an almost totally geometric picture of the dot product:  Given two vectors $\vec{v}$ and $\vec{w}$,  $|\vec{v}\cos(\theta)|$ can be viewed as the length of the
 projection of $\vec{v}$ onto the line containing $\vec{w}$.  So $|\vec{v}||\vec{w}|\cos(\theta)$ is the ``length of the projection of $\vec{v}$ in the direction of $\vec{w}$ times the
 length of $\vec{w}$''.
 
 BADBAD PICTURE
 
 As mentioned above, this theorem is really being used to \textit{define} the angle between two vectors.  This is not quite rigorous:  how do we even know that 
 $\frac{v \cdot w}{|v||w|}$ is even between $-1$ and $1$, so that it could be the cosine of an angle?  This is clear from the ``Euclidean Geometry'' perspective, 
 but not as clear from the ``Cartesian Geometry'' perspective.  To make sure that everything is okay, we prove the ``Cauchy-Schwarz'' theorem which reconciles these two
 worlds:
 
 \begin{theorem}
 	$v \cdot w \leq |v||w|$ for any two vectors $v,w \in \R^n$
 \end{theorem} 
 
 \begin{proof}
 	If $\vec{v}$ or $\vec{w}$ is the zero vector, the result is trivial.  So assume $\vec{v} \neq \vec{0}$ and $\vec{w} \neq \vec{0}$
	Start by noting that $\langle v-w,v-w\rangle \geq 0 $.  Expanding this out, we have:
	\[
		\langle v,v \rangle - 2 \langle v,w \rangle +\langle  w,w\rangle \geq 0
		2\langle v,w \rangle \leq \langle v,v \rangle  + \langle w,w\rangle
	\]
	
	Now, if $\vec{v}$ and $\vec{w}$ are unit vectors, this says that
	
	\[
	2\langle \vec{v}, \vec{w} \rangle \leq 2
	\langle \vec{v},\vec{w}\rangle \leq 1
	\]
	
	Now to prove the result for any pair of nonzero vectors, simply scale them to make them unit vectors:
	
	\[
	\langle \frac{1}{|\vec{v}|}\vec{v},\frac{1}{|\vec{w}|}\vec{w}\rangle \leq 1
	\langle v,w\rangle \leq  |v||w|
	\]
 \end{proof}
   
In the next question, we ask you to fill in the details of an alternative proof which, while a little harder than the one above, is at least as beautiful.

\begin{question}
	Start by noting that $\langle v-w,v-w\rangle \geq 0 $.  Expanding this out, we have:
	\[
		\langle v,v \rangle - 2 \langle v,w \rangle +\langle  w,w\rangle \geq 0
		2\langle v,w \rangle \leq \langle v,v \rangle  + \langle w,w\rangle
	\]
	
	Now notice that the left hand side is unaffected by scaling $v$ by a scalar $\lambda$ and $w$ by $\frac{1}{\lambba}$, but the left hand side is!
	This allows us to breath new life into the inequality:  we know that for every scalar $\lambda \in (0,\infty)$ 
	
	\[
		\langle  v,w \rangle \leq \lambda^2 |v|^2 + \frac{1}{\lambda^2} |w|^2
	\] 
	
	This is somewhat miraculous:  we have a stronger inequality than the one we started with ``for free".
	
	This new inequality is strongest when the left hand side (LHS) is minimized.  As it stands the LHS is just a function of one real variable $\lamba$.
	
	What is the value of $\lambda$ which minimizes the LHS? 
	\begin{answer} \sqrt{|w|/|v|}\end{answer}
	
	Conclude that the Cauchy-Schwarz theorem is true!
	
	Credit for this beautiful line of reasoning goes to Terry Tao at \url{https://terrytao.wordpress.com/2007/09/05/amplification-arbitrage-and-the-tensor-power-trick/}.
	
\end{question}


 \begin{question}
 	What is the angle between the vectors $BLAH$  and $BLAH$?
 \end{question}
	 	
 	\begin{question}
 		Find a vector which is perpendicular to $\verticalvector(2,3,1)$.  
 	\end{question}
 	
 	\begin{question}
 		Find a vector which is perpendicular to both $\verticalvector(2,3,1)$ and $\verticalvector(5,9,2)$
 	\begin{question}
	
\begin{question}
	Prove the ``Triangle inequality'':  For any two vector $\vec{v},\vec{w} \in \R^n$, $|\vec{v}+\vec{w}|\leq |\vec{v}+\vec{w}|$.  Draw a picture.  Why is this called the triangle inequality?
\end{question}
 	

\end{document}