\begin{document}
\section{The Single Variable Derivative, revisited}
	
	Our goal is to define the derivative of a multivariable function, but first we will recast the derivative of a single variable function in a
	manner which is ripe for generalization.
	
	The derivative of a function $f:\R \to \R$ at an abscissa $x = a$ is the "instantaneous rate of change" of $f(x)$ with respect to $x$.
	In other words, 
	
	$f(a + \Delta x) \approx f(x) +f'(a)\Delta x$.
	
	This is really the essential thing to understand about the derivative.
	
	BADBAD conceptual mooculus questions on derivative here.
	
	We have not made precise what we mean the approximate sign.  After all, if $\Delta x$ is small enough and $f$ is continuous, 
	$f(a+\Delta x)$ will be close to $f(x)$, but we do not want to say that the derivative is always $0$!  We will make the  $\approx$ sign precise by asking that the difference 
	between the actual value and the estimated value go to $0$ faster than $\Delta x$ does:
	
	\begin{definition}
		Let $f: \R \to \R$ be a function, and let $a \in \R$.  $f$ is said to be differentiable at $x=a$ if there is a number $m$ such that 
		
		\[ f(a+\Delta x) = f(a) + m\Delta x + Error_a(\Delta x)\]
		
		\[ \lim_{\Delta x \to 0} \frac{\left|Error_a(\Delta x)|}{\left|\Delta x\right|} = 0 \].
		
		If $f$ is differentiable at $a$, there is only one such number $m$, which we call the derivative of $f$ at $a$.  
		
		Verbally,  $m$ is the number which makes the error between the function value $f(a+\Delta x)$ and the linear approximation $f(a)+m\Delta x$ go to zero 
		"faster than $\Delta x$" does.
	\end{definition}
	
	This definition looks more complicated than the usual definition (and it is!), but it has the advantage that it will 
	generalize directly to the derivative of a multivariable function.
	
	\begin{question}
		Confirm that for $f(x)=x^2$, $f'(2)=4$ using our definition of the derivative.
	\end{question}
	
	\begin{question}
		Show the equivalence of our definition of the derivative with the ``usual'' definition.  That is, show that the number $m$ in our definition satisfies
		$m = \lim_{\Delta x \to 0}\frac{f(a+\Delta x)-f(a)}{\Delta x}$.  This also shows the uniqueness of $m$.
	\end{question}
	
	\end{document}