\documentclass{ximera}
  
\title{Covectors}

\begin{document}

\begin{abstract}
  A covector eats vectors and provides numbers.
\end{abstract}

\begin{definition}
  A \textit{covector} on $\R^n$ is a linear map from $\R^n \to R$.

  As a matrix, it is a single row of length $n$.
\end{definition}

\begin{example}
  $\begin{bmatrix} 2 & -1 & 3 \end{bmatrix}$ is the matrix of a
  covector on $\R^3$.
\end{example}

\begin{question}
  \begin{solution}
    \begin{hint}
      $\begin{bmatrix} 2 & -1 & 3 \end{bmatrix} \begin{bmatrix} 3\\5\\7 \end{bmatrix} = 2(3)+-1(5)+3(7) = 22$
    \end{hint}
    $\begin{bmatrix} 2 & -1 & 3 \end{bmatrix} \begin{bmatrix} 3\\5\\7 \end{bmatrix} = $\answer{22}
  \end{solution}

  Now we can do this a bit more abstractly.

  \begin{hint}
    $\begin{bmatrix} x & y & z \end{bmatrix} \begin{bmatrix} a \\b\\c\end{bmatrix} = ax+by+cz$ 
  \end{hint}
  $\begin{bmatrix} x & y & z \end{bmatrix} \begin{bmatrix} a \\b\\c\end{bmatrix} =$ \answer{$ax+by+cz$}
\end{question}

There is a natural way to turn a vector into a covector, or a covector into a vector:  just turn the matrix $90^\circ$ one direction or the other!

\begin{definition}
  We define the transpose of a vector $v = \begin{bmatrix} x_1 \\x_2\\ \vdots \\ x_n\ \end{bmatrix}$ to be the covector $v^\top$ with matrix 
  $\begin{bmatrix} x_1 &x_2& \cdots &x_n \end{bmatrix}$.
	
  Similarly we define the transpose of a covector $\omega: \begin{bmatrix} x_1 &x_2& \cdots &x_n\ \end{bmatrix}$ to be the vector $\omega^\top$ with matrix
  $\begin{bmatrix} x_1 \\x_2\\ \vdots \\ x_n\ \end{bmatrix}$.  
\end{definition}

\begin{question}
  Suppose $\vec{v} = \begin{bmatrix} 1 \\ 4 \\ 3 \end{bmatrix}$.  What is $(\vec{v}^\top)^\top$?

  \begin{solution}
    \begin{multiple-choice}
      \choice[correct]{$\vec{v}^\top)^\top = \begin{bmatrix} 1 \\ 4 \\ 3 \end{bmatrix}$}
      \choice{$\vec{v}^\top)^\top = \begin{bmatrix} 1 & 4 & 3 \end{bmatrix}$}
    \end{multiple-choice}
  \end{solution}

  Indeed, $(\vec{v}^\top)^\top = \vec{v}$ and $(\omega^\top)^\top = \omega$ for any vector $\vec{v}$ and covector $\omega$.

  Let $v = \begin{bmatrix}  5 \\ 3 \\ 1\end{bmatrix}$ and $w = \begin{bmatrix}  2 \\ -2 \\ 7\end{bmatrix}$ 
  \begin{solution}
    \begin{hint}
      $v^\top(w) = \begin{bmatrix} 5 & 3 & 1\end{bmatrix} \verticalvector{2\\-2\\7} = 5(2)+3(-2)+1(7) = 11$
    \end{hint}
    $v^\top(w)=$ \answer{11}?
  \end{solution}

  \begin{solution}
    \begin{hint}
      \begin{align*}
        w(v^\top) &=  \verticalvector{2\\-2\\7}\begin{bmatrix} 5 & 3 & 1\end{bmatrix}\\
        &=\begin{bmatrix}
          10  & 6 & 2\\
          -10 & -6& -2\\
          35  & 21& 7
        \end{bmatrix}
      \end{align*}
    \end{hint}
    What is $wv^\top$? 
    \begin{matrix-answer}
      correctMatrix = [['10','6','2'],['-10','-6','-2'],['35','21','7']]
    \end{matrix-answer}
  \end{solution}
\end{question}


\end{document}

%%% Local Variables: 
%%% mode: latex
%%% TeX-master: t
%%% End: 
