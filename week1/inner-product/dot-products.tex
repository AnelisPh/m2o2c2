\documentclass{ximera}



\title{Dot product}

\begin{document}

\begin{abstract}
  The standard inner product is the dot product.
\end{abstract}
\maketitle

\begin{definition}
  Given two vectors $\vec{v},\vec{w} \in \R^n$, we define their \textit{standard inner product} $\langle \vec{v}, \vec{w}\rangle$ by $\langle \vec{v},\vec{w} \rangle = \vec{v}^\top(\vec{w}) \in \R$.  We sometimes use the notation 
  $\vec{v} \cdot \vec{w}$ for $\langle \vec{v} , \vec{w} \rangle$, and call the operation the \textit{dot product}. 
\end{definition}

\begin{warning}
  Note that $\vec{v}^\top(\vec{w}) \neq \vec{w}(\vec{v}^\top)$: one is a number, while the other is an $n\times n$ matrix.
\end{warning}

\begin{question}
  Make sure for yourself, by using the definition, that 
  \[\begin{bmatrix} x_1 \\x_2\\ \vdots \\ x_n\ \end{bmatrix} \cdot \begin{bmatrix} y_1 \\y_2\\ \vdots \\ y_n\ \end{bmatrix}  = x_1y_1+x_2y_2+x_3y_3 + \cdots +x_ny_n.\]
\end{question}

Prove the following facts about the dot product.  $\vec{u},\vec{v},\vec{w} \in \R^n$ and $a \in \R$
\begin{enumerate}
\item $\vec{v} \cdot \vec{w} = \vec{w} \cdot \vec{v}$ (The dot product is commutative)
			
\item $(\vec{u}+\vec{v})\cdot \vec{w} = \vec{u}\cdot \vec{w} + \vec{v}\cdot \vec{w}$ 	and $(a\vec{v})\cdot \vec{w} = a(\vec{v} \cdot \vec{w})$ (The dot product is linear in the first argument)
			
\item $\vec{u} \cdot (\vec{v}+\vec{w}) = \vec{u}\cdot \vec{v}+ \vec{u}\cdot \vec{w}$ and  $\vec{v} \cdot (a\vec{w}) = a(\vec{v} \cdot  \vec{w})$ (The dot product is linear in the second argument)
			
\item $\vec{v}\cdot \vec{v} \geq 0$ (We say that the dot product is ``positive definite'')
			
\item if $\vec{v} \cdot \vec{z} = 0$ for all $\vec{z} \in \R^n$, then $\vec{v} =\vec{0}$ (The dot product is nondegenerate)
\end{enumerate}
	
\begin{free-response}
  $1.$  $\vec{v} \cdot \vec{w} = v_1w_1+v_2w_2+...+v_nw_n = w_1v_1+w_2v_2+...+w_nv_n = w \cdot v$, so the dot product is commutative.
  \\
  \\
  (skipping item $2$ for now)
  
  $3.$ \begin{align*}
    \vec{u} \cdot (v+\vec{w}) &= \vec{u}^\top(v +\vec{w}) \text{ by definition}\\
    &=\vec{u}^\top(v) + \vec{u}^\top(\vec{w}) \text{ since $\vec{u}^\top: \R^n \to \R$ is linear}\\
    &=\vec{u} \cdot v+\vec{u}\cdot \vec{w} \text{ by definition}
  \end{align*}
  
  and 
  
  \begin{align*}
    \vec{u}\cdot(a\vec{w}) &=\vec{u}^\top(a\vec{w}) \text{ by definition}\\
    &= a\vec{u}^\top(\vec{w}) \text{ since $\vec{u}^\top: \R^n \to \R$ is linear}\\
    &=a\vec{u}\cdot \vec{w} \text{ by definition}
  \end{align*}
  \\
  \\
  $2.$ follows from $3$ and $1$
  \\
  \\
  $4.$ $\vec{v} \cdot \vec{v} = v_1^2+v_2^2+v_3^2+...+v_n^2$, and the square of a real number is nonnegative, so the sum of these squares is also nonnegative.
  \\
  \\
  $5.$ is perhaps the trickiest fact to prove.  Observe that if $\vec{v} \cdot \vec{z} =0$ for every $\vec{z} \in \R^n$, then this formula is true
  in particular for $z=\vec{e}_j$.  But $\vec{v} \cdot \vec{e}_j = v_j$.  Thus, by dotting with all of the standard basis vectors, we see that every coordinate of $\vec{v}$ must
  be $0$.  Thus $\vec{v}$ is the zero vector
\end{free-response}
	
The fact that the dot product is linear in two separate vector
variables means that it is an example of a ``bilinear form''.  We will
make a careful study of bilinear forms later in this course: it will
turn out that the second derivative of a multivariable function gives
a bilinear form at each point.

So far, the inner product feels like it belongs to the realm of pure
algebra.  In the next few exercises, we will start to see some hints
of its geometric meaning.

\begin{question}
  Let $v  = \begin{bmatrix}  5  \\ 1\end{bmatrix}$.   
	
  \begin{solution}
    \begin{hint}
      $\langle \vec{v},\vec{v} \rangle = 5^2+1^2 = 26$
    \end{hint}
    $\langle \vec{v},\vec{v} \rangle = $ \answer{26}
  \end{solution}

  Let's think about this a bit more abstractly.
  Set $v  = \begin{bmatrix}  x  \\ y\end{bmatrix}$. 
	
  \begin{solution}
    \begin{hint}
      $\langle \vec{v},\vec{v} \rangle = x^2+y^2$
    \end{hint}
    $\langle \vec{v},\vec{v} \rangle = $ \answer{$x^2+y^2$}    
    
  \end{solution}
  
  Notice that the length of the line segment from $(0,0)$ to $(x,y)$ is $\sqrt{x^2+y^2}$ by the Pythagorean theorem.
	
\end{question}

\end{document}
%%% Local Variables: 
%%% mode: latex
%%% TeX-master: t
%%% End: 
