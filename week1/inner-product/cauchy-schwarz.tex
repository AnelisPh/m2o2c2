\documentclass{ximera}

\newcommand{\R}{\mathbb R}

\newcommand{\href}[2]{#2\footnote{\url{#1}}}
\newcommand{\verticalvector}[1]{\begin{bmatrix}#1\end{bmatrix}}
\newcommand{\gt}{>}


\title{Cauchy-Schwarz}

\begin{document}

\begin{abstract}
  The Cauchy-Schwarz inequality relates the inner product and the norm of the two vectors.
\end{abstract}
\maketitle

This is the Cauchy-Schwarz inequality.

\begin{theorem}
  $|v \cdot w| \leq |v||w|$ for any two vectors $v,w \in \R^n$
\end{theorem} 
 
\begin{proof}
  If $\vec{v}$ or $\vec{w}$ is the zero vector, the result is trivial.  So assume $\vec{v} \neq \vec{0}$ and $\vec{w} \neq \vec{0}$
  Start by noting that $\langle v-w,v-w\rangle \geq 0 $.  Expanding this out, we have:
  \begin{align*}
  \langle v,v \rangle - 2 \langle v,w \rangle +\langle  w,w\rangle &\geq 0 \\
  2\langle v,w \rangle &\leq \langle v,v \rangle  + \langle w,w\rangle
  \end{align*}
  
  Now, if $\vec{v}$ and $\vec{w}$ are unit vectors, this says that
  
  \begin{align*}
  2\langle \vec{v}, \vec{w} \rangle &\leq 2 \\
  \langle \vec{v},\vec{w}\rangle &\leq 1
  \end{align*}
  
  Now to prove the result for any pair of nonzero vectors, simply scale them to make them unit vectors:
  
  \begin{align*}
  \langle \frac{1}{|\vec{v}|}\vec{v},\frac{1}{|\vec{w}|}\vec{w}\rangle &\leq 1\\
  \langle v,w\rangle &\leq  |v||w|
  \end{align*}
  
\end{proof}

We are not quite done with the proof, because we have not proven that $v \cdot w \geq -|v||w|$.  
Following the same basic outline, try to prove the other half of this inequality below.
\begin{free-response}
  Start by noting that $\langle v+w,v+w\rangle \geq 0 $.  Expanding this out, we have:
  \begin{align*}
  \langle v,v \rangle + 2 \langle v,w \rangle +\langle  w,w\rangle &\geq 0\\
  2\langle v,w \rangle &\geq -\langle v,v \rangle  + -\langle w,w\rangle
  \end{align*}
  
  Now, if $\vec{v}$ and $\vec{w}$ are unit vectors, this says that
  
  \begin{align*}
  2\langle \vec{v}, \vec{w} \rangle &\geq -2\\
  \langle \vec{v},\vec{w}\rangle &\geq -1
  \end{align*}
  
  Now to prove the result for any pair of nonzero vectors, simply scale them to make them unit vectors:
  
  \begin{align*}
  \langle \frac{1}{|\vec{v}|}\vec{v},\frac{1}{|\vec{w}|}\vec{w}\rangle &\geq -1\\
  \langle v,w\rangle &\leq  -|v||w|
  \end{align*}
\end{free-response}

In the next question, we ask you to fill in the details of an alternative proof which, while a little harder than the one above, is at least as beautiful.

\begin{question}
  Start by noting that $\langle v-w,v-w\rangle \geq 0 $.  Expanding this out, we have:
  \begin{align*}
  \langle v,v \rangle - 2 \langle v,w \rangle +\langle  w,w\rangle &\geq 0 \\
  2\langle v,w \rangle &\leq \langle v,v \rangle  + \langle w,w\rangle
  \end{align*}
  
  Now notice that the left hand side is unaffected by scaling $v$ by a scalar $\lambda$ and $w$ by $\frac{1}{\lambda}$, but the right hand side is!
  This allows us to breathe new life into the inequality:  we know that for every scalar $\lambda \in (0,\infty)$ 
  
  \[
  \langle  v,w \rangle \leq \lambda^2 |v|^2 + \frac{1}{\lambda^2} |w|^2
  \] 
  
  This is somewhat miraculous:  we have a stronger inequality than the one we started with ``for free.''
  
  This new inequality is strongest when the right hand side (RHS) is minimized.  As it stands the RHS is just a function of one real variable $\lambda$.
  
  \begin{solution}
    \begin{hint}
      We can minimize the right hand side using single variable calculus.
    \end{hint}
    \begin{hint}
      Let $f(\lambda) = \lambda^2 |v|^2 + \frac{1}{\lambda^2} |w|^2$.
     \\
     \\
      Then $f'(\lambda) = 2\lambda |v|^2 -2\frac{|w|^2}{\lambda^3}$
      \\
      \\
      The minimum must occur where $f'$ vanishes
    \end{hint}
    \begin{hint}
      \begin{align*}
        f'(\lambda) &= 0\\
        2\lambda |v|^2 -2\frac{|w|^2}{\lambda^3} &= 0\\
        \lambda^4 |v|^2 &=|w|^2\\
        \lambda = \sqrt{\frac{|w|}{|v|}}
      \end{align*}
    \end{hint}
    \begin{hint}
      You can type $|w|$ by writing \verb|abs(w)|.
    \end{hint}
    The value of $\lambda$ which minimizes the left hand side is \answer{sqrt(abs(w)/abs(v))}	
  \end{solution}
  
  Conclude that the Cauchy-Schwarz theorem is true!
  
  Credit for this beautiful line of reasoning goes to Terry Tao at 
  \href{https://terrytao.wordpress.com/2007/09/05/amplification-arbitrage-and-the-tensor-power-trick/}{this blog post}.
  
\end{question}

\begin{question}
  \begin{solution}
    \begin{hint}
      We know that $\vec{v}\cdot\vec{w} =  |\vec{v}||\vec{w}|\cos{\theta}$
    \end{hint}
    \begin{hint}	
      $\verticalvector{2\\3\\1} \cdot \verticalvector{1\\1\\1} = 2(1)+3(1)+1(1) = 6$
    \end{hint}
    \begin{hint}
      $|\vec{v}| = \sqrt{\vec{v}\cdot\vec{v}} = \sqrt{14}$
    \end{hint}
    \begin{hint}
      $|\vec{w}| = \sqrt{\vec{w}\cdot\vec{w}} = \sqrt{3}$
    \end{hint}
    \begin{hint}
      Thus, $6 = \sqrt{14}\sqrt{3}\cos(\theta)$
    \end{hint}
    \begin{hint}
      Therefore, $\theta = \arccos(\frac{6}{\sqrt{42}})$
    \end{hint}
    The angle between the vectors $\vec{v} = \verticalvector{2\\3\\1}$  and $\vec{w}=\verticalvector{1\\1\\1}$ is \answer{arccos(6/(sqrt(14)*sqrt(3)))}
  \end{solution}
  This problem probably would have stumped you before you started this activity!
\end{question}

\begin{question}
  Find a vector which is perpendicular to $\vec{w} = \verticalvector{2\\3\\1}$.  
  \begin{solution}
    \begin{hint}
      For $\vec{v}$ to be perpendicular to $\verticalvector(2,3,1)$, we would need that the angle between $\vec{v}$ and $\vec{w}$ is $\frac{\pi}{2}$ (or $\frac{-\pi}{2}$).
      In either case $\vec{v}\cdot\vec{w} = |\vec{v}||\vec{w}|\cos(\frac{\pm \pi}{2}) = 0$
      So we need to find a vector for which $\vec{v}\cdot \vec{w}=0$
    \end{hint}
    \begin{hint}
      Let $\vec{v} = \verticalvector{x\\y\\z}$.  Then 
      \begin{align*}
        \vec{v}\cdot\vec{w} &= 0\\
        \verticalvector{x\\y\\z}\cdot\verticalvector{2\\3\\1} &= 0\\
        2x+3y+z &= 0
      \end{align*}
    \end{hint}
    \begin{hint}
      There are a whole lot of choices for $x,y$, and $z$ that fit these criteria (In fact there is an entire \textit{plane} of vectors perpendicular to $\vec{w}$)
    \end{hint}
    \begin{hint}
      $\verticalvector{0\\1\\-3}$ works for instance.
    \end{hint}
    \begin{matrix-answer}
    function validator(m) {
      if (isWrongSize(m, 3, 1)) return false;

      var result = matrixProduct([['2','3','1']],unpackMatrix(m));

      return (isMatrixCorrect(result,[['0']]));
    }
    \end{matrix-answer}
  \end{solution}
\end{question}

\begin{question}
  Find a vector $\vec{u}$ which is perpendicular to both $\vec{v} = \verticalvector{2\\3\\1}$ and $\vec{w} = \verticalvector{5\\9\\2}$
  \begin{solution}
    \begin{hint}
      We need both $\vec{u} \cdot \vec{v}=0 $ and $\vec{u}\cdot \vec{w}=0$
    \end{hint}
    \begin{hint}
      Letting $\vec{u} = \verticalvector{x\\y\\z}$, we have the conditions
      \[
      \begin{cases}
        2x+3y+z=0\\
        5x+9y+2z=0
      \end{cases}
      \]
    \end{hint}
    \begin{hint}
      \begin{align*}
        \begin{cases}
          4x+6y+2z=0\\
          5x+9y+2z=0
        \end{cases}
        \\
        \begin{cases}
          x+3y = 0\\
          5x+9y+2z=0
        \end{cases}
      \end{align*}
    \end{hint}
    \begin{hint}
      Picking whatever you like for $x$, you should be able to find the other values now.  Try $x=3$.
    \end{hint}
    \begin{hint}
      $\verticalvector{3\\-1\\3}$ works.
    \end{hint}
    \begin{matrix-answer}
    function validator(m) {
      if (isWrongSize(m, 3, 1)) return false;

      var result = matrixProduct([['2','3','1']],unpackMatrix(m));

      if (! isMatrixCorrect(result,[['0']])) {
        feedback('This is not perpendicuar to [[2],[3],[1]]');
        return false;
      }

      var result = matrixProduct([['5','9','2']],unpackMatrix(m));

      if (isMatrixCorrect(result,[['0']]))
        return true;
      
      feedback('This is not perpendicular to [[5],[9],[2]]');
      return false;
    }
    \end{matrix-answer}
  \end{solution}
\end{question}
    
Prove the ``Triangle inequality'':  For any two vectors $\vec{v},\vec{w} \in \R^n$, $|\vec{v}+\vec{w}|\leq |\vec{v}|+|\vec{w}|$.  Draw a picture.  Why is this called the triangle inequality?


\begin{free-response}
  The inequality is equivalent to $|\vec{v}+\vec{w}|^2\leq \left||\vec{v}|+|\vec{w}|\right|^2$, which is easier to handle because it does not involve square roots.
  
  \begin{align*}
    |\vec{v}+\vec{w}|^2  &= \langle \vec{v}+\vec{w},\vec{v}+\vec{w}\rangle\\
    &= |v|^2+2\langle v,w\rangle + |w|^2\\
    &\leq |v|^2+2|v||w|+|w|^2 \text{ by the Cauchy-Schwarz inequality}\\
    &= (|v|+|w|)^2
  \end{align*}
\end{free-response} 	
    
\end{document}
%%% Local Variables: 
%%% mode: latex
%%% TeX-master: t
%%% End: 
