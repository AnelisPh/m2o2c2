\documentclass{ximera}

\newcommand{\R}{\mathbb R}

\newcommand{\href}[2]{#2\footnote{\url{#1}}}
\newcommand{\verticalvector}[1]{\begin{bmatrix}#1\end{bmatrix}}
\newcommand{\gt}{>}


\title{Length}

\begin{document}

\begin{abstract}
  The inner product provides a way to measure the length of a vector.
\end{abstract}
\maketitle


You should have discovered that $v\cdot v$ is the square of the length
of the vector $v$ when viewed as an arrow based at the origin.  So
far, you have only shown this in the $2$-dimensional case.  See if you
can do it in three dimensions.

Show that the length of the line segment from $(0,0,0)$ to $(x,y,z)$
is $\sqrt{\vec{v} \cdot \vec{v}}$, where $\vec{v} = \begin{bmatrix} x
  \\ y \\ z\end{bmatrix}$.

\begin{free-response}
  
\end{free-response}

Until now, you may not have seen a treatment of length in higher
dimensions.  Generalizing the results above, we define:

\begin{definition}
  The \textit{length} of a vector $\vec{v} \in \R^n$ is defined by $|v| = \sqrt{v \cdot v}$.
\end{definition}

\begin{question}
  \begin{solution}
    The length of the vector $\verticalvector{6\\2\\3\\1}=$ \answer{$sqrt(6^2+2^2+3^2+1^2)$}
  \end{solution}
\end{question}

\begin{question}
  \begin{solution}
    \begin{hint}
      By the Pythagorean theorem, we can see that  the distance is $\sqrt{(5-2)^2+(9-3)^2}$
    \end{hint}
    \begin{hint}
      We could also view this as the length of the vector $\verticalvector{3\\6}$ which ``points'' from $(2,3)$ to $(5,9)$.
    \end{hint}
    The distance between the points $(2,3)$ and $(5,9)$ is \answer{$sqrt(3^2+6^2)$}
  \end{solution}
\end{question}

\begin{definition}
  The distance between two points $\mathbf{p}$ and $\mathbf{q}$ in $\R^n$ is defined to be the length of the ``displacement'' vector $\vec{p} - \vec{q}$.
\end{definition}

\begin{question}
  \begin{solution}
    \begin{hint}
      The displacement vector between these points is $\verticalvector{5-2\\6-7\\9-3\\8-1} = \verticalvector{3\\1\\6\\7}$
    \end{hint}
    \begin{hint}
      The length of the displacement vector is $\sqrt{3^2+1^2+6^2+7^2}$
    \end{hint}
    The distance between the points $(2,7,3,1)$ and $(5,6,9,8)$ is \answer{$sqrt(3^2+1+6^2+7^2)$}
  \end{solution}
\end{question}

\begin{question}
  Write an equation for the sphere centered at $(0,0,0,0)$ in $\R^4$ of radius $r$ using the coordinates $x,y,z,w$ on $\R^4$.
  \begin{solution}
    \begin{hint}
      For a point $\mathbf{p}=(x,y,z,w)$ to be on the sphere of radius $r$ centered at $(0,0,0,0)$, the distance from $\mathbf{p}$
      to the origin must be $r$
    \end{hint}
    \begin{hint}
      $r = \sqrt{x^2+y^2+z^2+w^2}$
    \end{hint}
    \begin{hint}
      $x^2+y^2+z^2+w^2=r^2$
    \end{hint}
    \answer{$x^2+y^2+z^2+w^2$} $= r^2$
  \end{solution}
\end{question}

\begin{question}
  Write an inequality stating that the point $(x,y,z,w)$ is more than $4$ units away from the point $(2,3,1,9)$
  \begin{solution}
    \begin{hint}
      The distance between the point $(x,y,z,w)$ and $(2,3,1,9)$ is $\sqrt{(x-2)^2+(y-3)^2+(z-1)^2+(w-9)^2}$.
    \end{hint}
    \begin{hint}
      So we need $\sqrt{(x-2)^2+(y-3)^2+(z-1)^2+(w-9)^2} > 4$
    \end{hint}
    \answer{$sqrt((x-2)^2+(y-3)^2+(z-1)^2+(w-9)^2)$} $>4$ 
  \end{solution}
\end{question}

Prove that $|a\vec{v}| = |a| |\vec{v}|$ for every $a \in \R$. 

\begin{warning}
  These two uses of $|\cdot|$ are distinct: $|a|$ means the absolute
  value of $a$, and $|\vec{v}|$ is the length of $\vec{v}$.
\end{warning}

\begin{free-response}
  \begin{align*}
    |a\vec{v}|
    &= \sqrt{\langle a\vec{v},a\vec{v}\rangle} \text{ by definition}\\
    &= \sqrt{a^2\langle \vec{v},\vec{v}\rangle} \text{ by the linearity of the inner product in each slot }\\
    &= \sqrt{a^2} \sqrt{\langle \vec{v},\vec{v}\rangle}\\
    &= |a| |\vec{v}|
  \end{align*}
\end{free-response}

\end{document}
