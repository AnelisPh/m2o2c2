\documentclass{ximera}

\begin{document}

Programming Prompts:

#See if you can formalize the following functions in python

#Model the function f(x)=x^2 as a python function

def f(x):
	return #your code here
	
#Model the function g(x)=-1 if x<=0 , g(x)=1 if x>0 as a python function

def g(x):
	#Your code here
	
#Model the function h(x,y)=xysin(1/y) if y=/=0, 0 if y=0

def h(x,y):
	#Your code here
	
# Here is an example of a higher order function horizontalShift.  It takes a function f of one variable, and a horizontal shift H, and returns the function whose graph is the same as f, only shifted H units.

def horizontalShift(f,H):
	# first we define a new function shiftF which is the appropriate shift of f
	def shiftedF(x):  
		return f(x-H) 
	# then we return that function
	return shiftedF

#Try  letting  k= horizontalShift(g,3) 	, and try evaluating k(3), k(4), k(-2).  Also try horizontalShift(f,2)(3).

# Write a function forwardDifference which takes a function f:R --> R and returns another function forwardDifference(f):R -->R defined 
# by forwardDifference(f)(x) = f(x+1)-f(x).

def forwardDifference(f):
	#Your code here
	
# Write a function approximateDerivative which takes a function f:R -->R  and a number h and returns another function approximateDerivative(f,h)(x) = [f(x+h)-f(x-h)]/(2h)

def approximateDerivative(f,h):
	#Your code here
	
#Try letting f(x) = x^2, and g = approximateDerivative(f,0.001).  Then try evaluating g(0),g(1),g(2),g(3),g(4).  Does this agree with what you know about differentiation?

#Python supports lists

	myList = [3,6,9,12]

#Try print myList[2] to see how lists work

#Model the function F(x,y,z)=(x+y,z^2x) as a function which takes the list [x,y,z] and returns the list [x+y,z*z*x]

def F(inputList):
	#your code here
	
#Model dot:R^2 x R^2 --> R as a function which takes two lists, X = [x_1,x_2] and Y=[y_1,y_2], and returns x_1y_1+ x_2y_2

def dot(X,Y):
	#your code here
	


\end{document}

%%% Local Variables: 
%%% mode: latex
%%% TeX-master: t
%%% End: 
