\documentclass{ximera}

\title{Calculus}

\begin{document}

\begin{abstract}
  We can do some calculus with Python, too.
  FIXME This page does not show anything to do - see comments.
\end{abstract}

Let's try doing some single-variable calculus with a bit of Python.

% def approximateDerivative(f,h):
% #Your code here

% #Try letting f(x) = x^2, and g = approximateDerivative(f,0.001). Then try evaluating g(0),g(1),g(2),g(3),g(4). Does this agree with what you know about differentiation?

% #Python supports lists

% myList = [3,6,9,12]

% #Try print myList[2] to see how lists work

% #Model the function F(x,y,z)=(x+y,z^2x) as a function which takes the list [x,y,z] and returns the list [x+y,z*z*x]

% def F(inputList):
% #your code here

% #Model dot:R^2 x R^2 --> R as a function which takes two lists, X = [x_1,x_2] and Y=[y_1,y_2], and returns x_1y_1+ x_2y_2

% def dot(X,Y):
% #your code here

\end{document}

%%% Local Variables: 
%%% mode: latex
%%% TeX-master: t
%%% End: 
