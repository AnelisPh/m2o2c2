\documentclass{ximera}

\title{Calculus}

\begin{document}

\begin{abstract}
  We can do some calculus with Python, too.
\end{abstract}

Let's try doing some single-variable calculus with a bit of Python.

Let \texttt{epsilon} be a small, but positive number.  Suppose $f:\R
\to \R$ has been coded as a Python function \texttt{f} which takes a
real number and returns a real number.  Seeing as
$$
f'(x) = \lim_{h \to 0} \frac{f(x+h) - f(x)}{h},
$$
can you find a Python function which approximates $f'(x)$?

Given a Python function \texttt{f} which takes a real number and
returns a real number, we can approximate $f'(x)$ by using
\texttt{epsilon}.  Write a Python function \texttt{derivative} which
takes a function \texttt{f} and returns an approximation to its
derivative.

\begin{solution}
  \begin{hint}
    To approximate this, use \texttt{(f(x+epsilon) - f(x))/epsilon}.
  \end{hint}
\begin{python}
epsilon = 0.0001
def derivative(f):
  def df(x): return (f(blah blah) - f(blah blah)) / blah blah
  return df

def validator():
  df = derivative(lambda x: 1+x**2+x**3)
  if abs(df(2) - 16) > 0.01:
    return False
  df = derivative(lambda x: (1+x)**4)
  if abs(df(-2.642) - -17.708405152) > 0.01:
    return False
  return True
\end{python}
This is great!  In the future, we'll review this activity, and then extend it to a multivariable setting.
\end{solution}

% def approximateDerivative(f,h):
% #Your code here

% #Try letting f(x) = x^2, and g = approximateDerivative(f,0.001). Then try evaluating g(0),g(1),g(2),g(3),g(4). Does this agree with what you know about differentiation?

% #Python supports lists

% myList = [3,6,9,12]

% #Try print myList[2] to see how lists work

% #Model the function F(x,y,z)=(x+y,z^2x) as a function which takes the list [x,y,z] and returns the list [x+y,z*z*x]

% def F(inputList):
% #your code here

% #Model dot:R^2 x R^2 --> R as a function which takes two lists, X = [x_1,x_2] and Y=[y_1,y_2], and returns x_1y_1+ x_2y_2

% def dot(X,Y):
% #your code here

\end{document}

%%% Local Variables: 
%%% mode: latex
%%% TeX-master: t
%%% End: 
