\documentclass{ximera}

\title{Higher-order python}

\begin{document}

\begin{abstract}
  One nice feature of Python is that we can play with functions which act on functions.
\end{abstract}

\begin{question}
  Here is an example of a higher order function
  \verb|horizontal_shift|.  It takes a function $f$ of one variable,
  and a horizontal shift $H$, and returns the function whose graph is
  the same as $f$, only shifted horizontally by $H$ units.


  \begin{solution}
    Find a function $f$ so that \verb|horizontal_shift(f,2)| is the squaring function.
  
    \begin{python}
def horizontal_shift(f,H):
  # first we define a new function shifted_f which is the appropriate shift of f
  def shifted_f(x):  
    return f(x-H)
  # then we return that function
  return shifted_f
def f(x):
  return # a function so that horizontal_shift(f,2) is the squaring function

def validator():
  return (f(1) == 9) and (f(0) == 4) and (f(-3) == 1)
    \end{python}
  \end{solution}


  \begin{solution}
    Write a function \verb|forward_difference| which takes a function $f : \R \to \R$ and returns another real-valued function defined 
    by $\texttt{forward\_difference}(f)(x) = f(x+1)-f(x)$.

    \begin{python}
def forward_difference(f):
  # Your code here

def validator():
  def f(x):
    return x**2
  def g(x):
    return x**3
  return (forward_difference(f)(3) == 7) and (forward_difference(g)(4) == 61)
    \end{python}
  \end{solution}
\end{question}


\end{document}




	
% # Write a function approximateDerivative which takes a function f:R -->R  and a number h and returns another function approximateDerivative(f,h)(x) = [f(x+h)-f(x-h)]/(2h)

% def approximateDerivative(f,h):
% 	#Your code here
	
% #Try letting f(x) = x^2, and g = approximateDerivative(f,0.001).  Then try evaluating g(0),g(1),g(2),g(3),g(4).  Does this agree with what you know about differentiation?

% #Python supports lists

% 	myList = [3,6,9,12]

% #Try print myList[2] to see how lists work

% #Model the function F(x,y,z)=(x+y,z^2x) as a function which takes the list [x,y,z] and returns the list [x+y,z*z*x]

% def F(inputList):
% 	#your code here
	
% #Model dot:R^2 x R^2 --> R as a function which takes two lists, X = [x_1,x_2] and Y=[y_1,y_2], and returns x_1y_1+ x_2y_2

% def dot(X,Y):
% 	#your code here
	
\end{document}


%%% Local Variables: 
%%% mode: latex
%%% TeX-master: t
%%% End: 
