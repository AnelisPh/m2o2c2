\documentclass{ximera}

\title{Currying}

\begin{document}

\begin{abstract}
  Higher-order functions provide a different perspective on functions that take many inputs.
\end{abstract}
\maketitle
\begin{definition}
  Let $A$ and $B$ be two sets.  The \textit{product} $A\times B$ of
  the two sets is the set of all ordered pairs $A \times B = \{ (a,b):
  a \in A \text{ and } b \in B\}$.
\end{definition}

\begin{example}
  If $A = \{ 1,2,\text{Wolf}\}$ and $B = \{4,5\}$, then $A \times B =
  \{(1,4),(1,5),(2,4),(2,5),(\text{Wolf},4),(\text{Wolf},5)\}$
\end{example}

\begin{example}
  We write $\R^2$ for pairs of real numbers, but we could have written
  $\R \times \R$ instead.
\end{example}

\begin{question}
  Let $\text{Func}(\R,\R)$ be the set of all functions from $R$ to
  $R$.  Define $\text{Eval}: \R \times \text{Func}(\R,\R) \to \R$ by
  $\text{Eval}(x,f) = f(x)$.
\begin{solution}
\begin{hint}
	$\text{Eval}(-3,g) = g(-3)=|-3|=3$
\end{hint}
If $g(x) = |x|$, then $\text{Eval}(-3,g)=$ \answer{3}?
\end{solution}
\end{question}

\begin{question}
  Let $\text{Func}(A,B)$ be the set of all functions from $A$ to $B$ for any two sets $A$ and $B$.  
  
  Let $\text{Curry}:\text{Func}(\R^2,\R) \to  \text{Func}(\R,\text{Func}(\R,\R))$ be defined by $\text{Curry}(f)(x)(y) = f(x,y)$.

  Let $h:\R^2 \to \R$ be defined by $h(x,y) = x^2 +xy$.  
  \begin{solution}
    \begin{hint}
      \begin{align*}
        G(3) &= \text{Curry}(h)(2)(3)\\
        &= h(2,3)\\
        &= 2^2+2(3)\\
        &=10
      \end{align*}
    \end{hint}
    Let $G = \text{Curry}(h)(2)$.  Then $G(3) =$ \answer{10}
  \end{solution}
  
  This wacky way of thinking is helpful when thinking about the
  \href{http://en.wikipedia.org/wiki/Lambda_calculus}{$\lambda$-calculus}.
  It also helps a lot if you ever want to learn to program in
  Haskell---which is one of the languages that Ximera was written in.
  
\end{question}

\end{document}


%%% Local Variables: 
%%% mode: latex
%%% TeX-master: t
%%% End: 
