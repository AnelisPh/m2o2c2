\documentclass{ximera}

\title{Linear maps}

\begin{document}

\begin{abstract}
  Linear maps respect addition and scalar multiplication.
\end{abstract}

We begin by defining linear maps.

\begin{definition}
  A function $L: \R^n \to \R^m$ is called a \textit{linear map} if it
  ``respects addition and scalar multiplication.''

  Symbolically, for a map to be linear, we must have that
  $L(\vec{v}+\vec{w}) = L(\vec{v})+L(\vec{w})$ for all
  $\vec{v},\vec{w} \in \R^n$ and also $L(a\vec{v}) = a L(\vec{v})$ for
  all $a \in \R$ and $\vec{v}\in \R^n$.
\end{definition}


\begin{definition}
	\textit{Linear Algebra} is the branch of mathematics concerning vector spaces and linear mappings between such spaces.
\end{definition}

\begin{question}
  Which of the following functions are linear?
  \begin{solution}
    \begin{hint}
    	For a function to be linear, it must respect scalar multiplication.  Lets see what how $f\left(5 \verticalvector{1\\1}\right)$ compares to $f\left(\verticalvector{1\\1}\right)$, and also how
        $h\left(5 \verticalvector{1\\1}\right)$ compares to $h\left(\verticalvector{1\\1}\right)$.  
	
        \begin{question}
        	\begin{solution}
		\begin{hint}
			Remember $f$ is defined by $f\left( \verticalvector{x\\y}\right) =x+2y$, so
			$f\left(5 \verticalvector{1\\1}\right) = f\left(\verticalvector{5\\5}\right) = 5+2(5) = 15$
		\end{hint}
        	 What is $f\left(5 \verticalvector{1\\1}\right)$?
        	 \answer{15}
        	\end{solution}
        	\begin{solution}
		\begin{hint}
			Remember $f$ is defined by $f\left( \verticalvector{x\\y}\right) =x+2y$, so
			$f\left(\verticalvector{1\\1}\right) = 1+2\left(1\right) = 3$
		\end{hint}
        	 What is $f\left(\verticalvector{1\\1}\right)$?
        	 \answer{3}
        	\end{solution}
        	\begin{solution}
        		Is $f\left(5 \verticalvector{1\\1}\right) = 5 f\left(\verticalvector{1\\1}\right)$?
        		\begin{multiple-choice}
        		\choice[correct]{Yes}
        		\choice{No}
        		\end{multiple-choice}
        	\end{solution}
        	Great!  So $f$ has a chance of being linear, since it is respecting scalar multiplication in this case.
        	What about $h$?
        	\begin{solution}
		\begin{hint}
			Remember $h$ is defined by $h\left( \verticalvector{x\\y}\right) =\verticalvector{17\\x}$, so
			$h\left(5 \verticalvector{1\\1}\right) = h\left(\verticalvector{5\\5}\right) = \verticalvector{17\\5}$
		\end{hint}
        	 What is $h\left(5 \verticalvector{1\\1}\right)$?
        	\begin{matrix-answer}[name=v]
    			  correctMatrix = [['17'],['5']]
        	 \end{matrix-answer}
        	\end{solution}
        	\begin{solution}
		\begin{hint}
			Remember $h$ is defined by $h\left( \verticalvector{x\\y}\right) =\verticalvector{17\\x}$, so
			$h\left(\verticalvector{1\\1}\right) =  \verticalvector{17\\1}$
		\end{hint}
        	 What is $h\left(\verticalvector{1\\1}\right)$?
        	 \begin{matrix-answer}[name=v]
    			  correctMatrix = [['17'],['1']]
        	 \end{matrix-answer}
        	\end{solution}
        	\begin{solution}
        		Is $h\left(5 \verticalvector{1\\1}\right) = 5 h\left(\verticalvector{1\\1}\right)$?
        		\begin{multiple-choice}
        		\choice{Yes}
        		\choice[correct]{No}
        		\end{multiple-choice}
        	\end{solution}
        	Great!  So $h$ is not linear:  by looking at this particular example, we can see that $h$ does not always respect scalar multiplication.  So $h$ is not linear.
		
        	Since we know one of the two functions is linear, we can already answer the question:  The answer is $f$.  To be thorough, lets check that $f$ really is linear.
		
        	First we check that $f$ really does respect scalar multiplication:
		
        	Let $a \in \R$ be an arbitrary scalar and $\verticalvector{x\\y} \in \R^2$ be an arbitrary vector.  Then
		
        	\begin{align*}
        	 f\left(a\verticalvector{x\\y}\right) &= f\left(\verticalvector{ax\\ay}\right)\\
        	 &= ax+2ay\\
        	 &= a\left(x+2y\right)\\
        	 &=af\left(\verticalvector{x\\y}\right) 		
        	 \end{align*}
		 
        	 Now we check that $f$ really does respect vector addition:
		 
        	 Let $\verticalvector{x_1\\y_1}$ and $\verticalvector{x_2\\y_2}$ be arbitrary vectors in $\R^2$.  Thn
		 
        	 \begin{align*}
        	 f\left(\verticalvector{x_1\\y_1}+\verticalvector{x_2\\y_2}\right)&= f\left(\verticalvector{x_1+x_2\\y_1+y_2}\right)\\
        	 &= \left(x_1+x_2\right)+2\left(y_1+y_2\right)\\
        	 &= x_1+x_2+2y_1+2y_2\\
        	 &=\left(x_1+2y_1\right)+\left(x_2+2y_2\right)\\
        	 &=f\left(\verticalvector{x_1\\y_1}\right)+f\left(\verticalvector{x_2,y_2}\right)
        	 \end{align*}
		 
        	 This proves that $f$ is linear!
		
        \end{question}
	
    \end{hint}

    \begin{multiple-choice}
      \choice[correct]{\(f: \R^2 \to \R^1\) defined by \(f\left(\verticalvector{x \\ y}\right) = x+2y\)}

      \choice{$h:\R^2 \to \R^2$ defined by $h\left( \verticalvector{x \\y} \right) = \verticalvector{17 \\ x}$}
    \end{multiple-choice}
  \end{solution}

  What about these two functions?  Which of them is a linear map?
  \begin{solution}
 
    \begin{hint}
    	For a function to be linear, it must respect scalar addition.  Lets see what how $h(5+2)$ compares to 
    	$h(5)+h(2)$ and also how $g\left( \verticalvector{2\\3\\1}+\verticalvector{1\\4\\5}\right)$ compares to 
    	$g\left(\verticalvector{2\\3\\1}\right)+g\left( \verticalvector{1\\4\\5}\right)$.
	
        \begin{question}
        	
       
        	
        	\begin{solution}
		\begin{hint}
			Remember $h$ is defined by $h(x) = \verticalvector{x\\x\\x\\4x}$, so
			$h(5+2) = h(7)= \verticalvector{7\\7\\7\\28}$
		\end{hint}
        	 What is $h(5+2)$?
        	\begin{matrix-answer}[name=v]
    			  correctMatrix = [['7'],['7'],['7'],['28']]
        	 \end{matrix-answer}
        	\end{solution}
        	\begin{solution}
		\begin{hint}
			Remember $h$ is defined by $h(x) =\verticalvector{x\\x\\x\\4x}$, so
			$h(5)+h(2) =  \verticalvector{5\\5\\5\\20}+\verticalvector{2\\2\\2\\8} = \verticalvector{7\\7\\7\\28}$
		\end{hint}
        	 What is $h(5) + h(2)$?
        	 \begin{matrix-answer}[name=v]
    			  correctMatrix = [['7'],['7'],['7'],['28']]
        	 \end{matrix-answer}
        	\end{solution}
        	\begin{solution}
        		Is $h(5+2) = h(5)+h(2)$?
        		\begin{multiple-choice}
        		\choice[correct]{Yes}
        		\choice{No}
        		\end{multiple-choice}
        	\end{solution}
        	
        	Great!  So $h$ has a chance of being linear, since it is respecting vector addition in this case.
        	What about $g$?        	
        	
        	\begin{solution}
		\begin{hint}
			Remember $g$ is defined by $g\left( \verticalvector{x\\y\\z}\right) =\verticalvector{x\\xy}$, so
			$g\left( \verticalvector{2\\3\\1}+\verticalvector{1\\4\\5}\right) = g\left( \verticalvector{3\\7\\6}\right) = \verticalvector{3\\3(7)} =\verticalvector{3\\21}$
		\end{hint}
        	 What is $g\left( \verticalvector{2\\3\\1}+\verticalvector{1\\4\\5}\right)$?
        	 \begin{matrix-answer}[name=v]
    			  correctMatrix = [['3'],['21']]
        	 \end{matrix-answer}
        	\end{solution}
        	\begin{solution}
		\begin{hint}
			Remember $g$ is defined by $g\left( \verticalvector{x\\y\\z}\right) =\verticalvector{x\\xy}$, so
			\begin{align*}
			g\left( \verticalvector{2\\3\\1}\right)+g\left(\verticalvector{1\\4\\5}\right) &= \verticalvector{2\\2(3)}+\verticalvector{1\\1(4)}\\
			&=\verticalvector{2\\6}+\verticalvector{1\\4}\\
			&=\verticalvector{3\\10}
			\end{align*}
		\end{hint}
        	 What is $g\left(\verticalvector{2\\3\\1}\right)+g\left( \verticalvector{1\\4\\5}\right)$?
        	 \begin{matrix-answer}[name=v]
    			  correctMatrix = [['3'],['10']]
        	 \end{matrix-answer}
        	\end{solution}
        	\begin{solution}
        		Is $g\left( \verticalvector{2\\3\\1}+\verticalvector{1\\4\\5}\right) = g\left(\verticalvector{2\\3\\1}\right)+g\left( \verticalvector{1\\4\\5}\right)$
        		\begin{multiple-choice}
        		\choice{Yes}
        		\choice[correct]{No}
        		\end{multiple-choice}
        	\end{solution}
        	
        	Great!  So $g$ is not linear:  by looking at this particular example, we can see that $g$ does not always respect vector addition.  So $g$ is not linear.
		
        	Since we know one of the two functions is linear, we can already answer the question:  The answer is $h$.  To be thorough, lets check that $h$ really is linear.
		
        	First we check that $h$ really does respect scalar multiplication:
		
        	Let $a \in \R$ be an arbitrary scalar and $x \in \R$ be an arbitrary vector.  Then
		
        	\begin{align*}
        	 h\left(ax\right) &= \verticalvector{ax\\ax\\ax\\4ax}\\
        	 &= a\verticalvector{x\\x\\x\\4x}\\
        	 &= ah(x)
        	 \end{align*}
		 
        	 Now we check that $h$ really does respect vector addition:
		 
        	 Let $x$ and $y$ be arbitrary vectors in $\R^1$.  Then
		 
        	 \begin{align*}
        	 h\left(x+y\right)&= \verticalvector{x+y\\x+y\\x+y\\4(x+y)}\\
        	 &= \verticalvector{x+y\\x+y\\x+y\\4x+4y}\\
        	 &= \verticalvector{x\\x\\x\\4x}+\verticalvector{y\\y\\y\\4y}\\
        	 &=h(x)+h(y)
        	 \end{align*}
		 
        	 This proves that $h$ is linear!
		
        \end{question}
	
    \end{hint}
    \begin{multiple-choice}
    \choice{$g: \R^3 \to R^2$ defined by $g\left(\verticalvector{x\\y\\z}\right) = \verticalvector{x\\xy}$}
    \choice[correct]{$h:\R \to \R^4$ defined by $h(x) = \verticalvector{x\\x\\x\\4x}$}
    \end{multiple-choice}
  \end{solution}

  And finally, which of the following functions are linear?
  
  \begin{solution}
  
    \begin{hint}
    	For a function to be linear, it must respect scalar multiplication.  Lets see what how $A\left( 2\verticalvector{2\\3}\right)$ compares to 
    	$2A\left(\verticalvector{2\\3}\right)$ and also how $G\left(2\verticalvector{1\\2\\3\\4}\right)$ compares to 
    	$2G\left(\verticalvector{1\\2\\3\\4}\right)$.
	
        \begin{question}
        	
        	\begin{solution}
		\begin{hint}
			Remember $A$ is defined by $A\left(\verticalvector{x\\y}\right) = \verticalvector{0\\0}$, so
			$A\left(2\verticalvector{2\\3}\right) = A\left(\verticalvector{4\\6}\right)=\verticalvector{0\\0}$
		\end{hint}
        	 What is $A\left(2\verticalvector{2\\3}\right)$?
        	\begin{matrix-answer}[name=v]
    			  correctMatrix = [['0'],['0']]
        	 \end{matrix-answer}
        	\end{solution}
        	\begin{solution}
		\begin{hint}
			Remember $A$ is defined by $A\left(\verticalvector{x\\y}\right) = \verticalvector{0\\0}$, so
			$2A\left(\verticalvector{2\\3}\right) = 2\verticalvector{0\\0} = \verticalvector{0\\0}$
		\end{hint}
        	 What is $2A\left(\verticalvector{2\\3}\right)$?
        	 \begin{matrix-answer}[name=v]
    			  correctMatrix = [['0'],['0']]
        	 \end{matrix-answer}
        	\end{solution}
        	\begin{solution}
        		Is $A\left(2\verticalvector{2\\3}\right) = 2A\left(\verticalvector{2\\3}\right))$?
        		\begin{multiple-choice}
        		\choice[correct]{Yes}
        		\choice{No}
        		\end{multiple-choice}
        	\end{solution}
        	
        	Great!  So $A$ has a chance of being linear, since it is respecting vector addition in this case.
        	What about $G$?        	
        	
        	\begin{solution}
		\begin{hint}
			Remember $G$ is defined by $G\left( \verticalvector{x\\y\\z\\t}\right) =\verticalvector{e^{x+y}\\x+z\\ \sin(x+t)}$, so
			\begin{align*}
				G\left( 2\verticalvector{1\\2\\3\\4} \right) &= G\left( \verticalvector{2\\4\\6\\8}\right)\\
					&=\verticalvector{e^{2+4}\\2+6\\ \sin(2+8)}\\
					&=\verticalvector{e^6\\8\\ \sin(10)}
			\end{align*}
		\end{hint}
        	 What is $G\left( 2\verticalvector{1\\2\\3\\4}\right)$?
        	 \begin{matrix-answer}[name=v]
    			  correctMatrix = [['e^6'],['8'],['sin(10)']]
        	 \end{matrix-answer}
        	\end{solution}
        	\begin{solution}
		\begin{hint}
				Remember $G$ is defined by $G\left( \verticalvector{x\\y\\z\\t}\right) =\verticalvector{e^{x+y}\\x+z\\ \sin(x+t)}$, so
			\begin{align*}
				2G\left( \verticalvector{1\\2\\3\\4} \right) &= 2\verticalvector{e^{1+2}\\1+3\\ \sin(1+4)}\\
					&=2\verticalvector{e^{3}\\4\\ \sin(5)}\\
					&=\verticalvector{2e^3\\8\\ 2\sin(5)}
			\end{align*}
		\end{hint}
        	 What is $2G\left(\verticalvector{1\\2\\3\\4}\right)$?
        	 \begin{matrix-answer}[name=v]
    			  correctMatrix = [['2e^3'],['8'],['2sin(5)']]
        	 \end{matrix-answer}
        	\end{solution}
        	\begin{solution}
        		Is $G\left( 2\verticalvector{1\\2\\3\\4}\right) =  2G\left(\verticalvector{1\\2\\3\\4}\right)$?
        		\begin{multiple-choice}
        		\choice{Yes}
        		\choice[correct]{No}
        		\end{multiple-choice}
        	\end{solution}
        	
        	Great!  So $G$ is not linear:  by looking at this particular example, we can see that $G$ does not always respect scalar multiplication.  So $G$ is not linear.
		
        	Since we know one of the two functions is linear, we can already answer the question:  The answer is $A$.  To be thorough, lets check that $A$ really is linear.
		
        	First we check that $A$ really does respect scalar multiplication:
		
        	Let $c \in \R$ be an arbitrary scalar and $\verticalvector{x\\y} \in \R^2$ be an arbitrary vector.  Then
		
        	\begin{align*}
        	 A\left(c\verticalvector{x\\y}\right) &= A\left( \verticalvector{ax\\ay}\right)\\
        	 	&=\verticalvector{0\\0}\\
        	 	&=a\verticalvector{0\\0}
        	 \end{align*}
		 
        	 Now we check that $A$ really does respect vector addition:
		 
        	 Let $\verticalvector{x_1\\y_1}$ and $\verticalvector{x_2\\y_2}$ be arbitrary vectors in $\R^2$.  Then
		 
        	 \begin{align*}
        	 A\left( \verticalvector{x_1\\y_1}+\verticalvector{x_2\\y_2}\right) &= A\left( \verticalvector{x_1+x_2\\y_1+y_2}\right)\\
        	 	&=\verticalvector{0\\0}\\
        	 	&=\verticalvector{0\\0}+\verticalvector{0\\0}\\
        	 	&=A\left(\verticalvector{x_1\\y_1}\right)+A\left(\verticalvector{x_2\\y_2}\right)
        	 \end{align*}
		 
        	 This proves that $A$ is linear!
		
        \end{question}
	
    \end{hint}
    \begin{multiple-choice}
      \choice{$G: \R^4 \to  \R^3$ defined by $G\left(\verticalvector{x\\y\\z\\t}\right) = \verticalvector{e^{x+y}\\x+z\\ \sin(x+t)}$}
      \choice[correct]{$A: \R^2 \to R^2$ defined by $A\left(\verticalvector{x\\y}\right)=\verticalvector{0\\0}$}
    \end{multiple-choice}
  \end{solution}

  \begin{warning}
    Note that the function which sends every vector to the zero vector
    \textit{is} linear.
  \end{warning}
\end{question}
	
\begin{question}
  Let $L:\R^3 \to \R^2$ be a linear function.  Suppose
  $L\left(\verticalvector{1\\0\\0}\right) = \verticalvector{3\\4}$,
  $L\left(\verticalvector{0\\1\\0}\right) = \verticalvector{-2\\0}$,
  and $L\left(\verticalvector{0\\0\\1}\right) =
  \verticalvector{1\\-1}$.
  
  \begin{solution}
  
  	\begin{hint}	
  		The only thing we know about linear maps is that they respect scalar multiplication and vector addition.  So we need to somehow rewrite the vector
  		$\verticalvector{4\\-1\\2}$ in terms of the vectors $\verticalvector{1\\0\\0}$, $\verticalvector{0\\1\\0}$ and $\verticalvector{0\\0\\1}$, scalar multiplication,
  		and vector addition, to exploit what we know about $L$.
  		
  		
  		
  		\begin{question}
  			Can you rewrite $\verticalvector{4\\-1\\2}$ in the form $a\verticalvector{1\\0\\0}+b\verticalvector{0\\1\\0}+c\verticalvector{0\\0\\1}$?
  			\begin{solution}
  				\begin{hint}
  					Observe that $\verticalvector{4\\-1\\2} = 4\verticalvector{1\\0\\0}+-1\verticalvector{0\\1\\0}+2\verticalvector{0\\0\\1}$.
  				\end{hint}
  				\begin{hint}
  					Consider the coefficient on $\verticalvector{1\\0\\0}$.
  				\end{hint}
  				\begin{hint}
  					In this case, $a = 4$.
  				\end{hint}
  				\begin{hint}
  					Moreover, $b = -1$.
  				\end{hint}
  				\begin{hint}
  					Finally, $c = 2$.
  				\end{hint}
  				$a =$ \answer{4}
  			\end{solution}
  			\begin{solution}
  				$b=$ \answer{-1}
  			\end{solution}
  			\begin{solution}
  				$c=$ \answer{2}
  			\end{solution}
  		
  		Now using the linearity of $L$, we can see that 
  		
  		\begin{align*}
  			L\left(\verticalvector{4\\-1\\2}\right) &= L\left( 4\verticalvector{1\\0\\0}+-1\verticalvector{0\\1\\0}+2\verticalvector{0\\0\\1} \right)\\
  			&= 4L\left(\verticalvector{1\\0\\0}\right)+-1L\left(\verticalvector{0\\1\\0}\right)+2L\left(\verticalvector{0\\0\\1}\right)
  		\end{align*}
  		
  		Can you finish off the computation?
              \end{question}
  	\end{hint}
  	\begin{hint}
  		\begin{align*}
  			 L \left(\verticalvector{4\\-1\\2}\right) &= 4L\left(\verticalvector{1\\0\\0}\right)+-1L\left(\verticalvector{0\\1\\0}\right)+2L\left(\verticalvector{0\\0\\1}\right)\\
  			&= 4\verticalvector{3\\4}+-1\verticalvector{-2\\0}+2\verticalvector{1\\-1}\\
  			&= \verticalvector{12\\16}+\verticalvector{2\\0}+\verticalvector{2\\-2}\\
  			&= \verticalvector{16\\14}
  		\end{align*}
  	\end{hint}

    Let $\vec{v} = L \left(\verticalvector{4\\-1\\2}\right)$.  What is $\vec{v}$?

    \begin{matrix-answer}[name=v]
      correctMatrix = [['16'],['14']]
    \end{matrix-answer}
    
  \end{solution}

  Can you generalize this?

  \begin{solution}
  \begin{hint}	
  		The only thing we know about linear maps is that they respect scalar multiplication and vector addition.  So we need to somehow rewrite the vector
  		$\verticalvector{x\\y\\z}$ in terms of the vectors $\verticalvector{1\\0\\0}$, $\verticalvector{0\\1\\0}$ and $\verticalvector{0\\0\\1}$, scalar multiplication,
  		and vector addition, to exploit what we know about $L$.
  		
  		
  		
  		\begin{question}
  			Can you rewrite $\verticalvector{x\\y\\z}$ in the form $a\verticalvector{1\\0\\0}+b\verticalvector{0\\1\\0}+c\verticalvector{0\\0\\1}$?
  			\begin{solution}
  				\begin{hint}
  					$\verticalvector{x\\y\\z} = x\verticalvector{1\\0\\0}+y\verticalvector{0\\1\\0}+z\verticalvector{0\\0\\1}$
  				\end{hint}
  				$a =$ \answer{x}
  			\end{solution}
  			\begin{solution}
  				$b=$ \answer{y}
  			\end{solution}
  			\begin{solution}
  				$c=$ \answer{z}
  			\end{solution}
  		\end{question}
  	\end{hint}
  	\begin{hint}
  		
  		Now using the linearity of $L$, we can see that 
  		
  		\begin{align*}
  			L\left(\verticalvector{x\\y\\z}\right) &= L\left( x\verticalvector{1\\0\\0}+y\verticalvector{0\\1\\0}+z\verticalvector{0\\0\\1} \right)\\
  			&= xL\left(\verticalvector{1\\0\\0}\right)+yL\left(\verticalvector{0\\1\\0}\right)+zL\left(\verticalvector{0\\0\\1}\right)
  		\end{align*}
  		
  		Can you finish off the computation?
  	\end{hint}
  	\begin{hint}
  		\begin{align*}
  			L\left(\verticalvector{4\\-1\\2}\right) &= xL\left(\verticalvector{1\\0\\0}\right)+yL\left(\verticalvector{0\\1\\0}\right)+zL\left(\verticalvector{0\\0\\1}\right)\\
  			&= x\verticalvector{3\\4}+y\verticalvector{-2\\0}+z\verticalvector{1\\-1}\\
  			&= \verticalvector{3x\\4x}+\verticalvector{-2y\\0}+\verticalvector{z\\-z}\\
  			&= \verticalvector{3x-2y+z\\4x-z}
  		\end{align*}
  	\end{hint}
    Let $\vec{v} = L\left(\verticalvector{x\\y\\z}\right)$?  What is $\vec{v}$?

    \begin{matrix-answer}[name=v]
      correctMatrix = [['3*x - 2*y + z'],['4*x - z']]
    \end{matrix-answer}
    
  \end{solution}

  As you have already discovered a linear map $L: \R^n \to \R^m$ is
  fully determined by its action on the ``standard basis vectors''
  $e_1 = \verticalvector{1\\0\\0\\\vdots\\0}$, $e_2 =
  \verticalvector{0\\1\\0\\\vdots\\0}$, and so on, until we reach $e_n
  = \verticalvector{0\\0\\\vdots\\0\\1}$.

  Argue convincingly that if $L:\R^n \to\R^m$ is a linear map and you know $L(\vec{e}_i)$ for $i=1,2,3,...,n$, then you could figure out $L(\vec{v})$ for
  any $\vec{v} \in \R^n$.
  \begin{free-response}
    I want to determine what $L$ does to any vector $\vec{v} = \verticalvector{x_1\\x_2\\x_3\\.\\.\\.\\x_n} \in \R^n$.  
    I can rewrite $\vec{v}$ as $x_1\vec{e_1}+x_2\vec{e_2}+x_3\vec{e_3}+...+\x_n\vec{e_n}$.  By the linearity of $L$,
    $L(\vec{v}) = x_1L(\vec{e_1})+x_2L(\vec{e_2})+x_3L(\vec{e_3})+...+x_nL(\vec{e_n})$.  Since I already know the value of 
    $L(\vec{e_i})$ for all $i=1,2,3,...,n$, this allows me to compute $L(\vec{v})$.  So $L$ is completely determined once I know what it does to each of the
    standard basis vectors. 
  \end{free-response}
\end{question}

\youtube{http://www.youtube.com/watch?v=8BFsz1FCdxM}

\end{document}
