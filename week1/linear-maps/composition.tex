\documentclass{ximera}

\title{Composition}

\begin{document}

\begin{abstract}
  The composition of linear maps can be computed with matrices.
\end{abstract}

Prove that if $S:\R^n \to \R^m$ is a linear map, and $T:\R^m \to \R^k$ is a linear map, then the composite function $T\circ S:\R^n \to R^k$ is also linear.
  
\begin{free-response}
	We need to show that $T \circ S$ respects scalar multiplication and vector addition:
	
	Scalar multiplication:
	For every scalar $a \in \R$ and every vector $\vec{v} \in \R^n$, we have:
	
	\begin{align*}	
		(T\circ S)(a\vec{v}) &= T\left(S(a\vec{v})\right)\\
		&= T(aS(\vec{v})) \text{ because $S$ respects scalar multiplication}\\
		&= aT(S(\vec{v})) \text{ because $T$ respects scalar multiplication}\\
		&=a(T \circ S)(\vec{v})
	\end{align*}
	
	Vector addition:
	For every two vectors $\vec{v},\vec{w} \in \R^n$, we have:
		
		\begin{align*}
			(T \circ S)(\vec{v}+\vec{w}) &= T\left(S(\vec{v}+\vec{w})\right) \\
			&= T(S(\vec{v}+S(\vec{w}))) \text{because $S$ respects vector addition}\\
			&= T(S(\vec{v}))+T(S(\vec{w})) \text{because $T$ respects vector addition}\\
			&= (T \circ S)(\vec{v})+(T\circ S)(\vec{w})
		\end{align*}
\end{free-response}
	
\begin{question}
  Suppose the matrix of $S$ is $M_S = \begin{bmatrix}
2 & 0 & -1 \\
-1 & 1 & 1
\end{bmatrix}$ and the matrix of $T$ is $M_T = \begin{bmatrix}
-1 & -1 \\
0 & 2 \\
-1 & 1
\end{bmatrix}$.

\begin{solution}
\begin{hint}
	Remember that the matrix for $S \circ T$ will have columns given by $(S \circ T)\left(\verticalvector{1\\0}\right)$ and $(S \circ T)\left(\verticalvector{0\\1}\right)$
\end{hint}
\begin{hint}
	\begin{question}
		\begin{solution}
		\begin{hint}
			\begin{align*}
				(S \circ T)\left(\verticalvector{1\\0}\right) &= S\left(T\left(\verticalvector{1\\0}\right)\right)\\
				&= S\left(\verticalvector{-1\\0\\-1}\right) \text{ because by definition, $T\left(\verticalvector{1\\0}\right)$ is the first column of the matrix of $T$}\\
				&=-1S\left(\verticalvector{1\\0\\0}\right) +-1 S\left(\verticalvector{0\\0\\1}\right) \text{by the linearity of $S$}\\
				&=-1\verticalvector{2\\-1}+-1\verticalvector{-1\\1} \text{ because ???}\\
				&=\verticalvector{-1\\0}
			\end{align*}
		\end{hint}
		What is  $(S \circ T)\left(\verticalvector{1\\0}\right)$?
			\begin{matrix-answer}[name=M]
 			   correctMatrix = [['-1'],['0']]
 			 \end{matrix-answer}
		\end{solution}
		
	\end{question}
	\begin{question}
		\begin{solution}
		\begin{hint}
			\begin{align*}
				(S \circ T)\left(\verticalvector{0\\1}\right) &= S\left(T\left(\verticalvector{0\\1}\right)\right)\\
				&= S\left(\verticalvector{-1\\2\\1}\right) \text{ because by definition, $T\left(\verticalvector{0\\1}\right)$ is the second column of the matrix of $T$}\\
				&=-1S\left(\verticalvector{1\\0\\0}\right) +2 S\left(\verticalvector{0\\1\\0}\right) + S\left(\verticalvector{0\\0\\1}\right)\text{by the linearity of $S$}\\
				&=-1\verticalvector{2\\-1}+2\verticalvector{0\\1} +\verticalvector{-1\\1}\text{ because ???}\\
				&=\verticalvector{-3\\4}
			\end{align*}
		\end{hint}
		What is  $(S \circ T)\left(\verticalvector{0\\1}\right)$?
			\begin{matrix-answer}[name=M]
 			   correctMatrix = [['-3'],['4']]
 			 \end{matrix-answer}
		\end{solution}
		
	\end{question}
\end{hint}
\begin{hint}
	The matrix of $(S \circ T)$ is $\begin{bmatrix} -1 & -3\\0 & 4 \end{bmatrix}$
\end{hint}
  What is the matrix of $S \circ T$?

  \begin{matrix-answer}[name=M]
    correctMatrix = [['-1','-3'],['0','4']]
  \end{matrix-answer}
\end{solution}

\begin{solution}
\begin{hint}
	Remember that the matrix for $T \circ S$ will have columns given by $(T \circ S)\left(\verticalvector{1\\0\\0}\right)$,  $(S \circ T)\left(\verticalvector{0\\1\\0}\right)$ and 
	$(S \circ T)\left(\verticalvector{0\\0\\1}\right)$
\end{hint}
\begin{hint}
	\begin{question}
		\begin{solution}
		\begin{hint}
			\begin{align*}
				(T \circ S)\left(\verticalvector{1\\0\\0}\right) &= T\left(S\left(\verticalvector{1\\0\\0}\right)\right)\\
				&= T\left(\verticalvector{2\\-1}\right) \text{ because by definition, $S\left(\verticalvector{1\\0\\0}\right)$ is the first column of the matrix of $S$}\\
				&=2T\left(\verticalvector{1\\0}\right) +-1 T\left(\verticalvector{0\\1}\right) \text{by the linearity of $T$}\\
				&=2\verticalvector{-1\\0\\-1}+-1\verticalvector{-1\\2\\1} \text{ because ???}\\
				&=\verticalvector{-1\\-2\\-3}
			\end{align*}
		\end{hint}
		What is  $(S \circ T)\left(\verticalvector{1\\0\\0}\right)$?
			\begin{matrix-answer}[name=M]
 			   correctMatrix = [['-1'],['-2'],['-3']]
 			 \end{matrix-answer}
		\end{solution}
		
	\end{question}
	\begin{question}
		\begin{solution}
		\begin{hint}
			\begin{align*}
				(T \circ S)\left(\verticalvector{0\\1\\0}\right) &= T\left(S\left(\verticalvector{0\\1\\0}\right)\right)\\
				&= T\left(\verticalvector{0\\1}\right) \text{ because by definition, $S\left(\verticalvector{1\\0\\0}\right)$ is the first column of the matrix of $S$}\\
				&=\verticalvector{-1\\2\\1} \text{ we got lucky: by definition $T\left(\verticalvector{0\\1}\right)$ is the second column of the matrix of $T$ }\\
			\end{align*}
		\end{hint}
		What is  $(S \circ T)\left(\verticalvector{0\\1\\0}\right)$?
			\begin{matrix-answer}[name=M]
 			   correctMatrix = [['-1'],['2'],['1']]
 			 \end{matrix-answer}
		\end{solution}
		
	\end{question}
	\begin{question}
		\begin{solution}
		\begin{hint}
			\begin{align*}
				(T \circ S)\left(\verticalvector{0\\0\\1}\right) &= T\left(S\left(\verticalvector{0\\0\\1}\right)\right)\\
				&= T\left(\verticalvector{-1\\1}\right) \text{ because by definition, $S\left(\verticalvector{0\\0\\1}\right)$ is the third column of the matrix of $S$}\\
				&=-1T\left(\verticalvector{1\\0}\right) + T\left(\verticalvector{0\\1}\right) \text{by the linearity of $T$}\\
				&=-1\verticalvector{-1\\0\\-1}+\verticalvector{-1\\2\\1} \text{ because ???}\\
				&=\verticalvector{0\\2\\2}
			\end{align*}
		\end{hint}
		What is  $(S \circ T)\left(\verticalvector{0\\0\\1}\right)$?
			\begin{matrix-answer}[name=M]
 			   correctMatrix = [['0'],['2'],['2']]
 			 \end{matrix-answer}
		\end{solution}
		
	\end{question}
	
	
\end{hint}
\begin{hint}
	The matrix of $(T \circ S)$ is $\begin{bmatrix} -1 & -1 & 0\\-2 & 2&2\\-3&1&2 \end{bmatrix}$
\end{hint}
  What is the matrix of $T \circ S$?

  \begin{matrix-answer}[name=M]
    correctMatrix = [['-1', '-1', '0'],['-2','2','2'],['-3','1','2']]
  \end{matrix-answer}
\end{solution}

\end{question}
	
\begin{definition}
  If $M$ is a $m\times n$ matrix and $N$ is a $k \times m$ matrix,
  then the \textit{product} $NM$ of the matrices is defined as the
  matrix of the composition of the linear maps defined by $M$ and $N$.

  In other words, $NM$ is the matrix of $L_N \circ L_M$.
\end{definition}

\begin{warning}
  You may have seen another definition for matrix multiplication in
  the past.  That definition could be seen as a shortcut for how to
  compute the product, but it is usually presented devoid of
  mathematical meaning.

  Hopefully our definition seems properly motivated: \textit{matrix
    multiplication is just what you do to compose linear maps.} We
  suggest working out the problems here using our definition: you will
  develop your own efficient shortcuts in time.
\end{warning}

You have already multiplied two matrices, even though you didn't know it, above.  Take some time now to get a whole lot of practice.  You do not need
us to prompt you:  invent your own matrices and try to multiply them, on paper.  What condition is needed on the rows and columns of the two matrices for matrix
multiplication to even make sense? You can check your work using a computer algebra system, like 
\href{http://www.sagemath.org/}{SAGE} or you can use a free web hosted app like \href{http://matrix.reshish.com/}{Reshih}.  Use our definition, and think through it 
each time.  Try to get faster and more efficient.  Eventually you should be able to do this quite rapidly.

\begin{question}
  Suppose $B = \begin{bmatrix} 1 & 2 \\ 3 & 4 \end{bmatrix}$.  Find a $2 \times 2$ matrix $A$ so that $AB \neq BA$.  Play around!  Can you find more than one?

  \begin{solution}
  	\begin{hint}
  		There is no systematic way to answer this question:  you just have to play around, and see what you discover!
  	\end{hint}
  	\begin{hint}
  		\begin{question}
  			\begin{solution}
  				\begin{hint}
  					$\begin{bmatrix} 1 & 2 \\ 3 & 4 \end{bmatrix} \begin{bmatrix}  1 &0 \\ 0&0\end{bmatrix}  = \begin{bmatrix} 1 & 0\\ 3&0 \end{bmatrix}$
  				\end{hint}
  				What is $\begin{bmatrix} 1 & 2 \\ 3 & 4 \end{bmatrix} \begin{bmatrix}  1 &0 \\ 0&0\end{bmatrix}$?
  					\begin{matrix-answer}[name=A]
  						correctMatrix = [['1','0'],['3','0']]
  					\end{matrix-answer}
  			\end{solution}
  		\end{question}
  		\begin{question}
  			\begin{solution}
  				\begin{hint}
  					$\begin{bmatrix}  1 &0 \\ 0&0\end{bmatrix}  \begin{bmatrix} 1 & 2 \\ 3 & 4 \end{bmatrix}  = \begin{bmatrix} 1 & 2\\ 0&0 \end{bmatrix}$
  				\end{hint}
  				What is $\begin{bmatrix}  1 &0 \\ 0&0\end{bmatrix} \begin{bmatrix} 1 & 2 \\ 3 & 4 \end{bmatrix}$?
  					\begin{matrix-answer}[name=A]
  						correctMatrix = [['1','2'],['0','0']]
  					\end{matrix-answer}
  			\end{solution}
  		\end{question}
  	\end{hint}
  	
    \begin{matrix-answer}[name=A]
    function validator(m) {
      if (isWrongSize(m, 2, 2)) return false;
      
      var b = [['1','2'],['3','4']]
      var ab = matrixProduct(m,b);
      var ba = matrixProduct(b,m);
      if (isMatrixCorrect(ab,ba))
        return false;

      return true;
    }
    \end{matrix-answer}  
  \end{solution}
\end{question}
	
\begin{question}
  Find $A \neq 0$ with $AA = 0$.
	\begin{hint}
		Try some simple matrices.  Maybe limit yourself to $2\times2$ matrices?
	\end{hint}
	\begin{hint}
		One simple linear map which would work is $L\left(\verticalvector{x\\y}\right) = \verticalvector{y\\0}$.  Applying this twice to any vector would 
		give you the zero vector.   This linear map is great for cooking up counterexamples to all sorts of naive things you might think about matrices!  See 
		this \href{http://mathoverflow.net/questions/16829/what-are-your-favorite-instructional-counterexamples/16841#16841}{Mathoverflow  answer} (you will understand
		more and more of these terms as the course progresses).
		
		\begin{question}
			\begin{hint}
				The matrix of $L$ is $\begin{bmatrix} 0&1\\0&0 \end{bmatrix}$
			\end{hint}
			What is the matrix of this linear map?
			\begin{matrix-answer}[name=A]
				correctMatrix = [['0','1'],['0','0']]
			\end{matrix-answer}
		\end{question}
	\end{hint}
	
  \begin{solution}
    \begin{matrix-answer}[name=A]
    function validator(m) {
      if (rows(m) != columns(m)) {
        feedback( 'You should try using a square matrix.' );
        return false;
      }
      
      var m2 = matrixProduct(m,m);
      
      var zeroMatrix = [];
      var i;
      for( i=0; i<rows(m); i++ ) {
        zeroMatrix[i] = [];
        var j;
        for( j=0; j<columns(m); j++ ) {
          zeroMatrix[i][j] = '0';
        }
      }

      if (isMatrixCorrect(m2,zeroMatrix))
        return false;

      return true;
    }
    \end{matrix-answer}
  \end{solution}
\end{question}
	
\begin{question}
  If $A = \begin{bmatrix} 2 & 8\\3&12\end{bmatrix}$, find $v \neq 0$ with $Av = \vec{0}$.
  \begin{solution}
  \begin{hint}
  	Let $\vec{v} = \verticalvector{x\\y}$, and solve a system of equations
  \end{hint}
  \begin{hint}
  	\begin{align*}
  		A(\vec{v}) &= \vec{0}\\
  		\begin{bmatrix} 2 & 8\\3&12\end{bmatrix}\verticalvector{x\\y}  &=\verticalvector{0\\0}\\
  		\verticalvector{2x+8y\\3x+12y} = \verticalvector{0\\0}
  	\end{align*}
  \end{hint}
  \begin{hint}
  	Both of these conditions ($2x+8y  =0$ and $3x+12y = 0$) are saying the same thing:  $x = -4y$.
  \end{hint}
  \begin{hint}
  	So $\verticalvector{-4\\1}$ works, for example.
  \end{hint}
  	\begin{matrix-answer}[name=v]
  		function validator(m){
  		if ((rows(m) != 2 )||(columns(m) != 1)){
  			feedback('This needs to be a column vector of length two');
  			return false;
  			}
  		var zeroMatrix = [['0'],['0']]
  		
  		if (isMatrixCorrect(m,zeroMatrix)){
  			feedback('You are not allowed to use the zero vector');
  			return false;
  			}
  		
  		var A = [['2',8],['3','12']];
  		
  		if !(isMatrixCorrect(matrixProduct(A,m),zeroMatrix)){
  			return false;
  			}
  		
  		return true;
  		}
  	\end{matrix-answer}
  \end{solution}
\end{question}
	
\begin{question}
  If $A = \begin{bmatrix} 1 &3\\2&4\end{bmatrix} $, find $\vec{v}$ with $A\vec{v} = \verticalvector{0\\8}$.
  \begin{solution}
  \begin{hint}
  	Let $\vec{v} = \verticalvector{x\\y}$ and solve a system of equations.
  \end{hint}
  \begin{hint}
  	\begin{align*}
  		A\vec{v} &= \verticalvector{0\\8}\\
  		\begin{bmatrix} 1 &3\\2&4\end{bmatrix}\verticalvector{x\\y} &= \verticalvector{0\\8}\\
  		\verticalvector{x+3y\\2x+4y} = \verticalvector{0\\8}
  	\end{align*}
  \end{hint}
  \begin{hint}
	\begin{align*}
    \begin{cases}
  			x+3y = 0\\
  			2x+4y=8
  		\end{cases}
  		\\
  		\begin{cases}
  			x+3y = 0\\
  			x+2y=4
  		\end{cases}
  		\\
  		\begin{cases}
  			x+3y = 0\\
  			y=-4
  		\end{cases}
  		\\
  		\begin{cases}
  			x=12\\
  			y=-4
  		\end{cases}
\end{align*}
 \end{hint}
  	\begin{matrix-answer}[name=v]
  		correctMatrix=[['12'],['-4']]
  	\end{matrix-answer}
  \end{solution}
\end{question}
	
In the last two exercises, you found that solving matrix equations is equivalent to solving systems of linear equations.

\begin{question}
  Rewrite  $\begin{cases}4x+7y+z =3\\ -x+8y-z =2\end{cases}$ as $A\verticalvector{x\\y\\z} = \verticalvector{3\\2}$.
  \begin{solution}
  	\begin{hint}
  		$A = \begin{bmatrix} 4 & 7 &1 \\ -1&8&-1\end{bmatrix}$
  	\end{hint}
  	\begin{matrix-answer}[name = A]
  		correctMatrix = [['4','7','1'],['-1','8','-1']]
  	\end{matrix-answer}
  \end{solution}
\end{question}


\end{document}
%%% Local Variables: 
%%% mode: latex
%%% TeX-master: t
%%% End: 
