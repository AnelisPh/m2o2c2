\documentclass{ximera}

\newcommand{\R}{\mathbb R}

\newcommand{\href}[2]{#2\footnote{\url{#1}}}
\newcommand{\verticalvector}[1]{\begin{bmatrix}#1\end{bmatrix}}
\newcommand{\gt}{>}


\title{Python}

\begin{document}

\begin{abstract}
  Build up some linear algebra in python.
\end{abstract}
\maketitle

\begin{exercise}
  We will store a vector as a list.  So the vector $\begin{bmatrix} 1 \\
    2 \\ 3\end{bmatrix}$ will be stored as \verb|[1,2,3]|.  Let's try
  to write some Python code for working with lists as if they were
  vectors.

\begin{solution}
  \begin{hint}
    This was discussed on \url{http://stackoverflow.com/questions/14050824/add-sum-of-values-of-two-lists-into-new-list}{StackOverflow}.
  \end{hint}

  Write a ``vector add'' function.  Your function may assume that the
  two vectors have the same number of entries.

\begin{python}
# write a function vector_sum(v,w) which takes two vectors v and w,
# and returns the sum v + w.
#
# For example, vector_sum([1,2], [4,1]) equals [5,3]
#
		
def vector_sum(v,w):
  # your code here
  return # the sum v+w

def validator():
  # It would be better to try more cases
  if vector_sum([-5,23],[10,2])[0] != 5:
    return False
  if vector_sum([1,5,6],[2,3,6])[1] != 8:
    return False
  return True

\end{python}
\end{solution}


\begin{solution}
  \begin{hint}
    Try a Python ``list comprehension''
  \end{hint}

  \begin{hint}
    For example, \verb|return [alpha * x for x in v]|
  \end{hint}

Next, write a scalar multiplication function.

\begin{python}
# write a function scale_vector(alpha, v) which takes a number alpha and a vector v
# and returns alpha * v
#
# For example, scale_vector(5,[1,2,3]) equals [5,10,15]
		
def scale_vector(alpha, v):
  # your code here
  return # the scaled vector alpha * v

def validator():
  # It would be better to try more cases
  if scale_vector(-3,[2,3,10])[1] != -9:
    return False
  if scale_vector(10,[4,3,2,1])[2] != 20:
    return False
  return True

\end{python}
\end{solution}

Let's write a dot product function.

\begin{solution}
\begin{python}
# Write a function dot_product(v,w) which takes two vectors v and w,
# and returns the dot product of v and w.
#
# For example, dot_product([1,2],[0,3]) is 6.
		
def dot_product(v,w):
  # your code here
  return # the dot product "v dot w"

def validator():
  if dot_product([1,2],[-3,5]) != 7:
    return False
  if dot_product([0,4,2],[2,3,-7]) != -2:
    return False
  return True
\end{python}
\end{solution}

\end{exercise}

And we will store a matrix as a list of lists.  For example the list
\verb|[[1,3,5],[2,4,6]]| will represent the matrix
\[
\begin{bmatrix}
 1 & 3 & 5 \\
 2 & 4 & 6  
\end{bmatrix}.
\]
Note that there are two different conventions that we could have
chosen: the innermost lists could be the rows, or the columns.  There
are good reasons to have chosen the opposite convention: after all,
when thinking of a matrix as a linear map, we should be paying
attention to the columns, since the $i$th column tells us what the
corresponding linear map does when applied to $\vec{e}_i$.

Nevertheless, \textbf{the innermost lists are rows in our chosen
  representation}.  This way, to talk about the entry $m_{ij}$, we
write \verb|m[i][j]|.  Had we made the other choice, the $m_{ij}$
entry would have been accessed by writing $j$ and $i$ in the other
order.  This is also the same convention used by the computer algebra
system, Sage.

\begin{exercise}
Write a ``matrix multiplication'' function.

\begin{solution}
\begin{python}
# write a function multiply(A,B) which takes two matrices A and B stored in the above format,
# and returns the matrix of their product
		
def multiply(A,B):
  # your code here
  return # the product AB

def validator():
  # It would be better to try more cases
  a = [[-2, 0], [-2, -3], [-1, 3]]
  b = [[-3, 2, -1, -2], [3, 2, 1, 3]]
  result = multiply(a,b)
  if (len(result) != 3):
    return False
  if (len(result[0]) != 4):
    return False
  if (result[2][1] != 4):
    return False
  return True
\end{python}
\end{solution}

Fantastic!

Next, let's think more about how matrices and linear maps are related.

\begin{solution}
  \begin{hint}
  \begin{warning}
    This is a function whose output is a function.
  \end{warning}
  \end{hint}

  \begin{hint}
    Try using \verb|lambda|.
  \end{hint}

  Write a function \verb|matrix_to_function| which takes a matrix
  $M_L$ representing the linear map $L$, and returns a Python
  function.  The returned Python function should take a vector
  $\vec{v}$ and send it to $L(\vec{v})$.

\begin{python}
# For example, if M = [[1,2],[3,4]], then matrix_to_function(M)([0,1]) should be [2,4]

def matrix_to_function(M):
  #your code here
  return # the function which sends v to M(v)

def validator():
  if matrix_to_function([[-3,2,4],[5,-7,2]])([5,3,2])[0] != -1:
    return False
  if matrix_to_function([[4,3],[2,-1],[-5,3]])([2,-4])[2] != -22:
    return False
  return True
\end{python}
\end{solution}

Now you can go back and check---for some examples of $A$, $B$, and $\vec{v}$---that the following is true:
\verb|matrix_to_function(A)(matrix_to_function(B)(v))| is the same as \verb|matrix_to_function(multiply(A,B))(v)|.

\begin{solution}
  Now let's go the other way.  Write a function
  \verb|function_to_matrix| which takes a Python function
  \verb|f|---assumed to be a linear map from $\R^2$ to $\R^2$---and
  returns the $2 \times 2$ matrix representing that linear map.

\begin{python}
# For example if you had defined
# 
# def L(v):
#   return [2*v[0]+3*v[1], -4*v[0]]
#
# Then function_to_matrix(L) is 
		
# You may assume that L takes [x,y] to another list with two entries
# and you may assume that L is linear

def function_to_matrix(L):
  #your code here
  return # the matrix

def validator():
  M = function_to_matrix( lambda v: [3*v[0]+5*v[1], -2*v[0] + 4*v[1]] )
  if (M[0][0] != 3):
    return False
  M = function_to_matrix( lambda v: [2*v[0]-3*v[1], -7*v[0] - 5*v[1]] )
  if (M[1][0] != -7):
    return False
  M = function_to_matrix( lambda v: [v[0]+7*v[1], 3*v[0] - 2*v[1]] )
  if (M[1][1] != -2):
    return False
  return True
\end{python}
\end{solution}

Great work!  If you like, you can try to compute \verb|function_to_matrix(matrix_to_function(M))|.  You should get back $M$.
\end{exercise}

\end{document}
