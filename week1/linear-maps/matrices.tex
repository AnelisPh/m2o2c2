\documentclass{ximera}



\title{Matrices}

\begin{document}

\begin{abstract}
  Matrices are a way to represent linear maps.
\end{abstract}
\maketitle


To make writing a linear map a little less cumbersome, we will develop a compact notation for linear maps using our previous observation that a linear 
map is determined by its action on the standard basis vectors.
	
\begin{definition}
  An $m \times n$ \textit{matrix} is an array of numbers which has $m$ rows and $n$ columns.  The numbers in a matrix are called \textit{entries}.

  When $A$ is a matrix, we write $A = (a_{ij})$, meaning that $a_{i,j}$ is the entry in the $i^{th}$  row and $j^{th}$ column of the matrix.  Note:  We start counting
  with $1$ not $0$.  So the upper lefthand entry of the matrix is $a_{1,1}$.
\end{definition}

\begin{question}
  The matrix $A = \begin{bmatrix}
    1&-1\\2&\phantom{-}4\\3&-5
  \end{bmatrix}$
  is an $n \times m$ matrix.  

  \begin{solution}
    \begin{hint}
      Note that this is $n \times m$ whereas the definition above used $m \times n$.
    \end{hint}
    \begin{hint}
    	$n$ is the number of rows, and $m$ is the number of columns 
    \end{hint}
    \begin{hint}
    	$n=3$  and $m=2$
    \end{hint}

    In this case, $n$ is \answer{$3$}.
  \end{solution}

  \begin{solution}
    And $m$ is \answer{$2$}.
  \end{solution}

  Remember, we write $a_{i,j}$ for the entry in the $i^{th}$ row and $j^{th}$ column of the matrix.


  \begin{solution}
  	\begin{hint}
  		$a_{3,2}$ is the entry in the $3^{rd}$ row and the $2^{nd}$ column.
  	\end{hint}
  	\begin{hint}
  		$a_{3,2} = -5$
  	\end{hint}
    Therefore $a_{3,2}$ is \answer{$-5$}.
  \end{solution}

  Next, suppose the $3 \times 4$ matrix $B$ has $b_{i,j} = i+j$.

  \begin{solution}
   \begin{hint}
   		\begin{question}
   			\begin{solution}
   			\begin{hint}
   				$b_{1,2} = 1+2 = 3$
   			\end{hint}
   			According to this rule, $b_{1,2}$ is \answer{$3$}
   			\end{solution} 
   		\end{question}
   		
   		So the entry in the first row and second column of this matrix should be  $3$.
   \end{hint}
    \begin{hint}
    	$B = \begin{bmatrix}
    		2 &3 & 4&5\\
    		3&4&5&6\\
    		4&5&6&7\\
    	\end{bmatrix}$
    \end{hint}
    What is $B$?
	
    \begin{matrix-answer}[name=B]
      correctMatrix = [['2','3','4','5'],['3','4','5','6'],['4','5','6','7']]
    \end{matrix-answer}
  \end{solution}
\end{question}

\begin{definition}
  To each linear map $L: \R^n \to \R^m$ we associate a $m \times n$
  matrix $A_L$ called the \textit{matrix of the linear map} with
  respect to the standard coordinates.  It is defined by setting
  $a_{i,j}$ to be the $i^{\text{th}}$ component of $L(e_j)$.  In other words,
  the $j^{\text{th}}$ column of the matrix $A_L$ is the vector $L(e_j)$.

  Going the other way, we likewise associate to each matrix $m \times
  n$ matrix $M$ a linear map $L_M: \R^n \to \R^m$ by requiring that
  $L(e_j)$ be the $j^{th}$ column of the matrix $M$.
\end{definition}

\begin{question}
  The linear map $L:\R^2\to\R^3$ satisfies
  $L\left(\verticalvector{1\\0}\right) = \verticalvector{3\\-5\\2}$ and
  $L\left(\verticalvector{0\\1}\right) = \verticalvector{-1\\1\\1}$.  What is the
  matrix of $L$?

  \begin{solution}
  	\begin{hint}
  		Remember that, by definition, the first column of this matrix should be $L\left(\verticalvector{1\\0}\right)$ and
  		the second column should be $L\left( \verticalvector{0\\1}\right)$.
  	\end{hint}
  	\begin{hint}
  		The matrix of $L$ is 
  		\[
  			\begin{bmatrix}
  				3&-1\\
  				-5&1\\
  				2&1
  			\end{bmatrix}
  			\]
  	\end{hint}
    \begin{matrix-answer}[name=L]
      correctMatrix = [['3','-1'],['-5','1'],['2','1']]
    \end{matrix-answer}    
  \end{solution}
\end{question}

Let's do another example.

\begin{question}
  Suppose $L$ is a linear map represented by the matrix
  $A = \begin{bmatrix}
    1&-1\\2&4\\3&-5
  \end{bmatrix}.$
  
  \begin{solution}
  \begin{hint}
  	$A$ should have one column for each basis vector of the domain.
  \end{hint}
  \begin{hint}
  	$A$ has $2$ columns, so the dimension of the domain is $2$.
  \end{hint}
  The dimension of the domain of $L$ is \answer{2}.
  \end{solution}

  \begin{solution}
  	\begin{hint}
  		Each column of $A$ is the image of a basis vector under the action of $L$
  	\end{hint}
  	\begin{hint}
  		Since the columns are of length $3$, that means $L$ is spitting out vectors of length $3$.
  	\end{hint}
  	\begin{hint}
  		The codomain of $L$ is $\R^3$ which is $3$ dimensional.
  	\end{hint}
    The dimension of the codomain of $L$ is \answer{3}.
  \end{solution}
  
  Suppose $\vec{v} = L\left(\verticalvector{0,1}\right)$.  What is $\vec{v}$?
    
  \begin{solution}
  	\begin{hint}
  		Remember that, by definition, the $i^{th}$ column of $A$ is $L(\vec{e_i})$. 
  	\end{hint}
  	\begin{hint}
  		So, by definition, $L\left(\verticalvector{0\\1}\right)$ is the second column of the matrix $A$.
  	\end{hint}
  	\begin{hint}
  		So $L\left(\verticalvector{0\\1}\right) = \verticalvector{-1\\4\\-5}$
  	\end{hint}
    \begin{matrix-answer}[name=v]
      correctMatrix = [['-1'],['4'],['-5']]
    \end{matrix-answer}          
  \end{solution}

  Suppose $\vec{w} = L\left(\verticalvector{4\\5}\right)$.  What is $\vec{w}$?
    
  \begin{solution}
  	\begin{hint}
  		By definition of the matrix associated to a linear map, we know that $L\left(\verticalvector{1\\0}\right) = \verticalvector{1\\2\\3}$
  		and $L\left(\verticalvector{0\\1}\right) = \verticalvector{-1\\4\\-5}$.
  	\end{hint}
  	\begin{hint}
  		Can you rewrite $\verticalvector{4\\5}$ in terms of $\verticalvector{1\\0}$ and $\verticalvector{0\\1}$ so that you can use the linearity
  		of $L$ to compute $L\left(\verticalvector{4\\5}\right)$?
  	\end{hint}
  	\begin{hint}
  		$L\left(\verticalvector{4\\5}\right) = L\left(4\verticalvector{1\\0}+5\verticalvector{0\\1}\right)$
  	\end{hint}
  	\begin{hint}
  		\begin{align*}
  		L\left(\verticalvector{4\\5}\right) &= L\left(4\verticalvector{1\\0}+5\verticalvector{0\\1}\right)\\
  		&= 4L\left(\verticalvector{1\\0}\right)+5L\left(\verticalvector{0\\1}\right)\\
  		&=4\verticalvector{1\\2\\3}+5\verticalvector{-1\\4\\-5}\\
  		&=\verticalvector{4\\8\\12}+\verticalvector{-5\\20\\-25}\\
  		&=\verticalvector{-1\\28\\-13}
  		\end{align*}
  	\end{hint}
    \begin{matrix-answer}[name=w]
      correctMatrix = [['-1'],['28'],['-13']]
    \end{matrix-answer}          
  \end{solution}
  
  
  What is $L\left(\verticalvector{x\\y}\right)$?

  \begin{solution}
  \begin{hint}
  		By definition of the matrix associated to a linear map, we know that $L\left(\verticalvector{1\\0}\right) = \verticalvector{1\\2\\3}$
  		and $L\left(\verticalvector{0\\1}\right) = \verticalvector{-1\\4\\-5}$.
  	\end{hint}
  	\begin{hint}
  		Can you rewrite $\verticalvector{x\\y}$ in terms of $\verticalvector{1\\0}$ and $\verticalvector{0\\1}$ so that you can use the linearity
  		of $L$ to compute $L\left(\verticalvector{4\\5}\right)$?
  	\end{hint}
  	\begin{hint}
  		$L\left(\verticalvector{x\\y}\right) = L\left(x\verticalvector{1\\0}+y\verticalvector{0\\1}\right)$
  	\end{hint}
  	\begin{hint}
  		\begin{align*}
  		L\left(\verticalvector{x\\y}\right) &= L\left(x\verticalvector{1\\0}+y\verticalvector{0\\1}\right)\\
  		&= xL\left(\verticalvector{1\\0}\right)+yL\left(\verticalvector{0\\1}\right)\\
  		&=x\verticalvector{1\\2\\3}+y\verticalvector{-1\\4\\-5}\\
  		&=\verticalvector{x\\2x\\3x}+\verticalvector{-y\\4y\\-5y}\\
  		&=\verticalvector{x-y\\2x+4y\\3x-5y}
  		\end{align*}
  	\end{hint}
    \begin{matrix-answer}[name=w]
      correctMatrix = [['x-y'],['2*x + 4*y'],['3*x - 5*y']]
    \end{matrix-answer}          
  \end{solution}
\end{question}

As an antidote to the abstraction, let's take a look at a simplistic ``real world'' example.

\begin{question}
  In the local barter economy, there is an exchange where you can 
  \begin{itemize}
  \item trade $1$ spoon for $2$ apples and $1$ orange,
  \item trade $1$ knife for $2$ oranges, and
  \item trade $1$ fork for $3$ apples and $4$ oranges.
  \end{itemize}
  Model this as a linear map from $L:\R^3 \to \R^2$, where the coordinates on $\R^3$ are $\verticalvector{\text{spoons}\\\text{knives}\\\text{forks}}$ and the coordinates on $\R^2$ are
  $\verticalvector{\text{apples}\\\text{oranges}}$.

  \begin{solution}
  \begin{hint}
  	Remember the matrix of a linear map is defined by the fact the the $kth$ column of the matrix is
  	the image of the $kth$ standard basis vector.
  \end{hint}
  \begin{hint}
  	$\verticalvector{1\\0\\0}$ represents one spoon in the codomain.  Its image under this linear map is $2$ apples and $1$ orange, which is 
  	represented by the vector $\verticalvector{2\\1}$ in the codomain.  So the first column of the matrix should be $\verticalvector{2\\1}$ 
  \end{hint}
  \begin{hint}
  	The full matrix is 
  	\[
  		\begin{bmatrix}
  			2&0&3\\
  			1&2&4
  		\end{bmatrix}
  	\]
  \end{hint}
    What is the matrix of the linear map $L$?

    \begin{matrix-answer}[name=w]
      correctMatrix = [['2','0','3'],['1','2','4']]
    \end{matrix-answer}              
  \end{solution}

  \begin{solution}
  \begin{hint}
  		\begin{align}
  		L\left(\verticalvector{3\\0\\4}\right) & = L\left(3 \verticalvector{1\\0\\0}+4\verticalvector{0\\0\\1}\right)\\
  		&= 3L\left(\verticalvector{1\\0\\0}\right)+4L\left(\verticalvector{0\\0\\1}\right)\\
  		&=3\verticalvector{2\\1}+4\verticalvector{3\\4}\\
  		&=\verticalvector{6\\3}+\verticalvector{12\\16}\\
  		&=\verticalvector{18\\19}
  		\end{align}
  		
  		So you would be able to get $18$ apples and $19$ oranges.
  \end{hint}
  \begin{hint}
  	Now the ``5 year old'' solution: If you have $3$ spoons, $0$ knives, and $4$ forks, and you traded them all in for fruit, how many apples would you have?
  \end{hint}
  \begin{hint}
  	$3$ spoons would get you $6$ apples, and $4$ forks get you $12$ apples, so you would have a total of $18$ apples.
  \end{hint}
    The first (``apples'') entry of $L\left(\verticalvector{3\\0\\4}\right)$ is \answer{18}.

    Try to answer this question both by applying the matrix to the
    vector, but also as a $5$ year old would solve it.
  \end{solution}
\end{question} 

Prove the following statement: if $S:\R^n \to\R^m$ and $T:\R^n \to \R^m$ are both linear maps, 
then the map $(S+T):\R^n \to\R^m$ defined by $(S+T)(\vec{v}) = S(\vec{v})+T(\vec{v})$ is also linear.

\begin{free-response} 
 We need to check that $(S+T)$ respects both scalar multiplication and vector addition.
 
 Scalar multiplication:
 
 Choose and arbitrary scalar $c \in \R$ and an arbitrary vector $\vec{v} \in \R^n$.  Then
 	\begin{align*}
 	(S+T)(c\vec{v}) &=S(c\vec{v})+T(c\vec{v}) \text{ by definition of } (S+T)\\
 	&= cS(\vec{v})+cT(\vec{v}) \text{ by the linearity of } S \text{ and } T\\
 	&=c\left( S(\vec{v}) + T(\vec{v})\right) \text{ by the distributivity of scalar multiplication over addition in } \R^m\\
 	&=c(S+T)(\vec{v}) \text{ by definition of } (S+T)
 	\end{align*}
 	
 	Vector addition:
 	Choose two arbitrary vectors $\vec{v}$ and $\vec{w}$ in $\R^n$.  Then
 	
 	\begin{align*}
 		(S+T)(\vec{v} +\vec{w}) &= S(\vec{v}+\vec{w})+T(\vec{v}+\vec{w}) \text{ by definition of } S+T\\
 		&= S(\vec{v})+S(\vec{w})+T(\vec{v})+T(\vec{w}) \text{ by the linearity of } S \text{ and } T\\
 		&= S(\vec{v})+T(\vec{v}) + S(\vec{w})+T(\vec{w}) \text{ by the commutativity of vector addition in } \R^m\\
 		&=(S+T)(\vec{v})+(S+T)(\vec{w}) \text{ by the definition of } S+T.
 	\end{align*}
\end{free-response}

%\begin{question}
%  If the matrix of $S$ is $BLAH$ and the matrix of $T$ is $BLAH$, what is the matrix of $S+T$?

%  \begin{free-response}
%  \end{free-response}
%\end{question}

Prove that if $T:\R^n \to \R^m$ is a linear map and $c \in \R$ is a scalar, then the map $cT:\R^n \to \R^m$,  defined by 
\[(cT)(\vec{v}) = cT(\vec{v})\] is also a linear map.
  
\begin{free-response}
We need to check that $cT$ respects both scalar multiplication and vector addition.
 
 Scalar multiplication:
 
 Choose and arbitrary scalar $a \in \R$ and an arbitrary vector $\vec{v} \in \R^n$.  Then
 	\begin{align*}
 		(cT)(a\vec{v}) &= cT(a\vec{v})\\
 		&= acT(\vec{v})\\
 		&=a(cT)(\vec{v})
 	\end{align*}
 	
 	Vector addition:
 	Choose two arbitrary vectors $\vec{v}$ and $\vec{w}$ in $\R^n$.  Then
 	
 	\begin{align*}
 		(cT)(\vec{v}+\vec{w}) &= cT(\vec{v}+\vec{w})\\
 		&= c\left(T(\vec{v})+T(\vec{w})\right)\\
 		&=cT(\vec{v})+cT(\vec{w})\\
 		&=(cT)(\vec{v})+(cT)(\vec{w})
 	\end{align*}
\end{free-response}

%\begin{question}
%  If the matrix of $T$ is $BLAH$, what is the matrix of $5T$?
%\end{question}

\begin{observation}
  The last two exercises show that we have a nice way to both add
  linear maps and multiply linear maps by scalars.  So linear maps
  themselves ``feel'' a bit like vectors.  You do not have to worry
  about this now, but we will see that the linear maps from $\R^n \to
  \R^m$ form an ``abstract vector space.''  Much of the power of
  linear algebra is that we can apply linear algebra to spaces of
  linear maps!
\end{observation}
	
\end{document}
%%% Local Variables: 
%%% mode: latex
%%% TeX-master: t
%%% End: 
