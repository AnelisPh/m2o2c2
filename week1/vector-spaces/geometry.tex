\documentclass{ximera}

\title{Geometry}

\begin{document}

\begin{abstract}
  Vectors can be viewed geometrically.
\end{abstract}

	Graphically, we depict a vector $\verticalvector{x_1\\x_2\\.\\.\\x_n}$ in $\R^n$ as an arrow whose base is at the origin and whose head is at  the point $(x_1,x_2,...,x_n)$.  
	For example, in $\R^2$ we would depict the vector $\verticalvector{3,4}$ as follows
 	 
 	 BADBAD PICTURE
 	 
 	 \begin{question}
 	 	What is the vector $v$ pictured below?
 	 	BADBAD PICTURE
  	 \end{question}
  	 
  	 \begin{question}
  	 	\begin{hint}
  	 		BADBAD picture
  	 	\end{hint}
  	 	On a sheet of paper, draw the vector $\vec{3}{1}$. Click the hint to see if you got it right.
  	 	%BADBAD INTERACTIVE
  	 \end{question}
  	 
  	 \begin{question}
  	 \begin{hint}
  	 	BADBAD PICTURE
  	 \end{hint}
  	 	 $\vec{v}_1$ and $\vec{v}_2$ are drawn below.  Redraw them on a sheet of paper, and also draw their sum $\vec{v}_1+\vec{v}_2$.
  	 	 Click the hint to see if you got it right.
  	 	
  	 	%BADBAD INTERACTIVE  
  	 \end{question}
  	 
  	 \begin{question}
  	 \begin{hint}
  	 BADBAD PICTURE
  	 \end{hint}
  	 	$\vec{v}$ is drawn below.  Redraw it on a sheet of paper, and also draw $3\vec{v}$.  Click the hint to see if you got it right
  	 	BADBAD INTERACTIVE
  	 \end{question}
  	 
  	% \begin{question}
  	 	%Below $v_1$ and $v_2$ are fixed, but $v_3$ is draggable.  Drag $v_3$ so that $2v_1+3v_2 = v_3$.
  	 	%BADBAD INTERACTIVE  
  	 %\end{question}
  	 
  		By playing around above, you may have noticed that you can sum vectors graphically by forming a parallelogram as in the picture below.
  		
  		BADBAD PICTURE
  		
  		You also may have noticed that multiplying a vector by a scalar leaves the vector pointing in the same direction but "scales" its length.  That is the reason we call real
  		numbers  "scalars" when they are coefficients of vectors:  it is to remind us that they act geometrically by scaling the vector.
  		
\end{document}


%%% Local Variables: 
%%% mode: latex
%%% TeX-master: t
%%% End: 
