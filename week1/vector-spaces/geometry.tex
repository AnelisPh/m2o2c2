\documentclass{ximera}

\title{Geometry}

\begin{document}

\begin{abstract}
  Vectors can be viewed geometrically.
\end{abstract}

Graphically, we depict a vector $\verticalvector{x_1\\x_2\\.\\.\\x_n}$
in $\R^n$ as an arrow whose base is at the origin and whose head is at
the point $(x_1,x_2,...,x_n)$.  For example, in $\R^2$ we would depict
the vector $\vec{v} = \verticalvector{3,4}$ as follows

\begin{tikzpicture}
  \draw[color=gray,->] (-5,0) -- (5,0);
  \draw[color=gray,->] (0,-5,0) -- (0,5);
  
  \draw[->] (0,0) -- (3,4);
  \node[anchor=south west] () at (3,4) {$\vec{v}$};
\end{tikzpicture}

\begin{question}
  What is the vector $\vec{w}$ pictured below?
  \begin{tikzpicture}
    \draw[color=gray,->] (-5,0) -- (5,0);
    \draw[color=gray,->] (0,-5,0) -- (0,5);
    
    \draw[->] (0,0) -- (-4,2);
    \node[anchor=south east] () at (-4,2) {$\vec{w}$};
  \end{tikzpicture}

  \begin{solution}
    \begin{hint}
      Consider whether the $x$ and $y$ coordinates are positive or negative.
    \end{hint}
    \begin{multiple-choice}
      \choice[correct]{$(-4,2)$}
      \choice{$(3,-3)$}
      \choice{$(-4,-2)$}
      \choice{$(4,2)$}
    \end{multiple-choice}
  \end{solution}
\end{question}
  	 
\begin{question}
  \begin{hint}
  \begin{tikzpicture}
    \draw[color=gray,->] (-5,0) -- (5,0);
    \draw[color=gray,->] (0,-5,0) -- (0,5);
    
    \draw[->] (0,0) -- (3,1);
    \node[anchor=south west] () at (3,1) {$\vec{v}$};
  \end{tikzpicture}
  \end{hint}
  On a sheet of paper, draw the vector $\vec{v} = \vec{3}{1}$. Click the hint to see if you got it right.
  
\end{question}
  	 
\begin{question}
  \begin{hint}
  \begin{tikzpicture}
    \draw[color=gray,->] (-1,0) -- (8,0);
    \draw[color=gray,->] (0,-1,0) -- (0,8);
    
    \draw[->] (0,0) -- (2,3);
    \node[anchor=south west] () at (2,3) {$\vec{v}_1$};

    \draw[->] (0,0) -- (4,1);
    \node[anchor=south west] () at (4,1) {$\vec{v}_2$};

    \draw[->] (2,3) -- (6,4);
    \node[anchor=south west] () at (4,1) {$\vec{v}_2$};

    \draw[->] (4,1) -- (6,4);
    \node[anchor=south west] () at (4,1) {$\vec{v}_2$};

    \draw[->] (0,0) -- (6,4);
    \node[anchor=south west] () at (4,1) {$\vec{v}_1 + \vec{v}_2$};
  \end{tikzpicture}
  \end{hint}
  $\vec{v}_1$ and $\vec{v}_2$ are drawn below.  Redraw them on a sheet of paper, and also draw their sum $\vec{v}_1+\vec{v}_2$.
  Click the hint to see if you got it right.

  \begin{tikzpicture}
    \draw[color=gray,->] (-1,0) -- (8,0);
    \draw[color=gray,->] (0,-1,0) -- (0,8);
    
    \draw[->] (0,0) -- (2,3);
    \node[anchor=south west] () at (2,3) {$\vec{v}_1$};

    \draw[->] (0,0) -- (4,1);
    \node[anchor=south west] () at (4,1) {$\vec{v}_2$};
  \end{tikzpicture}
  
\end{question}
  	 
\begin{question}
  \begin{hint}
    \begin{tikzpicture}
      \draw[color=gray,->] (-1,0) -- (8,0);
      \draw[color=gray,->] (0,-1,0) -- (0,8);
      
      \draw[->] (0,0) -- (1,2);
      \node[anchor=south] () at (2,3) {$\vec{v}$};
      \draw[->] (0,0) -- (3,6);
      \node[anchor=south] () at (3,6) {$3 \vec{v}$};
    \end{tikzpicture}
  \end{hint}
  $\vec{v}$ is drawn below.  Redraw it on a sheet of paper, and also draw $3\vec{v}$.  Click the hint to see if you got it right

  \begin{tikzpicture}
    \draw[color=gray,->] (-1,0) -- (8,0);
    \draw[color=gray,->] (0,-1,0) -- (0,8);
    
    \draw[->] (0,0) -- (1,2);
    \node[anchor=south west] () at (1,2) {$\vec{v}$};
  \end{tikzpicture}
\end{question}

You may have noticed that you can sum vectors graphically by forming a
parallelogram.

  		You also may have noticed that multiplying a vector by a scalar leaves the vector pointing in the same direction but "scales" its length.  That is the reason we call real
  		numbers  "scalars" when they are coefficients of vectors:  it is to remind us that they act geometrically by scaling the vector.

\end{document}


%%% Local Variables: 
%%% mode: latex
%%% TeX-master: t
%%% End: 
