\documentclass{ximera}

\title{Functions}

\begin{document}

\begin{abstract}
  A function relates inputs and outputs.
\end{abstract}

\begin{definition}
  A function $f$ from a set $A$ to a set $B$ is an assignment of
  exactly one element of $B$ to each element of $A$.  If $a$ is an
  element of $A$, we write $f(a)$ for the element of $B$ which is
  assigned to $a$ by $f$.
\end{definition}

We call $A$ the domain of $f$, and $B$ the codomain of $f$.  We will
also commonly write $f:A \to B$ which we read outloud as ``$f$ from
$A$ to $B$'' or ``$f$ maps $A$ to $B$.''

\begin{example}
  Let $W =\{ \text{yes},\text{no}\}$ and $A = \{ \text{Dog},
  \text{Cat}, \text{Walrus}\}$.  Let $f:A \to W$ be the function which
  assigns to each animal in $Q$ the answer to the question ``Is this
  animal commonly a pet?.''  Then $f(\text{Dog}) = \text{yes}$,
  $f(\text{Cat}) = \text{yes}$, and $f(\text{Walrus}) =
  \text{no}$.  $A$ is the domain, and $W$ is the codomain.
\end{example}

In these activities, we mostly study functions from $\R^n$ to $\R^m$.

\begin{question}
  Let $g:\R^1 \to \R^2$ be defined by $g(\theta) = (cos(\theta),\sin(\theta))$.
  \begin{solution}
    \begin{hint}
      \begin{warning}
        In everything that follows, $\cos$ and $\sin$ are in terms of radians.
      \end{warning}
    \end{hint}
    \begin{hint}
      $g(\frac{\pi}{6}) = (\cos(\frac{\pi}{6}),\sin(\frac{\pi}{6}))$
    \end{hint}
    \begin{hint}
      If you remember your trig facts, this is
      $(\frac{\sqrt{3}}{2},\frac{1}{2})$.  Format this as
      $\verticalvector{\frac{\sqrt{3}}{2} \\ \frac{1}{2}}$ for
      this question.
    \end{hint}
    What is $g(\frac{\pi}{6})$ ?  Give your answer as a vertical column of numbers.
    \begin{matrix-answer}
      correctMatrix = [['cos(pi/6)'],['sin(pi/6)']]
    \end{matrix-answer}
  \end{solution}
  
  Can you imagine what would happen to the point $g(\theta)$ as $\theta$ moved from $0$ to $2\pi$?
\end{question}

\begin{question}
  Let $h:\R^2 \to \R^2$ be defined by $h(x,y) = (x,-y)$.
  \begin{solution}
    \begin{hint}
      Consider $h(2,1) = (2,-1)$.
    \end{hint}
    \begin{hint}
      \begin{tikzpicture}
        \draw[color=gray,->] (-3,0) -- (3,0);
        \draw[color=gray,->] (0,-3) -- (0,3);
        \draw[->] (0,0) -- (2,1);
        \node[anchor=south] () at (2,1) {$(2,1)$};
        \node[anchor=north] () at (2,-1) {$h(2,1)$};
        \draw[->] (0,0) -- (2,-1);
      \end{tikzpicture}
    \end{hint}
    \begin{hint}
      $h$ takes any point $(x,y)$ to its reflection in the $x-$axis.
    \end{hint}
    \begin{hint}
    	Format your answer as $\verticalvector{2\\-1}$
    \end{hint}
    
    What is $h(2,1)$?  Format your answer as a vertical column of numbers.
    
    Try to understand this function graphically.  How does it transform the plane?  The hint reveals the answer to this question.
    
    \begin{matrix-answer}
      correctMatrix = [['2'],['-1']]
    \end{matrix-answer}
  \end{solution}
  
  % To visualize this function, we have a pair of coordinate planes.  Move the point $(x,y)$ around in the first plane, and see what happens to $h(x,y)$
  % in the second plane.  Why does the function behave this way?
\end{question}
  
\begin{question}
  Let $f:\R^4 \to \R^2$ be defined by $f((x_1,x_2,x_3,x_4)) = (x_1 \, x_2+x_3,{x_4}^2+x_1)$.
  \begin{solution}
    \begin{hint}
      $f(3,4,1,9) = (3\cdot 4+1,9^2+3) = (13,84)$.
    \end{hint}
    \begin{hint}
      Format this as $\verticalvector{13\\84}$.
    \end{hint}
    What is $f(3,4,1,9)$? Format your answer as a vertical column of numbers.
    \begin{matrix-answer}
      correctMatrix = [['13'],['84']]
    \end{matrix-answer}
  \end{solution}

  Note that this function has too many inputs and outputs to visualize
  easily.  That certainly does not stop it from being a useful and meaningful
  function; this is a ``massively multivariable'' course.
\end{question}

\end{document}
