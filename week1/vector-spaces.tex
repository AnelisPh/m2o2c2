\documentclass{ximera}



\title{Vector spaces}

\begin{document}

\begin{abstract}
  Vector spaces are where vectors live.
\end{abstract}
\maketitle

It will be convenient for us to equip $\R^n$ with two algebraic
operations: ``vector addition'' and ``scalar multiplication'' (to be
defined soon).  This additional structure will transform $\R^n$ from a
mere set into a ``vector space.''  To distinguish between $\R^n$ as a
set and $\R^n$ as a vector space, we think of elements of $\R^n$ as a
set as being ordered lists, such as
\[
\mathbf{p} = (x_1,x_2,x_3,\dots,x_n),
\]
but elements of $\R^n$ the vector space will be written
typographically as vertically oriented lists flanked with square
brackets, like this
\[ 
\vec{v}= \verticalvector{x_1\\x_2\\x_3\\\vdots\\x_n}
\]
 
We will try to stick to the convention that bold letters like $\mathbf{p}$ represent points, while letters with little arrows above them (like $\vec{v}$) represent vectors. 

Unfortunately (like practically everybody else in the world), we use
the same symbol $\R^n$ to refer to both the vector space $\R^n$ and
the underlying set of points $\R^n$.
 
Vector addition is defined as follows:
 
\[\verticalvector{x_1\\x_2\\\vdots\\x_n} +\verticalvector{y_1\\y_2\\\vdots\\y_n}=\verticalvector{x_1+y_1\\x_2+y_2\\\vdots\\x_n+y_n}\]

\begin{warning}
  You cannot add vectors in $\R^n$ and $\R^m$ unless $n = m$.
\end{warning}

An element of $\R$ is a number, but it is also called a ``scalar'' in this context, and vectors can be multiplied by scalars as follows:
	
\[ c\verticalvector{x_1\\x_2\\\vdots\\x_n} = \verticalvector{cx_1\\cx_2\\\vdots\\cx_n} \]
 	 
\begin{warning}
  We have not yet defined a notion of multiplication for vectors.  You might think it is reasonable to define 
\[
\verticalvector{x_1\\x_2\\\vdots\\x_n} \verticalvector{y_1\\y_2\\\vdots\\y_n}=\verticalvector{x_1y_1\\x_2y_2\\\vdots\\x_ny_n},
\] but 
actually this operation is not especially useful, and will
\textit{never} be utilized in this course.  We will have a notion of
``vector multiplication'' called the \textit{dot product}, but that is
not the (faulty) definition above.
\end{warning}

\begin{question}
  \begin{solution}
    \begin{hint}
      $\verticalvector{1\\2\\3}+\verticalvector{3\\-2\\4} = \verticalvector{1+3\\2+-2\\3+4} = \verticalvector{4\\0\\7}$
    \end{hint}
    What is $\verticalvector{1\\2\\3}+\verticalvector{3\\-2\\4}$?
    
    \begin{matrix-answer}[name=v]
      correctMatrix = [['4'],['0'],['7']]
    \end{matrix-answer}
  \end{solution}
\end{question}
 	
\begin{question}
  \begin{solution}
    \begin{hint}
      $3\verticalvector{3\\-2\\4} = \verticalvector{3(3)\\3(-2)\\3(4)}  = \verticalvector{9\\6\\12}$
    \end{hint}
    What is $3\verticalvector{3\\-2\\4}$?
    \begin{matrix-answer}[name=v]
      correctMatrix = [['9'],['-6'],['12']]
    \end{matrix-answer}
  \end{solution}
\end{question}

\begin{question}
  
  If $\vec{v}_1 = \verticalvector{3\\-2} $, $\vec{v}_2 = \verticalvector{1\\5}$,  and  $\vec{v}_3 = \verticalvector{1\\1}$
  can you find $a,b \in \R$ so that $a\vec{v}_1+b\vec{v}_2=v_3$?
  \begin{solution}
    \begin{hint}
      \begin{align*}
        a\vec{v}_1+b\vec{v}_2&=v_3\\
        a\verticalvector{3\\-2} +b\verticalvector{1\\5} &=  \verticalvector{1\\1}\\
        \verticalvector{3a\\-2a}+\verticalvector{b\\5b} &= \verticalvector{1\\1}\\
        \verticalvector{3a+b\\-2a+5b} &= \verticalvector{1\\1}
      \end{align*}
      
      Can you turn this into a system of two equations?
    \end{hint}
    \begin{hint}	
      \begin{align*}
        &\begin{cases}
          3a+b=1\\
          -2a+5b=1
        \end{cases}\\
        &\begin{cases}
          15a+5b= 5\\
          -2a+5b=1
        \end{cases}\\
        &\begin{cases}
          17a = 4\\
          -2a+5b=1
        \end{cases}\\
        &\begin{cases}
          a = \frac{4}{17}\\
          -2(\frac{4}{17})+5b =1
        \end{cases}\\
        &\begin{cases}
          a = \frac{4}{17}\\
          b = \frac{5}{17}
        \end{cases}
      \end{align*}
    \end{hint}
    $a = $\answer{4/17}
  \end{solution}
  \begin{solution}
    $b=$ \answer{5/17}
  \end{solution}
\end{question}
 	
\end{document}
