\documentclass{ximera}

\title{Linear maps}

\begin{document}

\begin{abstract}
  Linear maps respect addition and scalar multiplication.
\end{abstract}

\begin{definition}
  A function $L: \R^n \to \R^m$ is called a \textit{linear map} if it
  ``respects addition and scalar multiplication.''  Symbolically, for
  a map to be linear, we must have that $L(v+w) = L(v)+L(w)$ for all
  $v,w \in \R^n$ and also $L(av) = a L(v)$ for all $a \in \R$ and
  $v\in \R^n$.
\end{definition}

\begin{exercise}
  Here is a multiple choice problem.

  \begin{solution}
    \begin{multiple-choice}
      \choice{Wrong answer A}
      \choice{Wrong answer B}
      \choice{Wrong answer C}
      \choice[correct]{Correct answer D}
    \end{multiple-choice}
  \end{solution}

  Here is some text after the multiple choice.
\end{exercise}


\begin{question}
  Which of the following functions are linear?
  \begin{solution}
    \begin{multiple-choice}
    \choice[correct] $f: \R^2 \to \R^1$ defined by $f(\verticalvector(x,y)) = x+2y$
    \choice $h:\R^2 \to \R^2$ defined by $h(x,y) = \verticalvector(17,x)$
    \end{multiple-choice}
  \end{solution}
\end{question}

\begin{question}
  Which of the following functions are linear?
  \begin{solution}
    \begin{multiple-choice}
    \choice $g: \R^3 \to R^2$ defined by $g(\verticalvector(x,y,z)) = \verticalvector(x,xy)$
    \choice[correct] $h:\R \to \R^4$ defined by $h(x) = \verticalvector(x,x,x,4x)$
    \end{multiple-choice}
  \end{solution}
\end{question}

\begin{question}
  Which of the following functions are linear?
  \begin{solution}
    \begin{multiple-choice}
      \choice $G: \R^4 \to  \R^3$ defined by $G(\verticalvector(x,y,z,t)) = \verticalvector(e^{x+y},x+y,sin(x+y))$
      \choice $A: \R^2 \to R^2$ defined by $A(\verticalvector(x,y))=\verticalvector(0,0)$
    \end{multiple-choice}
  \end{solution}
\end{question}
	
\begin{question}
  Let $L:\R^3 \to \R^2$ be a linear function.  Suppose $L(\verticalvector(1,0,0)) = \verticalvector(3,4)$, 
  $L(\verticalvector(0,1,0)) = \verticalvector(-2,0)$,  and  $L(\verticalvector(0,0,1)) = \verticalvector(1,-1)$.
  
  \begin{solution}
    What is $L (4,-1,2)$?

    \begin{matrix-answer}[name=M]
      function validator(m) {
        console.log( "validating ", m );
        return m[0][0].v == '17';
      }
    \end{matrix-answer}
    
    After the matrix answer.
  \end{solution}
\end{question}


\begin{question}
  Let $L:\R^3 \to \R^2$ be a linear function.  Suppose $L(\verticalvector(1,0,0)) = \verticalvector(3,4)$, 
  $L(\verticalvector(0,1,0)) = \verticalvector(-2,0)$,  and  $L(\verticalvector(0,0,1)) = \verticalvector(1,-1)$.
  
  \begin{solution}
    What is $L(x,y,z)$?

    \begin{matrix-answer}[name=M]
      function validator(m) {
        console.log( "validating ", m );
        return m[0][0].v == '17';
      }
    \end{matrix-answer}
    
    After the matrix answer.
  \end{solution}

\end{question}

As you have already discovered a linear map $L: \R^n \to \R^m$ is
fully determined by its action on the "standard basis vectors" $e_1 =
\verticalvector(1,0,0,...,0), e_2 = \verticalvector(0,1,0,...,0), e_3
= (0,0,1,0,...,0)$, and $e_n = (0,0,0,...,0,1)$.

\begin{free-response}
  Argue convincingly that if $L:\R^n \to\R^m$ is a linear map and you know $L(\vec{e_i})$ for $i=1,2,3,...,n$, then you could figure out $L(\vec{v})$ for
  any $\vec{v} \in \R^n$.
\end{free-response}

\end{document}
