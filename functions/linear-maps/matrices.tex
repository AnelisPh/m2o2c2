\documentclass{article}

\title{Matrices}

\begin{document}

\begin{abstract}
  Matrices are a way to represent linear maps.
\end{abstract}

To make writing a linear map a little less cumbersome, we will develop a compact notation for them using the observation above. 
	
\begin{definition}
  A $m \times n$ \textit{matrix} is an array of numbers which has $m$ rows and $n$ columns.  The numbers in a matrix are called \textit{enteries}. If $A$ is a matrix, 
  we will often write $a_{i,j}$ for the entry in the $i^{th}$  row and $j^{th}$ column of the matrix.
\end{definition}

\begin{definition}
  To each linear map $L: \R^n \to \R^m$  we associate a $m \times n$ matrix $A_L$ called the \textit{matrix of the linear map}.  It is defined 
  by letting $a_{i,j}$ be the $i^{th}$ component of $L(e_j)$.  In other words, the $j^{th}$ column of the matrix $A_L$ is the vector $L(e_j)$.  We also associate to each 
  matrix $m \times n$ matrix $M$ a linear map $L_M: \R^n \to \R^m$ by requiring that $L(e_j)$ is the $j^{th}$ column of the matrix $M$. 
\end{definition}

\begin{question}
  The matrix $A = \begin{bmatrix}
    1&-1\\2&4\\3&-5
  \end{bmatrix}$
  is an $n \times m$ matrix.  

  \begin{solution}
    In this case, $n = \answer{3}$.

    And $m = \answer{2}$.
  \end{solution}
\end{question}
	
\begin{question}
  The $3 \times 4$ matrix $A$ has $a_{i,j} = i+j$.  What is $A$?
\end{question}

\begin{question}
  $A = \begin{bmatrix}
    1&-1\\2&4\\3&-5
  \end{bmatrix}$
  What is $a_{3,2}$?
\end{question}

\begin{question}
  The linear map $L:\mathbb{R^2}\to\mathbb{R^3}$ satisfies
  $L(\verticalvector(1,0)) = \verticalvector(3,-5,2)$ and
  $L(\verticalvector(0,1)) = \verticalvector(1,1,1)$.  What is the
  matrix of $L$?
\end{question}

\begin{question}
  The matrix of $L$ is 
  $A = \begin{bmatrix}
    1&-1\\2&4\\3&-5
  \end{bmatrix}$
  What is the dimension of the domain of  $L$?  The codomain?
\end{question}

\begin{question}
  The matrix of $L$ is 
  $A = \begin{bmatrix}
    1&-1\\2&4\\3&-5
  \end{bmatrix}$
  
  What is $L(0,1)$?
\end{question}

\begin{question}
  The matrix of $L$ is 
  $A = \begin{bmatrix}
    1&-1\\2&4\\3&-5
  \end{bmatrix}$
  
  What is $L(4,5)$?
\end{question}

\begin{question}
  The matrix of $L$ is 
  $A = \begin{bmatrix}
    1&-1\\2&4\\3&-5
  \end{bmatrix}$
  
  What is $L(x,y)$?
\end{question}

As an antidote to the abstraction of these examples, lets look at a simplistic "real world" example:

\begin{question}
  In the local barter economy, there is an exchange rate :
  $1$ spoon $\mapsto$ $2$ apples and $1$ orange
  $1$ knife $\mapsto$  $2$ oranges
  $1$ fork $\mapsto$  $3$ apples and $4$ oranges
  
  Model this as a linear map from $L:\R^3 \to \R^2$, where the coordinates on $\R^3$ are $\verticalvector(\text{spoons},\text{knives},\text{forks}) $ and the coordinates on $\R^2$ are
  $\verticalvector(\text{apples},\text{oranges})$.
  
  What is the matrix of this linear map?  
  What is $L(\verticalvector(3,4))$?  Try to solve  this problem both by applying the matrix to the vector, and also as a $5$ year old would solve it. 
\end{question} 

\begin{question}
  Prove that if $S:\R^n \to\R^m$ and $T:\R^n \to \R^m$ are both linear maps, 
  then the map $(S+T):\R^n \to\R^m$ defined by $(S+T)(\vec{v}) = S(\vec{v})+T(\vec{v})$ is also linear.
\end{question}

\begin{question}
  If the matrix of $S$ is $BLAH$ and the matrix of $T$ is $BLAH$, what is the matrix of $S+T$?
\end{question}

\begin{question}
  Prove that if $T:\R^n \to \R^m$ is a linear map and $c \in \R$ is a scalar, then the map $cT:\R^n \to \R^m$  defined by $(cT)(\vec{v}) = cT(\vec{v})$ is also 
  a linear map
\end{question}

\begin{question}
  If the matrix of $T$ is $BLAH$, what is the matrix of $5T$?
\end{question}

\begin{observation}
The last two exercises show that we have a nice way to both add linear
maps and multiply linear maps by scalars.  So linear maps themselves
kind of ``feel'' like vectors.  You do not have to worry about this
now, but we will see next week that the linear maps from $\R^n \to
\R^m$ form an ``abstract vector space.''  Part of the power of linear
algebra is that we can apply linear algebra to spaces of linear
operators!
\end{observation}
	
\end{document}
%%% Local Variables: 
%%% mode: latex
%%% TeX-master: t
%%% End: 
