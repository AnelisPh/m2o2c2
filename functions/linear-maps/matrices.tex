\documentclass{ximera}

\title{Matrices}

\begin{document}

\begin{abstract}
  Matrices are a way to represent linear maps.
\end{abstract}

To make writing a linear map a little less cumbersome, we will develop a compact notation for linear maps using our previous observations.
	
\begin{definition}
  A $m \times n$ \textit{matrix} is an array of numbers which has $m$ rows and $n$ columns.  The numbers in a matrix are called \textit{entries}.

  When $A$ is a matrix, we write $A = (a_{ij})$, meaning that $a_{i,j}$ is the entry in the $i^{th}$  row and $j^{th}$ column of the matrix.
\end{definition}

\begin{question}
  The matrix $A = \begin{bmatrix}
    1&-1\\2&\phantom{-}4\\3&-5
  \end{bmatrix}$
  is an $n \times m$ matrix.  

  \begin{solution}
    \begin{hint}
      Note that this is $n \times m$ whereas the definition above used $m \times n$.
    \end{hint}

    In this case, $n$ is \answer{$3$}.
  \end{solution}

  \begin{solution}
    And $m$ is \answer{$2$}.
  \end{solution}

  Remember, we write $a_{i,j}$ for the entry in the $i^{th}$ row and $j^{th}$ column of the matrix.

  \begin{solution}
    Therefore $a_{3,2}$ is \answer{$-5$}.
  \end{solution}

  Next, suppose the $3 \times 4$ matrix $B$ has $b_{i,j} = i+j$.

  \begin{solution}
    What is $B$?

    \begin{matrix-answer}[name=B]
      correctMatrix = [['1','2','3','4'],['2','3','4','5'],['3','4','5','6']]
    \end{matrix-answer}
  \end{solution}
\end{question}

\begin{definition}
  To each linear map $L: \R^n \to \R^m$ we associate a $m \times n$
  matrix $A_L$ called the \textit{matrix of the linear map} with
  respect to the standard coordinates.  It is defined by setting
  $a_{i,j}$ to be the $i^{\text{th}}$ component of $L(e_j)$.  In other words,
  the $j^{\text{th}}$ column of the matrix $A_L$ is the vector $L(e_j)$.

  Going the other way, we likewise associate to each matrix $m \times
  n$ matrix $M$ a linear map $L_M: \R^n \to \R^m$ by requiring that
  $L(e_j)$ be the $j^{th}$ column of the matrix $M$.
\end{definition}

\begin{question}
  The linear map $L:\mathbb{R^2}\to\mathbb{R^3}$ satisfies
  $L(\verticalvector(1,0)) = \verticalvector(3,-5,2)$ and
  $L(\verticalvector(0,1)) = \verticalvector(1,1,1)$.  What is the
  matrix of $L$?

  \begin{solution}
    \begin{matrix-answer}[name=L]
      correctMatrix = [['3','-1'],['-5','1'],['2','1']]
    \end{matrix-answer}    
  \end{solution}
\end{question}

Let's do another example.

\begin{question}
  Suppose $L$ is a linear map represented by the matrix
  $A = \begin{bmatrix}
    1&-1\\2&4\\3&-5
  \end{bmatrix}.$
  
  \begin{solution}
  The dimension of the domain of $L$ is \answer{2}.
  \end{solution}

  \begin{solution}
    The dimension of the codomain of $L$ is \answer{3}.
  \end{solution}
  
  Suppose $\vec{v} = L\left(0,1\)$.  What is $\vec{v}$?
    
  \begin{solution}
    \begin{matrix-answer}[name=v]
      correctMatrix = [['-1'],['4'],['-5']]
    \end{matrix-answer}          
  \end{solution}

  Suppose $\vec{w} = L\left(4,5\)$.  What is $\vec{w}$?
    
  \begin{solution}
    \begin{matrix-answer}[name=w]
      correctMatrix = [['-1'],['4'],['-5']]
    \end{matrix-answer}          
  \end{solution}
  
  What is $L(4,5)$?

  \begin{solution}
    \begin{matrix-answer}[name=w]
      correctMatrix = [['-1'],['28'],['-13']]
    \end{matrix-answer}          
  \end{solution}
  
  What is $L(x,y)$?

  \begin{solution}
    \begin{matrix-answer}[name=w]
      correctMatrix = [['x-y'],['2*x + 4*y'],['3*x - 5*y']]
    \end{matrix-answer}          
  \end{solution}
\end{question}

As an antidote to the abstraction, let's take a look at a simplistic ``real world'' example.''

\begin{question}
  In the local barter economy, there is an exchange, where you can 
  \begin{itemize}
  \item trade $1$ spoon for $2$ apples and $1$ orange,
  \item trade $1$ knife for $2$ oranges, and
  \item trade $1$ fork for $3$ apples and $4$ oranges.
  \end{itemize}
  Model this as a linear map from $L:\R^3 \to \R^2$, where the coordinates on $\R^3$ are $\verticalvector{\text{spoons}\\\text{knives}\\\text{forks}}$ and the coordinates on $\R^2$ are
  $\verticalvector{\text{apples}\\\text{oranges}}$.

  \begin{solution}
    What is the matrix of the linear map $L$?

    \begin{matrix-answer}[name=w]
      correctMatrix = [['2','0','3'],['1','2','4']]
    \end{matrix-answer}              
  \end{solution}

  \begin{solution}
    The first (``apples'') entry of $L\left(\verticalvector{3\\0\\4}\right)$ is \answer{18}.

    Try to answer this question both by applying the matrix to the
    vector, but also as a $5$ year old would solve it.
  \end{solution}
\end{question} 

\begin{question}
  Prove the following statement: if $S:\R^n \to\R^m$ and $T:\R^n \to \R^m$ are both linear maps, 
  then the map $(S+T):\R^n \to\R^m$ defined by $(S+T)(\vec{v}) = S(\vec{v})+T(\vec{v})$ is also linear.

  \begin{free-response}
  \end{free-response}
\end{question}

%\begin{question}
%  If the matrix of $S$ is $BLAH$ and the matrix of $T$ is $BLAH$, what is the matrix of $S+T$?

%  \begin{free-response}
%  \end{free-response}
%\end{question}

\begin{question}
  Prove that if $T:\R^n \to \R^m$ is a linear map and $c \in \R$ is a scalar, then the map $cT:\R^n \to \R^m$,  defined by 
  \[(cT)(\vec{v}) = cT(\vec{v})\] is also a linear map.

  \begin{free-response}
  \end{free-response}
\end{question}

%\begin{question}
%  If the matrix of $T$ is $BLAH$, what is the matrix of $5T$?
%\end{question}

\begin{observation}
  The last two exercises show that we have a nice way to both add
  linear maps and multiply linear maps by scalars.  So linear maps
  themselves ``feel'' a bit like vectors.  You do not have to worry
  about this now, but we will see that the linear maps from $\R^n \to
  \R^m$ form an ``abstract vector space.''  Much of the power of
  linear algebra is that we can apply linear algebra to spaces of
  linear maps!
\end{observation}
	
\end{document}
%%% Local Variables: 
%%% mode: latex
%%% TeX-master: t
%%% End: 
