\begin{document}
\section{$\R^n$ as a vector space}
It will be convenient for us to equip $\R^n$ with two algebraic operations:  "vector addition" and "scalar multiplication" (to be defined soon).
This additional structure will transform $\R^n$ from a mere set into a "vector space".  To distinguish between $\R^n$ the set and $\R^n$ the vector space,
 we think of elements of $\R^n$ the set as being ordered lists , like this 
 
 $$\mathbf{p} = (x_1,x_2,x_3, ...,x_,n)$$, 
 
 but elements of $\R^n$ the vector space will be written typographically as vertically oriented lists flanked with square brackets, like this 
 
  $$ \vec{v}= \begin{bmatrix}
 		x_1\\
 		x_2\\
 		.\\
 		.\\
 		.\\
 		x_n
 	 \end{bmatrix}
	 	$$
 
We will try to stick to the convention that bold letters like $\mathbf{p}$ represent points, while letters with little arrows above them (like $\vec{v}$) represent vectors. Unfortunately, we use the same symbol $\R^n$ to refer to both the vector space $\R^n$ and the underlying set of points $\R^n$.
 
 Vector addition is defined as follows:
 
 $$\begin{bmatrix}
 		x_1\\
 		x_2\\
 		.\\
 		.\\
 		.\\
 		x_n
 	 \end{bmatrix} +
 		\begin{bmatrix}
 		y_1\\
 		y_2\\
 		.\\
 		.\\
 		.\\
 		y_n
 	 \end{bmatrix}=
 	 \begin{bmatrix}
 		x_1+y_1\\
 		x_2+y_2\\
 		.\\
 		.\\
 		.\\
 		x_n+y_n
 	 \end{bmatrix}
	 	$$
	 	
	An element of $\R$ is also called a "scalar" in this context, and vectors can be multiplied by scalars as follows:
	
	$$ c\begin{bmatrix}
 		x_1\\
 		x_2\\
 		.\\
 		.\\
 		.\\
 		x_n
 	 \end{bmatrix} = 
 	 \begin{bmatrix}
 		cx_1\\
 		cx_2\\
 		.\\
 		.\\
 		.\\
 		cx_n
 	 \end{bmatrix}$$
 	 
 	Important:  We do not define multiplication of vectors!
 	
 	\begin{question}
 		What is $\begin{bmatrix}
 		1\\2\\3
 	 \end{bmatrix}+\begin{bmatrix}
 		3\\-2\\4
 	 \end{bmatrix}?$
 	
 	\end{question}
 	
 	\begin{question}
 		What is $3\begin{bmatrix}
 		3\\-2\\4
 	 \end{bmatrix}?$
 	 
 	 \begin{question}
 	 	If $v_1 = \begin{bmatrix}
 		3\\-2
 	 \end{bmatrix} $,
 	  $v_2 = \begin{bmatrix}
 		1\\5
 	 \end{bmatrix}$,  and 
 	 $v_3 = \begin{bmatrix}
 		1\\1
 	 \end{bmatrix}$
 	 can you find $a,b \in \R$ so that $a_1v_1+a_2v_2=v_3$?
 	 \end{question}
 	
	Graphically, we depict a vector $\begin{bmatrix}
 		x_1\\
 		x_2\\
 		.\\
 		.\\
 		.\\
 		x_n
 	 \end{bmatrix}$ in $\R^n$ as an arrow whose base is at the origin and whose head is at  the point $(x_1,x_2,...,x_n)$.  For example, in $\R^2$ 
 	 we would depict the vector $\begin{bmatrix}3\\4\end{bmatrix}$ as follows
 	 
 	 BADBAD PICTURE
 	 
 	 \begin{question}
 	 	What is the vector $v$ pictured below?
 	 	BADBAD PICTURE
  	 \end{question}
  	 
  	 \begin{question}
  	 	Drag the vector below so that it represents the vector \begin{bmatrix}-2\\5\end{bmatrix}.
  	 	BADBAD INTERACTIVE
  	 \end{question}
  	 
  	 \begin{question}
  	 	Below $v_1$ and $v_2$ are fixed, but $v_3$ is draggable.  Drag $v_3$ so that $v_1+v_2 = v_3$.
  	 	BADBAD INTERACTIVE  
  	 \end{question}
  	 
  	 \begin{question}
  	 	Below $v$ is fixed , but $w$ is draggable.  Drag $w$ so that $w = 3v$.
  	 	BADBAD INTERACTIVE
  	 \end{question}
  	 
  	 \begin{question}
  	 	Below $v_1$ and $v_2$ are fixed, but $v_3$ is draggable.  Drag $v_3$ so that $2v_1+3v_2 = v_3$.
  	 	BADBAD INTERACTIVE  
  	 \end{question}
  	 
  		By playing around above, you may have noticed that you can sum vectors graphically by placing them "head to tail" as in the picture below.
  		
  		BADBAD PICTURE
  		
  		You also may have noticed that multiplying a vector by a scalar leaves the vector pointing in the same direction but "scales" its length.  That is the reason we call real 		numbers  "scalars" when they are coefficients of vectors:  it is to remind us that they act geometrically by scaling the vector.
  		
  		\begin{definition}
  			We say that a vector $w$ is a \textit{linear combination} of the vectors $v_1,v_2,v_3,...,v_k$ if there are scalars $a_1,a_2,...a_k$ 
  			so that $w = a_1v_1+a_2v_2+...+a_kv_k$
  		\end{definition}
  		
  		\begin{definition}
  			The \textit{span} of a set of vectors $v_1,v_2, ..,v_k \in \R^n$ is the set of all linear combinations of the vectors.
  			Symbolically, $\vspan(v_1,v_2,...,v_k) = \{ a_1v_1+a_2v_2+...+a_kv_k : a_1,a_2, ...,a_k \in \R^n\}$
  		\end{definition}
  	 
  	 	\begin{question}
  	 		Is $\begin{bmatrix} 3//4//2\end{bmatrix}$ in the span of $\begin{bmatrix} 1\\2\\0\end{bmatrix}$ and $\begin{bmatrix}  3\\-3\\0\begin{bmatrix}$?
  	 	\end{question}
  	 	
  	 	\begin{question}
  	 		Write $u= \begin{bmatrix} 14 \\15\\-4\end{bmatrix}$ as a linear combination of 
  	 		$v = \begin{bmatrix} 1\\0\\1\end{bmatrix}$ and $w = \begin{bmatrix} 2\\3\\0\end{bmatrix}$
  	 		BADBAD FORMAT?
  	 	\end{question}
  	 	 
\end{document}