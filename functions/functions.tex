\begin{document}
\section{Functions}

\begin{definition}
A function $f$ from a set $A$ to a set $B$ is an assignment of exactly one element of $B$ to each element of $A$.  If $a$ is an element of $A$, we write $f(a)$ for the element of 
$B$ which is uniquely assigned to $a$ by $f$. 
\end{definition}

We call $A$ the domain of $f$, and $B$ the codomain of $f$.  We will also commonly write $f:A \to B$ which reads "$f$ from $A$ to $B$"

\begin{example}
Let  $W =\{ \text{``yes"},\text{``no"}\}$ and $A = \{ \text{Dog}, \text{Cat}, \text{Walrus}\}$.  Let $f:Q \to D$ be the function which assigns to each animal in $Q$ the answer to the question "Is this animal commonly a pet?".
Then $f(\text{Dog}) = \text{``yes''}$, $f(\text{Cat}) = \text{``yes''}$, and $f(\text{Walrus}) = \text{``no''}$.  $A$ is the domain, and $W$ is the codomain. 
\end{example}

In this class we will mostly study functions  from $\R^n$ to $\R^m$.

\begin{question}
	Let $g:\R^1 \to \R^2$ be defined by $g(\theta) = (cos(\theta),\sin(\theta))$.
	\begin{solution}
	\begin{hint}
		$g(\frac{\pi}{6}) = (\cos(\frac{\pi}{6}),\sin(\frac{\pi}{6}))$
	\end{hint}
	\begin{hint}
		If you remember your trig facts, this is $(\frac{\sqrt{3}}{2},\frac{1}{2})$.  Format this as $\verticalvector{\frac{\sqrt{3}}{2} \\ \frac{1}{2}}$ for this question.
	\end{hint}
	What is$g(\frac{\pi}{6})$ ?  Give your answer as a vertical column of numbers.
	\begin{matrix-answer}
		correctMatrix = [['cos(pi/6)'],[sin('pi/6')]]
	\end{matrix-answer}
	\end{solution}

	Can you imagine what would happen to the point $g(\theta)$ as $\theta$ moved from $0$ to $2\pi$?
\end{question}

\begin{question}
Let $h:\R^2 \to \R^2$ be defined by $h(x,y) = (x,-y)$.
\begin{solution}
		\begin{hint}
			$h(2,1)$
		\end{hint}
		\begin{hint}
			BADBAD PICTURE
			$h$ takes any point $(x,y)$ to its reflection in the $x-$axis.
		\end{hint}
		What is $h(2,1)$?  Format your answer as a vertical column of numbers.
		
		Try to understand this function graphically.  How does it transform the plane?  Click the hint twice for an answer to this question.
		\begin{matrix-answer}
		correctMatrix = [['2'],['-1']]
		\end{matrix-answer}
	\end{solution}

%To visualize this function, we have a pair of coordinate planes.  Move the point $(x,y)$ around in the first plane, and see what happens to $h(x,y)$
%in the second plane.  Why does the function behave this way?
\end{question}

\begin{question}
Let $f:\R^4 \to \R^2$ be defined by $f((x_1,x_2,x_3,x_4)) = (x_1*x_2+x_3,x_4^2+x_1)$.
\begin{solution}
\begin{hint}
	$f(3,4,1,9) = (3*4+1,9^2+3) = (13,84)$.  Format this as $\verticalvector{13\\84}$.
\end{hint}
What is $f(3,4,1,9)$? Format your answer as a vertical column of numbers.
\begin{matrix-answer}
	correctMatrix = [['13'],['84']]
\end{matrix-answer}
\end{solution}

Note that this function has too many dimensions to visualize easily.  That does not stop it from being a useful and meaningful function.
\end{question}

One of the most important things you can do with functions is compose them.

\begin{definition}
	Let $f:A \to B$ and $g:B \to C$.  Then there is another function $(g \circ f): A \to C$ defined by $(g \circ f)(a) = g\left[f(a)\right]$ for each $a \in A$.
	It is called the composition of $g$ with $f$.
\end{definition}

\begin{warning}
	The composition is only defined if the codomain of $f$ is the domain of $g$
\end{warning}

\begin{question}
		Let $A = {\text{cat},\text{dog}}$, $B = {(2,3),(5,6),(7,8)}$, $C = \R$. Let $f$ be defined by $f(\text{cat}) = (2,3)$ and $f(\text{dog} = (7,8))$.  Let $g$ be defined by
		the rule $g((x,y)) = x+y$.  
		\begin{solution}
			\begin{hint}
				$(g \circ f)(\text{cat}) = g\left[f(\text(cat))\right] =g((2,3)) = 2+3=5$
			\end{hint}
			 $(g \circ f)(\text{cat}) = \answer{5}$
		\end{solution} 
\end{question}

\begin{question}
	Let $h: \R^2 \to \R^3$ be defined by $h(x,y) = (x^2,xy,z+y)$, and let $\omega: \R^3 \to \R^2$ be defined by $\omega(x,y,z) = (\sin(xyz),z)$.
	\begin{solution}
		\begin{hint}
			\begin{align*}
			(\omega\circ h)(x,y) &= \omega\left[h(x,y)\right]\\
			&= \omega(x^2,xy,z+y)\\
			&= (\sin((x^2)(xy)(z+y)), z+y)\\
			&=(\sin(x^3yz+x^3y^2),z+y)
			\end{align*}
		\end{hint}
		What is $(\omega\circ h)(x,y)$?  Format your answer as a vertical column of formulas.
		\begin{matrix-answer}
			correctMatrix = [['sin(x^3*y*z+x^3*y^2)'],['z+y']]
		\end{matrix-answer}
	\end{solution}
\end{question}

Functions from $\R^n \to \R^m$ are not the only useful kind of function.  While such functions are our primary object of study in 
this multivariable calculus class,  it will often be helpful to think about "functions of functions".  The next examples might seem a 
bit peculiar, but later on in the course these kinds of mappings will become very important.

\begin{question}
Let $C_{[0,1]}$ be the set of all continuous functions from $[0,1]$ to $\R$.  Define $I:C_{[0,1]} \to \R$ by $I(f) = \displaystyle \int_0^1 f(x)dx$ 
	\begin{solution}
		\begin{hint}
			\begin{align*}
				I(g) &= \int_0^1 g(x)dx\\
				&= \int_0^1 x^2dx\\
				&= \frac{1}{3} x^3\big|_0^1\\
				&= \frac{1}{3}(1-0)\\
				&= \frac{1}{3}
			\end{align*}
		\end{hint}
		
		If g(x)  = x^2,  $I(g)$ = \answer{'1/3'}
	\end{solution}
\end{question}

\begin{question}
Let $C^{\infty}(\R)$ be the set of all infinitely differentiable functions on $\R$.  Define $Q: C^{\infty}(\R) \to C^{\infty}(\R)$ by $Q(f)(x) = f(0)+f'(0)x+\dfrac{f''(0)}{2}x^2$.
\begin{solution}
\begin{hint}
	\begin{question}
		\begin{solution}
			\begin{hint}
				$f(0)=\cos(0)=1$
			\end{hint}
			$f(0) = $ \answer{1}
		\end{solution}
	\end{question}
	\begin{question}
		\begin{solution}
			\begin{hint}
				$f'(x) = -\sin(x)$, so $f'(0)=-\sin(0)=0$
			\end{hint}
			$f'(0) = $ \answer{0}
		\end{solution}
	 \end{question}
	 \begin{question}
		\begin{solution}
			\begin{hint}
				$f''(x) = -\cos(x)$, so $f''(0)=-\cos(0)=-1$
			\end{hint}
			$f''(0) = $ \answer{-1}
		\end{solution}
	 \end{question}
\end{hint}
\begin{hint}
	So $Q(f)(x) = 1-\frac{x^2}{2}$
\end{hint}
If $f(x) = \cos(x)$, $Q(f)(x)=$ \answer{'1-x^2/2'}?
\end{solution}
This is an example of a function which eats a function and spits out another function. In particular, this takes a function and 
returns the second order MacLaurin polynomial of that function.
\end{question}

\begin{definition}
	Let $A$ and $B$ be two sets.  The \textit{product}  $A\times B$ of the two sets is the set of all ordered pairs $A \times B = \{ (a,b): a \in A and b \in B\}$.
\end{definition}

\begin{example}
	If $A = \{ 1,2,\text{Wolf}\}$  and $B = \{4,5\}$, then $A \times B = \{(1,4),(1,5),(2,4),(2,5),(\text{Wolf},4),(\text{Wolf},5)\}$
\end{example}

\begin{question}

Let $Func(\R,\R)$ be the set of all functions from $R$ to $R$.  Define $Eval: \R \times Func(\R,\R) \to \R$ by $Eval(x,f) = f(x)$.
\begin{solution}
\begin{hint}
	$Eval(-3,g) = g(-3)=|-3|=3$
\end{hint}
If $g(x) = |x|$,  $Eval(-3,g)=$ \answer{3}?
\end{solution}
\end{question}

\begin{question}
	Let $Func(A,B)$ be the set of all functions from $A$ to $B$ for any two sets $A$ and $B$.  
	Let $Curry:Func(\R^2,\R)\times \R \to  Func(\R,\R)$ be defined by $Curry(f,x)(y) = f(x,y)$.
	Let $h:\R^2 \times \R$ be defined by $h(x,y) = x^2 +xy$.  
	\begin{solution}
		\begin{hint}
			\begin{align*}
				G(3) &= Curry(h,2)(3)\\
				&= h(2,3)\\
				&= 2^2+2(3)\\
				&=10
			\end{align*}
		\end{hint}
		Let $G = Curry(h,2)$.  Then $G(3) =$ \answer{10}
	\end{solution}
	
	This wacky way of thinking is needed if you want to understand the \href{http://en.wikipedia.org/wiki/Lambda_calculus}{$\lambda$-calculus}.  It also helps a lot if
	you ever want to learn to program in Haskell (which this course is partially written in!)
	
\end{question}

\begin{question}
Define $dot_n:\R^n \times \R^n \to \R$ by $dot((x_1,x_2,...,x_n),(y_1,y_2,...,y_n))=x_1y_1+x_2y_2+x_3y_3+...+x_ny_n$.  
\begin{solution}
\begin{hint}
	$dot_3((2,4,5),(0,1,4))= 2(0)+4(1)+5(4) = 24$
\end{hint}
$dot_3((2,4,5),(0,1,4))=$ \answer{24}
\end{solution}
\end{question}

Programming Prompts:

#See if you can formalize the following functions in python

#Model the function f(x)=x^2 as a python function

def f(x):
	return #your code here
	
#Model the function g(x)=-1 if x<=0 , g(x)=1 if x>0 as a python function

def g(x):
	#Your code here
	
#Model the function h(x,y)=xysin(1/y) if y=/=0, 0 if y=0

def h(x,y):
	#Your code here
	
# Here is an example of a higher order function horizontalShift.  It takes a function f of one variable, and a horizontal shift H, and returns the function whose graph is the same as f, only shifted H units.

def horizontalShift(f,H):
	# first we define a new function shiftF which is the appropriate shift of f
	def shiftedF(x):  
		return f(x-H) 
	# then we return that function
	return shiftedF

#Try  letting  k= horizontalShift(g,3) 	, and try evaluating k(3), k(4), k(-2).  Also try horizontalShift(f,2)(3).

# Write a function forwardDifference which takes a function f:R --> R and returns another function forwardDifference(f):R -->R defined 
# by forwardDifference(f)(x) = f(x+1)-f(x).

def forwardDifference(f):
	#Your code here
	
# Write a function approximateDerivative which takes a function f:R -->R  and a number h and returns another function approximateDerivative(f,h)(x) = [f(x+h)-f(x-h)]/(2h)

def approximateDerivative(f,h):
	#Your code here
	
#Try letting f(x) = x^2, and g = approximateDerivative(f,0.001).  Then try evaluating g(0),g(1),g(2),g(3),g(4).  Does this agree with what you know about differentiation?

#Python supports lists

	myList = [3,6,9,12]

#Try print myList[2] to see how lists work

#Model the function F(x,y,z)=(x+y,z^2x) as a function which takes the list [x,y,z] and returns the list [x+y,z*z*x]

def F(inputList):
	#your code here
	
#Model dot:R^2 x R^2 --> R as a function which takes two lists, X = [x_1,x_2] and Y=[y_1,y_2], and returns x_1y_1+ x_2y_2

def dot(X,Y):
	#your code here
	




\end{document}