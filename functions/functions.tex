\begin{document}
\section{Functions}

\begin{definition}
A function $f$ from a set $A$ to a set $B$ is an assignment of exactly one element of $B$ to each element of $A$.  If $a$ is an element of $A$, we write $f(a)$ for the element of 
$B$ which is uniquely assigned to $a$ by $f$. 
\end{definition}

We call $A$ the domain of $f$, and $B$ the codomain of $f$.  We will also commonly write $f:A \to B$ which reads "$f$ from $A$ to $B$"

\begin{question}
Let  $D =\{ "yes","no"\}$ and $Q = \{ Dog, Cat, Walrus\}$.  Let $f:Q \to D$ be the function which assigns to each animal in $Q$ the answer to the question "Is this animal commonly a pet?".
Is this a function?
What is $f(Dog)$?
What is $f(Walrus)$?
\end{question}

In this class we will mostly study functions  from $\R^n$ to $\R^m$.

\begin{question}
	Let $g:\R^1 \to \R^2$ be defined by $g(\theta) = (cos(\theta),\sin(\theta))$.
What is $g(\pi/2)$?
What is $g(\pi/4)$?

To visualize this function, here is a number line for $\theta$.  As you move $\theta$, what happens to $g(\theta)$ in the plane?  Why?
\end{question}

\begin{question}
Let $h:\R^2 \to \R^2$ be defined by $h(x,y) = (x,-y)$.
What is $h((2,1))$?
What is $h((0,0))$?
If $h((x,y)) = (4,4)$, what is $(x,y)$?

To visualize this function, we have a pair of coordinate planes.  Move the point $(x,y)$ around in the first plane, and see what happens to $h(x,y)$
in the second plane.  Why does the function behave this way?
\end{question}

\begin{question}
Let $f:\R^4 \to \R^2$ be defined by $f((x_1,x_2,x_3,x_4)) = (x_1*x_2+x_3,x_4^2+x_1)$.
What is $f(3,4,1,9)$?

Note that this function has too many dimensions to visualize easily.  That does not stop it from being a useful and meaningful function.
\end{question}

Function from $\R^n \to \R^m$ are not the only useful kind of function.  While such functions are our primary object of study in 
this multivariable calculus class,  it will often be helpful to think about "functions of functions".  The next examples might seem a 
bit peculiar, but later on in the course these kinds of mappings will become very important.

\begin{question}
Let $C_{[0,1]}$ be the set of all continuous functions from $[0,1]$ to $\R$.  Define $I:C_{[0,1]} \to \R$ by $I(f) = \displaystyle \int_0^1 f(x)dx$ 
If g(x)  = x^2, what is $I(g)$?
\end{question}

\begin{question}
Let $C^{\infty}(\R)$ be the set of all infinitely differentiable functions on $\R$.  Define $Q: C^{\infty}(\R) \to C^{\infty}(\R)$ by $Q(f)(x) = f(0)+f'(0)x+\dfrac{f''(0)}{2}x^2$..

If $f(x) = \cos(x)$, what is $Q(f)(x)$?

What is $Q(f)(3)?$

This is an example of a function which eats a function and spits out another function.
\end{question}

\begin{definition}
	Let $A$ and $B$ be two sets.  The \textit{product}  $A\times B$ of the two sets is the set of all ordered pairs $A \times B = \{ (a,b): a \in A and b \in B\}$.
\end{definition}

\begin{example}
	If $A = \{ 1,2,\text{Wolf}\}$  and $B = \{4,5\}$, then $A \times B = \{(1,4),(1,5),(2,4),(2,5),(\text{Wolf},4),(\text{Wolf},5)\}$
\end{example}

\begin{question}
Let $Func(\R,\R)$ be the set of all functions from $R$ to $R$.  Define $Eval: \R \times Func(\R,\R) \to \R$ by $Eval(x,f) = f(x)$.
If $g(x) = |x|$, what is $Eval(-3,g)$?
\end{question}

\begin{question}
Define $dot_n:\R^n \times \R^n \to \R$ by $dot((x_1,x_2,...,x_n),(y_1,y_2,...,y_n))=x_1y_1+x_2y_2+x_3y_3+...+x_ny_n$.  What is $dot_3((2,4,5),(0,1,4))$?
Can you find  two different $(y_1,y_2)$ with $dot_2((1,1),(y_1,y_2))=0$?
\end{question}

Programming Prompts:

#See if you can formalize the following functions in python

#Model the function f(x)=x^2 as a python function

def f(x):
	return #your code here
	
#Model the function g(x)=-1 if x<=0 , g(x)=1 if x>0 as a python function

def g(x):
	#Your code here
	
#Model the function h(x,y)=xysin(1/y) if y=/=0, 0 if y=0

def h(x,y):
	#Your code here
	
# Here is an example of a higher order function horizontalShift.  It takes a function f of one variable, and a horizontal shift H, and returns the function whose graph is the same as f, only shifted H units.

def horizontalShift(f,H):
	# first we define a new function shiftF which is the appropriate shift of f
	def shiftedF(x):  
		return f(x-H) 
	# then we return that function
	return shiftedF

#Try  letting  k= horizontalShift(g,3) 	, and try evaluating k(3), k(4), k(-2).  Also try horizontalShift(f,2)(3).

# Write a function forwardDifference which takes a function f:R --> R and returns another function forwardDifference(f):R -->R defined 
# by forwardDifference(f)(x) = f(x+1)-f(x).

def forwardDifference(f):
	#Your code here
	
# Write a function approximateDerivative which takes a function f:R -->R  and a number h and returns another function approximateDerivative(f,h)(x) = [f(x+h)-f(x-h)]/(2h)

def approximateDerivative(f,h):
	#Your code here
	
#Try letting f(x) = x^2, and g = approximateDerivative(f,0.001).  Then try evaluating g(0),g(1),g(2),g(3),g(4).  Does this agree with what you know about differentiation?

#Python supports lists

	myList = [3,6,9,12]

#Try print myList[2] to see how lists work

#Model the function F(x,y,z)=(x+y,z^2x) as a function which takes the list [x,y,z] and returns the list [x+y,z*z*x]

def F(inputList):
	#your code here
	
#Model dot:R^2 x R^2 --> R as a function which takes two lists, X = [x_1,x_2] and Y=[y_1,y_2], and returns x_1y_1+ x_2y_2

def dot(X,Y):
	#your code here
	




\end{document}