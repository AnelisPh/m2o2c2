\documentclass{ximera}
\title{Multilinear forms}
\begin{document}
	\begin{abstract}
		Multilinear forms are separately linear in multiple vector variables 
	\end{abstract}
	
	\begin{definition}
		Let $X$ be a set.  We introduce the notation $X^k$ to stand for the set of all ordered $k$-tuples of elements of $X$.  In other words, $X^k = X \times X \times X...\times X$, 
		 with $k$ ``factors'' of $X$.
	\end{definition}	
	
	\begin{definition}
		A $k$-linear  form on a vector space $V$ is a function $T: V^k \to \R$ which is linear in each vector variable.  In other words, given $k-1$ vectors
		 $v_1,v_2,...,v_{i-1},v_{i+1},...,v_k$, the map $T_i: V \to \R$ defined by $T_i(v) = T(v_1,v_2,...,v_{i-1},v,v_{i+1},...,v_k)$ is linear.
	\end{definition}
	
	The $k$ linear forms on $V$ form a vector space.
	
	\begin{question}
		Let $T:\R^2\times \R^2 \times \R^2 \to \R$ be a trilinear form on $\R^2$.  Suppose we know that
			\begin{itemize}
				\item $T\left( \verticalvector{1\\0},\verticalvector{1\\0},\verticalvector{1\\0}\right) = 1$
				\item $T\left( \verticalvector{1\\0},\verticalvector{1\\0},\verticalvector{0\\1}\right) = 2$
				\item $T\left( \verticalvector{1\\0},\verticalvector{0\\1},\verticalvector{1\\0}\right) = 3$
				\item $T\left( \verticalvector{1\\0},\verticalvector{0\\1},\verticalvector{0\\1}\right) = 4$
				\item $T\left( \verticalvector{0\\1},\verticalvector{1\\0},\verticalvector{1\\0}\right) = 5$
				\item $T\left( \verticalvector{0\\1},\verticalvector{1\\0},\verticalvector{0\\1}\right) = 6$
				\item $T\left( \verticalvector{0\\1},\verticalvector{0\\1},\verticalvector{1\\0}\right) = 7$
				\item $T\left( \verticalvector{0\\1},\verticalvector{0\\1},\verticalvector{0\\1}\right) = 8$
			\end{itemize}
			
			\begin{solution}
				\begin{hint}
					\begin{align*}
						T\left( \verticalvector{1\\1},\verticalvector{1\\2},\verticalvector{1\\0}\right) 
						&= T\left(\verticalvector{1\\0},\verticalvector{1\\2},\verticalvector{1\\0}\right)+T\left(\verticalvector{0\\1},\verticalvector{1\\2},\verticalvector{1\\0}\right)\\
						&=T\left(\verticalvector{1\\0},\verticalvector{1\\0},\verticalvector{1\\0}\right)+2T\left(\verticalvector{1\\0},\verticalvector{0\\1},\verticalvector{1\\0}\right)+
						T\left(\verticalvector{0\\1},\verticalvector{1\\0},\verticalvector{1\\0}\right)+ 2T\left(\verticalvector{0\\1},\verticalvector{0\\1},\verticalvector{1\\0}\right)\\
						&=1+2(3)+5+2(7)\\
						&=26
					\end{align*}
				\end{hint}
					$T\left( \verticalvector{1\\1},\verticalvector{1\\2},\verticalvector{1\\0}\right) = $\answer{26}
			\end{solution}
			
			
	\end{question}
	
	From the last example, and by analogy with the bilinear case, it is clear that if you know the value of a $k-$linear form on all $(dim V)^k$ $k$-tuples of basis vectors of 
			$V$, then you can find the value of $T$ on any $k$-tuple of vectors.
			
			\begin{definition}
				Let $T:V^{k_1} \to \R$ and $S:V^{k_2} \to \R$ be multilinear forms.  Then we define their tensor product by $T \otimes S: V^{k_1 + k_2} \to \R$ by multiplication:
				$(T \otimes S)(v_1,v_2,...,v_{k_1+k_2}) = T(v_1,v_2,...,v_{k_1})S(v_{k_1+1},v_{k_1+2},...,v_{k_1+k_2})$.
			\end{definition}
			
			\begin{theorem}
				The $k$-linear forms $dx_{i_1} \otimes dx_{i_2} \otimes ... \otimes dx_{i_k}$ where $1<i_j<n $ form a basis for the space of all multilinear $k$-forms on $\R^n$.  In fact
				
				\[
					T = \sum T(e_{i_1},e_{i_2},...,e_{i_k}) dx_{i_1} \otimes dx_{i_2} \otimes ... \otimes dx_{i_k}
				\]
				
				Where the sum ranges of all $n^k$  $k$-tuples of basis vectors.
			\end{theorem}
			
			The proof is as straightforward as the corresponding proof for bilinear forms, but the notation is truly awful.
			
		\begin{question}
			\begin{solution}
				\begin{hint}
					\(
						dx_1 \otimes dx_2 \otimes dx_2 (\verticalvector{1\\2},\verticalvector{-2\\4},\verticalvector{5\\6}) &= 1 \cdot 4 \cdot 6 = 24
					\)
				\end{hint}
				$dx_1 \otimes dx_2 \otimes dx_2 (\verticalvector{1\\2},\verticalvector{-2\\4},\verticalvector{5\\6}) = $\answer{$24$}
			\end{solution}
		\end{question}
		
		\begin{question} 
		Let $T = dx_1 \otimes dx_1 \otimes dx_1 + 4 dx_2 \otimes dx_2\otimes dx_1$ be a trilinear form on $\R^2$.
		Let $v = \verticalvector{x\\y}$
			\begin{solution}
				\begin{hint}
					\begin{align*}T(v,v,v) &= 
					dx_1 \otimes dx_1 \otimes dx_1( \verticalvector{x\\y},\verticalvector{x\\y},\verticalvector{x\\y}) 
					+ 4 dx_2 \otimes dx_2\otimes dx_1(\verticalvector{x\\y},\verticalvector{x\\y},\verticalvector{x\\y})\\
					&=x \cdot x \cdot y+ 4y \cdot y \cdot x\\
					&=x^3+4y^2x
					\end{align*}
				\end{hint}
				As a function of $x,$ and $y$,   $T(v,v,v)=$ \answer{ $x^3+4y^2x$}
			\end{solution}
			As this example shows, applying a $k$-linear form to the same vector three times gives a homogeneous polynomial of degree $k$.
		\end{question}
\end{document}