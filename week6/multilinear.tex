\documentclass{ximera}
\title{Multilinear forms}
\begin{document}

\begin{abstract}
  Multilinear forms are separately linear in multiple vector variables.
\end{abstract}
	
\begin{definition}
  Let $X$ be a set.  We introduce the notation $X^k$ to stand for the set of all ordered $k$-tuples of elements of $X$.  In other words, $X^k = X \times X \times \cdots \times X$, 
  with $k$ ``factors'' of $X$.
\end{definition}	

For example, if $X = \{ \text{cat}, \text{dog} \}$, then $X^3$ is a
set with eight elements, consisting of $3$-tuples of either
$\text{cat}$ or $\text{dog}$.  For example,
$(\text{cat},\text{cat},\text{dog}) \in X^3$.
	
\begin{definition}
  A $k$-linear  form on a vector space $V$ is a function $T: V^k \to \R$ which is linear in each vector variable.  In other words, given $(k-1)$ vectors
  $v_1,v_2, \ldots ,v_{i-1},v_{i+1}, \ldots ,v_k$, the map $T_i: V \to \R$ defined by $T_i(v) = T(v_1,v_2, \ldots ,v_{i-1},v,v_{i+1}, \ldots ,v_k)$ is linear.
\end{definition}

The $k$-linear forms on $V$ form a vector space.

\begin{question}
  Let $T:\R^2\times \R^2 \times \R^2 \to \R$ be a trilinear form on $\R^2$.  Suppose we know that
  \begin{itemize}
  \item $T\left( \verticalvector{1\\0},\verticalvector{1\\0},\verticalvector{1\\0}\right) = 1$
  \item $T\left( \verticalvector{1\\0},\verticalvector{1\\0},\verticalvector{0\\1}\right) = 2$
  \item $T\left( \verticalvector{1\\0},\verticalvector{0\\1},\verticalvector{1\\0}\right) = 3$
  \item $T\left( \verticalvector{1\\0},\verticalvector{0\\1},\verticalvector{0\\1}\right) = 4$
  \item $T\left( \verticalvector{0\\1},\verticalvector{1\\0},\verticalvector{1\\0}\right) = 5$
  \item $T\left( \verticalvector{0\\1},\verticalvector{1\\0},\verticalvector{0\\1}\right) = 6$
  \item $T\left( \verticalvector{0\\1},\verticalvector{0\\1},\verticalvector{1\\0}\right) = 7$
  \item $T\left( \verticalvector{0\\1},\verticalvector{0\\1},\verticalvector{0\\1}\right) = 8$
  \end{itemize}
  
  \begin{solution}
    \begin{hint}
      \begin{align*}
        T\left( \verticalvector{1\\1},\verticalvector{1\\2},\verticalvector{1\\0}\right) 
        &= T\left(\verticalvector{1\\0},\verticalvector{1\\2},\verticalvector{1\\0}\right)+T\left(\verticalvector{0\\1},\verticalvector{1\\2},\verticalvector{1\\0}\right)\\
        &=T\left(\verticalvector{1\\0},\verticalvector{1\\0},\verticalvector{1\\0}\right)+2T\left(\verticalvector{1\\0},\verticalvector{0\\1},\verticalvector{1\\0}\right)+
        T\left(\verticalvector{0\\1},\verticalvector{1\\0},\verticalvector{1\\0}\right)+ 2T\left(\verticalvector{0\\1},\verticalvector{0\\1},\verticalvector{1\\0}\right)\\
        &=1+2(3)+5+2(7)\\
        &=26
      \end{align*}
    \end{hint}
    $T\left( \verticalvector{1\\1},\verticalvector{1\\2},\verticalvector{1\\0}\right) = $\answer{26}
  \end{solution}
  
\end{question}

From the last example---and by analogy with the bilinear case---it is
clear that if you know the value of a $k-$linear form on all
$k$-tuples of basis vectors of $V$ (there are $(dim V)^k$ of such),
then you can find the value of $T$ on any $k$-tuple of vectors.

\begin{definition}
  Let $T:V^{k_1} \to \R$ and $S:V^{k_2} \to \R$ be multilinear forms.  Then we define their tensor product by $T \otimes S: V^{k_1 + k_2} \to \R$ by multiplication:
  $(T \otimes S)(v_1,v_2, \ldots ,v_{k_1+k_2}) = T(v_1,v_2, \ldots ,v_{k_1})S(v_{k_1+1},v_{k_1+2}, \ldots ,v_{k_1+k_2})$.
\end{definition}

\begin{theorem}
  The $k$-linear forms $dx_{i_1} \otimes dx_{i_2} \otimes \cdots \otimes dx_{i_k}$ where $1<i_j<n $ form a basis for the space of all multilinear $k$-forms on $\R^n$.  In fact,
  \[
  T = \sum T(e_{i_1},e_{i_2}, \ldots ,e_{i_k}) dx_{i_1} \otimes dx_{i_2} \otimes \cdots \otimes dx_{i_k},
  \]
  where the sum ranges of all $n^k$  $k$-tuples of basis vectors.
\end{theorem}

The proof is as straightforward as the corresponding proof for bilinear forms, but the notation is something awful.

\begin{question}
  \begin{solution}
    \begin{hint}
      \(dx_1 \left(\verticalvector{1\\2}\right) = 1.\)
    \end{hint}
    \begin{hint}
      \(dx_2 \left(\verticalvector{-2\\4}\right) = 4.\)
    \end{hint}
    \begin{hint}
      \(dx_2 \left(\verticalvector{5\\6}\right)) = 6.\)
    \end{hint}
    \begin{hint}
      So putting this all together, we have
      \[
      \left(dx_1 \otimes dx_2 \otimes dx_2\right) \left(\verticalvector{1\\2},\verticalvector{-2\\4},\verticalvector{5\\6}\right) &= 1 \cdot 4 \cdot 6 = 24
      \]
    \end{hint}
    $\left( dx_1 \otimes dx_2 \otimes dx_2 \right) \left(\verticalvector{1\\2},\verticalvector{-2\\4},\verticalvector{5\\6}\right) = $\answer{$24$}.
  \end{solution}
\end{question}

\begin{question} 
  Let $T = dx_1 \otimes dx_1 \otimes dx_1 + 4 dx_2 \otimes dx_2\otimes dx_1$ be a trilinear form on $\R^2$.
  Let $\vec{v} = \verticalvector{x\\y}$
  \begin{solution}
    \begin{hint}
      \begin{align*}T(\vec{v},\vec{v},\vec{v}) &= 
        dx_1 \otimes dx_1 \otimes dx_1( \verticalvector{x\\y},\verticalvector{x\\y},\verticalvector{x\\y}) 
        + 4 dx_2 \otimes dx_2\otimes dx_1(\verticalvector{x\\y},\verticalvector{x\\y},\verticalvector{x\\y})\\
        &=x \cdot x \cdot x + 4y \cdot y \cdot x\\
        &=x^3+4y^2x
      \end{align*}
    \end{hint}
    As a function of $x$ and $y$,   $T(\vec{v},\vec{v},\vec{v})=$ \answer{ $x^3+4y^2x$}
  \end{solution}

  As this example shows, applying a trilinear form to the same vector
  three times gives a polynomial.  
  \begin{solution}
    \begin{hint}
      The monomial $x^3$ has degree three.
    \end{hint}
    \begin{hint}
      The monomial $4 x^2 x$ also has degree three.
    \end{hint}
    \begin{hint}
      So the total degree of each monomial is three.
    \end{hint}
    The total degree of each monomial is \answer{$3$}.
  \end{solution}

  What we are seeing is a special case of the following result.
  \begin{theorem}
    Applying a $k$-linear form to the same vector $k$-times gives a
    homogeneous polynomial of degree $k$.
  \end{theorem}
\end{question}

\end{document}
