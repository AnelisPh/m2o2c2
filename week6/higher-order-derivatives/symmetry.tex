\documentclass{ximera}

\title{Symmetry}

\begin{document}

\begin{abstract}
  In many nice situations, higher-order derivatives are symmetric.
\end{abstract}\maketitle

Recall that we once saw the following theorem.
\begin{theorem}
  Let $f:\R^n \to \R$ be a differentiable function.  Assume that the partial derivatives $f_{x_i}:\R^n \to \R$ are all differentiable, and the second partial derivatives
  $f_{x_i,x_j}$ are continuous.  Then $f_{x_i,x_j} = f_{x_j,x_i}$.
\end{theorem}
After interpreting the ``second derivative'' as a bilinear form, we
were then able to say something nicer (though the hypothesis is
stronger, so this is a weaker theorem).
\begin{theorem}
  Let $f:\R^n \to \R$ be a continuously twice differentiable function; then the bilinear form representing the second derivative is symmetric.
\end{theorem}

And finally, we are in a position to formulate the higher-order
version of this theorem.
\begin{theorem}
  Let $f:\R^n \to \R$ be a continuously $k$-times differentiable
  function; then the $k$-linear form representing the $k$-th order
  derivative is a symmetric form.
\end{theorem}

\end{document}
