\documentclass{ximera}
\title{Taylor's theorem}
\begin{document}
\begin{abstract}
	Higher order derivatives give rise to higher order polynomial approximations.
\end{abstract}

Here is the statement of a statement of Taylor's theorem for many
variables.

\begin{theorem}
	Let $f:\R^n \to \R$ be a $(k+1)$-times differentiable function.  Then
	$$f(\mathbf{p}+\vec{h}) 
	= f(\mathbf{p})+Df(\mathbf{p})(\vec{h})+D^2f(\mathbf{p})(\vec{h},\vec{h})+D^3f(\mathbf{p})(\vec{h},\vec{h},\vec{h})+ \cdots
	+D^kf(\mathbf{p})(\vec{h}^k) + D^{k+1}(\mathbf{p}+\xi\vec{h})(\vec{h}^{k+1})$$ for some $\xi \in [0,1]$,  where we have abbreviated the 
	ordered tuple of $i$ $\vec{h}'$s as $\vec{h}^i$
\end{theorem}

Let's apply this to a specific function.

\begin{question}
  Let $f:\R^2 \to \R$ be defined by $f(x,y) = e^{x+y}$.  
  \begin{solution}
    \begin{hint}
      The second order taylor approximation is $1+(x+y)+\frac{(x+y)^2}{2}$
    \end{hint}
    \begin{hint}
      Every partial derivative of this function is $e^{x+y}$, so all of the third partial derivatives are $1$
    \end{hint}
    \begin{hint}
      So the third derivative is the sum of all of the following terms 
      \begin{itemize}
      \item $dx \otimes dx \otimes dx$
      \item $dx \otimes dx \otimes dy$
      \item $dx \otimes dy \otimes dx$
      \item $dx \otimes dy \otimes dy$
      \item $dy \otimes dx \otimes dx$
      \item $dy \otimes dx \otimes dy$
      \item $dy \otimes dy \otimes dx$
      \item $dy \otimes dy \otimes dy$
      \end{itemize}
    \end{hint}
    \begin{hint}
      Applying this tensor to $(\verticalvector{x\\y},\verticalvector{x\\y},\verticalvector{x\\y})$ we get 
      $xxx+xxy+xyx+xyy+yxx+yxy+yyx+yyy = (x+y)^3$
    \end{hint}
    \begin{hint}
      So the third order taylor expansion is $1+(x+y)+\frac{(x+y)^2}{2}+\frac{(x+y)^3}{6}$
    \end{hint}
    The third order taylor series of $f$ about the point $(0,0)$ is \answer{$1+(x+y)+(x+y)^2/2+(x+y)^3/6$}
  \end{solution}
\end{question}

	
	
\end{document}