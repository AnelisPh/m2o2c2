\documentclass{ximera}
\title{Python}

\begin{document}

\begin{abstract}
  Build some self-adjoint operators in Python.
\end{abstract}	

\begin{question}
  We will represent a linear operator in Python as a function which takes as input a list of $n$ real numbers, and outputs a list of $n$ real numbers.

  Write down a linear operator \texttt{L} which is \textit{self-adjoint}.
  \begin{solution}
    \begin{hint}
      In this problem, $n = 4$.
    \end{hint}
    \begin{hint}
      So the input \texttt{v} will be a list of four numbers, namely \texttt{v[0]}, \texttt{v[1]}, \texttt{v[2]}, and \texttt{v[3]}.
    \end{hint}
    \begin{hint}
      The output should also be a list of four numbers.
    \end{hint}
    \begin{hint}
      We must make sure that the resulting operator is self-adjoint, which we can achieve if the corresponding matrix is symmetric.
    \end{hint}
    \begin{hint}
      Since we just need to write down one example, we could even get away with \texttt{return v}, namely, the identity operator.  But let's try to be fancier!
    \end{hint}
    \begin{hint}
      Let's make \texttt{L} into the linear operator represented by the matrix $\begin{bmatrix} 2 & 3 & 0 & 0 \\ 3 & 4 & 0 & 0 \\ 0 & 0 & 1 & 0 \\ 0 & 0 & 0 & 1 \end{bmatrix}$.
    \end{hint}
    \begin{hint}
      We can achieve this with \texttt{return [2*v[0] + 3*v[1], 3*v[0] + 4*v[1], v[2], v[3]]}.
    \end{hint}
    \begin{python}
n = 4
def L(v):
  return # the vector L(v), but make sure that L is self-adjoint

def validator():
  e = lambda i: [0] * i + [1] + [0] * (n-i-1)
  for i in range(0,4):
    for j in range(i,4):   
      if L(e(i))[j] != L(e(j))[i]:
        return False
  return True
    \end{python}
  \end{solution}

Fantastic!

\end{question}

\end{document}
