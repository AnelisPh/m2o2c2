\documentclass{ximera}
\title{Spectrum of the adjoint}

\begin{document}

\begin{abstract}
  Taking adjoints doesn't affect the spectrum.
\end{abstract}

The set of eigenvalues of a linear operator---what we call the
\textbf{spectrum} of the linear operator---is of fundamental
importance.  Taking adjoints is one way to build a new linear operator
from an old linear operator.  Fusing these two ideas together results
in a question: \textit{how does the spectrum of $L$ relate to the
  spectrum of its adjoint, $L^*$?}  Surprisingly, the spectrum of $L$
is the \textit{same} as the spectrum of $L^*$.
	
Let's get started: show that if $\vec{v}$ is a nonzero eigenvector of
$L:\R^n \to \R^n$, with eigenvalue $\lambda$ then there is a nonzero
eigenvector $\vec{u}$ of $L^*$ with eigenvalue $\lambda$.

\begin{warning}
  This is a very hard problem.
\end{warning}

\begin{hint}
  Consider the map $S: \R^n \to \R^n$ given by $S(\vec{v}) = L^*(\vec{v}) - \lambda \vec{v}$, or in other words $S = L^* - \lambda I$.  Showing that this map has a nontrivial kernel is the same as
  showing that $L^*$ has $\lambda$ as an eigenvector.
\end{hint}
\begin{hint}
  Notice that $L-\lambda I$ is adjoint to $S = L^* -\lambda I$
\end{hint}
\begin{hint}
  For all $\vec{w} \in \R^n$, we have $\langle S(\vec{w}),\vec{v}\rangle = \langle \vec{w}, L(\vec{v}) - \lambda \vec{v}\rangle = 0$
\end{hint}
\begin{hint}
  Thus $S(\vec{w})$ is in the subspace of vectors perpendicular to the eigenvector $\vec{v}$, which we denote $\vec{v}^\perp$.
\end{hint}
\begin{hint}
  Thus we have that $Im(S) \subset \vec{v}^\perp$.  This implies that $dim(Im(S)) \leq n-1$
\end{hint}
\begin{hint}
  By the rank nullity theorem, we have that $dim(ker(S)) \geq 1$
\end{hint}
\begin{hint}
  So $S$ has a nontrivial kernel, so $L^*$ has a nonzero eigenvector $\vec{u}$ with eigenvalue $\lambda$. 
\end{hint}
\begin{free-response}
  Consider the map $S: \R^n \to \R^n$ given by $S(\vec{v}) = L^*(\vec{v}) - \lambda \vec{v}$.  Showing that this map has a nontrivial kernel is the same as
  showing that $L^*$ has $\lambda$ as an eigenvector.
  
  Notice that $L-\lambda I$ is adjoint to $S = L^* -\lambda I$
  
  For all $\vec{w} \in \R^n$, we have $\langle S(\vec{w}),\vec{v}\rangle = \langle \vec{w}, L(\vec{v}) - \lambda \vec{v}\rangle = 0$
  
  Thus $S(\vec{w})$ is in the subspace of vectors perpendicular to the eigenvector $\vec{v}$, which we denote $\vec{v}^\perp$.
  
  Thus we have that $Im(S) \subset \vec{v}^\perp$.  This implies that $dim(Im(S)) \leq n-1$
  
  By the rank nullity theorem, we have that $dim(ker(S)) \geq 1$
  
  So $S$ has a nontrivial kernel, so $L^*$ has a nonzero eigenvector $\vec{u}$ with eigenvalue $\lambda$. 
\end{free-response}
		
		
	
\end{document}
