\documentclass{ximera}
\title{Python}

\begin{document}

\begin{abstract}
  Build some bilinear maps in Python.
\end{abstract}	

\begin{question}
  Suppose $\vec{v}$ and $\vec{w}$ are both vectors in $\R^4$, represented in Python as two lists of four real numbers called \texttt{v} and \texttt{w}.  Build a Python function \texttt{B} which represents some bilinear form $B : \mathbb{R}^4 \times \mathbb{R}^4 \to \R$.

  \begin{solution}
    \begin{hint}
      For example, you could try returning \texttt{17 * v[0] * w[3]}.
    \end{hint}
    \begin{python}
def B(v,w):
  return # the real number B(v,w)

def validator():
  if B([4,2,3,4],[6,5,4,3]) + B([6,2,3,4],[6,5,4,3]) != B([10,4,6,8],[6,5,4,3]):
    return False

  if B([1,2,3,4],[6,5,4,3]) + B([1,2,3,4],[6,3,4,3]) != B([1,2,3,4],[12,8,8,6]):
    return False

  if 2*B([1,2,3,4],[6,5,4,3]) != B([2,4,6,8],[6,5,4,3]):
    return False

  if 2*B([1,2,3,4],[6,5,4,3]) != B([1,2,3,4],[12,10,8,6]):
    return False

  return True
    \end{python}
  \end{solution}

Now let's write a Python function \texttt{tensor} which takes two covectors $\alpha$ and $\beta$, and returns their tensor product $\alpha \otimes \beta$.  

  \begin{solution}
    \begin{hint}
      The returned function should take two parameters (say \texttt{v} and \texttt{w}) and output $\alpha(\vec{v}) \cdot \beta(\vec{w})$.
    \end{hint}
    \begin{hint}
      Specifically, you could try \texttt{return lambda v,w: alpha(v) * beta(w)}
    \end{hint}
    \begin{python}
def tensor(alpha,beta):
  return # the bilinear form alpha tensor beta

def validator():
  return tensor(lambda x: 4*x[0] + 5*x[1], lambda y: 2*y[0] - 3*y[1])([1,3],[4,5]) == -133
    \end{python}
  \end{solution}

\end{question}

\end{document}