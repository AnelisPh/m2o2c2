
\documentclass{ximera}
\title{Duality}
\begin{document}
\section{Linear maps associated with bilinear forms}
	
	It turns out that we will be able to use the inner product on $\R^n$ to rewrite any bilinear form on $\R^n$ in a special form. 
	
	Given a bilinear map $B:V \times W \to \R$, we obtain a new map $B(\cdot, w):V \to \R$ for each vector $w \in W$.   $B(\cdot, w)$ is linear, since 
	by definition of bilinearity it is linear in the first slot for a fixed vector $\vec{w}$ in the second slot.  Thus we have a map $\textrm{Curry}(B): W \to V^*$ defined by
	$\textrm{Curry}(B)(w) = B(\cdot, w)$.
	
	If $V$ and $W$ are Euclidean spaces, then we have that every bilinear map $\R^n \times \R^m \to \R$ gives rise to a map $\R^m \to (\R^n)^*$.  But every element
	of $ \omega \in (\R^n)^*$ is just a row vector, and so can be represented as the dot product against the vector  $\omega^\top$.  
	Thus we obtain a map $L_B:\R^m \to \R^n$ defined by $L_B(w) = B(\cdot,w)^\top$. 
	This is called the \textbf{linear map associated to the bilinear form}. 
	We also call the matrix of $L_B$ the \textbf{matrix of $B$}.
	
	Hopefully computing some examples will make these definitions less abstract.
	
	\begin{question}
		Let  $B:\R^2 \times \R^3 \to R$ be a bilinear mapping, and you know the following values of $B$:
			\begin{itemize}
				\item $B\left(\verticalvector{1\\0},\verticalvector{1\\0\\0}\right) = 2$
				\item $B\left(\verticalvector{1\\0},\verticalvector{0\\1\\0}\right) = 1$
				\item $B\left(\verticalvector{1\\0},\verticalvector{0\\0\\1}\right) = -3$
				\item $B\left(\verticalvector{0\\1},\verticalvector{1\\0\\0}\right) = 3$
				\item $B\left(\verticalvector{0\\1},\verticalvector{0\\1\\0}\right) = 5$
				\item $B\left(\verticalvector{0\\1},\verticalvector{0\\0\\1}\right) = 4$ 
			\end{itemize}
			
			What is the matrix of $L_B$?
			
			\begin{solution}
				\begin{hint}
					$L_B:\R^3 \to \R^2$
				\end{hint}
				\begin{hint}
					$L_B(e_1) = B(\cdot, e_1)^\top$
				\end{hint}
				\begin{hint}
					To fin the matrix of $B(\cdot,e_1):\R^2 \to \R$, we need to see its effect on basis vectors.
					\begin{align*}
						B(e_1,e_1) &= 2\\
						B(e_2,e_1) &=3
					\end{align*}
					
					so the matrix of $B(\cdot,e_1)$ is \(\begin{bmatrix} 2 &3\end{bmatrix}\)
				\end{hint}
				\begin{hint}
					Thus $L_B(e_1) = B(\cdot, e_1)^\top = \verticalvector{2\\3}$
				\end{hint}
				\begin{hint}
					Similarly, $L_B(e_2) = \verticalvector{1\\5}$ and $L_B(e_3)=\verticalvector{-3\\4}$
				\end{hint}
				\begin{hint}
					Thus the matrix of $L_B$ is \(\begin{bmatrix} 2&1&-3\\3&5&4\end{bmatrix}\)
				\end{hint}
				\begin{matrix-answer}
					correctMatrix = [['2','1','-3'],['3','5','4']]
				\end{matrix-answer}
			\end{solution}
	\end{question}
	
	\begin{question}
		If $B: \R^3 \times \R^3 \to \R$ is a bilinear map, and the matrix of $B$ is 
		\(\begin{bmatrix} 2 & 3 & 1 \\ -2  & 1& 5\\ 3&2 & 1\end{bmatrix}\)
		\begin{solution}
			\begin{hint}
				By definition, $B\left( \verticalvector{1\\2\\0}, \verticalvector{0\\0\\1}\right) = \verticalvector{1\\2\\0} \cdot L_B(\verticalvector{0\\0\\1})$
			\end{hint}
			\begin{hint}
				Thus $B\left( \verticalvector{1\\2\\0}, \verticalvector{0\\0\\1}\right) = \verticalvector{1\\2\\0} \cdot \verticalvector{1\\5\\1} = 11$
			\end{hint}
			Then $B\left( \verticalvector{1\\2\\0}, \verticalvector{0\\0\\1}\right) =$ \answer{'11'}
		\end{solution}
	\end{question}
	
	Show that the matrix of the bilinear form $\sum a_{i,j} dx_i \otimes dx_j$ is the matrix $[a_{i,j}]$.
	
	\begin{free-response}
		Let $M$ be the matrix of $L_B$.
		
		Following the same line of reasoning as in \href{http://ximera.osu.edu/course/kisonecat/m2o2c2/course/activity/week1/inner-product/multiply-dot/}{this activity}, we 
		know that $M_{i,j} = e_i^\top L_B(e_j)$.  But by definition, this is 	$B(e_i,e_j)$, which plainly evaluates to $a_{i,j}$.   The claim is proven.
	\end{free-response}
	
	To every linear map  $L:\R^m \to \R^n$ we also obtain a bilinear map $B_L : \R^n \times \R^m \to \R$ defined by $B_L(v,w) = v^\top L(w)$.
	
	To summarize, we have a really nice story about bilinear maps $B:\R^n \times \R^m \to \R$:  Every single one of them can be written as $B(v,w) = v^\top L(w)$ for some 
	unique linear map $L:\R^m \to \R^n$. Also every linear map $\R^m \to \R^n$ gives rise to a bilinear form by defining $B(v,w) = v^\top L(w)$.  On the level of matrices,
	we just have that $B(v,w) = v^\top M w $ where $M$ is the matrix of the linear map $L_B$.  We will sometimes say talk about "using a matrix as a bilinear form":  this is what we
	mean by that.  This will be very important to us when we start talking about the second derivative.
	
	In this activity we have shown that for bilinear maps $\R^n \times \R^m \to \R$, there is a useful notion of a linear map $\R^m \to \R^n$ associated to it.  
	If the codomain of the original bilinear map had been anything other than $\R$ we would not have such luck: our work depended crucially on
	the ability to turn covectors into vectors using the inner product on $\R^n$.
	
	

\end{document}