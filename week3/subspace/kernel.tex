\documentclass{ximera}
\title{Kernel}

\begin{document}
\begin{abstract}
	A kernel is everything sent to zero.
\end{abstract}

There are some special subspaces that we will want to pay attention to.

\begin{theorem}
  If $L:V \to W$ is a linear transformation, then the \textbf{kernel} of $L$, defined by $$\ker(L) = \{\vec{v} \in V:L(\vec{v}) = \vec{0}\}$$ is a subspace of $V$.
\end{theorem}

You may also hear this referred to as the \textbf{null space} of $L$.

Prove this theorem that $\ker L$ is a subspace.

\begin{free-response}
We only need to show that $ker(L)$ is closed under scalar multiplication and vector addition.

For any $\vec{v} \in ker(L)$ and $c \in \R$,

\begin{align*}
	L(c\vec{v}) &=cL(\vec{v})\\
		&=c\vec{0}\\
		&=0
\end{align*}

so $c\vec{v} \in ker(L)$.

If $\vec{v},\vec{w} \in ker(L)$, then

\begin{align*}
	L(\vec{v}+\vec{w}) &= L(\vec{v})+L(\vec{w})\\
	&= \vec{0}+\vec{0}\\
	&=\vec{0}
\end{align*}

so $\vec{v}+\vec{w} \in ker(L)$

Thus $ker(L)$ is a subspace!

\end{free-response}

\begin{question}
	Let $L:\R^3 \to \R^2$ be the linear map whose matrix is 
	\( \begin{bmatrix} 
		2 & 3 & \phantom{-}1\\
		1 & 0 & -1
	\end{bmatrix}\)
	\begin{solution}
		\begin{hint}
			Just by evaluating all three, we see the only one which gets sent to $\verticalvector{0\\0}$ by $L$ is $\verticalvector{1\\-1\\1}$
		\end{hint}
		Which of the following vectors is in the kernel of $L$?
		\begin{multiple-choice}
			\choice[correct]{$\verticalvector{1\\-1\\1}$}
			\choice{$\verticalvector{3\\2\\0}$}
			\choice{$\verticalvector{0\\0\\2}$}
		\end{multiple-choice}
	\end{solution}
\end{question}

\begin{theorem}
  A linear map $L:V \to W$ is injective if and only if $\ker(L) = \{\vec{0}\}$.
\end{theorem}

\begin{definition}
  The word ``injective'' is an adjective meaning the same thing as ``one to one.''  In other words, a function $f:A \to B$ is injective if $f(a_1)=f(a_2)$ implies $a_1=a_2$.  
\end{definition}

Prove this theorem.
	
\begin{free-response}
	Let $L$ be injective.  Then $L(\vec{v}) = \vec{0}$ implies $L(\vec{v}) = L(\vec{0})$.  Since $L$ is injective, this implies $\vec{v} = \vec{0}$.  Thus the only element
	of the kernel is $\vec{0}$.
	
	On the other hand, if $ker(L) = \{\vec{0}\}$, then if $L(\vec{v_1}) = L(\vec{v_2})$, then $L(\vec{v_1}-\vec{v_2})= \vec{0}$, so $\vec{v_1}-\vec{v_2}$
	 is in the null space, and hence must be equal to $\vec{0}$.  But then we can conclude that $\vec{v_1} = \vec{v_2}$
	\end{free-response}

\begin{definition}
  The dimension of the kernel of $L$ is the \textbf{nullity} of $L$.        
\end{definition}

Be careful to observe that $\ker L$ is a subspace, while $\dim \ker L$
is a number, so the nullity of $L$ is just a number.

\end{document}
