\documentclass{ximera}
\title{Image}

\begin{document}

\begin{abstract}
  The ``image'' is every actual output.
\end{abstract}

\begin{definition}
  If $L:V \to W$ is a linear transformation, then the \textbf{image}
  of $L$ is $$\text{Imag}(L) = \{\vec{w} \in W: \text{ there exists
    $\vec{v} \in V $ with }L(\vec{v}) = \vec{w}\}$$.
\end{definition}

(Some people may call this the ``range.''  Some other people use the
word ``range'' for what we've been calling the codomain.  The result
is that, in my opinion, the word ``range'' is now overused, so we give
up and never use the word.)

\begin{question}
  Suppose $L : \R^2 \to \R^3$, and suppose that
  \begin{align*}
    L\left(\begin{bmatrix} 1 \\ 0 \end{bmatrix}\right) &= \begin{bmatrix} 3 \\ 2 \\ 1 \end{bmatrix} \\
    L\left(\begin{bmatrix} 0 \\ 1 \end{bmatrix}\right) &= \begin{bmatrix} 1 \\ 1 \\ 1 \end{bmatrix}
  \end{align*}

  What is a vector $\vec{v} \in \R^2$ so that $L(\vec{v}) = \begin{bmatrix} 2 \\ 1 \\ 0 \end{bmatrix}$?

  \begin{solution}
    \begin{hint}
      Use the fact that $\begin{bmatrix} 2 \\ 1 \\ 0 \end{bmatrix} = \begin{bmatrix} 3 \\ 2 \\ 1 \end{bmatrix} - \begin{bmatrix} 1 \\ 1 \\ 1 \end{bmatrix}$.
    \end{hint}

    \begin{hint}
      In other words, $\begin{bmatrix} 2 \\ 1 \\ 0 \end{bmatrix} = L(\vec{e}_1) - L(\vec{e}_2)$.
    \end{hint}

    \begin{hint}
      By linearity of $L$, we have $L(\vec{e}_1) - L(\vec{e}_2) = L(\vec{e}_1 - \vec{e}_2)$.
    \end{hint}

    \begin{hint}
      And so a vector in the domain which is sent to $\begin{bmatrix} 2 \\ 1 \\ 0 \end{bmatrix}$ is the vector $\begin{bmatrix} 1 \\ -1 \end{bmatrix}$.
    \end{hint}

    \begin{matrix-answer}[name=v]
      correctMatrix = [['1'],['-1']]
    \end{matrix-answer}
  \end{solution}

  This is a special case of a general fact: if we have two vectors in the image, then their sum is in the image, too.

  \begin{theorem}
    The image of a linear map is a subspace of the codomain.
  \end{theorem}

  Prove this.

  \begin{free-response}
    If $w \in \text{Imag}(L)$, then there is a $v \in V$ with $L(v) =w$.  $L(cv) = cL(v)=cw$, so $cw \in \text{Imag}(L)$ for any $c\in \R$. 
    Thus $\text{Imag}(L)$ is closed under scalar multiplication.
    
    If $w_1$, $w_2 \in \text{Imag}(L)$, then there are $v_1,v_2 \in V$ with $L(v_1)=w_1$ and $L(v_2)=w_2$.  $L(v_1+v_2)=L(v_1)+L(v_2)=w_1+w_2$, so $w_1+w_2 \in \text{Imag}(L)$.
    Thus $\text{Imag}(L)$ is closed under vector addition.
  \end{free-response}

\end{question}

\hrule

We finish with some terminology.

\begin{definition}
  The dimension of the image of $L$ is the \textbf{rank} of $L$.
\end{definition}

Be careful to observe that the image of $L$ is a subspace, while the dimension of the image of $L$
is a number, so the rank of $L$ is just a number.

\begin{question}
  Consider the linear map $L : \R^2 \to \R^3$ given by the matrix
  $$
  \begin{bmatrix}
    2 & 1 \\
    4 & 2 \\
    6 & 3
  \end{bmatrix}.
  $$
  
  \begin{solution}
    \begin{hint}
      Not every vector in $\R^3$ is the image of $L$.
    \end{hint}

    \begin{hint}
      Let's think about which vectors are in the image of $L$.

      \begin{question}
        Is $\begin{bmatrix} 2 \\ 4 \\ 6 \end{bmatrix}$ in the image of $L$?

        \begin{solution}
          \begin{multiple-choice}
            \choice[correct]{Yes.}
            \choice{No.}
          \end{multiple-choice}
        \end{solution}

        In fact, $\begin{bmatrix} 2 \\ 4 \\ 6 \end{bmatrix} = L \left( \begin{bmatrix} 1 \\ 0 \end{bmatrix} \right)$.

        Is $\begin{bmatrix} 1 \\ 1 \\ 1 \end{bmatrix}$ in the image of $L$?

        \begin{solution}
          \begin{multiple-choice}
            \choice{Yes.}
            \choice[correct]{No.}
          \end{multiple-choice}
        \end{solution}

        But how can we tell?  The only things in the image of $L$ are
        vectors of the form $\begin{bmatrix} x \\ 2x \\
          3x \end{bmatrix}$ for some $x \in \R$.  This is the span of $\left(\begin{bmatrix} 1 \\ 2 \\
          3 \end{bmatrix}\right)$.

        So what is the dimension of the vector space spanned by this single vector?
        \begin{solution}
          \begin{multiple-choice}
            \choice{$0$}
            \choice[correct]{$1$}
            \choice{$2$}
          \end{multiple-choice}
        \end{solution}        
        
        And so the rank is one.
      \end{question}
    \end{hint}
    
    The rank of $L$ is \answer{$1$}.
  \end{solution}
\end{question}

\end{document}