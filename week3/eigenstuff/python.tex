\documentclass{ximera}

\title{Python}

\begin{document}

\begin{abstract}
  We can find eigenvectors in Python.
\end{abstract}

Let's suppose I have an $n \times n$ matrix $M$, expressed in Python as a list of lists.  For example, suppose
$$
M = \begin{bmatrix}
6 & 4 & 3 \\
4 & 5 & 2 \\
3 & 2 & 7 \\
\end{bmatrix} = \text{[[6,4,3],[4,5,2],[3,2,7]]}.
$$
Further suppose that the matrix $M = (m_{ij})$ is \textbf{symmetric},
meaning that $m_{ij} = m_{ji}$.  I'd like to compute an eigenvector of
$M$ quickly.

\begin{question}
  Here's a procedure that I'd like you to code in Python:
  \begin{enumerate}
  \item Start with some vector $\vec{v}$.
  \item Replace $\vec{v}$ with the result of applying the linear map $L_M$ to the vector $\vec{v}$.
  \item Normalize $\vec{v}$ so that it has unit length.
  \item Repeat many times.
  \end{enumerate}

  You can try \texttt{print eigenvector([[6,4,3],[4,5,2],[3,2,7]])} to
  see what happens in the case of the matrix above.

  \begin{solution}
    \begin{python}
def eigenvector(M):
  # start with a random vector v
  v = [1] * len(M[0])
  # for many, many times
  #   replace v with Mv
  #   normalize v
  # return v 

def validator():
  v = eigenvector([[6, 5, 5], [5, 2, 3],[5, 3, 8]])
  if abs((v[1] / v[0]) - 0.6514182851) > 0.01:
    return False
  if abs((v[2] / v[0]) - 1.0603152077) > 0.01:
    return False
  return True
    \end{python}
  \end{solution}

Can you use your program to find, numerically, an eigenvector of the matrix $M$?

\end{question}

\end{document}

%%% Local Variables: 
%%% mode: latex
%%% TeX-master: t
%%% End: 
