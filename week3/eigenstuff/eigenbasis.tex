\documentclass{ximera}

\title{Eigenbasis}

\begin{document}

\begin{abstract}
  An eigenbasis is a basis of eigenvectors.
\end{abstract}

\begin{observation}
  If $(v_1,v_2,...,v_n)$ is a basis of eigenvectors of a linear
  operator $L$, then the matrix of $L$ with respect to that basis is
  diagonal, with the eigenvalues of $L$ appearing along the diagonal.
\end{observation}

\begin{theorem}
  Let $L:V \to W$ be a linear map. If $v_1,v_2,...,v_n$ are nonzero eigenvectors of $L$ with distinct eigenvalues $\lambda_1,\lambda_2,...\lambda_n$, then $(v_1,v_2,...,v_n)$
  are linearly independent.
\end{theorem}

Prove this theorem.

\begin{free-response}
  Assume to the contrary that the list is linearly dependent.  Let $v_k$ be the first vector in the list which is in the span of the preceding vectors, 
  so that the vectors $(v_1,v_2,...,v_{k-1})$  are linearly dependent.  
  Let $a_1v_1+a_2v_2+...+a_{k-1}v_{j-1} = v_k$.  Then applying $L$ to both sides of this equation we have
  $a_1\lambda_1 v_1+a_2\lambda_2 v_2+ ...+a_{k-1}\lambda_{k-1} v_{k-1} = \lambda_k v_k$.  If we multiply the first equation by $\lambda_k$ we also have
  $a_1\lambda_1v_1+a_2\lambda_1v_2+...+a_{k-1}\lambda_1v_{j-1} = \lambda_1v_k$.  Subtracting these two equations we have
  
  \(
  a_1(\lambda_k-\lambda_1)v_1+a_2(\lambda_k-\lambda_2)v_2+...+a_3(\lambda_k-\lambda_{k-1})v_k=0.
  \)
  
  Since the vectors   $(v_1,v_2,...,v_{k-1})$  are linearly dependent, we must have that $a_i(\lambda_k - \lambda_i) = 0$.  But $\lambda_k \neq \lambda_i$, so 
  $a_i=0$ for each $i$.  Looking back at where the $a_i$ came from, we see that this implies that $v_k=0$.  This contradicts the assumption that the $v_j$ were all nonzero.
  
  So our assumption that the list was linearly dependent was absurd, hence the list is linearly dependent.
\end{free-response}

A corollary of this theorem is that if $V$ is $n$ dimensional and $L: V \to V$ has $n$ distinct eigenvalues, then the eigenvectors of $L$ form a basis of $V$.
The matrix of the operator with respect to this basis is diagonal.
	

\end{document}