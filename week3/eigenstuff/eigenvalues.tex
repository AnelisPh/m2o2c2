\documentclass{ximera}

\title{Eigenvectors}

\begin{document}

\begin{abstract}
  Eigenvectors are mapped to multiples of themselves.
\end{abstract}

\begin{definition}
  Let $L : V \to V$ be a linear operator (NB: linear maps with the
  same domain and codomain are called linear \textit{operators}).  The
  set of all eigenvalues of $L$ is the \textbf{spectrum} of $L$.
\end{definition}

Let's try finding the spectrum.

\begin{question}
  Let $L :\R^2 \to \R^2$ be the linear map whose matrix is
  \(\begin{bmatrix} 1 & 2 \\2& 1\end{bmatrix}\) with respect to the
  standard basis.  $L$ has two different eigenvalues.  What are they?
  Give your answer in the form of a matrix $\verticalvector{\lambda_1 \\
    \lambda_2}$, where $\lambda_1 \leq \lambda_2$.
	
  \begin{solution}
    \begin{hint}
      For $lambda$ to be an eigenvalue we need
      \(\begin{bmatrix} 1 & 2 \\2& 1\end{bmatrix} \verticalvector{x\\y} = \lambda \verticalvector{x\\y}\)
    \end{hint}
    \begin{hint}
      This is the same as \(\begin{cases}
        x+2y = \lambda x \\
        2x+y =\lambda y
      \end{cases}\)
      
      or
      
      \(\begin{cases}
        (1-\lambda)x+2y = 0 \\
        2x+(1-\lambda)y =0
      \end{cases}\)
      
    \end{hint}
    \begin{hint}
      These are two lines passing through the origin.  To have more than just the origin as a solution, we need that the slope of the two lines is the same.  So
      
      \(
      \frac{1-\lambda}{2} = \frac{2}{1-\lambda}
      \)
    \end{hint}
    \begin{hint}
      \begin{align*}
        \frac{1-\lambda}{2} &= \frac{2}{1-\lambda}\\
        (1-\lambda)^2 &= 4\\
        1-\lambda &= \pm 2\\
        lambda &= -1 \text{ or } 3
      \end{align*}
    \end{hint}
    \begin{hint}
      Let us now check that these really are eigenvalues:
      
      If we let $\lambda = -1$, we have the equation $2x+2y=0$.  Check that $\verticalvector{1\\-1}$ is an eigenvector with eigenvalue $-1$
      
      If we let $\lambda = 3$, we have the equation $2x-2y=0$.  Check that $\verticalvector{1\\1}$ is an eigenvector with eigenvalue $3$
      
    \end{hint}
    
    \begin{matrix-answer}
      correctMatrix =[['-1'],['3']]
    \end{matrix-answer}
  \end{solution}
\end{question}

\begin{question}
  Let's try another example.  Suppose $F : \R^2 \to \R^2$ is the linear map represented by the matrix
  $$
  \begin{bmatrix}
    0 & -1 \\
    1 & 0
  \end{bmatrix}.
  $$

  Which of these numbers is an eigenvalue of $F$?
  \begin{solution}
    \begin{hint}
      Let's suppose that $\begin{bmatrix} x \\ y \end{bmatrix}$ is an eigenvector.
    \end{hint}

    \begin{hint}
      Then there is some $\lambda \in \R$ so that $\begin{bmatrix}
    0 & -1 \\
    1 & 0
  \end{bmatrix} \begin{bmatrix} x \\ y \end{bmatrix} = \lambda \begin{bmatrix} x \\ y \end{bmatrix}$.
    \end{hint}

    \begin{hint}
      But $\begin{bmatrix}
    0 & -1 \\
    1 & 0
  \end{bmatrix} \begin{bmatrix} x \\ y \end{bmatrix} = \begin{bmatrix} -y \\ x \end{bmatrix}$.
    \end{hint}

    \begin{hint}
      And so $\begin{bmatrix} -y \\ x \end{bmatrix} = \lambda \begin{bmatrix} x \\ y \end{bmatrix}$.
    \end{hint}

    \begin{hint}
      This means that $-y = \lambda x$ and $x = \lambda y$.
    \end{hint}

    \begin{hint}
      Putting this together, $-y = \lambda^2 y$ and $x = -\lambda^2 x$.
    \end{hint}

    \begin{hint}
      Since we are looking for a nonzero eigenvector (in order to have an eigenvalue), we must have that either $x \neq 0$ or $y \neq 0$.
    \end{hint}

    \begin{hint}
      Consequently, $\lambda^2 = -1$.
    \end{hint}

    \begin{hint}
      But there is no real number $\lambda \in \R$ so that $\lambda^2 = -1$, since the square of any real number is nonnegative.
    \end{hint}

    \begin{hint}
      Therefore, there is no real eigenvalue.
    \end{hint}

    \begin{multiple-choice}
      \choice[correct]{There is no real eigenvalue.}
      \choice{$-1$}
      \choice{$\sqrt{2}$}
      \choice{$1$}
    \end{multiple-choice}
  \end{solution}

  Perhaps surprisingly, not every linear operator from $\R^n$ to $\R^n$ has \textit{any} real eigenvalues.

  Geometrically, what is this linear map $F$ doing?
  \begin{solution}
    \begin{multiple-choice}
      \choice[correct]{Rotation by $90^\circ$ counterclockwise.}
      \choice{Rotation by $90^\circ$ clockwise.}
      \choice{Rotation by $180^\circ$.}
    \end{multiple-choice}
  \end{solution}
    
  This geometric fact also explains why there is no eigenvalue: what
  would be the corresponding eigenvector whose direction is unchanged
  by applying $F$?  Every vector is moved by a rotation!

  The additional fact that there are imaginary solutions to $\lambda^2
  = -1$ is hinting that $i$ should have something to do with rotation,
  too.
\end{question}

\end{document}