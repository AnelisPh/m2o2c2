\documentclass{ximera}

\title{Eigenspace}

\begin{document}

\begin{abstract}
  An eigenspace collects together all the eigenvectors for a given eigenvalue.
\end{abstract}

\begin{theorem}
  Let $\lambda$ be an eigenvalue of a linear operator $L:V\to V$.
  Then the set $E_\lambda(L) = \{ v \in V: L(v) = \lambda v\}$ of all
  (including zero) eigenvectors with eigenvalue $\lambda$ forms a
  subspace of $V$.

  This subspace is the \textbf{eigenspace} associated to the
  eigenvalue $\lambda$.
\end{theorem}

Prove this theorem.
\begin{free-response}
	We need to check that $E_\lambda(L)$ is closed under scalar multiplication and vector addition
	
	If $v \in E_\lambda(L)$, and $c \in \R$, then $L(cv) = cL(v) = c\lambda v = \lambda( cv)$, so $cv$ is also an eigenvector of $L$.
	
	If $v_1,v_2 \in E_\lambda(L)$, then $L(v_1+v_2) = \lambda v_1+\lambda v_2 = \lambda( v_1 + v_2)$, so $v_1+v_2$ is also an eigenvector of $L$.
\end{free-response}

The kernel of $L$ is the eigenspace of the eigenvalue $0$.




\end{document}
