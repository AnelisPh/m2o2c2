\documentclass{ximera}
\title{Matrix of a linear map}

\begin{document}
\begin{abstract}
	Matrices record where basis vectors go
\end{abstract}

In our first brush with linear algebra, we only dealt with linear maps between the spaces spaces $\R^n$.  For those spaces, the convenient 
standard basis allowed us to record linear maps using the finite data of a matrix.  In this section we will see that a similar story plays out for
maps between finite dimensional vector spaces:  they too can be described by matrices with respect to a choice of basis on the codomain and codomain.

\begin{question}
	Let $V$ be a vector space with basis $\vec{v_1},\vec{v_2},\vec{v_3}$ and $W$ be a vector space with basis $\vec{w_1},\vec{w_2}$.
	Moreover, suppose there is a linear map $L: V \to W$ for which $L(\vec{v_1}) = 3\vec{w_1} + 2\vec{w_3}$, 
	$L(\vec{v_2}) = 3\vec{w_1} - 2\vec{w_3}$, and $L(\vec{v_3}) = \vec{w_1} + \vec{w_3}$.
	
	What is $L(2\vec{v_1}+3\vec{v_2}-4\vec{v_3})$?
\end{question}

\begin{question}
	Show that if $L:V \to W$ is a linear map, and you know the value of $L(v_i)$ for some basis $(\vec{v_1},\vec{v_2},...,\vec{v_n})$ of $V$, 
	then you can compute $L(\vec{v})$ for any $\vec{v} \in V$.
\end{question}

\begin{definition}
	Let $L:V \to W$ be a linear map between finite dimensional vector spaces,  let $\mathcal{B}_V  =(\vec{v_1},\vec{v_2},...,\vec{v_n}) $ be
	a basis for $V$, and let $\mathcal{B}_W = (\vec{w_1},\vec{w_2},...,\vec{w_m})$  be a basis for $W$.  
	Then $L(\vec{v_i}) = a_{i,1}\vec{w_1}+a_{i,2}\vec{w_2} +...+ a_{i,m}\vec{w_m}$.
	Then the \textit{matrix of $L$ with respect to the bases
	$\mathcal{B}_V and \mathcal{B}_V$} is the matrix $M$ whose entry in the $i^{th}$ column and $j^{th}$ row is $a_{i,j}$
\end{definition}

\begin{problem}
	Let $\mathcal{B}_1 = (\verticalvector{1,1},\verticalvector{1,0})$ and $\mathcal{B}_2 = (\verticalvector{1,0,0},\verticalvector{0,0,1},\verticalvector{0,1,0})$ be bases
	for $\R^2$ and $\R^3$ respectively. 
	\begin{solution}
	\begin{hint}
		The first column of the matrix will be the image of $\verticalvector{1\\1}$ expressed in the new basis.  Note that the order of the basis matters!
	\end{hint}
	\begin{hint}
		\begin{align*}
			L\left(\verticalvector{1\\1}\right) &= \verticalvector{2\\1\\0}\\
				&= 2\verticalvector{1\\0\\0}+0\verticalvector{0\\0\\1}+1\verticalvector{0\\1\\0}\\
		\end{align*}
	\end{hint}
	\begin{hint}
		So the first column of the matrix is $\verticalvector{2\\0\\1}$.
	\end{hint}
	\begin{hint}
		Similarly, the second column is $\verticalvector{1\\0\\1}$
	\end{hint}
	\begin{hint}
		So the matrix of this linear map is \(\begin{bmatrix} 2 & 1 \\0& 0\\ 1 & 1\end{bmatrix}\)
	\end{hint}
	 What is the matrix for the linear map $L(\verticalvector{x,y} = \verticalvector{x+y,x,0})$ with respect to the 
	bases $\mathcal{B}_1$  and $\mathcal{B}_2$?
	\begin{matrix-answer}
		correctMatrix = [['2','1'],['0','0'],['1','1']]
	\end{matrix-answer}
	\end{solution}
\end{problem}

\begin{problem}
	Let $P_2$ be the space of polynomials of degree at most $2$.  Let $\mathcal{B}_0 = (1,x,x^2)$ and $\mathcal{B}_1 = (1,(x-1),(x-1)^2)$.  
	Consider the map $L:P_2 \to \R$ given by $L(p) = p(1)$.  
	\begin{solution}
	\begin{hint}
		\begin{align*}
			L(1) &= 1\\
			L(x) &= 1\\
			L(x^2)&=1^2=1
		\end{align*}
	\end{hint}
	\begin{hint}
	\begin{bmatrix}
		The matrix of $L$ with respect to $\mathcal{B}_0$ is \( \begin{bmatrix} 1&1&1 \end{bmatrix}\)
	\end{bmatrix}
	\end{hint}
	What is the matrix of this linear map with respect to the basis $\mathcal{B}_0$?  
	\begin{matrix-answer}
		correctMatrix = [['1','1','1']]
	\end{matrix-answer}
	\end{solution}
		
	\begin{solution}
	\begin{hint}
		\begin{align*}
			L(1) &= 1\\
			L(x-1) &= 1-1=0\\
			L((x-1)^2)&=(1-1)^2=0
		\end{align*}
	\end{hint}
	\begin{hint}
	\begin{bmatrix}
		The matrix of $L$ with respect to $\mathcal{B}_1$ is \( \begin{bmatrix} 1&0&0 \end{bmatrix}\)
	\end{bmatrix}
	\end{hint}
	What is the matrix of this linear map with respect to the basis $\mathcal{B}_1$?  
	
	\begin{matrix-answer}
		correctMatrix = [['1','0','0']]
	\end{matrix-answer}
	\end{solution}
\end{problem}

\begin{problem}
	Let $P_3$ be the space of polynomials of degree at most $3$.  Let $\mathcal{B} = (1,x,x^2,x^3)$.  Consider the map $L:P_3 \to P_3$ given by 
	$L(p(x)) = \frac{d}{dx} p(x)$. This map is linear (why?).  
	\begin{solution}
		\begin{hint}
			$L$ is linear because the derivative of a sum of two functions is the sum of the derivative of the two functions, and since the derivative of a constant
			times a function is the constant times the derivative of the function.
		\end{hint}
		\begin{hint}
			\begin{align*}
				L(1) &=0\\
				L(x) &=1\\
				L(x^2)&=2x\\
				L(x^3)&=3x^2
			\end{align*}
		\end{hint}
		\begin{hint}
			The matrix of $L$ is \(\begin{bmatrix} 0&1&0&0\\0&0&2&0\\0&0&0&3\\0&0&0&0\end{bmatrix}\)
		\end{hint}
	What is the matrix for $L$ with respect to the basis $\mathcal{B}$?
	\begin{matrix-answer}
		correctMatrix = [['0','1','0','0'],['0','0','2','0'],['0','0','0','3'],['0','0','0','0']]
	\end{matrix-answer}
	\end{solution}
\end{problem}

\end{document}