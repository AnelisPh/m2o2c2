\documentclass{ximera}
\title{Matrix of a linear map}

\begin{document}
\begin{abstract}
  Matrices record where basis vectors go.
\end{abstract}

In our first brush with linear algebra, we only dealt with linear maps
between the spaces spaces $\R^n$ for varying $n$.  For those maps and
those spaces, the convenient standard basis allowed us to record
linear maps using the finite data of a matrix.

In this section we will see that a similar story plays out for maps
between finite dimensional vector spaces: they too can be described by
a matrix, \textbf{but only after making a choice} of ``basis'' on the
domain and codomain.

\begin{question}
  Let $V$ be a vector space with basis $(\vec{v}_1,\vec{v}_2,\vec{v}_3)$ and $W$ be a vector space with basis $(\vec{w}_1,\vec{w}_2)$.

  Suppose there is a linear map $L: V \to W$ for which
  
  \begin{align*}
    L(\vec{v}_1) &= 3\vec{w}_1 + 2\vec{w}_2, \\ 
    L(\vec{v}_2) &= 3\vec{w}_1 - 2\vec{w}_2, \text{ and} \\
    L(\vec{v}_3) &= \vec{w}_1 + \vec{w}_2.
  \end{align*}
  
  In light of all this, compute $L(2\vec{v}_1)$.  But how will we write down our answer?

  Where does $L(2\vec{v}_1)$ live?

  \begin{solution}
    \begin{multiple-choice}
      \choice[correct]{In $W$.}
      \choice{In $V$.}
    \end{multiple-choice}
  \end{solution}

    And since we know that $L(2\vec{v}_1) \in W$, we'll write our
    answer as $\alpha \vec{w}_1 + \beta \vec{w}_2$ for some numbers
    $\alpha$ and $\beta$.

    So say $L(2\vec{v}_1) = \alpha \vec{w}_1 + \beta \vec{w}_2$.  

    \begin{solution}
      In this case, $\alpha = $ \answer{$6$}.
    \end{solution}

    \begin{solution}
      And $\beta = $ \answer{$4$}.
    \end{solution}

    Next compute $L(2\vec{v}_1+3\vec{v}_2-4\vec{v}_3) = \alpha \vec{w}_1 + \beta \vec{w}_2$.
    \begin{solution}
      \begin{hint}
        Use the fact that $L(2\vec{v}_1+3\vec{v}_2-4\vec{v}_3) = L(2\vec{v}_1)+L(3\vec{v}_2)-L(4\vec{v}_3)$.
      \end{hint}

      \begin{hint}
        Further use the fact that $L(2\vec{v}_1)+L(3\vec{v}_2)-L(4\vec{v}_3) = 2 \, L(\vec{v}_1)+3\,L(\vec{v}_2)-4\,L(\vec{v}_3)$.
      \end{hint}

      In this case, $\alpha = 11$ but $\beta =$ \answer{$-6$}.
    \end{solution}

    What we are seeing is an instance of the following observation.

    \begin{observation}
      If $L:V \to W$ is a linear map, and you know the value of $L(v_i)$ for each vector in the basis $(\vec{v}_1,\vec{v}_2,\ldots,\vec{v}_n)$ of $V$, 
      then you can compute $L(\vec{v})$ for any $\vec{v} \in V$.

      And by ``compute,'' I mean you can write down $L(\vec{v})$ in terms of a basis of $W$.
    \end{observation}

\end{question}

\begin{definition}
  Let $L:V \to W$ be a linear map between finite dimensional vector
  spaces, let $\mathcal{B}_V =(\vec{v}_1 ,\vec{v}_2,\ldots,\vec{v}_n) $
  be a basis for $V$, and let $\mathcal{B}_W =
  (\vec{w}_1,\vec{w}_2,\ldots,\vec{w}_m)$ be a basis for $W$.  Then
  $L(\vec{v_i}) = a_{i,1}\vec{w}_1+a_{i,2}\vec{w}_2 + \cdots +
  a_{i,m}\vec{w}_m$.

  Then the \textit{matrix with respect to the bases} $\mathcal{B}_V$
  and $\mathcal{B}_V$ is the matrix $M$ whose entry in the $i^{th}$
  column and $j^{th}$ row is $a_{i,j}$
\end{definition}

\begin{problem}
  Let $\mathcal{B}_1 = \left(\verticalvector{1\\1}_{\mathcal{E}_2},\verticalvector{1\\0}_{\mathcal{E}_2}\right)$ and $\mathcal{B}_2 = \left(\verticalvector{1\\0\\0}_{\mathcal{E}_3},\verticalvector{0\\0\\1}_{\mathcal{E}_3},\verticalvector{0\\1\\0}_{\mathcal{E}_3}\right)$ be bases
  for $\R^2$ and $\R^3$, respectively. 

  \begin{solution}
    \begin{hint}
      The first column of the matrix will be $\verticalvector{1\\1}_{\mathcal{E}_2}$ but written with respect to the basis $\mathcal{B}_2$.

      Remember the \textit{order} of vectors in the basis matters.
    \end{hint}

    \begin{hint}
      \begin{align*}
        L\left(\verticalvector{1\\1}_{\mathcal{E}_2}\right) &= \verticalvector{2\\1\\0}_{\mathcal{E}_3}\\
        &= 2\verticalvector{1\\0\\0}_{\mathcal{E}_3}+0\verticalvector{0\\0\\1}_{\mathcal{E}_3}+1\verticalvector{0\\1\\0}_{\mathcal{E}_3}\\
      \end{align*}
    \end{hint}
    \begin{hint}
      So the first column of the matrix is $\verticalvector{2\\0\\1}_{\mathcal{B}_2}$.
    \end{hint}
    \begin{hint}
      Similarly, the second column is $\verticalvector{1\\0\\1}_{\mathcal{B}_2}$.
    \end{hint}
    \begin{hint}
      So the matrix of this linear map is \(\begin{bmatrix} 2 & 1 \\0& 0\\ 1 & 1\end{bmatrix}\)
    \end{hint}

    What is the matrix for the linear map $L\left(\verticalvector{x \\ y}_{\mathcal{E}_2}\right) = \verticalvector{x+y \\x \\0}_{\mathcal{E}_3}$ with respect to the 
    bases $\mathcal{B}_1$  and $\mathcal{B}_2$?

    \begin{matrix-answer}
      correctMatrix = [['2','1'],['0','0'],['1','1']]
    \end{matrix-answer}
  \end{solution}
\end{problem}

\begin{problem}
  Let $P_2$ be the space of polynomials of degree at most $2$.  Let $\mathcal{B}_0 = (1,x,x^2)$ and $\mathcal{B}_1 = (1,(x-1),(x-1)^2)$.  
  Consider the map $L:P_2 \to \R$ given by $L(p) = p(1)$.  
  \begin{solution}
    \begin{hint}
      \begin{align*}
        L(1) &= 1\\
        L(x) &= 1\\
        L(x^2)&=1^2=1
      \end{align*}
    \end{hint}
    \begin{hint}
      \begin{bmatrix}
        The matrix of $L$ with respect to $\mathcal{B}_0$ is \( \begin{bmatrix} 1&1&1 \end{bmatrix}\)
      \end{bmatrix}
    \end{hint}
    What is the matrix of this linear map with respect to the basis $\mathcal{B}_0$?  
    \begin{matrix-answer}
      correctMatrix = [['1','1','1']]
    \end{matrix-answer}
  \end{solution}
  
  \begin{solution}
    \begin{hint}
      \begin{align*}
        L(1) &= 1\\
        L(x-1) &= 1-1=0\\
        L((x-1)^2)&=(1-1)^2=0
      \end{align*}
    \end{hint}
    \begin{hint}
      \begin{bmatrix}
        The matrix of $L$ with respect to $\mathcal{B}_1$ is \( \begin{bmatrix} 1&0&0 \end{bmatrix}\)
      \end{bmatrix}
    \end{hint}
    What is the matrix of this linear map with respect to the basis $\mathcal{B}_1$?  
    
    \begin{matrix-answer}
      correctMatrix = [['1','0','0']]
    \end{matrix-answer}
  \end{solution}
\end{problem}

\begin{problem}
  Let $P_3$ be the space of polynomials of degree at most $3$.  Let $\mathcal{B} = (1,x,x^2,x^3)$.  Consider the map $L:P_3 \to P_3$ given by 
  $L(p(x)) = \frac{d}{dx} p(x)$. This map is linear (why?).  
  \begin{solution}
    \begin{hint}
      $L$ is linear because the derivative of a sum of two functions is the sum of the derivative of the two functions, and since the derivative of a constant
      times a function is the constant times the derivative of the function.
    \end{hint}
    \begin{hint}
      \begin{align*}
        L(1) &=0\\
        L(x) &=1\\
        L(x^2)&=2x\\
        L(x^3)&=3x^2
      \end{align*}
    \end{hint}
    \begin{hint}
      The matrix of $L$ is \(\begin{bmatrix} 0&1&0&0\\0&0&2&0\\0&0&0&3\\0&0&0&0\end{bmatrix}\)
    \end{hint}
    What is the matrix for $L$ with respect to the basis $\mathcal{B}$?
    \begin{matrix-answer}
      correctMatrix = [['0','1','0','0'],['0','0','2','0'],['0','0','0','3'],['0','0','0','0']]
    \end{matrix-answer}
  \end{solution}
\end{problem}

\end{document}