\documentclass{ximera}
\title{Python}

\begin{document}
\begin{abstract}
  Certain Python functions form a vector space?
\end{abstract}

Let $\mathcal{P}$ be the collection of all Python functions \texttt{f} with the properties that
\begin{itemize}
\item \texttt{f} accepts a single numeric parameter,
\item \texttt{f} returns a single numeric parameter, and
\item no matter what number \texttt{x} is, the function call \texttt{f(x)} successfully returns a number.
\end{itemize}
We'll say that two Python functions are ``equal'' if they produce the same outputs for the same inputs.

\hrule

Now the collection $\mathcal{P}$ (arguably) forms a vector space.  I
say ``arguably'' because ``numbers'' in Python aren't real numbers,
but let's just play along and pretend that they are.

\begin{question}
  What function plays the role of $\vec{0}$ in $\mathcal{P}$?
  \begin{solution}
    \begin{python}
def zero(x):
  # return ?  
    
def validator():
  return (zero(17) == 0)
    \end{python}
  \end{solution}

  Suppose we have two functions \texttt{f} and \texttt{g}.  What is their sum?
  \begin{solution}
    \begin{python}
def vector_sum(f,g):
  # return a new Python function which is the sum of f and g
    
def validator():
  return (vector_sum(lambda x: x**2, lambda x: x**3)(3) == 36)
    \end{python}
  \end{solution}

  \hrule

  Now suppose we have a function \texttt{f} and a scalar \texttt{c}.  What is \texttt{c} times \texttt{f}?
  \begin{solution}
    \begin{python}
def scalar_multiple(c,f):
  # return a new Python function which is c*f
    
def validator():
  return (scalar_multiple(17, lambda x: x**2)(2) == 68)
    \end{python}
  \end{solution}

  \hrule

    Now suppose we have a function \texttt{f} and a point \texttt{a}.  The map \texttt{evaluation_map} sends $\texttt{f} \in \mathcal{P}$ to the value $\texttt{f}(\texttt{a})$.
  \begin{solution}
    \begin{python}
def scalar_multiple(c,f):
  return (lambda x: c * f(x))
def vector_sum(f,g):
  return (lambda x: f(x) + g(x))
def evaluation_map(a,f):
  # return the value of f at the point a
#
# Now note that evaluation_map(a,vector_sum(f,g)) = evaluation_map(a,f) + evaluation_map(a,g)
#
f = lambda x: x**2
g = lambda x: x**3
a = 3
print evaluation_map(a,vector_sum(f,g))
print evaluation_map(a,f) + evaluation_map(a,g)
    
def validator():
  return (evaluation_map(17,(lambda x: x**2)) == 289)
    \end{python}
  \end{solution}

  This is an example of the fact that ``evaluation at \texttt{a}'' is a linear map from $\mathcal{P}$ to the underlying number system.

  \hrule

  Finally, some food for thought: in a little while, we'll be thinking
  about ``dimension.''  Keep in mind the following question: what is
  the dimension of the vector space $\mathcal{P}$?  You may want to
  think about this in the ideal world where ``Python functions'' take
  honest real numbers as their input and outputs.
\end{question}

\end{document}

%%% Local Variables: 
%%% mode: latex
%%% TeX-master: t
%%% End: 
