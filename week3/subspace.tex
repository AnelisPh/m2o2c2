\documentclass{ximera}
\title{Subspaces}

\begin{document}
\begin{abstract}
  A subspace is a subset of a vector space which is also a vector space.
\end{abstract}

\begin{definition}
  A subset $U$ of a vector space $V$ is a \textbf{subspace} of $V$ if
  $U$ is a vector space with respect to the scalar multiplication and
  the vector addition inherited from $V$.
\end{definition}

\begin{question}
  Which of the following is a subspace of $\R^2$?
  \begin{solution}
    \begin{hint}
      The vectors $\begin{bmatrix} 1 \\ 2\end{bmatrix}$ and $\begin{bmatrix} 0 \\ 1\end{bmatrix}$ are both on the line $\ell$, but the sum $\begin{bmatrix} 1 \\ 2\end{bmatrix} + \begin{bmatrix} 0 \\ 1\end{bmatrix}$ is \textbf{not} on the line $\ell.$
    \end{hint}

    \begin{hint}
      So $\ell$ is not a subspace.
    \end{hint}

    \begin{hint}
      The set $P$ consists of a single vector.
    \end{hint}

    \begin{hint}
      But the vector in $P$ is not the origin $\begin{bmatrix} 0 \\ 0 \end{bmatrix}$.
    \end{hint}

    \begin{hint}
      So $\begin{bmatrix} 1 \\ 2 \end{bmatrix} \in P$ but $10 \cdot \begin{bmatrix} 1 \\ 2 \end{bmatrix} \not\in P$.
    \end{hint}

    \begin{hint}
      So $P$ is not a subspace.
    \end{hint}

    \begin{hint}
      By process of elimination, the $x$-axis must be a subspace.  Is it really?
    \end{hint}

    \begin{hint}
      Yes, if I multiply any vector $\begin{bmatrix} x \\ 0 \end{bmatrix}$ by a scalar, it is still on the $x$-axis.
    \end{hint}

    \begin{hint}
      And if I add together two vectors of the form $\begin{bmatrix} x \\ 0 \end{bmatrix}$, the result is still on the $x$-axis.
    \end{hint}

    \begin{multiple-choice}
      \choice[correct]{The $x$-axis.}
      \choice{The set $P = \left\{\begin{bmatrix} 1 \\ 2 \end{bmatrix}\right\}$.}
      \choice{The line $\ell = \left\{\begin{bmatrix} x \\ y \end{bmatrix} \in \R^2 : y = x + 1\right\}$.}
    \end{multiple-choice}
  \end{solution}
  
  Let's look at some more examples!  Which of the following is a subspace of $\R^2$?
  \begin{solution}
    \begin{hint}
      The set $A$ is not a subspace because $\verticalvector{1\\0} \in A$, but $-1 \cdot \verticalvector{1\\0}$ is not in $A$, so a scalar multiple of something in $A$ need not be in $A$.
    \end{hint}
    \begin{hint}
      The set $C$ is not a subspace because even though it is closed
      under scalar multiplication (check this!) it is not closed under
      vector addition, since $\verticalvector{1\\-2}$ and
      $\verticalvector{1\\2}$ are both in $C$, but their sum
      $\verticalvector{2\\0}$ is not (draw a picture of this
      example!).
    \end{hint}
    \begin{hint}
      As the only choice left, $B$ must be a subspace.

      The reason is that it is just the span of the vector
      $\verticalvector{2\\1}$, and as such, is closed under scalar
      multiplication and vector addition.
    \end{hint}
    \begin{multiple-choice}
      \choice{ The set $A = \{ \verticalvector{x\\y} : x>0 \text{ and } y>0  \}$}
      \choice[correct]{ The set $B = \verticalvector{x\\y}: x=2y$}
      \choice{ The set $C = \{ \verticalvector{x\\y} : |y| < |x| \}$}
    \end{multiple-choice}
  \end{solution}


  \begin{solution}
    \begin{hint}
      \begin{question}
        What about the line $y = 3$?  Does it form a subspace of $\R^2$?
        
        \begin{solution}
          \begin{multiple-choice}
            \choice{Yes.}
            \choice[correct]{No.}
          \end{multiple-choice}
        \end{solution}

        That's right; the tip of the vector $\begin{bmatrix} 0 \\
          3 \end{bmatrix}$ is on that line, but the scalar multiple of
        that vector, like $2 \cdot \begin{bmatrix} 0 \\
          3 \end{bmatrix} = \begin{bmatrix} 0 \\ 6 \end{bmatrix}$, is
        not on the line.
      \end{question}
    \end{hint}

    So when do the points on a line in $\R^2$ form a subspace?
    \begin{multiple-choice}
      \choice[correct]{When the line passes through the point $(0,0)$.}
      \choice{When the line is parallel to the $x$-axis.}
    \end{multiple-choice}
  \end{solution}

  This is an important observation.

  \begin{observation}
    Suppose $U$ is a subspace of a vector space $V$.  Then the ``zero vector'' is in $U$.
  \end{observation}

\end{question}

\end{document}
