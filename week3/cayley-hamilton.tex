\documentclass{ximera}

\title{Cayley-Hamilton theorem}

\begin{document}

\begin{abstract}
  Sometimes eigen-information reveals quite a bit about linear
  operators.
\end{abstract}

We will not be proving---or even stating!---the
\href{http://en.wikipedia.org/wiki/Cayley–Hamilton_theorem}{Cayley-Hamilton
  theorem}, but there is one very special case which provides a nice
activity.  This activity will force us to think about bases and about
eigenvectors and eigenvalues.

Here's the setup: suppose $L : \R^2 \to \R^2$ is a linear map, and it
has an eigenvector $\vec{u}$ (with eigenvalue $2$) and an eigenvector
$\vec{w}$ (with eigenvalue $3$).

\begin{question}
  Now suppose $\vec{v} \in \R^2$ is some arbitrary vector.  How does
  $L(L(\vec{v}))$ compare to $-6 \vec{v} + 5 \cdot L(\vec{v})$?  

  \begin{solution}
    \begin{multiple-choice}
      \choice[correct]{$L(L(\vec{v})) = -6 \vec{v} + 5 \cdot L(\vec{v})$}
      \choice{$L(L(\vec{v})) \neq -6 \vec{v} + 5 \cdot L(\vec{v})$}
      \choice{It cannot be determined from the information given.}
    \end{multiple-choice}
  \end{solution}

  Why is this the case?
  
  \begin{solution}
    \begin{hint}
      The vectors $\vec{u}$ and $\vec{w}$ together form a basis for $\R^2$.
    \end{hint}

    Can we write $\vec{v}$ as $\alpha \vec{u} + \beta \vec{w}$?

    \begin{multiple-choice}
      \choice[correct]{Yes.}
      \choice{No.}
    \end{multiple-choice}    
  \end{solution}
  
  What is $L(\vec{v})$ in terms of $\alpha$, $\beta$, $\vec{u}$, and $\vec{w}$?
  \begin{solution}
    \begin{multiple-choice}
      \choice[correct]{$\alpha L(\vec{u}) + \beta L(\vec{w})$}
      \choice{$\alpha L(\vec{w}) + \beta L(\vec{u})$}
    \end{multiple-choice} 
  \end{solution}
  
  But what is $L(\vec{u})$?
  \begin{solution}
    \begin{hint}
      \begin{question}
        \begin{solution}
          Remember that $\vec{u}$ is an eigenvector with eigenvalue \answer{$2$}.
        \end{solution}

        Consequently $L(\vec{u}) = 2 \vec{u}$.
      \end{question}
    \end{hint}
    
    \begin{multiple-choice}
      \choice[correct]{$2 \vec{u}$}
      \choice{$3 \vec{u}$}
      \choice{$2 \vec{w}$}
      \choice{$3 \vec{w}$}
    \end{multiple-choice}   
  \end{solution}
  
  And what is $L(\vec{w})$?
  \begin{solution}
    \begin{hint}
      \begin{question}
        \begin{solution}
          Remember that $\vec{w}$ is an eigenvector with eigenvalue \answer{$3$}.
        \end{solution}

        Consequently $L(\vec{w}) = 3 \vec{w}$.
      \end{question}
    \end{hint}
    
    \begin{multiple-choice}
      \choice{$2 \vec{u}$}
      \choice{$3 \vec{u}$}
      \choice{$2 \vec{w}$}
      \choice[correct]{$3 \vec{w}$}
    \end{multiple-choice}   
  \end{solution}
  
  Using these facts, what is $L(\vec{v})$ in terms of $\alpha$, $\beta$, $\vec{u}$, and $\vec{w}$?
  \begin{solution}
    \begin{multiple-choice}
      \choice[correct]{$2\alpha\,\vec{u} + 3\beta\,\vec{w}$}
      \choice{$3\alpha\,\vec{u} + 2\beta\,\vec{w}$}
      \choice{$2\alpha\,\vec{w} + 3\beta\,\vec{u}$}
      \choice{$3\alpha\,\vec{w} + 2\beta\,\vec{u}$}
    \end{multiple-choice} 
  \end{solution}
  
  \begin{solution}
    \begin{hint}
      \begin{question}
        \begin{solution}
          Using linearity of $L$, what is $L(L(\vec{v}))$?

          \begin{multiple-choice}
            \choice[correct]{$2\alpha\,L(\vec{u}) + 3\beta\,L(\vec{w})$}
            \choice{$3\alpha\,L(\vec{u}) + 2\beta\,L(\vec{w})$}
            \choice{$2\alpha\,L(\vec{w}) + 3\beta\,L(\vec{u})$}
            \choice{$3\alpha\,L(\vec{w}) + 2\beta\,L(\vec{u})$}
          \end{multiple-choice} 
        \end{solution}
      \end{question}
    \end{hint}

    \begin{hint}
      \begin{question}
        \begin{solution}
          But what is $L(\vec{u}))$?

          \begin{multiple-choice}
            \choice[correct]{$L(\vec{u})) = 2 \vec{u}$}
            \choice{$L(\vec{u})) = 3 \vec{u}$}
            \choice{$L(\vec{u})) = 2 \vec{w}$}
            \choice{$L(\vec{u})) = 3 \vec{w}$}
          \end{multiple-choice} 
        \end{solution}
      \end{question}
    \end{hint}

    \begin{hint}
      \begin{question}
        \begin{solution}
          And what is $L(\vec{w}))$?

          \begin{multiple-choice}
            \choice[correct]{$L(\vec{w})) = 3 \vec{w}$}
            \choice{$L(\vec{w})) = 2 \vec{w}$}
            \choice{$L(\vec{w})) = 2 \vec{u}$}
            \choice{$L(\vec{w})) = 3 \vec{u}$}
          \end{multiple-choice} 
        \end{solution}
      \end{question}
    \end{hint}

    \begin{hint}
      Try substituting the facts that $L(\vec{u})) = 2 \vec{u}$ and $L(\vec{w})) = 3 \vec{w}$ into $2\alpha\,L(\vec{u}) + 3\beta\,L(\vec{w})$.
    \end{hint}

    What is $L(L(\vec{v}))$?
    \begin{multiple-choice}
      \choice[correct]{$4\alpha\,\vec{u} + 9\beta\,\vec{w}$}
      \choice{$4\alpha\,\vec{w} + 9\beta\,\vec{u}$}
      \choice{$9\alpha\,\vec{u} + 4\beta\,\vec{w}$}
      \choice{$9\alpha\,\vec{w} + 4\beta\,\vec{u}$}
    \end{multiple-choice} 
  \end{solution}

  What is $-6 \vec{v} + 5 \cdot L(\vec{v})$ in terms of $\alpha$, $\beta$, $\vec{u}$, and $\vec{w}$?
  \begin{solution}
    \begin{hint}
      Earlier we wrote $\vec{v} = \alpha \vec{u} + \beta \vec{w}$.
    \end{hint}

    \begin{hint}
      Since $L$ is a linear map, we have $L(\vec{v}) = \alpha L(\vec{u}) + \beta L(\vec{w})$.
    \end{hint}

    \begin{multiple-choice}
      \choice[correct]{$-6 \left( \alpha \vec{u} + \beta \vec{w} \right) + 5 \alpha L(\vec{u}) + 5 \beta L(\vec{w})$}
      \choice{$-6 \left( \alpha \vec{u} + \beta \vec{w} \right) + 5 \beta L(\vec{u}) + 5 \alpha L(\vec{w})$}
      \choice{$-6 \left( \alpha \vec{u} + \beta \vec{w} \right) + 3 \alpha L(\vec{u}) + 3 \beta L(\vec{w})$}
      \choice{$-6 \left( \alpha \vec{u} + \beta \vec{w} \right) + 3 \beta L(\vec{u}) + 3 \alpha L(\vec{w})$}
    \end{multiple-choice} 
  \end{solution}

  \begin{question}
    \begin{hint}
      \begin{question}
        \begin{solution}
          But what is $L(\vec{u}))$?

          \begin{multiple-choice}
            \choice[correct]{$L(\vec{u})) = 2 \vec{u}$}
            \choice{$L(\vec{u})) = 3 \vec{u}$}
            \choice{$L(\vec{u})) = 2 \vec{w}$}
            \choice{$L(\vec{u})) = 3 \vec{w}$}
          \end{multiple-choice} 
        \end{solution}
      \end{question}
    \end{hint}

    \begin{hint}
      \begin{question}
        \begin{solution}
          And what is $L(\vec{w}))$?

          \begin{multiple-choice}
            \choice[correct]{$L(\vec{w})) = 3 \vec{w}$}
            \choice{$L(\vec{w})) = 2 \vec{w}$}
            \choice{$L(\vec{w})) = 2 \vec{u}$}
            \choice{$L(\vec{w})) = 3 \vec{u}$}
          \end{multiple-choice} 
        \end{solution}
      \end{question}
    \end{hint}

    \begin{hint}
      Try substituting the facts that $L(\vec{u})) = 2 \vec{u}$ and $L(\vec{w})) = 3 \vec{w}$ into $-6 \left( \alpha \vec{u} + \beta \vec{w} \right) + 5 \alpha L(\vec{u}) + 5 \beta L(\vec{w})$.
    \end{hint}    

    \begin{hint}
      Then we get $-6 \alpha \vec{u} - 6 \beta \vec{w} + (5 \cdot 2) \alpha \vec{u} + (5 \cdot 3) \beta \vec{w}$.
    \end{hint}    

    \begin{hint}
      But $-6 + 10 = 4$ and $-6 + 15 = 9$.
    \end{hint}    

    \begin{hint}
      Consequently, this simplifies to $4\alpha\,\vec{u} + 9\beta\,\vec{w}$.
    \end{hint}    

    Now write $-6 \vec{v} + 5 \cdot L(\vec{v})$ but without referring to $L$.
    \begin{multiple-choice}
      \choice[correct]{$4\alpha\,\vec{u} + 9\beta\,\vec{w}$}
      \choice{$4\alpha\,\vec{w} + 9\beta\,\vec{u}$}
      \choice{$9\alpha\,\vec{u} + 4\beta\,\vec{w}$}
      \choice{$9\alpha\,\vec{w} + 4\beta\,\vec{u}$}
    \end{multiple-choice} 
  \end{question}

  And so, after all this, we see that $L(L(\vec{v})) = -6 \vec{v} + 5 \cdot L(\vec{v})$.

  What happens if you try this in higher dimensions?  Suppose you have
  a map $L : \R^3 \to \R^3$ and it has three eigenvectors with three
  different eigenvalues.  Can you rewrite $L(L(L(\vec{v})))$ in terms
  of $\vec{v}$ and $L(\vec{v})$ and $L(L(\vec{v}))$ in that case?

\end{question}


\end{document}

%%% Local Variables: 
%%% mode: latex
%%% TeX-master: t
%%% End: 
