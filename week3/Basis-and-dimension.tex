\begin{document}
\begin{Basis and Dimension}

In our study of the vector spaces $\R^n$, we have relied quite heavily on the ``standard basis vectors'' $\vec{e_1}  =\verticalvector{1,0,0,...,0}, 
\vec{e_2}  =\verticalvector{0,1,0,...,0}, \vec{e_3}  =\verticalvector{0,0,1,...,0},..., \vec{e_n}  =\verticalvector{0,0,0,...,1}$.  

A great feature of these vectors is that they \textit{span} all of $\R^n$:  every vector $\vec{v} \in \R^n$ can be written in the form
 $v = x_1\vec{e_1} +x_2\vec{e_2}+...+x_n\vec{e_n}$.  What is more this representation is unique:  if I also have $v = y_1\vec{e_1} +y_2\vec{e_2}+...+y_n\vec{e_n}$
 then $x_1 = y_1, x_2 = y_2, ..., x_n=y_n$.
 
 Our goal in this section will be to find similarly nice sets of vectors in an abstract vector space.
 
 \begin{definition}
 	The \textit{span} of an ordered list of vectors $(v_1,v_2, ...,v_n)$ is the set of all \textit{linear combinations} of the $v_i$:
 	$\textrm{Span}(v_1,v_2, ...,v_n) = \{a_1v_1+a_2v_2+...+a_nv_n: a_i \in \R\}$
 \end{definition}
 
 \begin{question}
 	Which of the following vectors are in the span of BLAH?
 \end{question}
 
 \begin{question}
 	Show that $\verticalvector{1,1}$ and $\verticalvector{1,0}$ span the entire space $\R^2$.
 \end{question}
 
  \begin{question}
 	Show that $(\verticalvector{2,3},\verticalvector{1,0},\verticalvector{4,5})$ span the space $\R^2$.
 \end{question}
 
 \begin{question}
 	Show that $1$, $x-1$, $(x-1)^2$ and $(x-1)^3$ span the space of polynomials of degree at most $3$.
 \end{question}
 
  \begin{definition}
 	A vector space is called \textit{finite dimensional} if it has a finite list of spanning vectors.  A space which is not finite dimensional is called \textit{infinite dimensional}.
 \end{definition}
 
 \begin{question}
 	Show that the space of all polynomials of one variable is infinite dimensional
 \end{question}
 
 \begin{question}
 	Show that the space of all continuous functions on the interval $[0,1]$ is infinite dimensional
 \end{question}
 
 \begin{definition}
 	Let $V$ be a vector space. An ordered list of vectors $(v_1,v_2,...,v_n)$ where all the $v_i \in V$ is called \textit{linearly independent} if
 	$a_1v_1+a_2v_2 + ...+a_nv_n = b_1v_1 + b_2v_2 + ... + b_nv_n$ implies that $a_1  = b_1$, $a_2 = b_2$, ...,$a_n=b_n$.  In other words,
 	every vector in the span of $(v_1,v_2,...,v_n)$ can be expressed as a linear combination of the $v_i$ in only one way.  
 	If the set of vectors is not linearly independent it is called \textit{linearly dependent}.
 \end{definition}
 
 \begin{question}
 	Is BLAH linearly independent?
 \end{question}
 
 \begin{question}
 	Show that the following alternative definition for linear independence is equivalent to our definition:
 	
 	\begin{definition}
 		Let $V$ be a vector space. An ordered list of vectors $(v_1,v_2,...,v_n)$ where all the $v_i \in V$ is called \textit{linearly independent} if
 	$a_1v_1+a_2v_2 + ...+a_nv_n = \vec{0}$ implies that $a_i = 0 $ for all $i=1,2,3,...,n$.
 	\end{definition}
 \end{question}
 
 \begin{question}
 	Prove that any ordered list of vectors containing the zero vector is linearly dependent. 
 \end{question}
 
 \begin{question}
 	Prove that an ordered list of length $2$ (i.e. $(v_1,v_2)$) is linearly dependent if and only if one vector is a scalar multiple of the other.
 \end{question}

 \begin{theorem}
 	If $(\vec{v_1},\vec{v_2},\vec{v_3}, ..., \vec{v_n})$ is linearly dependent in $V$ and $\vec{v_1} \neq 0$, then one of the vectors $v_j$ is in the 
 	span of $\vec{v_1},\vec{v_2},...,\vec{v_{j-1}}$
 \end{theorem}
 
 \begin{proof}
 	Since $(\vec{v_1},\vec{v_2},\vec{v_3}, ..., \vec{v_n})$ is linearly dependent, by definition there are scalars $a_i \in \R$ with 
 	$a_1\vec{v_1}+a_2\vec{v_2}+ ...+a_n\vec{v_n} = 0$, and not all of the $a_j =0$.  Let $j$ be the largest element of ${2,3,...,n}$ so that $a_j$ is not equal to $0$.
 	Then we have 
 	$v_j = -\frac{a_1}{a_j}\vec{v_1}-\frac{a_2}{a_j}\vec{v_2} -\frac{a_3}{a_j}\vec{v_3} - ...-\frac{a_{j-1}}{a_{j-1}}\vec{v_{j-1}}$.
 	So $v_j$ is in the span of $\vec{v_1},\vec{v_2},...,\vec{v_{j-1}}$.
 \end{proof}
 
 \begin{question}
 	If $2$ vectors $\vec{v_1},\vec{v_2}$ span $V$, is it possible that the three vectors $\vec{w_1},\vec{w_2},\vec{w_3}$ are linearly independent?
 \end{question}

 
 \begin{question}
 	Prove that the length of a linearly independent list of spanning vectors is less than the length of any spanning list of vectors. (i.e. generalize the problem above)
 \end{question}
 
 \begin{definition}
 	An ordered list of vectors $\mathcal{B} = (\vec{v_1},\vec{v_2},\vec{v_3},...,\vec{v_n})$ is called a \textit{basis} of the vector space $V$ if 
 	$\mathcal{B}$ is both spans $V$ and is linearly independent.  
 \end{definition}
 
 \begin{question}
 	Show that $1$, $x-1$, $(x-1)^2$ is a basis for the space of polynomials of degree at most $2$.
 \end{question}
 
 \begin{question}
 	Let $V$ be a finite dimensional vector space.  Show that $V$ has a basis.
 \end{question}
 
 \begin{question}
 	Let $V$ be a finite dimensional vector space.  Show that every basis of $V$ has the same length.  In other words, if
 	$\vec{v_1},\vec{v_2}, ..., \vec{v_n}$ is a basis and	$\vec{w_1},\vec{w_2}, ..., \vec{w_m}$ is a basis, then $n=m$.
  \end{question}
  
  \begin{definition}
  	We say that a finite dimensional vector space has dimension $n$ if it has a basis of length $n$.
  \end{definition}
  
  \begin{question}
  	Let $p_1,p_2,..,p_n$ be polynomials in the space $P_n$of all polynomials of degree at most $n$.  Assume $p_i(3) = 0$ for $i=1,2,...,n$.  Is it possible that
  	$p_1,p_2,...,p_n$ are all linearly independent?  Why or why not?
  \end{question}
  

\end{document}