\documentclass{ximera}

\begin{document}
\begin{Basis and Dimension}

In our study of the vector spaces $\R^n$, we have relied quite heavily on the ``standard basis vectors'' $\vec{e_1}  =\verticalvector{1,0,0,...,0}, 
\vec{e_2}  =\verticalvector{0,1,0,...,0}, \vec{e_3}  =\verticalvector{0,0,1,...,0},..., \vec{e_n}  =\verticalvector{0,0,0,...,1}$.  

A great feature of these vectors is that they \textit{span} all of $\R^n$:  every vector $\vec{v} \in \R^n$ can be written in the form
 $v = x_1\vec{e_1} +x_2\vec{e_2}+...+x_n\vec{e_n}$.  What is more this representation is unique:  if I also have $v = y_1\vec{e_1} +y_2\vec{e_2}+...+y_n\vec{e_n}$
 then $x_1 = y_1, x_2 = y_2, ..., x_n=y_n$.
 
 Our goal in this section will be to find similarly nice sets of vectors in an abstract vector space.
 
 \begin{definition}
 	The \textbf{span} of an ordered list of vectors $(v_1,v_2, ...,v_n)$ is the set of all \textbf{linear combinations} of the $v_i$.
 	\[\textrm{Span}(v_1,v_2, ...,v_n) = \{a_1v_1+a_2v_2+...+a_nv_n: a_i \in \R\}\]
 \end{definition}
 
% \begin{question}
% 	Which of the following vectors are in the span of BLAH?
% \end{question}
 
 	Show that $\verticalvector{1\\1}$ and $\verticalvector{1\\0}$ span the entire space $\R^2$.

\begin{free-response}
	Since we already know that $\verticalvector{1\\0}$ and $\verticalvector{0\\1}$ span $\R^2$,
	 it is enough to show that $\verticalvector{1\\1}$ and $\verticalvector{1\\0}$ span these two vectors, i.e.
	 we need only show that $\verticalvector{0\\1}$ is in the span of these two vectors.
	 
	 But $\verticalvector{0\\1} = \verticalvector{1\\1} +-1\verticalvector{1\\0}$, so we are done.
	 
	 To be a bit more explicit, we can write any vector \begin{align*}
	 \verticalvector{x\\y} &= x\verticalvector{1\\0}+ y\verticalvector{0\\1}\\
	 &= x\verticalvector{1\\0}+ y\left(\verticalvector{1\\1} +-1\verticalvector{1\\0}\right)\\
	 &=(x-y)\verticalvector{1\\0}+y\vecticalvector{1\\1}
	 \end{align*}
	 
	 So we have expressed any vector as a linear combination of $\verticalvector{1\\1}$ and $\verticalvector{1\\0}$.
\end{free-response} 

%  \begin{question}
 %	Show that $(\verticalvector{2,3},\verticalvector{1,0},\verticalvector{4,5})$ span the space $\R^2$.
 %\end{question}
 
 	Show that $1$, $x-1$, and $(x-1)^2$  span the space of polynomials of degree at most $2$.

\begin{free-response}
	Let $p(x)=a_0+a_1x+a_2x^2$. 
	
	Then \begin{align*}
		p(x) &= a_0+a_1[(x-1)+1]+a_2[(x-1)+1]^2\\
			   &= a_0+a_1(x-1)+a_1+a_2[(x-1)^2+2(x-1)+1]\\
			   &= (a_0+a_1+a_2)1+(a_1+2a_2)(x-1)+a_2(x-1)^2
		\end{align*}
	
	so we have expressed every polynomial of degree at most $2$ as a linear combination of $1,(x-1)$ and $(x-1)^2$.
	
	You could also solve this problem by appealing to Taylor's theorem in one variable calculus.  Can you see how?
\end{free-response}

 
  \begin{definition}
 	A vector space is called \textbf{finite dimensional} if it has a finite list of spanning vectors.  A space which is not finite dimensional is called \textit{infinite dimensional}.
 \end{definition}
 
 	Show that the space $P$ of all polynomials in one variable $x$ is infinite dimensional

\begin{free-response}
	 Let $p_1,p_2,...,p_n$ be a finite list of  vectors.  
	Since this list of polynomials is finite they must be bounded in degree, i.e.  the degree of $p_i$ must be less than some $k$ for each $i$.  But a linear combination
	of polynomials of degree at most $k$ is also of degree at most $k$.  So the polynomial $x^{k+1} \not\in  \textrm{Span}(p_1,p_2,...,p_n)$. Thus no finite list of 
	polynomials spans all of $P$.  So $P$ is infinite dimensional.
\end{free-response}

 \begin{definition}
 	Let $V$ be a vector space. An ordered list of vectors $(v_1,v_2,...,v_n)$ where all the $v_i \in V$ is called \textit{linearly independent} if
 	$a_1v_1+a_2v_2 + ...+a_nv_n = b_1v_1 + b_2v_2 + ... + b_nv_n$ implies that $a_1  = b_1$, $a_2 = b_2$, ...,$a_n=b_n$.  In other words,
 	every vector in the span of $(v_1,v_2,...,v_n)$ can be expressed as a linear combination of the $v_i$ in only one way.  
 	If the set of vectors is not linearly independent it is called \textit{linearly dependent}.
 \end{definition}
 
%\begin{question}
% 	Is BLAH linearly independent?
%\end{question}
 
 	Show that the following alternative definition for linear independence is equivalent to our definition:
 	
 	\begin{definition}
 		Let $V$ be a vector space. An ordered list of vectors $(v_1,v_2,...,v_n)$ where all the $v_i \in V$ is called \textit{linearly independent} if
 	$a_1v_1+a_2v_2 + ...+a_nv_n = \vec{0}$ implies that $a_i = 0 $ for all $i=1,2,3,...,n$.
 	\end{definition}
 	
 
 \begin{free-response}
	Let us say that our original definition is of  being linearly independent in the first sense, while this second definition is being linearly independent 
	in the second sense.  If a list of vectors $(v_1,v_2,...,v_n)$ is linearly independent in the first sense, then if $a_1v_1+a_2v_2 + ...+a_nv_n = \vec{0}$ we have
	$a_1v_1+a_2v_2 + ...+a_nv_n = 0v_1+0v_2+...+0v_n$, so by the definition of linear independence in the first sense, we have $a_1=a_2=...=a_n=0$.
	
	On the other hand, if $(v_1,v_2,...,v_n)$ are linearly independent in the second sense, then if $a_1v_1+a_2v_2 + ...+a_nv_n = b_1v_1 + b_2v_2 + ... + b_nv_n$ we have
	$(a_1-b_1)v_1+(a_2-b_2)v_2+...+(a_n-b_n)v_n = \vec{0}$, so $a_i-b_i=0$ for each $i$.  Thus $a_i=b_i$ for each $i$, proving that the list was linearly independent 
	in the first sense.
 \end{free-response}
 
  	Often this definition is easier to check, although it does not capture the "meaning" of linear independence as well as the first definition.

 
 	Prove that any ordered list of vectors containing the zero vector is linearly dependent. 
	\begin{free-response}
		We can see immediately from the second definition that since $1\vec{0} = \vec{0}$, but $1\neq 0$, that the list cannot be linearly independent
	\end{free-response}
 
 	Prove that an ordered list of length $2$ (i.e. $(v_1,v_2)$) is linearly dependent if and only if one vector is a scalar multiple of the other.
	\begin{free-response}
		For $v_1$ and $v_2$ to be linearly independent there must be two scalars $a,b \in V$ with $av_1+bv_2=0$ with at least one of $a$ or $b$ nonzero.
		Let us assume (without loss of generality) that $a \neq 0$.  Then $av_1=-bv_2$, so $v_1=\frac{-b}{a}v_2$.  Thus one vector is a scalar multiple of the other.	
	
		\end{free-response}


 \begin{theorem}
 	If $(\vec{v_1},\vec{v_2},\vec{v_3}, ..., \vec{v_n})$ is linearly dependent in $V$ and $\vec{v_1} \neq 0$, then one of the vectors $v_j$ is in the 
 	span of $\vec{v_1},\vec{v_2},...,\vec{v_{j-1}}$
 \end{theorem}
 
Prove this theorem.

\begin{free-response}
 	Since $(\vec{v_1},\vec{v_2},\vec{v_3}, ..., \vec{v_n})$ is linearly dependent, by definition there are scalars $a_i \in \R$ with 
 	$a_1\vec{v_1}+a_2\vec{v_2}+ ...+a_n\vec{v_n} = 0$, and not all of the $a_j =0$.  Let $j$ be the largest element of ${2,3,...,n}$ so that $a_j$ is not equal to $0$.
 	Then we have 
 	$v_j = -\frac{a_1}{a_j}\vec{v_1}-\frac{a_2}{a_j}\vec{v_2} -\frac{a_3}{a_j}\vec{v_3} - ...-\frac{a_{j-1}}{a_{j-1}}\vec{v_{j-1}}$.
 	So $v_j$ is in the span of $\vec{v_1},\vec{v_2},...,\vec{v_{j-1}}$.
 \end{free-response}
 
 
 	If $2$ vectors $\vec{v_1},\vec{v_2}$ span $V$, is it possible that the three vectors $\vec{w_1},\vec{w_2},\vec{w_3}$ are linearly independent?
 	
 	\begin{warning}
 		This is harder to prove than you might think!
 	\end{warning}
\begin{free-response}
	No!
	
	Assume to the contrary that $\vec{w_1},\vec{w_2},\vec{w_3}$ are linearly independent.
	
	Since the list $(v_1,v_2)$ spans $V$, the list $(w_1,v_1,v_2)$ is linearly independent.  Thus by the previous theorem, either $v_1$ is in the span of $w_1$,
	or $v_2$ is in the span of $(w_1,v_1)$.  In either case we get that $(w_1,v)$ spans $V$, where $v$ is either $v_1$ or $v_2$.
	
	Now apply the same trick:  $(w_2,w_1,v)$ must span $V$.  So by the previous theorem, either $w_1$ is in the span of $w_2$ or $v$ is in the span of $w_2,w_1$.  
	$w_2$ cannot be in the span of $w_1$ because the $w$'s are linearly independent.  So $v$ is in the span of $w_2,w_1$.  So $(w_2,w_1)$ spans $V$.  But then
	$w_3$ is in the span of $(w_2,w_1)$, contradicting the fact that it is linearly independent from those two vectors.  We have arrived at our contradiction.
	
	Therefore, $w_1,w_2,w_3$ cannot be linearly independent.
\end{free-response}

This problem generalizes:

 \begin{theorem}
 	The length of a linearly independent list of spanning vectors is less than the length of any spanning list of vectors.
	\end{theorem}
Prove this theorem
 \begin{free-response}
 	We will follow the same procedure that we did above.  Assume $(v_1,v_2,...,v_n)$ is a list of vectors which spans $V$, and $(w_1,w_2,...,w_m)$ is a linearly
 	independent list of vectors.  We must show that $m<n$.
 	
 	$(w_1,v_1,v_2,...,v_n)$ is linearly dependent since $w_1$ is in the span of the $v_i$.  
 	By the theorem above, we can remove on of the $v_i$ and still have a spanning list of length $n$.
 	
 	Repeating this, we can always add one $w$ vector to the beginning of the list, while deleting a $v$ vector from the end of the list.  This maintains a list of length 
 	$n$ which spans all of $V$.  We know that it must be a $v$ which gets deleted, because the $w$s are all linearly independent. 
 	 If $m>n$, then at the $n^{th]$ stage of this process we obtain that $(w_1,w_2,...,w_n)$ spans all of $V$, which contradicts the fact that $w_{n+1}$ is supposed 
 	 to be linearly independent from the rest of the $w$.
 \end{free-response}
 
 \begin{definition}
 	An ordered list of vectors $\mathcal{B} = (\vec{v_1},\vec{v_2},\vec{v_3},...,\vec{v_n})$ is called a \textit{basis} of the vector space $V$ if 
 	$\mathcal{B}$ is both spans $V$ and is linearly independent.  
 \end{definition}
 
 %Show that $1$, $x-1$, $(x-1)^2$ is a basis for the space of polynomials of degree at most $2$.
 %\begin{free-response}
 %	We already showed that the
 %\end{free-response}
 
 
 	Let $V$ be a finite dimensional vector space.  Show that $V$ has a basis.
\begin{free-response}
	Let $(v_1,v_2,...,v_n)$ be a spanning list of vectors (which exists and is finite since $V$ is finite dimensional).  If this list is linearly dependent we can 
	go through the following process:
	For each $i$ if $v_i \in Span(v_1,v_2,...,v_{i-1})$, delete $v_i$ from the list.  Note that this also covers the $1st$ case:  if $v_1=0$ delete it from the list.
	
	At the end of this process, we have a list of vectors which span $V$, and also no vector is the span of all the previos vectors.  By the theorem above, the list is linearly 
	independent.  So this new list is a basis for $V$.
\end{free-response}

 
 	Note: Let $V$ be a finite dimensional vector space.  Then  every basis of $V$ has the same length.  In other words, if
 	$\vec{v_1},\vec{v_2}, ..., \vec{v_n}$ is a basis and	$\vec{w_1},\vec{w_2}, ..., \vec{w_m}$ is a basis, then $n=m$.  This follows because we have already proven
 	that $n\leq m$ and $m\leq n$
  
  \begin{definition}
  	We say that a finite dimensional vector space has dimension $n$ if it has a basis of length $n$.
  \end{definition}
  
  	Let $p_1,p_2,..,p_n,p_{n+1}$ be polynomials in the space $P_n$of all polynomials of degree at most $n$.  Assume $p_i(3) = 0$ for $i=1,2,...,n$.  Is it possible that
  	$p_1,p_2,...,p_n,p_{n+1}$ are all linearly independent?  Why or why not?
\begin{free-response}
	No.  If $p_1,p_2,...,p_n,p_{n+1}$ were all linearly independent then they would form a basis of $P_n$, since $P_n$ has dimension $n+1$.  
	But every polynomial in the span of the $p_i$ must evaluate to $0$ at $x=3$, while some polynomials in $P_n$ do not evaluate to $0$ at $x=3$,  (for example, the polynomial $x$).
\end{free-response}  

\end{document}