\documentclass{ximera}
\title{Proof of the chain rule}

\begin{document}
	\begin{abstract}
		This section is optional
	\end{abstract}\maketitle
	
	\textbf{This section is optional}
	\\
	\\
	Before we beginning the proof of the chain rule, we need to introduce a new piece of machinery:
	\\
	\\
	Let $L:\R^n \to \R^m$ be a linear map.  Let $S^{n-1} = \{ \vec{v} \in \R^n: |\vec{v}|  = 1\}$ be the unit sphere in $\R^n$.  Then there is a function
		$F: S^{n-1} \to \R^m$ defined on this sphere which takes a vector and returns the length of its image under $L$, $F(\vec{v}) = |L(\vec{v})|$.  
		
	\begin{definition}
		The maximum value of the function $F$ is called the \textbf{operator norm} of $L$, and is written $|L|_{op}$.
	\end{definition}
	
	The fact that the operator norm of a linear transformation exists is a kind of deep (for this course) piece of analysis.  It follows from the fact that the sphere is
	\href{http://en.wikipedia.org/wiki/Compact_space}{compact}, and continuous functions on compact spaces must achieve a maximum value.  For example, every continuous function on the closed interval
	$[0,1] \subset \R$ has a maximum, although not every continuous function on the open interval $(0,1)$ does.  The essential difference between 
	these two intervals is that the first one is compact, while the second one is not.
	\\
	\\
	The essential property of the operator norm is that $|L(\vec{v})| \leq |L|_{op} |\vec{v}|$ for each vector $\vec{v} \in \R^n$.  This is true just because we have 
	$L(\frac{\vec{v}}{|\vec{v}|}) \leq |L|_{op}$ because $\frac{\vec{v}}{|\vec{v}|}$ is a unit vector, and by the definition of the operator norm.  This fact will be essential 
	to us as we prove the chain rule.
	\\
	\\
	Let $f:\R^n \to \R^m$ and $g:\R^m \to \R^k$ be differentiable functions.  
	We want to show that $D(g \circ f)(\mathbf{p}) = Dg(\mathbf{f(p)}) \circ Df(\mathbf{p})$.
	\\
	\\
	All we know at the beginning of the day is that  
	\[f(\mathbf{p}+\vec{h}) = f(\mathbf{p})+Df(\mathbf{p})(\vec{h})+\text{fError}(\vec{h})\]
	 with 
	\[\displaystyle\lim_{\vec{h} \to 0} \frac{\left| \text{fError}(\vec{h})\right|}{|\vec{h}|} = 0\]
	 and 
	 \[g(\mathbf{f(p)}+\vec{u}) = g(f(\mathbf{p}))+Dg(\mathbf{f(p)})(\vec{u})+\text{gError}(\vec{u})\]
	 with 
	\[ \displaystyle\lim_{\vec{u} \to 0} \frac{\left| \text{gError}(\vec{u})\right|}{|\vec{u}|} = 0\]
	\\
	\\
	We will simply compose these two formulas for $f$ and $g$, and try to get some control on the complicated error term which results.
	\begin{align*}
	g\left(f(\mathbf{p}+\vec{h})\right) &= g\left( f(\mathbf{p})+Df(\mathbf{p})(\vec{h})+\text{fError}_{\mathbf{p}}(\vec{h})\right)\\
		&= g(f(\mathbf{p}))+Dg(\mathbf{f(p)})\left(Df(\mathbf{p})(\vec{h})+ \text{fError}_{\mathbf{p}}(\vec{h})\right)\\
		&\hphantom{dsdsd}+\text{gError}\left(Df(\mathbf{p})(\vec{h})+\text{fError}_{\mathbf{p}}(\vec{h})\right)\\
		&= (g \circ f)(\mathbf{p})+Dg(\mathbf{f(p)}) \circ Df(\mathbf{p})(\vec{h})\\
		&\hphantom{dsdsd}+ \textcolor{red}{Dg(\mathbf{f(p)})\left(\text{fError}_{\mathbf{p}}(\vec{h})\right) + \text{gError}\left(Df(\mathbf{p})(\vec{h})+\text{fError}_{\mathbf{p}}(\vec{h})\right)}\\
	\end{align*}
	
	This looks pretty horrible, but at least we can see the error term in red that we have to get some control over.  
	In particular we will have proven the chain rule if we can show that
	\\
	\\
	\[
		\displaystyle\lim_{\vec{h} \to 0} \frac{\left|\textcolor{red}{Dg(\mathbf{f(p)})\left(\text{fError}_{\mathbf{p}}(\vec{h})\right) + \text{gError}\left(Df(\mathbf{p})(\vec{h})+\text{fError}_{\mathbf{p}}(\vec{h})\right)}\right|}{|\vec{h}|} = 0
	\]
	\\
	\\
	Since 
	\\
	\\
	\[0 \leq \left|Dg(\mathbf{f(p)})\left(\text{fError}_{\mathbf{p}}(\vec{h})\right) + \text{gError}\left(Df(\mathbf{p})(\vec{h})+\text{fError}_{\mathbf{p}}(\vec{h})\right) \right|\\\\ \leq \left|Dg(\mathbf{f(p)})\left(\text{fError}_{\mathbf{p}}(\vec{h})\right)\right| + \left| \text{gError}\left(Df(\mathbf{p})(\vec{h})+\text{fError}_{\mathbf{p}}(\vec{h})\right)\right|\]
	\\
	\\
	by the triangle inequality, we will have the result if we can prove separately that
	
	\[
		\displaystyle\lim_{\vec{h} \to 0} \frac{\left|Dg(\mathbf{f(p)})\left(\text{fError}_{\mathbf{p}}(\vec{h})\right)\right|}{\left|\vec{h}\right|}  = 0
	\]
	
	and 
	
	\[
	\displaystyle\lim_{\vec{h} \to 0} \frac{ \left| \text{gError}\left(Df(\mathbf{p})(\vec{h})+\text{fError}_{\mathbf{p}}(\vec{h})\right)\right|}{\vec{h}} = 0
	\]
	
	Lets prove the first limit.  This is where operator norms enter the picture.
	
\[
	\frac{\left|Dg(\mathbf{f(p)})\left(\text{fError}_{\mathbf{p}}(\vec{h})\right)\right|}{\left|\vec{h}\right|} \leq \frac{\left|Dg\right|_{op}|\text{fError}(\vec{h})|}{|\vec{h}|}
\]

Since $\frac{\left|\text{fError}\right|}{\left|\vec{h}\right|} \to 0$ as $\vec{h} \to 0$, then we see that $\frac{\left|Dg(\mathbf{f(p)})\left(\text{fError}_{\mathbf{p}}(\vec{h})\right)\right|}{\left|\vec{h}\right|}$ 
must as well, since it is bounded above by a constant multiple of something which goes to $0$ (and bounded below by $0$).
\\
\\
For the other part of the error,

\begin{align*}
& \frac{ \left| \text{gError}\left(Df(\mathbf{p})(\vec{h})+\text{fError}(\vec{h})\right)\right|}{\vec{h}}\\
&=\frac{\left| \text{gError}\left(Df(\mathbf{p})(\vec{h})+\text{fError}(\vec{h})\right)\right|}{\left|Df(\mathbf{p})(\vec{h})+\text{fError}_{\mathbf{p}}(\vec{h})\right|} \frac{\left|Df(\mathbf{p})(\vec{h})+\text{fError}(\vec{h})\right|)}{|\vec{h}|}\\
\end{align*}

The first factor in this expression goes to $0$ as $\vec{h} \to 0$ because $g$ is differentiable.  
So all we need to do is make sure that the second factor is bounded.

\begin{align*}
	&\frac{\left|Df(\mathbf{p})(\vec{h})+\text{fError}_{\mathbf{p}}(\vec{h})\right|}{|\vec{h}|} \\
	&\leq \frac{\left|Df(\mathbf{p})(\vec{h}) \right|}{|\vec{h}|}+ \frac{\left|\text{fError}(\vec{h})\right|}{|\vec{h}|} \text{by the triangle inequality}\\
	&\leq \frac{\left|Df(\mathbf{p})\right|_{op}\left|\vec{h}\right|}{|\vec{h}|}+ \frac{\left|\text{fError}(\vec{h})\right|}{|\vec{h}|}\\
	&= \left|Df(\mathbf{p})\right|_{op} + \frac{\left|\text{fError}(\vec{h})\right|}{|\vec{h}|}
\end{align*}

Now the second term in this expression goes to $0$ as $\vec{h} \to 0$ since $f$ is differentiable.  So the whole expression is bounded by, say, $|Df(\mathbf{p})|_{op}+\frac{1}{2}$
if $\vec{h}$ is small enough.  
\\
\\
Now we are done!  We have successfully shown that the nasty error term $\textcolor{red}{Error}$ 
satisfies \[\displaystyle\lim_{\vec{h} \to 0}\frac{\left|\textcolor{red}{Error}(\vec{h})\right|}{|\vec{h}|} = 0\]
\\
\\
Thus $g \circ f$ is differentiable at $\mathbf{p}$, and its derivative is given by $Dg(\mathbf{f(p)}) \circ Df(\mathbf{p})$.
\\
\\
QED


	
\end{document}
