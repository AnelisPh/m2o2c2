\documentclass{ximera}
\title{Rigorously}

\begin{document}
	\begin{abstract}
		The derivative approximates the changes in a function to first order accuracy
	\end{abstract}
	
	We are now ready to define the derivative rigorously.  Mimicking our development of the single variable derivative, we define:

\begin{definition}
	Let $f :\R^n \to \R^m$ be a function, and let $\mathbf{p} \in \R^n$.  
	 $f$ is said to be differentiable at $\mathbf{p}$ if there is a linear map $M:\R^n \to \R^m$ such that 
		
		\[ f(\mathbf{p}+\vec{h}) = f(\mathbf{p}) + M(\vec{h})+ \text{Error}_{\mathbf{p}}(\vec{h})\]
		
		with
		
		\[ \displaystyle\lim_{\vec{h} \to 0} \frac{\left|\text{Error}_{\mathbf{p}}(\vec{h})\right|}{\left|\vec{h}\right|} = 0 \].
		
		If $f$ is differentiable at $\mathbf{p}$, there is only one such linear map $M$, which we call the (total) derivative of $f$ at $\mathbf{p}$.  
		
		Verbally,  $M$ is the linear function which makes the error between the function value $f(\mathbf{p}+\vec{h})$ and the affine approximation 
		$f(\mathbf{a})+M(\vec{h})$ go to zero "faster than $\vec{h}$" does.
\end{definition}

This definition is great, but it doesn't tell us how to actually compute the derivative of a differentiable function!  Lets dig a little deeper:

\begin{example}
	Let $f:\mathbb{R^2} \to \mathbb{R^2}$ be defined by $f\left(\verticalvector{x\\y}\right) = \verticalvector{f_1(x,y)\\f_2(x,y)}$. 
	 Assuming $f$ is differentiable at the point $(1,2)$, lets try to compute the derivative there. 
	 Let $M$ be the derivative of $f$ at $(1,2)$.  Then 
	 \begin{align*}\displaystyle\lim_{h \to 0} \frac{\left|f((1,2)+h\verticalvector{1\\0}) - f(\verticalvector{1\\2}) - M(h\verticalvector{1\\0})\right|}{\left|h\verticalvector{1\\0}\right|} &= 0\\
	  \displaystyle\lim_{h \to 0}\left| \frac{f(1+h,2) - f(1,2) - hM(\verticalvector{1\\0})}{h} \right| &= 0\\
	  \displaystyle\lim_{h \to 0}\left| \frac{f(\verticalvector{1+h\\2}) - f(\verticalvector{1\\2})}{h} -M(\verticalvector{1\\0})\right| &= 0
	  \end{align*}
	  so
	  \begin{align*}
	  M(\verticalvector{1\\0}) &= \displaystyle\lim_{ h \to 0 } \frac{f(1+h,2) - f(1,2)}{h}\\
	  M(\verticalvector{1\\0}) &= \displaystyle\lim_{ h \to 0 } \verticalvector{\frac{f_1(1+h,2) - f_1(1,2)}{h}\\ \frac{f_2(1+h,2) - f_2(1,2)}{h}}
	 \end{align*}
	 
	 But each of the remaining quantities are derivatives of \textbf{one} variable functions!  In particular, we have that
	 
	 $M(\verticalvector{1\\0}) = \verticalvector{\frac{d}{dx}(f_1(x,2))\big|_{x=1} \\ \frac{d}{dx}(f_2(x,2))\big|_{x=1}}$.  
	 We call these kinds of quantities \textbf{partial derivatives}  because they are part of the derivative.  We will learn more about partial derivatives in the next section.
\end{example}

	Without copying the work in the example above (if you can) try to find $M(0,1)$.
	
\begin{free-response}

 \begin{align*}
 	  \displaystyle\lim_{h \to 0} \frac{\left|f((1,2)+h\verticalvector{0\\1}) - f(1,2) - M(h\verticalvector{0\\1})\right|}{\left|h \verticalvector{0\\1}\right|} &= 0\\
	  \displaystyle\lim_{h \to 0}\left| \frac{f(1,2+h) - f(1,2) - hM(\verticalvector{0\\1})}{h} \right| &= 0\\
	  \displaystyle\lim_{h \to 0}\left| \frac{f(1,2+h) - f(1,2)}{h} -M(\verticalvector{0\\1})\right| &= 0
	  \end{align*}
	  so
	  \begin{align*}
	  M(\verticalvector{0\\1}) &= \displaystyle\lim_{ h \to 0 } \frac{f(1,2+h) - f(1,2)}{h}\\
	  M(\verticalvector{0\\1}) &= \displaystyle\lim_{ h \to 0 } \verticalvector{\frac{f_1(1,2+h) - f_1(1,2)}{h} \\ \frac{f_2(1,2+h) - f_2(1,2)}{h}}\\
	  M(\verticalvector{0\\1}) &= \verticalvector{\frac{d}{dy} \left( f_1(1,y)\right)\big|_{y=2}\\ \frac{d}{dy} \left( f_2(1,y)\right)\big|_{y=2}}
	 \end{align*}
\end{free-response}

This question and the previous example show that the matrix of the derivative of $f$ at $(1,2)$ is 
\(
\begin{bmatrix} \frac{d}{dx}(f_1(x,2))\big|_{x=1} & \frac{d}{dy}(f_1(1,y))\big|_{y=2}
\\ 
\frac{d}{dx}(f_2(x,2))\big|_{x=1} & \frac{d}{dy}(f_2(1,y))\big|_{y=2}\end{bmatrix}
\)

\begin{question}
	Use the results of this question and the previous example  to find the matrix of the derivative of  $f\left(\verticalvector{x\\y}\right) = \verticalvector{x^2+y^2\\xy}$
	at the point \((1,2)\).
	\begin{solution}
		\begin{hint}
			In this case  $f_1(x,y) = x^2+y^2$ and $f_2(x,y) = xy$
		\end{hint}
		\begin{hint}
			By the result of the last two exercises, the matrix of the derivative is \(\begin{bmatrix} \frac{d}{dx}(f_1(x,2))\big|_{x=1} & \frac{d}{dy}(f_1(1,y))\big|_{y=2}
			\\ \frac{d}{dx}(f_2(x,2))\big|_{x=1} & \frac{d}{dx}(f_2(1,y))\big|_{y=2}\end{bmatrix}\)
		\end{hint}
		\begin{hint}
			\begin{align*}
			f_1(x,2) &= x^2+2^2\\
			  f_1(1,y) &= 1^2+y^2\\
			    f_2(x,2) &= 2x \\
			    f_2(1,y) &= y
			\end{align*}
		\end{hint}
		\begin{hint}
			\begin{align*}	
			\frac{d}{dx}(f_1(x,2))\big|_{x=1} &= \frac{d}{dx} \left( x^2+2^2\right)\big|_{x=1} = 2x\big|_{x=1}=2\\
			\frac{d}{dx}(f_2(x,2))\big|_{x=1} &= \frac{d}{dx} \left( 2x\right)\big|_{x=1} = 2\big|_{x=1}=2\\
			\frac{d}{dy}(f_1(1,y))\big|_{y=2} &= \frac{d}{dy} \left( 1^2+y^2\right)\big|_{y=2} = 2y\big|_{y=2}=4\\
			\frac{d}{dy}(f_2(1,y))\big|_{y=2} &= \frac{d}{dy} \left(y\right)\big|_{y=2} = 1\big|_{x=1}=1\\
			\end{align*}
		\end{hint}
		\begin{hint}
			Thus the matrix of the derivative is \(\begin{bmatrix} 2&4\\2&1 \end{bmatrix}\)
		\end{hint}
		\begin{matrix-answer}
			correctMatrix = [['2','4'],['2','1']]
		\end{matrix-answer}
	\end{solution}
\end{question}

\end{document}

