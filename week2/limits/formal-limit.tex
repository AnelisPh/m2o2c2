\documentclass{ximera}
\title{ The formal definition of the limit}

\begin{document}

\begin{abstract}
	Limits are defined by formalizing the notion of closeness.
\end{abstract}

This \texbf{optional} section explores limits from a formal and rigorous point of view.   
The level of mathematical maturity required to get through this section is much higher than others. 
If you get through it and understand everything, you can consider yourself ``hardcore.''

\begin{definition}
	Let $U \subset \R^n$.  The \textbf{closure} of $U$, written $\overline{U}$ is defined to be the set of all $\mathbf{p} \in \R^n$ such that
	every solid ball centered at $p$ contains at least one point of $U$.

        Symbolically,
	$$\overline{U} = \{ \mathbf{p} \in \R^n : \text{for all $r>0$ there exists $\mathbf{x} \in U$ so that $|x-p|<r$} \}.$$
\end{definition}

Prove that $U \subset \overline{U}$ for any subset $U$ of $\R^n$.
\begin{free-response}
	
	Let $\mathbf{p} \in U$.  Then for every $r >0$, $\mathbf{p}$ is an element of $U$ whose distance to $\mathbf{p}$ is less than $r$.  In other words,
	since every solid ball centered at $\mathbf{p}$ must contain $\mathbf{p}$, and $\mathbf{p}$ is in $U$, then $\mathbf{p}$ must be in the closure of $U$. 
	So $\mathbf{p} \in \overline{U}$.
	
\end{free-response}


Prove that the closure of the open unit ball is the closed unit ball.  That is, show that if $U = \{\mathbf{x}:|x|<1\}$, then $\overline{U}  = \{\mathbf{x}:|x|\leq 1\}$.

\begin{free-response}
	 Let $B = \{\mathbf{x}:|x|\leq 1\}$.  We need to see that $\overline{U} = B$.
	 
	 It is easy to see that $B \subset \overline{U}$, since for each point $\mathbf{p} \in B$, either $\mathbf{p} \in U$ (in which case it is in the closure), 
	 or $|\mathbf{p}| = 1$.  In this case, for every $r > 0$, the point $\mathbf{q} = \mathbf{p} - \frac{1}{2r}\mathbf{p}$ is in $U$ and satisfies 
	  $|\mathbf{p} - \mathbf{q}|<r$.
	  
	  On the other hand, if $|\mathbf{p}|>1$, then a solid ball of radius $\frac{|\mathbf{p}|-1}{2}$ centered at $\mathbf{p}$ will not intersect $U$.  So we are done.
\end{free-response}

\begin{definition}
	Let $f:U \to  V$  with $U \subset \R^n$, $V \subset \R^m$ and $\mathbf{p} \in \overline{U}$.  We say that $\lim_{\mathbf{x} \to \mathbf{p}}f(x) = \mathbf{L}$ if for every $\epsilon >0$ we can find a 
	$\delta>0$ so that if $0<|\mathbf{x}-\mathbf{p}|<\delta$ and $\mathbf{x} \in U$, then $|\mathbf{p}-\mathbf{x}|<\epsilon$.
\end{definition}

\begin{definition}
	 Let $f:U \to  V$  with $U \subset \R^n$ and $V \subset \R^m$.  We say that $f$ is \textbf{continuous} at $\mathbf{p} \in U$ if $\lim_{\mathbf{x} \to \mathbf{p}} f(\mathbf{x}) = f(\mathbf{p})$. 
\end{definition}

Prove, using the $\epsilon$-$\delta$ definition of the limit, that $f:\R^2 \to \R$ defined by $f(x,y) = xy$ is continuous everywhere.

\begin{free-response}
	Let $\mathbf{p} = (a,b)$.   Let $\epsilon>0$ be given.  Without loss of generality, assume $a,b \geq 0$.
	
	We work "backwards":
	
	\begin{align*}
		&|xy-ab|<\epsilon\\
		&\impliedby |[(x-a)+a][(y-b)+b]-ab| <\epsilon \\
		&\impliedby |(x-a)(y-b)+a(y-b) +b(x-a)| <\epsilon \\
		&\impliedby |x-a||y-b|+a|y-b|+a|x-a|< \epsilon \text{ by the triangle inequality}\\
		&\impliedby \begin{cases}
			|x-a||y-b| < \frac{\epsilon}{3}\\
			a|y-b| < \frac{\epsilon}{3}\\
			b|x-a| < \frac{\epsilon}{3}
			\end{cases}
	\end{align*}
	
	Now it is easy to arrange that $a|y-a|$  and $b|x-a|$ are less than $\frac{\epsilon}{3}$.  If $a=0$, or $b=0$, 
	you do not have to do anything to get that condition satisfied, but otherwise $|x-a| \leq \frac{\epsilon}{3b}$ is implied by $|(x,y) - (a,b)| \leq \frac{\epsilon}{3b\sqrt{2}}$
	and  $|y-a| \leq \frac{\epsilon}{3a}$ is implied by $|(x,y) - (a,b)| \leq \frac{\epsilon}{3a\sqrt{2}}$.
	
	$|x-a||y-b| < \frac{\epsilon}{3}$ is implied by $|(x,y)-(a,b)| \leq \frac{\sqrt{\epsilon}}{\sqrt{3}}$.  
	So if we let $\delta = min(\frac{\epsilon}{3b\sqrt{2}}, \frac{\epsilon}{3a\sqrt{2}}, \frac{\sqrt{\epsilon}}{\sqrt{3}})$ 
	we are done.
\end{free-response}

Of course, this fact---that $(x,y) \mapsto xy$ is continuous---is something you probably believe intuitively.  ``Wiggling two numbers by a little bit doesn't affect their product by very much.''  Making that intuition \textit{precise} obviously took some work in this activity.

\end{document}
