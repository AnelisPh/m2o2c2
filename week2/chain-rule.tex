\documentclass{ximera}

\title{The Chain Rule}

\begin{document}

\begin{abstract}
	Differentiating a composition of functions is the same as composing the derivatives of the functions.
\end{abstract}

The chain rule of single variable calculus tells you how the derivative of a composition of functions relates to the derivatives of each of the original functions.
The chain rule of multivariable calculus will work analogously.

\begin{question}
	Let $f:\R^2 \to \R^3$ and $g:\R^3 \to \R$.  The only things you know about $f$ are that $f(2,3) = (3,0,0)$ and $Df(2,3)$ 
	has matrix \(\begin{bmatrix}  4 & -1 \\2& 3\\ 0 &1\end{bmatrix}\).  The only thing you know about $g$ is that $g(3,0,0) = 4$ and $Dg(3,0,0)$ has matrix
	\(\begin{bmatrix} 4&5&6 \end{bmatrix}\).  
	\begin{solution}
		\begin{hint}
			$g(f(2+a,3+b)) \approx g\left(f(2,3)+Df(2,3)\left(\verticalvector{a\\b}\right)\right)$ using the linear approximation to $f$ at $(2,3)$ 
		\end{hint}
		\begin{hint}
			\begin{align*}
				g\left(f(2,3)+Df(2,3)\left(\verticalvector{a\\b}\right)\right) & = g((3,0,0)+\begin{bmatrix}  4 & -1 \\2& 3\\ 0 &1\end{bmatrix} \verticalvector{a\\b})\\
				&= g\left( (3,0,0)+ \verticalvector{4a-b\\2a+3b\\b}\right)\\
				&\approx g(3,0,0)+Dg(3,0,0)\verticalvector{4a-b\\2a+3b\\b} \text{ using the linear approximation to $g$ at $(3,0,0)$}\\
				&\approx 4+\begin{bmatrix} 4&5&6 \end{bmatrix}\verticalvector{4a-b\\2a+3b\\b}\\
				&=4+(16a-4b)+(10a+15b)+(6b)\\
				&=4+26a+17b
			\end{align*}
		\end{hint}
		Assuming $a,b \in \R^2$ are small,  $(g \circ f)(2+a,3+b) \approx$ \answer{$4+26a+17b$} 
	\end{solution}
	
	\begin{solution}
		So the matrix of $D(g \circ f)(2,3)$ is 
		\begin{matrix-answer}[name =J]
			correctMatrix  = [['26','17']]
		\end{matrix-answer}
	\end{solution}
	
	\begin{hint}
		$Dg\big|_{f(2,3)} \circ Df\big|_{(2,3)}$ has matrix 
		\begin{align*}
		\begin{bmatrix} 4&5&6 \end{bmatrix}\begin{bmatrix}  4 & -1 \\2& 3\\ 0 &1\end{bmatrix}  &= \begin{bmatrix} 4(4)+5(2)+6(0) & 4(-1)+5(3)+6(1)\end{bmatrix}\\
		&= \begin{bmatrix} 26& 17\end{bmatrix}
		\end{align*}
	\end{hint}
	Notice that $D(g \circ f)(2,3)$ is the same as $Dg\big|_{f(2,3)} \circ Df\big|_{(2,3)}$!  You should really check this to make sure.  
	Look at the hint to see how to do the composition if you need help.
	
\end{question}

The heuristic approximations in the last question lead us to expect the following theorem:

\begin{theorem}
	Let $f:\R^n \to \R^m$ and $g: \R^m \to \R^k$ be differentiable functions, and $\mathbf{p} \in \R^n$.  Then 
	\[
		D(g \circ f)(\mathbf{p}) = Dg(\mathbf{f(p)}) \circ Df(\mathbf{p})
	\]
	
	In other words, the derivative of a composition of functions is the composition of the derivatives of the functions.
\end{theorem}

The trickiest part of the above theorem is remembering that you need to apply $Dg$ at the point $f(\mathbf{p})$.

The proof of this theorem is a little bit beyond what we want to require you to think about in this course, but the essential idea of the proof is just that

\[
	(g \circ f)(\mathbf{p} +\vec{h}) \approx g(f(\mathbf{p}) + Df(\mathbf{p})(\vec{v})) \approx g(f(p)) + Dg(\mathbf{f(p)})(Df(\mathbf{p})(\vec{h}))
\]

You \textbf{should} understand this essential idea, even if you do not understand the full proof.  We cover the full proof in an \textbf{optional section} after this one.

\begin{question}
	Let $f:\R^2 \to \R^2$ be defined by $f(r,t) = (r\cos(t),r\sin(t))$.  Let $g:\R^2 \to \R$ be defined by $g(x,y) = x^2+y^2$. 
	
	Don't let the choice of variable names scare you.
	
	\begin{solution}
		\begin{hint}
			\[\begin{bmatrix}  
			\frac{\partial}{ \partial r} r\cos(t)& \frac{\partial}{ \partial t}  r\cos(t)
			\\  
			\frac{\partial}{ \partial r} r\sin(t)& \frac{\partial}{ \partial t} r\sin(t)
			\end{bmatrix}\]
		\end{hint}
		\begin{hint}
			\[\begin{bmatrix}  
			\cos(t) & -r\sin(t)
			\\  
			\sin(t) & r\cos(t)
			\end{bmatrix}\]
		\end{hint}
		What is the Jacobian of $f$ at $(r,t)$?
		\begin{matrix-answer}[name =J]
			correctMatrix = [['cos(t)', '-r(sin(t))'],['sin(t)','r(cos(t))']]
		\end{matrix-answer}
	\end{solution}
	
	\begin{solution}
		\begin{hint}
			 \[ \begin{bmatrix} \frac{\partial}{\partial x} (x^2+y^2)& \frac{\partial}{\partial y} (x^2+y^2) \end{bmatrix}\]
		\end{hint}
		\begin{hint}
			 \[ \begin{bmatrix} 2x & 2y \end{bmatrix}\]
		\end{hint}
		What is the Jacobian of $g$ at $(x,y)$?
		\begin{matrix-answer}
			correctMatrix = [['2x','2y']]
		\end{matrix-answer}
	\end{solution}
	
	\begin{solution}
		\begin{hint}
			Just plug $(r\cos(\theta),r\sin(\theta))$ into the formula for the Jacobian of $g$ you obtained above.
		\end{hint}
		\begin{hint}
			The answer is \[\begin{bmatrix} 2r\cos(t) &2r\sin(t)\end{bmatrix}\]
		\end{hint}
		What is the matrix of $Dg(\mathbf{r,t})$?
		\begin{matrix-derivative}
			correctMatrix = [['2r(cos(t))','2r(sin(t))']]
		\end{matrix-derivative}
	\end{solution}
	
	\begin{solution}
		What is the matrix of $Dg(\mathbf{(r,t)}) \circ Df(r,t)$?
			\begin{matrix-answer}
				correctMatrix = [['2r','0']]
			\end{matrix-answer}
	\end{solution}
	
	\begin{solution}
		\begin{hint}
			\begin{align*}
				(g \circ f)(r,t) &= g(r\cos(t),r\sin(t))\\
				 &=(r\cos(t))^2+(r\sin(t))^2\\
				 &=r^2(\cos^2(t)+\sin^2(t))\\
				 &=r^2
			\end{align*}
		\end{hint}
		Compute the composite directly: $(g \circ f)(r,t) =$\answer{$r^2$}
	\end{solution}
	
	\begin{solution}
		Compute $D(g \circ f)(r,t)$ directly from the formula for $(g \circ f)$.
		\begin{matrix-answer}
			correctMatrix = [['2r','0']]
		\end{matrix-answer}
	\end{solution}
	
	This example has demonstrated the chain rule in action!  Computing the derivative of the composite was the same as composing the derivatives.
\end{question}

The product rule from single variable calculus can be proved by invoking the chain rule for multivariable functions.  I will supply a basic outline of the proof below:
can you complete this outline to give a full proof?
\\
\\
Let $f,g:\R \to \R$ be two differentiable functions. Let $p:\R \to \R$ be defined by $p(t) = f(t)g(t)$. Let $\text{Both}:\R \to \R^2$ be defined by $\text{Both}(t) = (f(t),g(t))$.  
Finally let $\text{Multiply}:\R^2 \to \R$ be defined by $\text{Multiply}(x,y) = xy$.  Then we can differentiate $\text{Multiply}(\text{Both}(t)) = p(t)$ using the multivariable chain rule.
This should result in the product rule.

\begin{free-response}
	$D\left(\text{Both}\right)$ has the matrix $\verticalvector{f'(t) \\ g'(t)}$ at the point $t \in \R$.
	\\
	\\
	$D\left( \text{Multiply} \right)$ has the matrix \(\begin{bmatrix} y & x\end{bmatrix}\) at the point $(x,y)\in \R^2$
	\\
	\\
	So $p'(t) =  D\left( \text{Multiply} \right)\big|_{\text{Both(t)}}\circ D\left(\text{Both}\right)\big|_{t}$ by the multivariable chain rule.
	\\
	\\
	So
	\begin{align*}
		p'(t) &= D\left( \text{Multiply} \right)\big|_{\text{Both(t)}}\circ D\left(\text{Both}\right)\big|_{t}\\
			&= D\left( \text{Multiply} \right)\big|_{(f(t),g(t))}}\circ D\left(\text{Both}\right)\big|_{t}\\
			&= \begin{bmatrix} g(t) & f(t)\end{bmatrix} \verticalvector{f'(t)\\g'(t)}\\
			&= g(t)f'(t)+f(t)g'(t)
	\end{align*}
	
	This is the product rule from single variable differential calculus.
\end{free-response}


\end{document}