\documentclass{ximera}
\title{Partial Derivatives}

\begin{document}
	\begin{abstract}
		The entries in the Jacobian matrix are partial derivatives
	\end{abstract}
	
	There is a familiar looking formula for the derivative of a differentiable function:
	
	\begin{theorem}
		Let $f:\R^n \to \R^m$ be a differentiable function.  Then
		\[
			D(f)\big|_{\mathbf{p}}(\vec{v}) = \lim_{h \to 0} \frac{f(\mathbf{p}+h\vec{v}) -f(\mathbf{p})}{h}
		\]
	\end{theorem}
	
	Prove this theorem
	
	\begin{free-response}
		By the definition of the derivative, we have that
		
		\begin{align*}
			\lim_{h \to 0} \frac{\left|f(\mathbf{p}+h\vec{v}) - f(\mathbf{p}) - Df(\mathbf{p})(\vec{h\vec{v}})\right|}{|h\vec{v}|} &= 0\\
			\frac{1}{|\vec{v}|}\lim_{h \to 0} \left| \frac{f(\mathbf{p}+h\vec{v}) - f(\mathbf{p}) - hDf(\mathbf{p})(\vec{v})}{h} \right| &=0 \text{ since $\frac{1}{|\vec{v}|}$ is a constant}\\
			\lim_{h \to 0} \left| \frac{f(\mathbf{p}+h\vec{v}) - f(\mathbf{p}){h} - Df(\mathbf{p})(\vec{v})\right| &=0 \text{ since $\frac{1}{|\vec{v}|}$ is a constant}\\
		\end{align*}
		
		We conclude that
		\[
			D(f)\big|_{\mathbf{p}}(\vec{v}) = \lim_{h \to 0} \frac{f(\mathbf{p}+h\vec{v}) -f(\mathbf{p})}{h}
		\]
	\end{free-response}
	
	\begin{question}
		Let $f:\R^2 \to \R^2$ be defined by $f(x,y)=(x^2-y^2,2xy)$.
			\begin{solution}
				\begin{hint}
					\begin{align*}
						  Df\big|_{(3,4)}\left(\verticalvector{-1\\2}\right) &= \lim_{h \to 0} \frac{f\left((3,4)+h\verticalvector{-1\\2}\right) - f(3,4)}{h}\\
						  &= \lim_{h \to 0} \frac{f(3-h,4+2h)-f(3,4)}{h}\\
						  &= \lim_{h \to 0}\frac{1}{h} \left( ((3-h)^2 -(4+2h)^2,2(3-h)(4+2h)) - (-7,24)\right\\
						  &=\lim_{h \to 0}\frac{1}{h} \verticalvector{(3-h)^2 -(4+2h)^2+7\\2(3-h)(4+2h)-24}
					\end{align*}
				\end{hint}
				\begin{hint}
					\begin{align*}
						&=\lim_{h \to 0}\verticalvector{\frac{-22h-3h^2}{h} \\ \frac{4h-4h^2}{h}}\\
						&=\lim_{h \to 0}\verticalvector{-22-3h\\4-4h}\\
						&=\verticalvector{-22\\4}
					\end{align*}
				\end{hint}
				Using the theorem above, compute $Df\big|_{(3,4)}\left(\verticalvector{-1\\2}\right)$
				\begin{matrix-answer}
					correctMatrix = [['-22'],['4']]
				\end{matrix-answer}
			\end{solution}
	\end{question}

	Since the unit directions are especially important we define:
	
	\begin{definition}
		Let $f: \R^n \to \R$ be a (not necessarily differentiable) function.  We define its partial derivative with respect to $x_i$ by
		\[
			\frac{\partial f}{\partial x_i} (\mathbf{p}) = f_{x_i}(\mathbf{p}) := \lim_{h \to 0} \frac{f(\mathbf{p}+h\vec{e}_i)-f(\mathbf{p})}{h}
		\]
		
		In other words, $\frac{\partial f}{\partial x_i} (\mathbf{p})$ is the instantaneous rate of change in $f$ by moving only in the $\vec{e}_i$ direction.
	\end{definition}	
	
	\begin{example}
		There is really only a good visualization of the partial derivatives of a map  $f: \R^2 \to \R$, 
		because this is really the only type of higher dimensional function we can effectively graph.
		
		The partial derivative $\frac{\partial f}{\partial x_1} \left(a,b\right)$ is the slope of the line tangent to the slice of the graph of $f$ with the
		plane $x_2 = b$, as pictured below:
		
		BADBAD PICTURE
		
		Similarly, the partial derivative  $\frac{\partial f}{\partial x_1} \left(a,b\right)$ is the slope of the line tangent to the slice of the graph of $f$ with the
		plane $x_1=a$:
		
		BADBAD PICTURE
		
	\end{example}
	
	Computing partial derivatives is no harder than computing derivatives of single variable functions.  You take a 
	 partial derivative of a function with respect to $x_i$ just by treating all other variables as constants, and taking the derivative with respect to $x_i$.
	
	\begin{question}
		
	Let $f:\R^2 \to \R$ be defined by $f(x,y) = x\sin(y)$.  
	\begin{solution}
		\begin{hint}
			We are trying to compute $\frac{\partial}{\partial x} \left(x \sin(y)\right)\big|_{(a,b)}$
		\end{hint}
		\begin{hint}
			We just differentiate as if $y$ were a constant, so 
			 $\frac{\partial}{\partial x} \left(x \sin(y)\right)\big|_{(a,b)} = \sin(y)\big|_{(a,b)}$
		\end{hint}
		\begin{hint}
			$f_x(a,b) = \sin(b)$
		\end{hint}
		$f_x(a,b) = $ \answer{$sin(b)$}
	\end{solution}
	
	\begin{solution}
	\begin{hint}
			We are trying to compute $\frac{\partial}{\partial y} \left(x \sin(y)\right)\big|_{(a,b)}$
		\end{hint}
		\begin{hint}
			We just differentiate as if $x$ were a constant, so 
			 $\frac{\partial}{\partial y} \left(x \sin(y)\right)\big|_{(a,b)} = x\cos(y)\big|_{(a,b)}$
		\end{hint}
		\begin{hint}
			$f_x(a,b) = a\cos(b)$
		\end{hint}
		$f_y(a,b) = $ \answer{$a(cos(b))$}
	\end{solution}
	
	\end{question}
	
	We have already proven the following theorem in the special case $n=m=2$ in the previous activity.  Proving it in the general case requires no new ideas: 
	 only better notational bookkeeping.
	
	\begin{theorem}
		Let $f:\R^n \to R^m$ be a function with component functions $f_i:\R^n \to R$, for $i=1,2,3,...,m$.  In other words,
		$f(\mathbf{p}) = \verticalvector(f_1(\mathbf{p}),f_2(\mathbf{p}),f_3(\mathbf{p}),.,.,., f_m(\mathbf{p}))$.  If $f$ is differentiable at $\mathbf{p}$,
		 then its Jacobian matrix at $\mathbf{p}$ is 
                 \[
                 \begin{bmatrix}
                   \frac{\partial f_1}{\partial x_1} \left(\mathbf{p}\right) & \frac{\partial f_1}{\partial x_2} \left(\mathbf{p}\right) & \cdots & \frac{\partial f_1}{\partial x_n}\left(\mathbf{p}\right) \\
                   \frac{\partial f_2}{\partial x_1} \left(\mathbf{p}\right) & \frac{\partial f_2}{\partial x_2} \left(\mathbf{p}\right) & \cdots & \frac{\partial f_2}{\partial x_n}\left(\mathbf{p}\right) \\
                   \vdots                                                    & \vdots                                                    & \ddots & \vdots \\
                   \frac{\partial f_m}{\partial x_1} \left(\mathbf{p}\right) & \frac{\partial f_m}{\partial x_2} \left(\mathbf{p}\right) & \cdots & \frac{\partial f_m}{\partial x_n}\left(\mathbf{p}\right) 
                 \end{bmatrix}
                 \]
		  
		  More compactly, we might write
		  
		  \[
		  	\frac{\partial f_i}{\partial x_j} \left( \mathbf{p} \right)
		  \]
		  

	\end{theorem}
	
	Try to prove this theorem.  Using the more compact notation will be helpful.  Follow along the proof we developed together in the last section!
	\begin{free-response}
		By the definition of the derivative,  we have 
		\begin{align*}
		 \lim_{h \to 0} \frac{\left| f(\mathbf{p}+h\vec{e_i}) - f(\mathbf{p}) - M(h\vec{e_i}) \right|}{\left| h\vec{e_i}\right|} &= 0\\
		 \lim_{h \to 0} \frac{\left| f(\mathbf{p}+h\vec{e_i}) - f(\mathbf{p}) - hM(\vec{e_i}) \right|}{|h|} &= 0\\
		 \lim_{h \to 0} \left| \frac{f(\mathbf{p}+h\vec{e_i}) - f(\mathbf{p}) - hM(\vec{e_i})}{h} \right|&= 0\\
		  \lim_{h \to 0} \left| \frac{f(\mathbf{p}+h\vec{e_i}) - f(\mathbf{p})}{h} - M(\vec{e_i}) \right|&= 0
		 \end{align*}
		 
		 So $\lim_{h \to 0}  \frac{f(\mathbf{p}+h\vec{e_i}) - f(\mathbf{p})}{h}  = M (\vec{e}_i)$.  But for this to be true, the $j^{th}$ row of each side must be equal, so 
		 
		\[\lim_{h \to 0}  \frac{f_j(\mathbf{p}+h\vec{e_i}) - f_j(\mathbf{p})}{h}  = M _{ji}\]
		
		But the quantity on the left hand side is $\frac{\partial f_j}{\partial x_i}\big|_{\mathbf{p}}$
	\end{free-response}
	
	\begin{question}
		Let $f:\R^3 \to \R^2$ be defined by $f(x,y,z) = (x^2+y+z^3,xy+yz^2)$.  
		\begin{solution}
		\begin{hint}
				The Jacobian Matrix is \(\begin{bmatrix} \frac{\partial f_1}{\partial x} & \frac{\partial f_1}{\partial y} & \frac{\partial f_1}{\partial z} \\
																				\frac{\partial f_2}{\partial x} & \frac{\partial f_2}{\partial y} & \frac{\partial f_2}{\partial z}\end{bmatrix}\)
		\end{hint}
		\begin{hint}
			As an example, $\frac{\partial f_2}{\partial z}\end{bmatrix} = \frac{\partial}{\partial z} xy+yz^2 = 2yz$.  Remember that we 
			just differentiate with respect to $z$, treating $x$ and $y$ as constants. 
		\end{hint}
		\begin{hint}
			\begin{align*}
				\frac{\partial f_1}{\partial x} &=  2x\\ 
				\frac{\partial f_1}{\partial y} &= 1\\ 
				\frac{\partial f_1}{\partial z} &= 3z^2\\ 
				\frac{\partial f_2}{\partial x} &= y\\ 
				\frac{\partial f_2}{\partial y} &= x+z^2\\ 
				\frac{\partial f_2}{\partial z} &= 2yz
			\end{align*}
		\end{hint}
		\begin{hint}
			The Jacobian matrix is \(
			\begin{align*}
				2x & 1 & 3z^2\\
				y&x+z^2 & 2yz
			\end{align*}
			\)
		\end{hint}
		What is the Jacobian Matrix of $f$?  This should be a matrix valued function of $x,y,z$.
		\begin{matrix-answer}[name=M]
			correctMatrix = [['2x','1','3z^2'],['y','x+z^2','2yz']]
		\end{matrix-answer}
		\end{solution}
	\end{question}
		
   The formula for the derivative $Df(\mathbf{p})(\vec{v}) = \lim_{h \to 0} \frac{f(\mathbf{p}+h\vec{v})-f(\mathbf{p})}{h}$ looks a lot more familiar
   than our definition. You might be asking why we didn't take this formula as our definition of the derivative.  After all, we usually take something that looks like 
	this as our definition in single variable calculus.
	
	In the following two $\textbf{optional}$ exercises you will find out why.
	\\
	\\
	Find a function $f:\R^2\to \R$ such that at $(0,0)$, $M(\vec{v}) = \lim_{h \to 0}\frac{f((0,0)+h\vec{v}) - f(0,0)}{h}$ exists for every vector $\vec{v} \in \R^2$, but 
	$M$ is not a linear map!
	
	\begin{hint}
		Try showing that the function 
		
		\[f(x,y) = \begin{cases} 
			\frac{x^3}{x^2+y^2} \text{ if $(x,y) = (0,0)$} \\
			0 \text{ if $(x,y) = (0,0)$}
			\end{cases}\] 
			
			has the desired properties.
	\end{hint}
	\begin{free-response}
		Let 
		
		\[f(x,y) = \begin{cases} 
			\frac{x^3}{x^2+y^2} \text{ if $(x,y) = (0,0)$} \\
			0 \text{ if $(x,y) = (0,0)$}
			\end{cases}\] 
			
		Then for any vector $\vec{v} = \verticalvector{a\\b}$, we have 
		
		\begin{align*}
			M(\vec{v}) &= \lim_{h \to 0}\frac{f((0,0)+h\vec{v}) - f(0,0)}{h}\\
							&= \lim_{h \to 0}\frac{f(ha,hb)}{h}\\
							&=\lim_{h \to 0}\frac{h^3a^3}{h(h^2a^2+h^2b^2)}\\
							&=\lim_{h \to 0}\frac{a^3}{a^2+b^2}\\
							&=\frac{a^3}{a^2+b^2}
		\end{align*}
		
		So $M(\verticalvector{a\\b}) = \frac{a^3}{a^2+b^2}$.  This is certainly not a linear function from $\R^2 \to \R$
			
	\end{free-response}
	
	So this formula cannot serve as a good definition of the derivative, because it does not have to produce linear functions.  What if we require that the function is linear as well?
	Even then, it is no good:
	\\
	\\
	Find a function $f:\R^2 \to \R$ such that at $(0,0)$, $M(\vec{v}) = \lim_{h \to 0}\frac{f((0,0)+h\vec{v}) - f(0,0)}{h}$ exists for every vector $\vec{v} \in \R^2$ and 
	the function $M$ defined this way is linear, but never the less, $f$ is \textbf{not} differentiable at $(0,0)$.
	
	\begin{hint}
		Try the function \[f(x,y) = \begin{cases}
			1 \text{ if $y=x^2$ and $(x,y) \neq (0,0)$}\\
			0 \text{else}
		\end{cases}\]
	\end{hint}
	
	\begin{free-response}
	Let 
	\[f(x,y) = \begin{cases}
			1 \text{ if $y=x^2$ and $(x,y) \neq (0,0)$}\\
			0 \text{else}
		\end{cases}\]
		
	Let $\vec{v} = \verticalvector{a\\b}$.  Then 
	
	\begin{align*}
			M(\vec{v}) &= \lim_{h \to 0}\frac{f((0,0)+h\vec{v}) - f(0,0)}{h}\\
							&= \lim_{h \to 0}\frac{f(ha,hb)}{h}\\
		\end{align*}
		
		Now, the intersection of the line $t \mapsto (ta,tb)$ with the parabola $y=x^2$ happens when $tb = t^2a^2$, i.e. when $t = \pm \frac{\sqrt{b}}{|a|}$.  So as long as
		we choose $h$ smaller than that, we know that $(ha,hb)$ is not on the parabola $y=x^2$.  Hence $f(ha,hb) = 0$ for small enough $h$.
		
		Thus $M(\vec{v}) = 0$ for all $\vec{v}\in \R^2$.
		
		This definitely \textbf{is} a linear function, but $f$ is not differentiable at $(0,0)$ using our definition, since
		
		\[\lim_{\vec{h} \to 0} \frac{\left|f((0,0)+\vec{h}) - f(0,0) - M(\vec{h})\right|}{|\vec{h}|} =  \lim_{\vec{h} \to 0} \frac{\left|f(\vec{h})\right|}{|\vec{h}|} \] 
		
		does not exist, since taking $\vec{h}$ on the parabola $y=x^2$ yields the limit $\lim_{t \to 0} \left| \frac{f(t,t^2)}{t}\right| = \lim_{t \to 0} \frac{1}{|t|}$, which
		diverges to $\infty$. 
		
	\end{free-response}
\end{document}