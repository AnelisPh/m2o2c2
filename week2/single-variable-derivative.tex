\documentclass{ximera}

\title{Single variable derivative, redux}

\begin{document}

\begin{abstract}
The derivative is the slope of the best linear approximation.
\end{abstract}
	
Our goal is to define the derivative of a multivariable function, but first we will recast the derivative of a single variable function in a
manner which is ripe for generalization.
	
The derivative of a function $f:\R \to \R$ at a point $x = a$ is the ``instantaneous rate of change'' of $f(x)$ with respect to $x$.
In other words, 
	
$f(a + \Delta x) \approx f(a) +f'(a)\Delta x$.
	
This is really the essential thing to understand about the derivative.
	
\begin{question}
  Let $f$ be a function with $f(3)  = 2$, and $f'(3) = 5$.
  \begin{solution}
    \begin{hint}
      $f(3.01) \approx f(3)+f'(3)(0.01)$
    \end{hint}
    \begin{hint}
      $\approx 2+5(0.01)$
    \end{hint}
    \begin{hint}
      $\approx 2.05$
    \end{hint}
    Then $f(3.01) \approx $ \answer{2.05}
  \end{solution}
\end{question}
	
\begin{question}
  Let $f$ be a function with $f(4)=2$ and $f(4.2) = 2.6$.  
  
  \begin{solution}
    \begin{hint}
      $f(4.2) \approx f(4)+f'(4)(0.2)$
    \end{hint}
    \begin{hint}
      $2.6 \approx 2+f'(4)(0.2)$
    \end{hint}
    \begin{hint}
      $f'(4) \approx \frac{2.6 - 2}{0.2}$
    \end{hint}
    \begin{hint}
      $f'(4) \approx 3$
    \end{hint}
    Then $f'(4) \approx$ \answer{3}
  \end{solution}
\end{question}

We have not made precise what we mean the approximate sign.  After all, if $\Delta x$ is small enough and $f$ is continuous, 
$f(a+\Delta x)$ will be close to $f(a)$, but we do not want to say that the derivative is always zero.  We will make the  $\approx$ sign precise by asking that the difference 
between the actual value and the estimated value goes to zero faster than $\Delta x$ goes to zero.

\begin{definition}
  Let $f: \R \to \R$ be a function, and let $a \in \R$.  $f$ is said to be differentiable at $x=a$ if there is a number $m$ such that 
  
  \[ f(a+\Delta x) = f(a) + m\Delta x + \text{Error}_a(\Delta x)\]
  
  with
  
  \[ \lim_{\Delta x \to 0} \frac{\left|\text{Error}_a(\Delta x)\right|}{\left|\Delta x\right|} = 0 \].
  
  If $f$ is differentiable at $a$, there is only one such number $m$, which we call the derivative of $f$ at $a$.  
  
  Verbally,  $m$ is the number which makes the error between the function value $f(a+\Delta x)$ and the linear approximation $f(a)+m\Delta x$ go to zero 
  ``faster than $\Delta x$'' does.
\end{definition}

This definition looks more complicated than the usual definition (and it is!), but it has the advantage that it will 
generalize directly to the derivative of a multivariable function.
	

Confirm that for $f(x)=x^2$, we have $f'(2)=4$ using our definition of the derivative.
	
\begin{free-response}
  \[
  f(2+\Delta x) = (2+\Delta x)^2 = 2^2+2(2)\Delta x + (\Delta x)^2
  \]
  So we have $f(2+\Delta x) = f(2)+4\Delta x + \text{Error}(\Delta(x))$, where $\text{Error}(\Delta x) = (\Delta x)^2$
  
  \begin{align*}
    \lim_{\Delta x \to 0} \frac{\text{Error}(\Delta x)}{\Delta x}\\
    &=\lim_{\Delta x \to 0} \frac{(\Delta x)^2}{\Delta x} \\
    &= \lim_{\Delta x \to 0} \left(\Delta x  \right)\\
    &= 0
  \end{align*}
  
  Thus, 
  
  $f'(2) = 2(2) =4$, according to our new definition!
\end{free-response}

Show the equivalence of our definition of the derivative with the ``usual'' definition.  That is, show that the number $m$ in our definition satisfies
$m = \lim_{\Delta x \to 0}\frac{f(a+\Delta x)-f(a)}{\Delta x}$.  This also shows the uniqueness of $m$.

\begin{free-response}
  Let $f$ be differentiable (in the sense above) at $x=a$, with derivative $m$.  Then 
  
  \[ \lim_{\Delta x \to 0} \frac{\left|\text{Error}_a(\Delta x)\right|}{\left|\Delta x\right|} = 0 \]
  
  where $\text{Error}_a(\Delta x)$ is defined by $f(a+\Delta x) = f(a) + m\Delta x + \text{Error}_a(\Delta x)$, i.e. $\text{Error}_a(\Delta x) = f(a+\Delta x) - f(a) -m\Delta x$.
  
  So 
  
  \begin{align*}
    \lim_{\Delta x \to 0} \frac{\left|f(a+\Delta x) - f(a) -m\Delta x\right|}{\left|\Delta x\right|} &= 0\\
    \lim_{\Delta x \to 0} \left| \frac{f(x+\Delta x) -f(a)}{\Delta x} - m\right| &= 0\\
  \end{align*}
  
  But this implies that
  
  \[m = \lim_{\Delta x \to 0}\frac{f(a+\Delta x)-f(a)}{\Delta x}\]
  
  So our definition of the derivative agrees with the  ``usual'' definition
  
\end{free-response}

\end{document}
