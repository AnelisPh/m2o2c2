\documentclass{ximera}
\title{Numerical integration}
\begin{document}

\begin{abstract}
  Integrate a covector field.
\end{abstract}

\begin{exercise}
  Suppose we have a one-form expressed as a Python function, e.g., \texttt{omega} (which we will often write as $\omega$) which takes a point (expressed as a list) and returns a $1 \times n$ matrix.  For example, perhaps we have that \texttt{omega([7,2,5])} is \texttt{[[5,3,2]]}.

  Let's only consider the case $n = 2$, and suppose that $\omega$ is
  the derivative of some mystery function $f : \R^2 \to \R$.  If we
  have access to \texttt{omega} and we know that $f(3,2) = 5$, can we
  approximate $f(4,3)$?  How might we go about this?

  We can take a path from $(3,2)$ to $(4,3)$, and break it up into
  small pieces; on each piece, we can use the derivative to
  approximate how a small change to the input will affect the output.
  And repeat.

  Do this in Python.

  \begin{solution}
    \begin{python}
# suppose the derivative of f is omega, and f(3,2) = 5.
# so omega([3,2]) is (perhaps) [[-4,3]].
#
# integrate(omega) is an approximation to the value of f at (4,3).
#
def integrate(omega):
  return # the value

def validator():
  return abs(integrate( lambda p: [[2*p[0] - p[1], -p[0] + 1]] ) - 7.0) < 0.05
    \end{python}
  \end{solution}

How did you move from $(3,2)$ to $(4,3)$?  Did it matter which path you walked along?

\end{exercise}

\end{document}
