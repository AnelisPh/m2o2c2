\documentclass{ximera}
\title{One forms}

\begin{document}
	\begin{abstract}
		One forms are covector fields.
	\end{abstract}
	
	In this section we just want to introduce you to some new notation and terminology which will be helpful to keep in mind for the next course, which
	will cover multivariable integration theory.
	
	As we observed in the last section, the derivative of  a function $f:\R^n \to \R$ assigns a covector to each point in $\R^n$.  In particular,
	$Df\big|_{\mathbf{p}} : \R^n \to \R$ is the covector whose matrix is the row 
	\(\begin{bmatrix} \frac{\partial f}{\partial x_1}\big|_{\mathbf{p}} & \frac{\partial f}{\partial x_2}\big|_{\mathbf{p}} & .&.&.& \frac{\partial f}{\partial x_n}\big|_{\mathbf{p}}\end{bmatrix}\)
	
	\begin{definition}
		A covector field, also known as a differential $1$-form, is a function which takes points in $\R^n$ and returns a covector on $\R^n$.  In other words, it is a covector
		valued function.  We can always write any covector field $\omega$ as
		
		$\omega(\mathbf{x}) = \begin{bmatrix} f_1(\mathbf{x}) &  f_2(\mathbf{x}) & .& .& .&  f_n(\mathbf{x})  \end{bmatrix}$ for $n$ functions $f_i:\R^n \to \R$.
	\end{definition}
	
	The derivative of a function $f:\R^n \to \R$ is the quintessential example of a $1-$form on $\R^n$.
	
	\begin{question}
		Let $f:\R^3 \to \R$ be the function $f(x,y,z)= y$. 
		\begin{solution}
		\begin{hint}
			The Jacobian of $f$ is  \begin{bmatrix} 0&1&0 \end{bmatrix} everywhere
		\end{hint}
		 What is the matrix for $Df$ at the point $(a,b,c)$?
		 	\begin{matrix-answer}
		 		correctMatrix = [['0','1','0']]
		 	\end{matrix-answer}
		 \end{solution}
	\end{question}
	
	Generalizing the result of the previous question, we see that if $\pi_i:\R^n \to \R$ is defined by $\pi_i(x_1,x_2,...,x_n) = x_i$, then $D(\pi_i)$ will
	be the row $\begin{bmatrix} 0&0&.&.&0&1&0&.&.&0\end{bmatrix}$, where the $1$ appears in the $i^{th}$ slot.
	
	We introduce the notation $dx_i$ for the covector field $D(\pi_i)$.  So we can rewrite any covector field 
	$\omega(\mathbf{x}) = \begin{bmatrix} f_1(\mathbf{x}) &  f_2(\mathbf{x}) & .& .& .&  f_n(\mathbf{x})  \end{bmatrix}$ for $n$ functions $f_i:\R^n \to \R$
	in the form $\omega(\mathbf{x}) = f_1(\mathbf{x})dx_1+f_2(\mathbf{x})dx_2+...+f_3(\mathbf{x})dx_n$.
	
	It turns out that $1-$forms, not functions, are the appropriate objects to integrate along curves in $\R^n$. 
	The sequel to this course will focus on the integration of differential forms:  we will not touch on it in this course.
\end{document}